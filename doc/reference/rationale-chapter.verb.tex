There are hundreds of programming languages in active use, and many more are
being designed each year. It is therefore very hard to justify the development
of yet another language. Nevertheless, this is what I attempt to do here. My
argument rests on two claims:
\begin{itemize}
\item[] {\em Claim 1:} The raise in importance of web services and
other distributed software is a fundamental paradigm
shift in programming. It is comparable in scale to the shift 20 years ago
from character-oriented to graphical user interfaces.
\item[] {\em Claim 2:} That paradigm shift will provide demand
for new programming languages, just as graphical user interfaces
promoted the adoption of object-oriented languages.
\end{itemize}
To back up the first claim, one observes that web services and other
distributed software increasingly tend to communicate using structured or
semi-structured data. A typical example is the use of XML to describe data
managed by applications as well as the messages between applications. This
tends to affect the role of a program in a fundamental way. Previously,
programs could be seen as objects that reacted to method calls and in turn
called methods of other objects. Some of these method calls might originate
from users while others might originate from other computers via remote
invocations.  These method calls have simple unstructured parameters or object
references as arguments.  Web services, on the other hand, communicate with
each other by transmitting asynchronous messages that carry structured
documents, usually in XML format. Programs then conceptually become {\em tree
transformers} that consume incoming message documents and produce outgoing
ones.

To back up the second claim, one notes that today's object-oriented languages
are not very good tools for analyzing and transforming the trees which
conceptually model XML data. Because these trees usually contain data but no
methods, they have to be decomposed and constructed from the ``outside'', that is
from code which is external to the tree definition itself. In an
object-oriented language, the ways of doing so are limited. The most common
solution is to represent trees in a generic way, where all tree nodes are
values of a common type.  This approach is used in DOM, for instance. It
facilitates the implementation of generic traversal functions, but forces
applications to operate on a very low conceptual level, which often loses
important semantic distinctions that are present in the XML data. A
semantically more precise alternative would be to use different internal types
to model different kinds of nodes.  But then tree decompositions require the use
of run-time type tests and type casts to adapt the treatment to the kind of
node encountered. Such type tests and type casts are generally not considered
good object-oriented style. They are rarely efficient, nor easy to use.  By
contrast, tree transformation is the natural domain of functional
languages. Their algebraic data types, pattern matching and higher-order
functions make these languages ideal for the task. It's no wonder, then, that
specialized languages for transforming XML data such as XSLT are
functional.

Web services can be constructed using such a transformation language
together with some middleware framework such as Corba to handle
distribution and an object-oriented language for the ``application
logic''.  The downside of this approach is that the necessary amount
of cross-language glue can make applications cumbersome to write,
verify, and maintain.  Better productivity and trustworthiness is
achievable using a single notation and conceptual framework that would
express object-oriented, concurrent, as well as functional aspects of
an application.  Hence, a case for so called ``multi-paradigm''
languages can be made. But one needs to be careful not to simply
replace cross-language glue by awkward interfaces between different
paradigms within the language itself. The benefits of integration are
realized fully only if the common language achieves a meaningful
unification of concepts rather than being merely an agglutination of
different programming paradigms.  This is what we try to achieve with
Scala\footnote{Scala stands for ``Software Composition and
Architecture LAnguage''.}.

Scala is an object-oriented and functional language with clear
semantic foundations.
\begin{itemize}
\item[] {\em Object-oriented:}
Scala is a pure object-oriented language in
the sense that every value is an object. Types and behavior of objects are
described by classes. Classes can be composed using mixin composition.
Scala is designed to interact well with mainstream object-oriented
languages, in particular Java and C\#.
\item[] {\em Functional:}
Scala is a functional language in the sense that every function is a
value. Nesting of function definitions and higher-order functions are
naturally supported. Scala also supports a general notion of pattern
matching which can model the algebraic types used in many functional
languages.
%\item[] {\em Concurrent:}
%Scala is a concurrent language in the sense that it supports
%lightweight threads with flexible constructs for message passing and
%process synchronization. These constructs are based functional nets, a
%theory which combines the principles of functional programming and
%Petri nets in a small kernel language based on Join calculus.
\item[] {\em Clear semantic foundations:}
The operational semantics of a Scala program can be formulated as a
functional net. Scala has an expressive type system which combines
genericity, subtyping, and a form of intersection types based on mixin
classes. That type system is in turn based on the foundational type
theory of name dependent types.
\end{itemize}

The design of Scala is driven by the desire to unify object-oriented
and functional elements. Here are three examples how this is achieved:
\begin{itemize}
\item
Since every function is a value and every value is an object, it
follows that every function in Scala is an object. Indeed, there is a
root class for functions which is specialized in the Scala standard
library to data structures such as arrays and hash tables.
\item
Data structures in many functional languages are defined using
algebraic data types. They are decomposed using pattern matching.
Object-oriented languages, on the other hand, describe data with class
hierarchies. Algebraic data types are usually closed, in that the
range of alternatives of a type is fixed when the type is defined.  By
contrast, class hierarchies can be extended by adding new leaf
classes.  Scala adopts the object-oriented class hierarchy scheme for
data definitions, but allows pattern matching against values coming
from a whole class hierarchy, not just values of a single type.
This can express both closed and extensible data types, and also
provides a convenient way to exploit run-time type information in
cases where static typing is too restrictive.
\item
Software architecture languages are often based on a formalized notion
of component.  A component can be expressed in Scala as a mixin class
where provided services are defined fields and required services are
abstract fields. Components are assembled using mixin composition. To
make this work in all contexts, defined service fields need to be
constructed lazily on demand. This component/composition design
pattern relies in an essential way on the object-oriented concept of
mixin composition and the functional concept of lazy evaluation.
\end{itemize}
%The theory of functional nets lets us describe both functional and
%concurrent computations in one reduction rule.  Mutable variables can
%be described as special cases of concurrent computations, where a
%variable acts as a concurrent thread that keeps a variable's value
%until the next write operation. Thereby, a conceptual unification of
%the theories underlying functions, state, and concurrency is
%established. The unified theory does not preclude Scala compilers to
%choose specialized, more efficient implementation techniques, however.
%\end{itemize}

%The rest of this report is structured as follows. Chapters
%\ref{sec:simple-examples} to \ref{sec:concurrency} give an informal overview of
%Scala by means of a sequence of program examples.  The remaining
%chapters contain the language definition. The definition is formulated
%in prose but tries to be precise.

At present the report is still preliminary. Many examples remain to be filled
in, and the definition needs to be made more precise and legible. I am grateful
for all comments that help improve it.
