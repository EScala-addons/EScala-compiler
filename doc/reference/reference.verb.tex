% $Id$

\documentclass[11pt]{report}

\usepackage{fleqn,a4wide,modefs,math,prooftree,scaladefs,vquote}

\newcommand{\ifqualified}[1]{}
\newcommand{\iflet}[1]{}
\newcommand{\ifundefvar}[1]{}
\newcommand{\iffinaltype}[1]{}
\newcommand{\ifpackaging}[1]{}
\newcommand{\ifnewfor}[1]{}
\renewcommand{\todo}[1]{{$\clubsuit$\bf todo: #1$\spadesuit$}}
\newcommand{\notyet}{\footnote{not yet implemented.}}
\title{Report on the Programming Language Scala}

\date{\today}

\author{
Martin Odersky
}


\sloppy
\begin{document}
\maketitle

%\todo{`:' as synononym for $\EXTENDS$?}

\chapter{Rationale}

\input{rationale-chapter}

\comment{
\chapter{Change Log}

5 August 2001: Changed from join patterns in function definitions to
when clauses.

5 August 2001: Introduced guard part in case expressions.

5 August 2001: Dropped overload modifier.

5 August 2001: Dropped \verb@=>@ as an alternative for \verb@=@ in
definitions.

5 August 2001: Dropped \verb@nil@ type.

5 August Replaced \verb@&@ operator for parallel composition by
\verb@fork@ function.

5 August 2001: Added chapter on Concurrency (\sref{sec:concurrency}).

7 August 2001: Made explicit in the grammar that block-local
definitions do not have modifiers.

7 August 2001: Introduced permitted but redundant modifier `private'
for guards.

9 August 2001: Replaced qualified identifiers in class definitions by
modifiers \verb@new@ and \verb@override@.

9 August 2001: Tightened rules for member access, so that superclass
ordering is irrelevant. See definition of qualified expansion in
\sref{sec:names}.

9 August 2001: Guarded choice now always picks tectually first enabled
guard, rather than an arbitrary one.

16 August 2001: Unary postfix operators now have lower precedence than
unary infix.

16 August 2001: Use `=' instead of `:=' for assignment.

16 August 2001: Changed scope rules so that scope of a definition
always extends to the whole enclosing block or template, plus an
anti forward-dependency rule for value definitions.

16 August 2001: Changed Grammar to allow repeated data definitions
only with extends clauses of simple type, so that parameter names
cannot be used.

16 August 2001: Changed Grammar to drop NxStat, etc (the context-free language
is not affected by the change).

23 August 2001: Eliminated unboxed types.

23 August 2001: Eliminated recursive value definitions; changed rules
                for pattern value definitions.

26 August 2001: Clarified indentation rules.

26 August 2001: Adapted meaning of \verb@null@ to value types.

26 August 2001: Apply local type inference in class instance expressions.

23 Sep 2001: Changed $\arrow$ in closures to $\Arrow$.

23 Sep 2001: Changed record types to refinements.

25 Oct 2001: Simplified and generalized class and module
design. Classes and modules can now appear anywhere; modules are no
longer parameterized.

25 Oct 2001: Dropped aspects (for the time being).

29 Oct 2001:
  Tuple and function types are now shorthands for predefined
  classes. Tuples are no longer covariant.

29 Oct 2001:
  Introduced $n$-ary functions and changed syntax for function types.

29 Oct 2001:
  Dropped static part of classes. Classes now define a constructor
  function of the same name.

29 Oct 2001:
  Generalized rules for overloading slightly.

29 Oct 2001:
  Dropped modifiers for guards.

29 Oct 2001:
  Disambiguated syntax for pattern matching cases with guards.

2 Nov 2001: Changed private and protected to their meaning in Java.

2 Nov 2001: Introduced packagings.

5 Nov 2001: Dropped value parameters in class aliases.

14 Nov 2001: Fixed rule for subtyping of overloaded
function. Tightened scope of value parameters.

15 Nov 2001: Changed `with' to `if' as pattern guard and simplified
pattern syntax.

15 Nov 2001: Changed `data' to `case class'.

20 Nov 2001: Case classes need no longer be `final'.

30 Nov 2001: Introduced `qualified' classes, with unqualified classes
being the default.

4 Dec 2001: Introduced interfaces (\sref{sec:interfaces})

4 Dec 2001: Changed rules which members are inherited (\sref{sec:members}).

4 Dec 2001: Changed rules for compound types with refinements to match
those for templates (\sref{sec:compound-types}).

6 Dec 2001: Dropped modifiers `virtual' and `new'.

6 Dec 2001: `new' is again mandatory in instance creation expressions.

10 Dec 2001: Dropped concurrency constructs in language.

11 Dec 2001: Changed name from `Funnel' to `Scala'.

18 Dec 2001: Slight change of syntax for anonymous functions.

29 Dec 2001: Case classes define accessors for parameters

6 Feb 2002: Import clauses defined only for stable identifiers.

6 Feb 2002: Dropped restriction that classes may not have
inherited abstract members.

6 Feb 2002: Reclassified `new' as Expr4

6 Feb 2002: Dropped interface keyword

6 Feb 2002: Changed syntax of `if'.

6 Feb 2002: Introduced constructor types.

6 Feb 2002: Eliminated class aliases.

6 Feb 2002: Replaced user-defined value definitions by comprehensions.

6 Feb 2002: Tightened conditions on well-formed base class sequence.

6 Feb 2002: Changed root class hierarchy.

7 Feb 2002: Tightened rules on forward references.

7 Feb 2002: Only prefix operators are now `-', `+', `!'. They are
translated to method calls of their operand.
}

class List[a <: AnyRef](x: a, xs: List[a]) {
  def head = x;
  def tail = xs;
  def this(x: a) = { ... ; this(x, Nil); ... }
  def this()     = this(null, Nil);
}




% 16 Jul 2003 Regular pattern matching (Chapter 8)

\subsection*{Status of This Document}

The present document defines slightly more than what is implemented in
the current compiler. But I have tried to mark all omissions that
still exist.

\part{Language Definition}

\chapter{Lexical Syntax}

This chapter defines the syntax of Scala tokens. Tokens are
constructed from symbols in the following character sets:
\begin{enumerate}
\item Whitespace characters.
\item Lower case letters \verb@`a' | ... | `z'@ and
upper case letters \verb@`A' | ... | `Z' | `$\Dollar$' | `_'@.
\item Digits \verb@`0' | ... | `9'@.
\item Parentheses \verb@`(' | `)' | `[' | `]' | `{' | `}'@.
\item Delimiter characters \verb@`\' | `'' | `"' | `.' | `;' | `,'@.
\item Operator characters. These include all printable ASCII characters
which are in none of the sets above.
\end{enumerate}

These sets are extended in the usual way to unicode\notyet (i.e.\ as in Java).
Unicode encodings \verb@`\uXXXX'@ are also as in Java.

\section{Identifiers}

\syntax\begin{verbatim}
op        \=::= \= special {special} [`_' [id]]
varid     \>::= \> lower {letter $|$ digit} [`_' [id]]
id        \>::= \> upper {letter $|$ digit} [`_' [id]]
          \>  | \> varid
          \>  | \> op
\end{verbatim}

There are two ways to form an identifier. First, an identifier can
start with a letter which can be followed by an arbitrary sequence of
letters and digits. Second, an identifier can be start with a special
character followed by an arbitrary sequence of special characters.
In both cases, the identifier prefix may be immediately followed
by an underscore `\verb@_@' character and another string of characters
that by themselves make up an identifier.  As usual, a longest match
rule applies. For instance, the string

\begin{verbatim}
big_bob++=z3
\end{verbatim}

decomposes into the three identifiers \verb@big_bob@, \verb@++=@, and
\verb@z3@.  The rules for pattern matching further distinguish between
{\em variable identifiers}, which start with a lower case letter, and
{\em constant identifiers}, which do not.


The `\verb@\$@' character is reserved for compiler-synthesized identifiers.
User programs are not allowed to define identifiers which contain `\verb@\$@'
characters.

The following names are reserved words instead of being members of the
syntactic class \verb@id@ of lexical identifiers.

\begin{verbatim}
abstract    case    catch    class    def    
do    else    extends    false    final    
finally    for    if    import    new    
null    object    override    package    private    
protected    return    sealed    super    this    
trait    try    true    type    val    
var    while    with   yield
_    :    =    =>    <-    <:    >:    #    @
\end{verbatim}

The unicode operator `\verb@=>@' has the ascii equivalent
`$=>$', which is also reserved\notyet.

\example
Here are examples of identifiers:
\begin{verbatim}
    x                \=Object        \=maxIndex        \=p2p             \=empty_?
    +   \> +_field
\end{verbatim}

\section{Symbols}

\syntax\begin{verbatim}
symbolLit ::= `\'` id
\end{verbatim}

A symbol literal has the form \verb@'x@ where $x$ is an identifier.
Such a symbol literal is a  shorthand for the application
\begin{verbatim}
scala.Symbol("x")
\end{verbatim}
of the facotry method for the standard case class \verb@Symbol@ to the string "x".

\section{Braces and Semicolons}

A semicolon `\verb@;@' is implicitly inserted after every closing brace
if there is a new line character between closing brace and the next
regular token after it, except if that token cannot legally start a
statement.

The tokens which cannot legally start a statement
are the following delimiters and reserved words:
\begin{verbatim}
else    extends    with    yield    do    as    is
,    .    ;    :    =    =>    <-    <:    >:    #    @    )    ]    }
\end{verbatim}

\section{Literals}

There are literals for integer numbers (of types \verb@Int@ and \verb@Long@),
floating point numbers (of types \verb@Float@ and \verb@Double@), characters, and
strings.  The syntax of these literals is in each case as in Java.

\syntax\begin{verbatim}
literal      \=::= \= intLit
             \>  | \> floatLit
             \>  | \> charLit
             \>  | \> stringLit
	     \>  | \> symbolLit
intLit       \>::= \> ``as in Java''
floatLit     \>::= \> ``as in Java''
charLit      \>::= \> ``as in Java''
stringLit    \>::= \> ``as in Java''
\end{verbatim}

\section{Whitespace and Comments}

Tokens may be separated by whitespace characters (ascii codes 0 to 32)
and/or comments. Comments come in two forms:

A single-line comment is a sequence of characters which starts with
\verb@//@ and extends to the end of the line.

A multi-line comment is a sequence of characters between \verb@/*@ and
\verb@*/@. Multi-line comments may be nested.


\chapter{\label{sec:names}Identifiers, Names and Scopes}

\syntax\begin{verbatim}
  Id              \=::= \= id  |  `+'  |  `-'  |  `!'
  QualId          \>::> \= Id {`.' Id}
\end{verbatim}

Names in Scala identify types, values, functions, and classes which
are collectively called {\em entities}.  Names are introduced by
definitions, declarations (\sref{sec:defs}) or import clauses
(\sref{sec:import}), which are collectively called {\em binders}.

There are two different name spaces, one for types (\sref{sec:types})
and one for terms (\sref{sec:exprs}).  The same name may designate a
type and a term, depending on the context where the name is used.  

A definition or declaration has a {\em scope} in which the entity
defined by a single name can be accessed using a simple name. Scopes
are nested, and a definition or declaration in some inner scope {\em
shadows} a definition in an outer scope that contributes to the same
name space. Furthermore, a definition or declaration shadows bindings
introduced by a preceding import clause, even if the import clause is
in the same block. Import clauses, on the other hand, only shadow
bindings introduced by other import clauses in outer blocks.

\todo{Examples}

A reference to an unqualified (type- or term-) identifier $x$ is bound
by the unique binder, which
\begin{itemize}
\item defines an entity with name $x$ in the same namespace as the
identifier, and
\item shadows all other binders that define entities with name $x$ in that namespace.
\end{itemize}
It is an error if no such binder exists.  If $x$ is bound by an import
clause, then the simple name $x$ is taken to be equivalent to the
qualified name to which $x$ is mapped by the import clause. If $x$ is bound by a definition or declaration,
then $x$ refers to the entity introduced by that
binder. In that case, the type of $x$ is the type of the referenced
entity.

A reference to a qualified (type- or term-) identifier $e.x$ refers to
the member of the type $T$ of $e$ which has the name $x$ in the same
namespace as the identifier. It is an error if $T$ is not an object type
(\sref{def:object-type}). The type of $e.x$ is the member type of the
referenced entity in $T$.

\chapter{\label{sec:types}Types}

\syntax\begin{verbatim}
  Type          \=::= \= Type1 `=>' Type
                \>   |\> `(' [Types] `)' `=>' Type
	        \>   |\> Type1
  Type1         \>::= \> SimpleType {with SimpleType} [Refinement]
  SimpleType   	\>::= \> StableId
	        \>   |\> SimpleType `#' Id
	        \>   |\> Path `.' type
                \>   |\> SimpleType TypeArgs
		\>   |\> `(' Type ')'
  Types	       \>::= \> Type {`,' Type}
\end{verbatim}

We distinguish between first-order types and type constructors, which
take type parameters and yield types. A subset of first-order types
called {\em value types} represents sets of (first-class) values.
Value types are either {\em concrete} or {\em abstract}. Every
concrete value type can be represented as a {\em class type}, i.e.\ a
type designator (\sref{sec:type-desig}) that refers to a 
class\footnote{We assume that objects and packages also
implicitly define a class (of the same name as the object or package,
but inaccessible to user programs).} (\sref{sec:classes}), 
or as a {\em compound type} (\sref{sec:compound-types}) 
consisting of class types and possibly
also a refinement (\sref{sec:refinements}) that further constrains the
types of its members.

A shorthands exists for denoting function types
(\sref{sec:function-types}).  Abstract value types are introduced by
type parameters and abstract type bindings (\sref{sec:typedcl}).
Parentheses in types are used for grouping.

Non-value types capture properties of
identifiers that are not values
(\sref{sec:synthetic-types}).  There is no syntax to express these
types directly in Scala.

\section{Paths}\label{sec:paths}

\syntax\begin{verbatim}
  StableId             \=::= \= Id
                  \>  |\> Path `.' Id
                  \>  |\> [Ident '.'] super `.' Id
  Path            \>::=\> StableId
                  \>  |\> [Ident `.'] this
\end{verbatim}

Paths are not types themselves, but they can be a part of named types
and in that way form a central role in Scala's type system.

A path is one of the following.
\begin{itemize}
\item
The empty path $\epsilon$ (which cannot be written explicitly in user programs).
\item
\verb@C.this@, where \verb@C@ references a class. 
The path \verb@this@ is taken as a shorthand for \verb@C.this@ where 
\verb@C@ is the class directly enclosing the reference. 
\item
\verb@p.x@ where \verb@p@ is a path and \verb@x@ is a stable member of \verb@p@.
{\em Stable members} are members introduced by value or object
definitions, as well as packages.
\item
\verb@C.super.x@ where \verb@C@ references a class and \verb@x@ references a 
stable member of
one of the base classes of \verb@C@. 
The path \verb@super.x@ is taken as a shorthand for \verb@C.super.x@ where 
\verb@C@ is the class directly enclosing the reference. 
\end{itemize}
A {\em stable identifier} is a path which ends in an identifier.

\section{Value Types}

\subsection{Singleton Types}
\label{sec:singleton-type}

\syntax\begin{verbatim}
  SimpleType ::= Path `.' type
\end{verbatim}

A singleton type is of the form \verb@p.type@, where \verb@p@ is a
path.  The type denotes the set of values consisting of
exactly the value denoted by \verb@p@.

\subsection{Type Projection}
\label{sec:type-project}

\syntax\begin{verbatim} 
SimpleType ::=  SimpleType # Id
\end{verbatim}

A type projection \verb@T # x@ references the type member named 
\verb@x@ of type \verb@T@. \verb@T@ must be either a singleton type,
or a non-abstract class type, or a Java class type (in either of the
last two cases, it is guaranteed that \verb@T@ has no abstract type
members).

\subsection{Type Designators}
\label{sec:type-desig}

\syntax\begin{verbatim}
  SimpleType   	::=  StableId
\end{verbatim}

A type designator refers to a named value type. It can be simple or
qualified. All such type designators are shorthands for type projections.

Specifically, the unqualified type name $t$ where $t$ is bound in some
class, object, or package $C$ is taken as a shorthand for
\verb@C.this.type # t@.  If $t$ is not bound in a class, object, or
package, then \verb@t@ is taken as a shorthand for
\verb@$\epsilon$.type # t@.

A qualified type designator has the form \verb@p.t@ where \verb@p@ is
a path (\sref{}) and $t$ is a type name. Such a type designator is
equivalent to the type projection \verb@p.type # x@.

\example 
Some type designators and their expansions are listed below. We assume
a local type parameter \verb@t@, a value \verb@mytable@
with a type member \verb@Node@ and the standard class \verb@scala.Int@, 
\begin{verbatim}
  t                            \=$\epsilon$.type # t
  Int              \>scala.type # Int
  scala.Int        \>scala.type # Int
  mytable.Node     \>mytable.type # Node
\end{verbatim}

\subsection{Parameterized Types}
\label{sec:param-types}

\syntax\begin{verbatim}
  SimpleType      \=::= \= SimpleType TypeArgs
  TypeArgs        \>::= \> `[' Types `]'
\end{verbatim}

A parameterized type $T[U_1, ..., U_n]$ consists of a type designator
$T$ and type parameters $U_1 \commadots U_n$ where $n \geq 1$.  $T$
must refer to a type constructor which takes $n$ type parameters $a_1,
..., a_n$ with lower bounds $L_1, ..., L_n$ and upper bounds $U_1,
..., U_n$.

The parameterized type is well-formed if each actual type parameter
{\em conforms to its bounds}, i.e.\ $L_i\sigma <: T_i <: U_i\sigma$ where $\sigma$
is the substitution $[a_1 := T_1, ..., a_n := T_n]$.

\example\label{ex:param-types}
Given the partial type definitions:

\begin{verbatim}
  class HashMap[a <: Ord, b] { ... }
  class List[a] { ... }
  type I = Ord { ... }
\end{verbatim}

the following parameterized types are well formed:

\begin{verbatim}
  HashMap[I, String]
  List[I]
  List[List[Boolean]]
\end{verbatim}

\example Given the type definitions of \ref{ex:param-types},
the following types are ill-formed:

\begin{verbatim}
  HashMap[I]			                         \=// illegal: wrong number of parameters
  HashMap[List[I], Boolean]  \>// illegal: type parameter not within bound
\end{verbatim}

\subsection{Compound Types}
\label{sec:compound-types}

\syntax\begin{verbatim} 
  Type            \=::= \= SimpleType {with SimpleType} [Refinement]
  Refinement      \>::=\> `{' [RefineStat {`;' RefineStat}] `}'
  RefineStat      \>::=\> Dcl
                  \>  |\> type TypeDef {`,' TypeDef}
                  \>  |\>
\end{verbatim}

A compound type \verb@T$_1$ with ... with T$_n$ {R}@ represents
objects with members as given in the component types $T_1 \commadots
T_n$ and the refinement \verb@{R}@. Each component type $T_i$ must be a
class type and the base class sequence generated by types $T_1
\commadots T_n$ must be well-formed (\sref{sec:basetypes-wf}). A
refinement \verb@{R}@ contains declarations and type
definitions. Each declaration or definition in a refinement must
override a declaration or definition in one of the component types
$T_1 \commadots T_n$. The usual rules for overriding (\sref{})
apply. If no refinement is given, the empty refinement is implicitly
added, i.e. \verb@T$_1$ with ... with T$_n$@ is a shorthand for
\verb@T$_1$ with ... with T$_n$ {}@.
 
\subsection{Function Types}
\label{sec:function-types}

\syntax\begin{verbatim}
  SimpleType         \=::= \= Type1 `=>' Type
                \>   |\> `(' [Types] `)' `=>' Type
\end{verbatim}
The type \verb@(T$_1$, ..., T$_n$) => U@ represents the set of function
values that take arguments of types $T_1 \commadots T_n$ and yield
results of type $U$.  In the case of exactly one argument type
\verb@S => T@ is a shorthand for \verb@(S) => T@.  Function types
associate to the right, e.g.\ \verb@(S) => (T) => U@ is the same as
\verb@(S) => ((T) => U)@.

Function types are shorthands for class types that define \verb@apply@
functions.  Specifically, the $n$-ary function type $(T_1 \commadots
T_n)U$ is a shorthand for the class type
\verb@Function$\,n$[$T_1 \commadots T_n$,U]@. Such class
types are defined in the Scala library for \verb@n@ between 0 and 9 as follows.
\begin{verbatim}
package scala;
trait Function$\,n$[-T$_1$, ..., -T$_n$, +R] {
  def apply(x$_1$: T$_1$, ..., x$_n$: T$_n$): R;
  override def toString() = "<function>";
}
\end{verbatim}
Hence, function types are covariant in their result type, and
contravariant in their argument types.

\section{Non-Value Types}
\label{sec:synthetic-types}

The types explained in the following do not denote sets of values, nor
do they appear explicitely in programs. They are introduced in this
report as the internal types of defined identifiers.

\subsection{Method Types}
\label{sec:method-types}

A method type is denoted internally as $(ps)U$, where $(ps)$ is a
parameter section $(x_1: T_1 \commadots x_n: T_n)$ for some $n \geq 0$
and $U$ is a (value or method) type.  This type represents named
methods that take arguments $x_1 \commadots x_n$ of types $T_1
\commadots T_n$ and that return a result of type $U$.

Method types associate to the right: $(ps_1)(ps_2)U$ is treated as
$(ps_1)((ps_2)U)$.

A special case are types of methods without any parameters. They are
written here $[]T$, following the syntax for polymorphic method types
(\sref{sec:poly-types}). Parameterless methods name expressions that
are re-evaluated each time the parameterless method name is
referenced.

Method types do not exist as types of values. If a method name is used
as a value, its type is implicitly converted to a corresponding
function type (\sref{sec:impl-conv}).

\example The declarations
\begin{verbatim}
def a: Int
def b (x: Int): Boolean
def c (x: Int) (y: String, z: String): String
\end{verbatim}
produce the typings
\begin{verbatim}
a: [] Int
b: (x: Int) Boolean
c: (x: Int) (y: String, z: String) String
\end{verbatim}

\subsection{Polymorphic Method Types}
\label{sec:poly-types}

A polymorphic method type is denoted internally as \verb@[tps]T@ where
\verb@[tps]@ is a type parameter section 
\verb@[a$_1$ <: L$_1$ >: U$_1$, ..., a$_n$ <: L$_n$ >: U$_n$] T@ 
for some $n \geq 0$ and \verb@T@ is a
(value or method) type.  This type represents named methods that
take type arguments \verb@S$_1$, ..., S$_n$@ which
conform (\sref{sec:param-types}) to the lower bounds
\verb@S$_1$, ..., S$_n$@ and the upper bounds
\verb@U$_1$, ..., U$_n$@ and that yield results of type \verb@T@.

\example The declarations
\begin{verbatim}
def empty[a]: List[a]
def union[a <: Comparable[a]] (x: Set[a], xs: Set[a]): Set[a]
\end{verbatim}
produce the typings
\begin{verbatim}
empty \=: [a >: All <: Any] List[a]
union \>: [a >: All <: Comparable[a]] (x: Set[a], xs: Set[a]) Set[a]  .
\end{verbatim}

\subsection{Overloaded Types}
\label{sec:overloaded-types}
\newcommand{\overload}{\la\mbox{\sf and}\ra}


More than one values or methods are defined in the same scope with the
same name, we model

An overloaded type consisting of type alternatives $T_1 \commadots
T_n$ $(n \geq 2)$ is denoted internally $T_1 \overload \ldots
\overload T_n$.

\example The definitions
\begin{verbatim}
def println: Unit;
def println(s: String): Unit = ...;
def println(x: Float): Unit = ...;
def println(x: Float, width: Int): Unit = ...;
def println[a](x: a)(tostring: a => String): Unit = ...
\end{verbatim}
define a single function \verb@println@ which has an overloaded
type.
\begin{verbatim}
println: \= [] Unit $\overload$
	 \> (s: String) Unit $\overload$
	 \> (x: Float) Unit $\overload$
	 \> (x: Float, width: Int) Unit $\overload$
	 \> [a] (x: a) (tostring: a => String) Unit
\end{verbatim}

\example The definitions
\begin{verbatim}
def f(x: T): T = ...;
val f = 0
\end{verbatim}
define a function \verb@f@ which has type \verb@(x: T)T $\overload$ Int@.

\section{Base Classes and Member Definitions}
\label{sec:base-classes}

Types, bounds and base classes of class members depend on the way the
members are referenced.  Central here are three notions, namely:
\begin{enumerate}
\item the notion of the base class sequence of a type $T$,
\item the notion of a type $T$ seen as a member of some class $C$ from some 
      prefix type $S$,
\item the notion of a member binding of some type $T$.
\end{enumerate}
These notions are defined mutually recursively as follows.

1. The {\em base class sequence} of a type is a sequence of class types, 
given as follows.
\begin{itemize}
\item
The base classes of a class type \verb@C@ are the base classes of class
\verb@C@.
\item
The base classes of an aliased type are the base classes of its alias.
\item
The base classes of an abstract type are the base classes of its upper bound.
\item
The base classes of a parameterized type \verb@C[T$_1$, ..., T$_n$]@ are the base classes
of type \verb@C@, where every occurrence of a type parameter $a_i$ 
of \verb@C@ has been replaced by the corresponding parameter type $T_i$.
\item
The base classes of a singleton type \verb@p.type@ are the base classes of
the type of \verb@p@.
\item
The base classes of a compound type
\verb@T$_1$ with ... with T$_n$ with {R}@ is the concatenation of the
base classes of all \verb@T$_i$@'s, except that later base classes replace
earlier base classes which are instances of the same class.
\end{itemize}

2. The notion of a type \verb@T@
{\em seen as a member of some class \verb@C@ from some prefix type
\verb@S@} makes sense only if the prefix type \verb@S@
has a type instance of class \verb@C@ as a base class, say
\verb@S' # C[T$_1$, ..., T$_n$]@. Then we define as follows.
\begin{itemize}
 \item 
  If \verb@S = $\epsilon$.type@, then $T$ seen as a member of $C$ from $S$ is $T$ itself.
 \item Otherwise, if \verb@T@ is the $i$'th type parameter of some class \verb@D@, then
   \begin{itemize}
   \item
   If \verb@S@ has a base class \verb@D[U$_1$, ..., U$_n$]@, for some type parameters
   \verb@[U$_1$, ..., U$_n$]@, then \verb@T@ seen as a member of $C$ from $S$ is $U_i$.
   \item
   Otherwise, if $C$ is defined in a class $C'$, then
   \verb@T@ seen as a member of $C$ from $S$ is the same as $T$ seen as
   a member of $C'$ from $S'$.
   \item
   Otherwise, if $C$ is not defined in another class, then  
   \verb@T@ seen as a member of $C$ from $S$ is $T$ itself.
  \end{itemize}
\item
   Otherwise, 
   if \verb@T@ is the singleton type \verb@D.this.type@ for some class \verb@D@
   then
   \begin{itemize}
   \item
   If \verb@D@ is a subclass of \verb@C@ and 
   \verb@S@ has a type instance of class $D$ among its base classes.
   then \verb@T@ seen as a member of $C$ from $S$ is $S$.
   \item
   Otherwise, if $C$ is defined in a class $C'$, then
   \verb@T@ seen as a member of $C$ from $S$ is the same as $T$ seen as
   a member of $C'$ from $S'$.
   \item
   Otherwise, if $C$ is not defined in another class, then  
   \verb@T@ seen as a member of $C$ from $S$ is $T$ itself.
   \end{itemize}
\item
  If $T$ is some other type, then the described mapping is performed
  to all its type components.
\end{itemize}

If \verb@T@ is a possibly parameterized class type, where $T$'s class
is defined in some other class $D$, and \verb@S@ is some prefix type,
then we use ``\verb@T@ seen from \verb@S@'' as a shorthand for
``\verb@T@ seen as a member of $D$ from $S$.

3. The {\em member bindings} of a type $T$ are all bindings $d$ such that
there exists a type instance of some class $C$ among the base classes of $T$
and there exists a definition or declaration $d'$ in $C$
such that $d$ results from $d'$ by replacing every
type $T'$ in $d'$ by $T'$ seen as a member of $C$ from $T$.

The {\em definition} of a type projection \verb@S # t@ is the member
binding $d$ of the type \verb@t@ in \verb@S@. In that case, we also say
that \verb@S # t@ {\em is defined by} \verb@d@.

\section{Relations between types}

We define two relations between types.
\begin{quote}\begin{tabular}{l@{\tab}l@{\tab}l}
\em Type equivalence & $T \equiv U$ & $T$ and $U$ are interchangeable
in all contexts.
\\
\em Conformance & $T \conforms U$ & Type $T$ conforms to type $U$.
\end{tabular}\end{quote}

\subsection{Type Equivalence}
\label{sec:type-equiv}

Equivalence $(\equiv)$ between types is the smallest congruence\footnote{ A
congruence is an equivalence relation which is closed under formation
of contexts} such that the following holds:
\begin{itemize}
\item 
If $t$ is defined by a type alias \verb@type t = T@, then $t$ is
equivalent to $T$.
\item
If a path $p$ has a singleton type \verb@q.type@, then
\verb@p.type $\equiv$ q.type@.
\item
If \verb@O@ is defined by an object definition, and \verb@p@ is a path
consisting only of package or object selectors and ending in \verb@O@, then
\verb@O.this.type $\equiv$ p.type@.
\item
Two compound types are equivalent if their component types are
pairwise equivalent and their refinements are equivalent. Two
refinements are equivalent if they bind the same names and the
modifiers, types and bounds of every declared entity are equivalent in
both refinements.
\item
Two method types are equivalent if they have equivalent result
types, both have the same number of parameters, and corresponding
parameters have equivalent types as well as the same \verb@def@ or
\verb@*@ modifiers.  Note that the names of parameters do not matter
for method type equivalence.
\item
Two polymorphic types are equivalent if they have the same number of
type parameters, and, after renaming one set of type parameters by
another, the result types as well as lower and upper bounds of
corresponding type parameters are equivalent.
\item
Two overloaded types are equivalent if for every alternative type in
either type there exists an equivalent alternative type in the other.
\end{itemize}

\subsection{Conformance}
\label{sec:subtyping}

The conformance relation $(\conforms)$ is the smallest 
transitive relation that satisfies the following conditions.
\begin{itemize}
\item Conformance includes equivalence. If $T \equiv U$ then $T \conforms U$.
\item For every value type $T$, \verb@scala.All <: T <: scala.Any@. 
\item For every value type \verb@T <: scala.AnyRef@ one has \verb@scala.AllRef <: T@.
\item A type variable or abstract type $t$ conforms to its upper bound and
      its lower bound conforms to $t$. 
\item A class type or parameterized type $c$ conforms to any of its basetypes, $b$.
\item A type projection \verb@T # t@ conforms to \verb@U # t@ if 
      \verb@T@ conforms to \verb@U@.
\item A parameterized type \verb@T[T$_1$, ..., T$_n$]@ conforms to 
      \verb@T[U$_1$, ..., U$_n$]@ if
      the following three conditions hold for $i = 1 \commadots n$. 
      \begin{itemize}
      \item
      If the $i$'th type parameter of $T$ is declared covariant, then $T_i <: U_i$.
      \item
      If the $i$'th type parameter of $T$ is declared contravariant, then $U_i <: T_i$.
      \item
      If the $i$'th type parameter of $T$ is declared neither covariant 
      nor contravariant, then $U_i \equiv T_i$.
      \end{itemize}
\item A compound type \verb@T$_1$ with ... with T$_n$ {R}@ conforms to
      each of its component types \verb@T$_i$@.
\item If $T \conforms U_i$ for $i = 1 \commadots n$ and for every
      binding of a type or value $x$ in $R$ there exists a member binding of
      $x$ in $T$ which is more specific, then $T$ conforms to
      the compound type \verb@T$_1$ with ... with T$_n$ {R}@.  
\item If
	$T'_i$ conforms to $T_i$ for $i = 1 \commadots n$ and $U$ conforms to $U'$ 
        then the method type $(x_1: T_1 \commadots x_n: T_n) U$ conforms to
	$(x_1: T'_1 \commadots x_n: T'_n) U'$.
\item If, assuming 
$L'_1 \conforms a_1 \conforms U'_1 \commadots L'_n \conforms a_n \conforms U'_n$ 
one has $L_i \conforms L'_i$ and $U'_i \conforms U_i$
for $i = 1 \commadots n$, as well as $T \conforms T'$ then the polymorphic type
$[a_1 >: L_1 <: U_1 \commadots a_n >: L_n <: U_n] T$ conforms to the polymorphic type
$[a_1 >: L'_1 <: U'_1 \commadots a_n >: L'_n <: U'_n] T'$.
\item 
An overloaded type $T_1 \overload \ldots \overload T_n$ conforms to each of its alternative types $T_i$.
\item
A type $S$ conforms to the overloaded type $T_1 \overload \ldots \overload T_n$
if $S$ conforms to each alternative type $T_i$.
\end{itemize}

A declaration or definition in some compound type of class type $C$
is {\em more specific} than another
declaration of the same name in some compound type or class type $C'$.
\begin{itemize}
\item
A value declaration \verb@val x: T@ or value definition
\verb@val x: T = e@ is more specific than a value declaration
\verb@val x: T'@ if $T <: T'$.
\item
A type alias
$\TYPE;t=T$ is more specific than a type alias $\TYPE;t=T'$ if
$T \equiv T'$.
\item 
A type declaration \verb@type t >: L <: U@ is more specific that 
a type declaration \verb@type t >: L' <: U'@ if \verb@L' <: L@ and \verb@U <: U'@.
\item
A type or class definition of some type $t$ is more specific than an abstract
type declaration \verb@type t >: L <: U@ if
\verb@L <: t <: U@.
\end{itemize}

The \verb@<:@ relation forms a partial order between types. The {\em
least upper bound} or the {\em greatest lower bound} of a set of types
is understood to be relative to that order.

\section{Type Erasure}
\label{sec:erasure}

A type is called {\em generic} if it contains type arguments or type variables.
{\em Type erasure} is a mapping from (possibly generic) types to
non-generic types. We write $|T|$ for the erasure of type $T$.
The erasure mapping is defined as follows.
\begin{itemize}
\item The erasure of a type variable is the erasure of its upper bound.
\item The erasure of a parameterized type $T[T_1 \commadots T_n]$ is $|T|$.
\item The erasure of a singleton type \verb@p.type@ is the 
      erasure of the type of \verb@p@.
\item The erasure of a type projection \verb@T # x@ is \verb@|T| # x@.
\item The erasure of a compound type $T_1;\WITH;\ldots;\WITH;T_n\{R\}$ is $|T_1|$.
\item The erasure of every other type is the type itself.
\end{itemize}

\section{Implicit Conversions}
\label{sec:impl-conv}

If $S \conforms T$, then values of type $S$ are implicitly {\em
converted} to values type of $T$ in situations where a value of type
$T$ is required. A conversion between two number types in \verb@Int@,
\verb@Long@, \verb@Float@, \verb@Double@ creates a value of the target
type representing the same number as the source.  When used in an
expression, a value of type \verb@Byte@, \verb@Char@, \verb@Short@ is
always implicitly converted to a value of type \verb@Int@.

The following implicit conversions are applied to expressions of
method type that are used as values, rather than being applied to some
arguments.
\begin{itemize}
\item
A parameterless method $m$ of type $[] T$
is converted to type $T$ by evaluating the expression to which $m$ is bound.
\item
An expression $e$ of polymorphic type $[a_1 \extends S_1 \commadots
a_n \extends S_n]T$ which does not appear as the function part of
a type application is converted to type $T$
by determining with local type inference
(\sref{sec:local-type-inf}) instance types $U_1
\commadots U_n$ for the type variables $a_1 \commadots a_n$ and
implicitly embedding $e$ in the type application
$e[U_1 \commadots U_n]$ (\sref{sec:type-app}).
\item
An expression $e$ of monomorphic method type
$(ps_1) \ldots (ps_n) U$ of arity $n > 0$
which does not appear as the function part of an application is
converted to a function type by implicitly embedding $e$ in
the following term, where $x$ is a fresh variable:
\begin{verbatim}
(val $x$ = $e$ ; $(ps_1) \ldots \Arrow \ldots \Arrow (ps_n) \Arrow x;(ps_1);\ldots;(ps_n)$)
\end{verbatim}
This conversion is not applicable to functions with call-by-name
parameters (\sref{sec:parameters}) of type $[]T$, because its result
would violate the well-formedness rules for anonymous functions
(\sref{sec:closures}). Hence, methods with call-by-name
parameters always need to be applied to arguments immediately.
\end{itemize}

\chapter{Basic Declarations and Definitions}
\label{sec:defs}

\syntax\begin{verbatim}
  Dcl             \=::=\= val ValDcl {`,' ValDcl}
                  \>  |\> var VarDcl {`,' VarDcl}
                  \>  |\> def FunDcl {`,' FunDcl}
                  \>  |\> type TypeDcl {`,' TypeDcl}
  Def             \>::=\> val PatDef {`,' PatDef}
		  \>  |\> var VarDef {`,' VarDef}
  	          \>  |\> def FunDef {`,' FunDef}
                  \>  |\> type TypeDef {`,' TypeDef}
	          \>  |\> ClsDef
\end{verbatim}
\iflet{$\LET$ ValDef {`,' ValDef}}

A {\em declaration} introduces names and assigns them types. It can
appear as one of the statements of a class definition
(\sref{sec:templates}) or as part of a refinement in a compound
type (\sref{sec:refinements}).

A {\em definition} introduces names
that denote terms or types. It can form part of a module or class or
it can be local to a block.  Both declarations and definitions produce
{\em bindings} that associate type names with type definitions or
bounds, and that associate term names with types.

The scope of a name introduced by a declaration or definition is the
whole statement sequence containing the binding.  However, there is a
restriction on forward references: In a statement sequence $s_1 \ldots
s_n$, if a simple name in $s_i$ refers to an entity defined by $s_j$
where $j \geq i$, then all statements between and including $s_i$ and
$s_j$ must be pure statements (\sref{sec:statements}).

{\em Pure} definitions can be evaluated without any side effect.
Function, type, class, or object definitions are always pure. A value
definition is pure if its right-hand side expression is pure. Pure
expressions are paths, literals, as well as typed expressions
\verb@e: T@ where \verb@e@ is pure.

\comment{
Every basic definition may introduce several defined names, separated
by commas. These are expanded according to the following scheme:
\bda{lcl}
\VAL;x, y: T = e && \VAL; x: T = e \\
		 && \VAL; y: T = x \\[0.5em]

\LET;x, y: T = e && \LET; x: T = e \\
		 && \VAL; y: T = x \\[0.5em]

\DEF;x, y (ps): T = e &\tab\mbox{expands to}\tab& \DEF; x(ps): T = e \\
		      && \DEF; y(ps): T = x(ps)\\[0.5em]

\VAR;x, y: T := e && \VAR;x: T := e\\
		  && \VAR;y: T := x\\[0.5em]

\TYPE;t,u = T && \TYPE; t = T\\
	      && \TYPE; u = t\\[0.5em]
\eda
}

All definitions have a ``repeated form'' where the initial
definition keyword is followed by several constituent definitions
which are separated by commas.  A repeated definition is
always interpreted as a sequence formed from the
constituent definitions. E.g.\ the function definition
\verb@def f(x) = x, g(y) = y@ expands to
\verb@def f(x) = x; def g(y) = y@ and
the type definition
\verb@type T$_1$, T$_2$ <: B$_2$@ expands to
\verb@type T$_1$; type T$_2$ <: B$_2$@.

\section{Value Declarations and Definitions}
\label{sec:valdef}

\syntax\begin{verbatim}
  Dcl             \=::= \= val ValDcl {`,' ValDcl}
  ValDcl       \>::= \> Id `:' Type
  Def          \>::= \> val PatDef {`,' PatDef}
  PatDef       \>::= \> Pattern `=' Expr
\end{verbatim}

A value declaration \verb@val x: T@ introduces \verb@x@ as a name of a value of
type \verb@T@.  

A value definition \verb@val x: T = e@ defines $x$ as a name of the value
that results from the evaluation of $e$. The type $T$ may be omitted
if it can be determined using local type inference
(\sref{sec:local-type-inf}).  The type of the expression $e$ must
conform to type $T$.

Evaluation of the value definition implies evaluation of its
right-hand side $e$.  The effect of the value definition is to bind
$x$ to the value of $e$ converted to type $T$.

Value definitions can alternatively have a pattern
(\sref{sec:patterns}) as left-hand side.  If $p$ is some pattern other
than a simple name or a name followed by a colon and a type, then the
value definition \verb@val p = e@ is expanded as follows:

1. If the pattern $p$ has bound variables $x_1 \commadots x_n$, where $n > 1$:
\begin{verbatim}
val $\Dollar$x = e.match {case p => scala.Tuple$\,n$(x$_1$, ..., x$_n$)}
val x$_1$ = $\Dollar$x._1
...
val x$_n$ = $\Dollar$x._n  .
\end{verbatim}
Here, \verb@$\Dollar$x@ is a fresh name.  The class
\verb@Tuple$\,n$@ is defined for $n = 2,...,9$ in package
\verb@scala@.

2. If $p$ has a unique bound variable $x$:
\begin{verbatim}
val x = e.match { case p => x }
\end{verbatim}

3. If $p$ has no bound variables:
\begin{verbatim}
e.match { case p => ()}
\end{verbatim}

\example
The following are examples of value definitions
\begin{verbatim}
val pi = 3.1415;
val pi: double = 3.1415;     \=// equivalent to first definition
val Some(x) = f();        \>// a pattern definition
val x :: xs = mylist;     \>// an infix pattern definition
\end{verbatim}

The last two definitions have the following expansions.
\begin{verbatim}
val x = f().match { case Some(x) => x }

val x$\Dollar$ = mylist.match { case x :: xs => scala.Tuple2(x, xs) }
val x = x$\Dollar$._1;
val xs = x$\Dollar$._2;

\end{verbatim}

\iflet{
\section{Let Definitions}
\label{sec:letdef}

\syntax\begin{verbatim}
  PureDef      \=::= \= $\LET$ ValDef {`,' ValDef}
  ValDef       \>::= \> Id [`:' Type] `=' Expr
\end{verbatim}

A let definition $\LET;x: T = e$ defines $x$ as a name of the value
that results from the delayed evaluation of $e$. The type $T$ must be
a concrete value type (\sref{sec:types}) and the type of the
expression $e$ must conform to $T$. The effect of the let definition
is to bind the left-hand side $x$ to the result of evaluating $e$
converted to type $T$.  However, the expression $e$ is not evaluated
at the point of the let definition, but is instead evaluated the first
time $x$ is dereferenced during execution of the program (which might
be never at all). An attempt to dereference $x$ again in the course of
evaluation of $e$ leads to a run-time error.  Other threads trying to
dereference $x$ while $e$ is being evaluated block until evaluation is
complete.

The type $T$ may be omitted if it can be determined using local type
inference (\sref{sec:local-type-inf}).
}

\section{Variable Declarations and Definitions}
\label{sec:vardef}

\syntax\begin{verbatim}
  Dcl            \=::= \= var VarDcl {`,' VarDcl}
  Def            \>::= \> var ValDef {`,' ValDef}
  VarDcl         \>::=\> Id `:' Type
  VarDef          \>::=\> Id [`:' Type] `=' Expr
                  \>  |\> Id `:' Type `=' `_'
\end{verbatim}

A variable declaration \verb@var x: T@ is equivalent to declarations
of a {\em getter function} \verb@x@ and a {\em setter function}
\verb@x_=@, defined as follows:

\begin{verbatim}
  def x: T
  def x_= (y: T): unit
\end{verbatim}

An implementation of a class containing variable declarations
may define these variables using variable definitions, or it may
define setter and getter functions directly.

A variable definition \verb@var x: T = e@ introduces a mutable
variable with type \verb@T@ and initial value as given by the
expression \verb@e@. The type $T$ can be omitted, 
in which case the type of $e$ is assumed.

A variable definition \verb@var x: T = _@ introduces a mutable
variable with type \verb@T@ and a default initial value. 
The default value depends on the type \verb@T@ as follows:
\begin{quote}\begin{tabular}{ll}
\verb@0@ & if $T$ is \verb@int@ or one of its subrange types, \\
\verb@0L@ & if $T$ is \verb@long@,\\
\verb@0.0f@ & if $T$ is \verb@Float@,\\
\verb@0.0d@ & if $T$ is \verb@double@,\\
\verb@false@ & if $T$ is \verb@boolean@,\\
\verb@()@ & if $T$ is \verb@unit@, \\
\verb@null@ & for all other types $T$.
\end{tabular}\end{quote}

When they occur as members of a template, both forms of variable
defintion also introduce a getter function \verb@x@ which returns the
value currently assigned to the variable, as well as a setter function
\verb@x_=@ which changes the value currently assigned to the variable.
The functions have the same signatures as for a variable declaration.

\example The following example shows how {\em properties} can be
simulated in Scala. It defines a class \verb@TimeOfDayVar@ of time
values with updatable integer fields representing hours, minutes, and
seconds. Its implementation contains tests that allow only legal
values to be assigned to these fields. The user code, on the other
hand, accesses these fields just like normal variables.

\begin{verbatim}
class TimeOfDayVar {
  private var h: int = 0, m: int = 0, s: int = 0;
  def hours                   \==  h;
  def hours_= (h: int) \>=  \=if (0 <= h && h < 24) this.h = h
		       \>   \>else new DateError().throw;
  def minutes  \>=  m
  def minutes_= (m: int) \>=  if (0 <= m && m < 60) this.m = m
		       \>   \>else new DateError().throw;
  def seconds  \>=  s
  def seconds_= (s: int)  \>=  if (0 <= s && s < 60) this.s = s
		       \>   \>else new DateError().throw;
}
val t = new TimeOfDayVar;
d.hours = 8; d.minutes = 30; d.seconds = 0;
d.hours = 25;                 \=// throws a DateError exception
\end{verbatim}

\section{Type Declarations and Type Aliases}
\label{sec:typedcl}
\label{sec:typealias}

\syntax\begin{verbatim}
  Dcl             \=::= \= type TypeDcl {`,' TypeDcl}
  TypeDcl         \>::= \> Id [>: Type] [<: Type]
  Def             \>::= \> type TypeDef {`,' TypeDef}
  TypeDef         \>::= \> Id `=' Type
\end{verbatim}

A {\em type declaration} \verb@type t >: L <: U@ declares \verb@t@ to
be an abstract type with lower bound type \verb@L@ and upper bound
type \verb@U@.  If such a declaration appears as a member declaration
of a type, implementations of the type may implement \verb@t@ with any
type \verb@T@ for which \verb@L <: T <: U@. Either or both bounds may
be omitted.  If the lower bound \verb@L@ is missing, the bottom type
\verb@scala.All@ is assumed.  If the upper bound \verb@U@ is missing,
the top type \verb@scala.Any@ is assumed.

A {\em type alias} \verb@type t = T@ defines \verb@t@ to be an alias
name for the type \verb@T@. Type declarations and type aliases are
collectively called {\em type bindings}.

The scope rules for definitions (\sref{sec:defs}) and type parameters
(\sref{sec:funsigs}) make it possible that a type name appears in its
own bound or in its right-hand side.  However, it is a static error if
a type alias refers recursively to the defined type itself.  That is,
the type \verb@T@ in a type alias \verb@type t = T@ may not refer
directly or indirectly to the name \verb@t@.  It is also an error if
an abstract type is directly or indirectly its own bound.

\example The following are legal type declarations and definitions:
\begin{verbatim}
type IntList = List[Integer];
type T extends Comparable[T];
\end{verbatim}

The following are illegal:
\begin{verbatim}
type Abs = Comparable[Abs];	                 \=// recursive type alias

type S <: T;                       \>// S, T are bounded by themselves.
type T <: S;

type T <: Object with T;	    \>// T is abstract, may not be part of
                                    \>// compound type

type T >: Comparable[T.That];       \>// Cannot select from T.
			            \>// T is a type, not a value
\end{verbatim}

Neither type declarations nor type aliases may carry type
parameters. However, it is possible to alias a type constructor of a
parameterized type, as is shown in the following example.

\example The \verb@Predef@ module contains a definition which establishes \verb@Pair@ 
as an alias of the parameterized class \verb@Tuple2@:
\begin{verbatim}
type Pair = Tuple2;
\end{verbatim}
As a consequence, for any two types \verb@S@ and \verb@T@, the type
\verb@Pair[S, T]@ is equivalent to the type \verb@Tuple2[S, T]@.
\verb@Pair@ can also be used as a constructor instead of \verb@Tuple2@, as in
\begin{verbatim}
new Pair[Int, Int](1, 2) .
\end{verbatim}

\section{Function Declarations and Definitions}
\label{sec:defdef}
\label{sec:funsigs}

\syntax\begin{verbatim}
  Dcl                             \=::= \= def FunDcl {`,' FunDcl}
  FunDcl      \>::= \> Id [FunTypeParamClause] {ParamClause} `:' Type
  Def         \>::= \> def FunDef {`,' FunDef}
  FunDef     \>::= \> Id [FunTypeParamClause] {ParamClause} [`:' Type] 
                 \>\> `=' Expr
  FunTypeParamClause \>::=\> `[' TypeDcl {`,' TypeDcl} `]'
  ParamClause     \>::=\> `(' [Param {`,' Param}] `)'
  Param        	  \>::= \> [def] Id `:' Type [*]
\end{verbatim}

A function declaration has the form \verb@def f psig: T@, where
\verb@f@ is the function's name, \verb@psig@ is its parameter
signature and \verb@T@ is its result type. A function definition
\verb@f psig:T = e@ also includes a {\em function body} \verb@e@,
i.e.\ an expression which defines the function's result.  A parameter
signature consists of an optional type parameter clause \verb@[tps]@,
followed by zero or more value parameter clauses
\verb@(ps_1) \ldots (ps_n)@.  Such a declaration or definition
introduces a value with a (possibly polymorphic) method type whose
parameters and result type are as given.

A type parameter clause \verb@tps@ consists of one or more type
declarations (\sref{sec:typedcl}), which introduce type parameters,
possibly with bounds.  The scope of a type parameter \verb@a@ includes
the whole signature, including any of the type parameter bounds as
well as the function body, if it is present.  

A value parameter clause \verb@ps@ consists of zero or more formal
parameter bindings such as \verb@x: T@, which bind value
parameters and associate them with their types.  The scope of a formal
value parameter name \verb@x@ is the function body, if one is
given. Both type parameter names and value parameter names must be
pairwise distinct.

Value parameters may be prefixed by \verb@def@, e.g.\
\verb@\DEF;x:T@. The type of such a parameter is then the
parameterless method type \verb@[]T@. This indicates that the
corresponding argument is not evaluated at the point of function
application, but instead is evaluated at each use within the
function. That is, the argument is evaluated using {\em call-by-name}.

\example The declaration
\begin{verbatim}
def while (def cond: Boolean) (def stat: Unit): Unit
\end{verbatim}
produces the typing
\begin{verbatim}
while: (cond: [] Boolean) (stat: [] Unit) Unit
\end{verbatim}
which indicates that both parameters of \verb@while@ are evaluated using
call-by-name.

The type of the function body must conform to the function's declared
result type, if one is given. If the function definition is not
recursive, the result type may be omitted, in which case it is
determined from the type of the function body.


\section{Overloaded Definitions}
\label{sec:overloaded-defs}
\todo{change}

An overloaded definition is a set of \verb@n > 1@ value or function
definitions in the same scope that define the same name, binding it to
types \verb@T_1 \commadots T_n@, respectively.  The individual
definitions are called {\em alternatives}.  Alternatives always need
to specify the type of the defined entity completely.  All
alternatives must have the same modifiers. It is an error if the types
of two alternatives \verb@T_i@ and \verb@T_j@ have the same erasure
(\sref{sec:erasure}).  An overloaded definition defines a single
entity, of type \verb@T_1 \overload \ldots \overload T_n@
(\sref{sec:overloaded-types}).
%This must be a well-formed
%overloaded type

\section{Import Clauses}
\label{sec:import}

\syntax\begin{verbatim}
  Import                  \=::=\= import ImportExpr {`,' ImportExpr}
  ImportExpr      \>::=\> StableId `.' (Id | `_' | ImportSelectors)
  ImportSelectors \>::=\> `{' {ImportSelector `,'} (ImportSelector | `_') `}'
  ImportSelector  \>::=\> Id [`=>' Id | `=>' `_']
\end{verbatim}

An import clause has the form \verb@import p.I@ where \verb@p@ is a stable
identifier (\sref{sec:paths}) and \verb@I@ is an import expression.
The import expression determines a set of names of members of \verb@p@
which are made available without qualification. The most general form
of an import expression is a list of {\em import selectors}
\begin{verbatim}
{ x$_1$ => y$_1$, ..., x$_n$ => y$_n$, _ }
\end{verbatim}
for $n \geq 0$, where the final wildcard `\verb@_@' may be absent.  It
makes available each member \verb@p.x$_i$@ under the unqualified name
\verb@y$_i$@. I.e.\ every import selector \verb@x$_i$ => y$_i$@ renames
\verb@p.x$_i$@ to
\verb@y$_i$@.  If a final wildcard is present, all members \verb@z@ of
\verb@p@ other than \verb@x$_1$, ..., x$_n$@ are also made available
under their own unqualfied names.

Import selectors work in the same way for type and term members. For
instance, an import clause \verb@import p.{x => y}@ renames the term
name \verb@p.x@ to the term name \verb@y@ and the type name \verb@p.x@
to the type name \verb@y@. At least one of these three names must
reference a member of \verb@p@.

If the destination in an import selector is a wildcard, the import
selector hides access to the source member. For instance, the import
selector \verb@x => _@ ``renames'' \verb@x@ to the wildcard symbol
(which is unaccessible as a name in user programs), and thereby
effectively prevents unqualified access to \verb@x@. This is useful if
there is a final wildcard in the same import selector list, which
imports all members not mentioned in previous import selectors.

Several shorthands exist. An import selector may be just a simple name
\verb@x@. In this case, \verb@x@ is imported without renaming, so the
import selector is equivalent to \verb@x => x@. Furthermore, it is
possible to replace the whole import selector list by a single
identifier or wildcard. The import clause \verb@import p.x@ is
equivalent to \verb@import p.{x}@, i.e.\ it makes available without
qualification the member \verb@x@ of \verb@p@. The import clause
\verb@import p._@ is equivalent to
\verb@import p.{_}@, 
i.e.\ it makes available without qualification all members of \verb@p@
(this is analogous to \verb@import p.*@ in Java).

An import clause with multiple import expressions
\verb@import p$_1$.I$_1$, ..., p$_n$.I$_n$@ is interpreted as a
sequence of import clauses 
\verb@import p$_1$.I$_1$; ..., import p$_n$.I$_n$@.

\example Consider the object definition:
\begin{verbatim}
object M { 
  def z = 0, one = 1; 
  def add(x: Int, y: Int): Int = x + y 
}
\end{verbatim}
Then the block
\verb@{ import M.{one, z => zero, _}; add(zero, one) }@
is equivalent to the block \verb@{ M.add(M.z, M.one) }@.

\chapter{Classes and Modules}
\label{sec:globaldefs}

\syntax\begin{verbatim}
  ClsDef          \=::=\= ([case] class | trait) ClassDef {`,' ClassDef}
                  \>  |\> [case] object ObjectDef {`,' ObjectDef}
\end{verbatim}

Classes (\sref{sec:classes}) and modules
(\sref{sec:modules}) are both defined in terms of {\em templates}.

\section{Templates}
\label{sec:templates}

\syntax\begin{verbatim}
  Template        \=::=\= Constr {`with' Constr} [TemplateBody]
  TemplateBody    \>::=\> `{' [TemplateStat {`;' TemplateStat}] `}'
\end{verbatim}

A template defines the type signature, behavior and initial state of a
class of objects or of a single object. Templates form part of
instance creation expressions, class definitions, and module
definitions.  A template
\verb@sc with mc$_1$ with ... with mc$_n$ {stats}@ consists of a
constructor invocation \verb@sc@ which defines the template's {\em
superclass}, constructor invocations \verb@mc$_1$ \commadots mc$_n$@
$(n \geq 0)$, which define the template's {\em mixin classes}, and a
statement sequence \verb@stats@ which contains additional member
definitions for the template.  Superclass and mixin classes together
are called the {\em parent classes} of a template.  The superclass of
a template must be a subtype of the superclass of each mixin class.
The {\em least proper supertype} of a template is the class type or
compound type (\sref{sec:compound-types}) consisting of the its parent classes.

Member definitions define new members or overwrite members in the
parent classes.  If the template forms part of a class definition,
\verb@stats@ may also contain declarations of abstract members.
%The type of each non-private definition or declaration of a
%template must be equivalent to a type which does not refer to any
%private members of that template.

\paragraph{Connection with Java} A template may have a Java class as
its superclass and Java interfaces as its mixin classes. On the other
hand, it is not permitted to have a Java class as a mixin class, or a
Java interface as a superclass.

\subsection{Constructor Invocations}
\label{sec:constr-invoke}
\syntax\begin{verbatim}
  Constr \=::= \= StableId [TypeArgs] [`(' [Exprs] `)']  
\end{verbatim}

Constructor invocations define the type, members, and initial state of
objects created by an instance creation expression, or of parts of an
object's definition which are inherited by a class or module
definition. A constructor invocation is a function application
\verb@x.c(args)@, where \verb@x@ is a stable identifier
\sref{sec:stable-ids}), \verb@c@ is a type name which either
designates a class or defines an alias type for one, and \verb@(args)@
is an argument list, which matches one of the constructors of that
class. The prefix \verb@x.@ can be omitted. The class \verb@c@ must
conform to \verb@scala.AnyRef@, i.e.\ it may not be a value type.  The
argument list \verb@(args)@ can also be omitted, in which case an
empty argument list \verb@()@ is implicitly added.

\subsection{Base Classes}
\label{sec:base-classes}

For every template, class type and constructor invocation we define two
sequences of class types: the {\em base classes} and {\em mixin base
classes}. Their definitions are as follows.

The {\em mixin base classes} of a template
\verb@sc;\WITH;mc_1;\WITH;mc_n;\WITH;(stats)@ are obtained by
concatenating, for each $i = 1 \commadots n$, the mixin base classes
of the mixin \verb@mc_i@. The mixin base classes of a class type \verb@C@ are
the mixin base classes of the template represented by \verb@C@, followed by
\verb@C@ itself. The mixin base classes of a constructor invocation of type
\verb@T@ are the mixin base classes of class \verb@T@.

The {\em base classes} of a template consist of the base classes of
its superclass, followed by the template's mixin base classes.  The
base classes of class \verb@scala.Any@ consist of just the
class itself. The base classes of some other class type \verb@C@ are the
base classes of the template represented by \verb@C@, followed by \verb@C@
itself.  The base classes of a constructor invocation of type \verb@T@
are the base classes of \verb@T@.

The notions of mixin base classes and base classes are extended from
classes to arbitrary types following the definitions of
\sref{sec:base-classes}.
 
If two types in the base class sequence of a template refer to the
same class definition, then that definition must define a trait
(\sref{sec:traits}), and the type that comes later in the sequence must
conform to the type that comes first. 
\sref{sec:case-classes}.

\subsection{Evaluation}

The evaluation of a template or constructor invocation depends on
whether the template defines an object or is a superclass of a
constructed object, or whether it is used as a mixin for a defined
object.  In the second case, the evaluation of a template used as a
mixin depends on an {\em actual superclass}, which is known at the
point where the template is used in a definition of an object, but not
at the point where it is defined. The actual superclass is used in the
determination of the meaning of \verb@super@ (\sref{sec:this-super}).

We therefore define two notions of template evaluation: (Plain)
evaluation (as a defining template or superclass) and mixin evaluation
with a given superclass \verb@sc@. These notions are defined for templates
and constructor invocations as follows.

A {\em mixin evaluation with superclass \verb@sc@} of a template
\verb@sc';\WITH;mc_1;\WITH;mc_n;\WITH;(stats)@ consists of mixin
evaluations with superclass \verb@sc@ of the mixin constructor invocations
\verb@mc_1 \commadots mc_n@ in the order they are given, followed by an
evaluation of the statement sequence \verb@stats@.  Within \verb@stats@ the
actual superclass refers to \verb@sc@.  A mixin evaluation with superclass
\verb@sc@ of a class constructor invocation \verb@ci@ consists of an evaluation
of the constructor function and its arguments in the order they are
given, followed by a mixin evaluation with superclass \verb@sc@ of the
template represented by the constructor invocation.

An {\em evaluation} of a template
\verb@sc;\WITH;mc_1;\WITH;mc_n;\WITH;(stats)@ consists of an evaluation of
the superclass constructor invocation \verb@sc@ (of type \verb@S@, say),
followed by a mixin evaluation with superclass \verb@sc@ of the template. An
evaluation of a class constructor invocation \verb@ci@ consists of an
evaluation of the constructor function and its arguments in
the order they are given, followed by an evaluation of the template
represented by the constructor invocation.

\subsection{Template Members}

\label{sec:members}

The object resulting from evaluation of a template has directly bound
members and inherited members. Members can be abstract or concrete.
These are defined as follows.
\begin{enumerate}
\item
A {\em directly bound} member is an entity introduced by a member
definition or declaration in the template's statement sequence. The
member is called {\em abstract} if it is introduced by a declaration,
{\em concrete} otherwise.
\item
A {\em concrete inherited} member is a non-private, concrete member of
one of the template's base classes \verb@B@, except if a member with the
same \ifqualified{qualified} name is already directly bound in the template, or is
directly bound in a base class of the template which is a subclass of
\verb@B@, or is a directly bound, non-private, concrete member of a base
class which succeeds \verb@B@ in the base class sequence of the template.
\item
An {\em abstract inherited} member is a non-private, abstract member
of one of the template's base classes \verb@B@, except if a member with the
same \ifqualified{qualified} name is already directly bound in the template, or is a
concrete inherited member, or is a directly bound, non-private member
of a base class which succeeds \verb@b@ in the base class sequence of the
template.
\end{enumerate}

\comment{
The type of a member \verb@m@ is determined as follows: If \verb@m@ is defined
in \verb@stats@, then its type is the type as given in the member's
declaration or definition. Otherwise, if \verb@m@ is inherited from the
base class \verb@B[T1 \commadots T_n]@, \verb@B@'s class declaration has formal
parameters \verb@[a_1 \commadots a_n]@, and \verb@M@'s type in \verb@B@ is \verb@U@, then
\verb@M@'s type in \verb@C@ is \verb@U[a_1 := T_1
\commadots a_n := T_n]@.

\ifqualified{
Members of templates have internally qualified names $Q\qex x$ where
$x$ is a simple name and $Q$ is either the empty name $\epsilon$, or
is a qualified name referencing the module or class that first
introduces the member. A basic declaration or definition of $x$ in a
module or class $M$ introduces a member with the following qualified
name:
\begin{enumerate}
\item
If the binding is labelled with an \verb@override $Q$@\nyi{Override
  with qualifier} modifier,
where $Q$ is a fully qualified name of a base class of $M$, then the
qualified name is the qualified expansion (\sref{sec:names}) of $x$ in
$Q$.
\item
If the binding is labelled with an \verb@override@ modifier without a
base class name, then the qualified name is the qualified expansion
of $x$ in $M$'s least proper supertype (\sref{sec:templates}).
\item
An implicit \verb@override@ modifier is added and case (2) also
applies if $M$'s least proper supertype contains an abstract member
with simple name $x$.
\item
If no \verb@override@ modifier is given or implied, then if $M$ is
labelled \verb@qualified@, the qualified name is $M\qex x$. If $M$ is
not labelled \verb@qualified@, the qualified name is $\epsilon\qex x$.
\end{enumerate}
}
}

\example Consider the class definitions

\begin{verbatim}
class A { def f: Int = 1 ; def g: Int = 2 ; def h: Int = 3 }
abstract class B { def f: Int = 4 ; def g: Int }
abstract class C extends A with B { def h: Int }
\end{verbatim}

Then class \verb@C@ has a directly bound abstract member \verb@h@. It
inherits member \verb@f@ from class \verb@B@ and member \verb@g@ from
class \verb@A@.

\ifqualified{
\example\label{ex:compound-b}
Consider the definitions:
\begin{verbatim}
qualified class Root extends Any with { def r1: Root, r2: Int }
qualified class A extends Root with { def r1: A, a: String }
qualified class B extends A with { def r1: B, b: Double }
\end{verbatim}
Then \verb@A with B@ has members
\verb@Root::r1@ of type \verb@B@, \verb@Root::r2@ of type \verb@Int@,
\verb@A::a:@ of type \verb@String@, and \verb@B::b@ of type \verb@Double@,
in addition to the members inherited from class \verb@Any@.
}

\subsection{Overriding}
\label{sec:overriding}

A template member \verb@M@ that has the same \ifqualified{qualified}
name as a non-private member \verb@M'@ of a base class (and that
belongs to the same namespace) is said to {\em override} that member.
In this case the binding of the overriding member \verb@M@ must be
more specific (\sref{sec:subtyping}) than the binding of the
overridden member \verb@M'@.  Furthermore, the overridden definition
may not be a class definition.  Method definitions may only override
other method definitions (or the methods implicitly defined by a
variable definition). They may not override value let definitions.
Finally, the following restrictions on modifiers apply to \verb@M@ and
\verb@M'@:
\begin{itemize}
\item
\verb@M'@ must not be labelled \verb@final@.
\item
\verb@M@ must not be labelled \verb@private@.
\item
If \verb@M@ is labelled \verb@protected@, then \verb@M'@ must also be
labelled \verb@protected@.
\item
If \verb@M'@ is not an abstract member, then
\verb@M@ must be labelled \verb@override@.
\end{itemize}

\example\label{ex:compound-a}
Consider the definitions:
\begin{verbatim}
trait Root with { type T <: Root }
trait A extends Root with { type T <: A }
trait B extends Root with { type T <: B }
trait C extends A with B;
\end{verbatim}
Then the trait definition \verb@C@ is not well-formed because the
binding of \verb@T@ in \verb@C@ is
\verb@type T extends B@,
which fails to be more specific than the binding of same name in type
\verb@A@. The problem can be solved by adding an overriding 
definition of type \verb@T@ in class \verb@C@:
\begin{verbatim}
class C extends A with B { type T <: C }
\end{verbatim}

\subsection{Modifiers}
\label{sec:modifiers}

\syntax\begin{verbatim}
  Modifier            \=::=\= LocalModifier
		  \>  |\> private
                  \>  |\> protected
                  \>  |\> override 
  LocalModifier   \>::=\> abstract
                  \>  |\> final
		  \>  |\> sealed
\end{verbatim}

Member definitions may be preceded by modifiers which affect the
\ifqualified{qualified names, }accessibility and usage of the
identifiers bound by them.  If several modifiers are given, their
order does not matter, but the same modifier may not occur repeatedly.
Modifiers preceding a repeated definition apply to all constituent
definitions.  The rules governing the validity and meaning of a
modifier are as follows.
\begin{itemize}
\item
The \verb@private@ modifier can be used with any definition in a
template. Private identifiers can be accessed only from within the template
that defines them.  
%Furthermore, accesses are not permitted in
%packagings (\sref{sec:topdefs}) other than the one containing the
%definition. 
Private members are not inherited by subclasses and they
may not override definitions in parent classes.
\verb@private@ may not be applied to abstract members, and it
may not be combined in one modifier list with
\verb@protected@, \verb@final@ or \verb@override@.
\item
The \verb@protected@ modifier applies to class member definitions.
Protected members can be accessed from within the template of the defining
class as well as in all templates that have the defining class as a base class.
%Furthermore, accesses from the template of the defining class are not
%permitted in packagings other than the one
%containing the definition.  
A protected identifier \verb@x@ may be used as
a member name in a selection \verb@r.x@ only if \verb@r@ is one of the reserved
words \verb@this@ and
\verb@super@, or if \verb@r@'s type conforms to a type-instance of the class
which contains the access.
\item
The \verb@override@ modifier applies to class member definitions.  It
is mandatory for member definitions that override some other
non-abstract member definition in a super- or mixin-class. If an
\verb@override@ modifier is given, there must be at least one
overridden member definition.  Furthermore, the overridden definition
must be concrete \sref{sec:members}, unless the class containing the
overriding member is abstract.
\item
The \verb@abstract@ modifier is used in class definitions. It is
mandatory if the class has abstract members, or if the class has
members labelled \verb@override@ which override only abstract members
in a parent class.  Classes with \verb@abstract@ members
cannot be instantiated (\sref{sec:inst-creation}) with a constructor
invocation unless followed by mixin constructors or statements which
override all abstract members of the class.
\item
The \verb@final@ modifier applies to class member definitions and to
class definitions. A \verb@final@ class member definition may not be
overridden in subclasses. A \verb@final@ class may not be inherited by
a template. \verb@final@ is redundant for modules.  Members of final
classes or modules are implicitly also final, so the \verb@final@
modifier is redundant for them, too.  \verb@final@ may not be applied
to abstract members, and it may not be combined in one modifier list with
\verb@private@ or \verb@sealed@.
\item
The \verb@sealed@ modifier applies to class definitions. A
\verb@sealed@ class may not be inherited, except if either
\begin{itemize}
\item
the inheriting template is nested within the definition of the sealed
class itself, or
\item
the inheriting template belongs to a class or object definition which
forms part of the same statement sequence as the definition of the
sealed class.
\end{itemize}
\end{itemize}

\example A useful idiom to prevent clients of a class from
constructing new instances of that class is to declare the class
\verb@final@ and \verb@abstract@:

\begin{verbatim}
object m {
  abstract final class C (x: Int) {
    def nextC = C(x + 1) with {}
  }
  val empty = new C(0) {}
}
\end{verbatim}
For instance, in the code above clients can create instances of class
\verb@m.C@ only by calling the \verb@nextC@ method of an existing \verb@m.C@
object; it is not possible for clients to create objects of class
\verb@m.C@ directly. Indeed the following two lines are both in error:

\begin{verbatim}
  m.C(0)           \=// ERROR: C is abstract, so it cannot be instantiated.
  m.C(0) {}        \>// ERROR: illegal inheritance from final class.
\end{verbatim}

\section{Class Definitions}
\label{sec:classes}

\syntax\begin{verbatim}
  ClsDef          \=::=\= class ClassDef {`,' ClassDef}
  ClassDef        \>::=\> Id [TypeParamClause] [ParamClause] [`:' SimpleType] 
                  \>   \> ClassTemplate
  ClassTemplate   \>::=\> extends Template
	          \>  |\> TemplateBody
		  \>  |\>
\end{verbatim}

The most general form of class definition is 
\verb@class c[tps](ps): s extends t@.
Here,
\begin{itemize}
\item[]
\verb@c@ is the name of the class to be defined.
\item[] \verb@tps@ is a non-empty list of type parameters of the class
being defined.  The scope of a type parameter is the whole class
definition including the type parameter section itself.  It is
illegal to define two type parameters with the same name.  The type
parameter section \verb@[tps]@ may be omitted. A class with a type
parameter section is called {\em polymorphic}, otherwise it is called
{\em monomorphic}.
\item[] 
\verb@ps@ is a formal parameter clause for the {\em primary
constructor} of the class. The scope of a formal parameter includes
the template \verb@t@. However, the formal parameter may not form 
part of the types of any of the parent classes or members of \verb@t@.
It is illegal to define two formal parameters with the same name.
The formal parameter section \verb@(ps)@ may be omitted in which case
an empty parameter section \verb@()@ is assumed.
\item[] 
\verb@s@ is the {\em self type} of the class. Inside the
class, the type of \verb@this@ is assumed to be \verb@s@.  The self
type must conform to the self types of all classes which are inherited
by the template \verb@t@. The self type declaration \verb@:s@ may be
omitted, in which case the self type of the class is assumed to be
equal to \verb@c[tps]@.
\item[] 
\verb@t@ is a
template (\sref{sec:templates}) of the form
\verb@sc with mc$_1$ with ... with mc$_n$ { stats }@
which defines the base classes, behavior and initial state of objects of
the class. The extends clause \verb@extends sc with ... with mc$_n$@
can be omitted, in which case
\verb@extends scala.Object@ is assumed.  The class body
\verb@{stats}@ may also be omitted, in which case the empty body
\verb@{}@ is assumed.
\end{itemize}
This class definition defines a type \verb@c[tps]@ and a constructor
which when applied to parameters conforming to typles \verb@ps@
initializes instances of type \verb@c[tps]@ by evaluating the template
\verb@t@.

\subsection{Constructor Definitions}

\syntax\begin{verbatim}
  FunDef          \=::=\= this ParamClause `=' ConstrExpr
  ConstrExpr      \>::=\> this ArgumentExpr
                  \>  |\>  `{' {BlockStat `;'} ConstrExpr `}'
\end{verbatim}

A class may have additional constructors besides the primary
constructor.  These are defined by constructor definitions of the form
\verb@def this(ps) = e@.  Such a definition introduces an additional
constructor for the enclosing class, with parameters as given in the
formal parameter list \verb@ps@, and whose evaluation is defined by
the constructor expression \verb@e@.  The scope of each formal
parameter is the constructor expression \verb@e@.  A constructor
expression is either a self constructor invocation \verb@this(args)@
or a block which ends in a constructor expression.  In terms of
visibility rules, constructor definitions are conceptually outside
their enclosing class.  Hence, they can access neither value
parameters nor members of the enclosing class by simple name, and the
value \verb@this@ refers to an object of the class enclosing the class
of the object being constructed. However, constructor definitions can
access type parameters of the enclosing class.

If there are auxilary constructors of a class \verb@C@, they define
together with \verb@C@'s primary constructor an overloaded constructor
value. The usual rules for overloading resolution
\sref{sec:overloading} apply for constructor invocations of \verb@C@,
including the self constructor invocations in the constructor
expressions themselves. To prevent infinite cycles of constructor
invocations, there is the restriction that every self constructor
invocation must refer to a constructor definition which precedes it
(i.e. it must refer to either a preceding auxiliary constructor or the
primary constructor of the class).  The type of a constructor
expression must be always so that a generic instance of the class is
constructed.  I.e., if the class in question has name \verb@C@ and type
parameters \verb@[tps]@, then each constructor must construct an
instance of \verb@C[tps]@; it is not permitted to instantiate formal
type parameters.

\example Consider the class definition

\begin{verbatim}
class LinkedList[a <: AnyRef](x: a, xs: LinkedList[a]) {
  var head = x;
  var tail = xs;
  def isEmpty = tail != null;  
  def this() = this(null, null);
  def this(x: a) = { val empty = new LinkedList(); this(x, empty) }
}
\end{verbatim}
This defines a class \verb@LinkedList@ with an overloaded constructor of type
\begin{verbatim}
[a <: AnyRef](x: a, xs: LinkList[a]): LinkedList[a]   $\overload$
[a <: AnyRef](): LinkedList[a]   $\overload$
[a <: AnyRef](x: a): LinkedList[a] .
\end{verbatim}
The second constructor alternative constructs an empty list, while the
third one constructs a list with one element.

\subsection{Case Classes}
\label{sec:case-classes}

\syntax\begin{verbatim} ClsDef \=::=\= case class ClassDef {`,'
  ClassDef}
\end{verbatim}

If a class definition is prefixed with \verb@case@, the class is said
to be a {\em case class}.  The primary constructor of a case class may
be used in a constructor pattern (\sref{sec:patterns}).  That
constructor make not have any value parameters which are prefixed by
\verb@def@.  None of the base classes of a case class may be a case
class. Furthermore, no type may have two different case classes among
its basetypes.

A case class definition of \verb@c[tps](ps)@ with type
parameters \verb@tps@ and value parameters \verb@ps@ implicitly
generates a function definition for a {\em case class factory}
\begin{verbatim}
def c[tps](ps): c[tps] = new c[tps](ps)
\end{verbatim}
together with the class definition itself (if a type parameter section
is missing in the class, it is also missing in the factory
definition).  Also implicity defined are accessor member definitions
in the class that return its value parameters. Every binding
\verb@x: T@ in the parameter section leads to a value definition of
\verb@x@ that defines \verb@x@ to be an alias of the parameter.  
%Every
%parameterless function binding \verb@def x: T@ leads to a
%parameterless function definition of \verb@x@ which returns the result
%of invoking the parameter function.  
%The case class may not contain a
%directly bound member with the same simple name as one of its value
%parameters.

Every case class implicitly overrides some method definitions of class
\verb@scala.Object@ (\sref{sec:cls-object}) unless a definition of the same
method is already given in the case class itself or a non-abstract
definition of the same method is given in some base class of the case
class different from \verb@Object@. In particular:
\begin{itemize}
\item[] Method \verb@equals@ is structural equality, where two
instances are equal if they belong to the same class and
have equal (wrt \verb@equals@) primary constructor arguments.
\item[] Method \verb@hashCode@ computes a hash-code
depending on the data structure in a way which maps equal (wrt
\verb@equals@) values to equal hash-codes.
\item[] Method \verb@toString@ returns a string representation which
contains the name of the class and its primary constructor arguments.
\end{itemize}

\example Here is the definition of abstract syntax for lambda
calculus:

\begin{verbatim}
class Expr;
case class
  Var       \=(x: String)              \=extends Expr,
  Apply     \>(f: Expr, e: Expr)       \>extends Expr,
  Lambda    \>(x: String, e: Expr)     \>extends Expr;
\end{verbatim}
This defines a class \verb@Expr@ with case classes
\verb@Var@, \verb@Apply@ and \verb@Lambda@. A call-by-value evaluator for lambda
expressions could then be written as follows.

\begin{verbatim}
type Env = String => Value;
case class Value(e: Expr, env: Env);

def eval(e: Expr, env: Env): Value = e match {
  case Var (x) =>
    env(x)
  case Apply(f, g) =>
    val Value(Lambda (x, e1), env1) = eval(f, env);
    val v = eval(g, env);
    eval (e1, (y => if (y == x) v else env1(y)))
  case Lambda(_, _) =>
    Value(e, env)
}
\end{verbatim}

It is possible to define further case classes that extend type
\verb@Expr@ in other parts of the program, for instance
\begin{verbatim}
case class Number(x: Int) extends Expr;
\end{verbatim}

This form of extensibility can be excluded by declaring the base class
\verb@Expr@ \verb@sealed@; in this case, the only classes permitted to
extend \verb@Expr@ are those which are nested inside \verb@Expr@, or
which appear in the same statement sequence as the definition of
\verb@Expr@.

\section{Traits}

\label{sec:traits}

\syntax\begin{verbatim}
  ClsDef          \=::=\= trait ClassDef {`,' ClassDef}
\end{verbatim}

A class definition which starts with the reserved word \verb@trait@
instead of \verb@class@ defines a trait. A trait is a specific
instance of an abstract class, so the \verb@abstract@ modifier is
redundant for it.  The template of a trait must satisfy the following
four restrictions.
\begin{enumerate}
\item All base classes of the trait are traits.
\item All parent class constructors of a template
      must be primary constructors with empty value
      parameter lists. 
\item None of the statements in the template may be a variable definition.
\item For every value definition \verb@val x: T = e@ among the statements 
      of the template, the defining expression \verb@e@ must {\em pure}.      
\end{enumerate}
A pure expression is a literal, or an occurrence of \verb@this@,
or a stable identifier, or a typed expression \verb@e: T@ where
\verb@e@ is a pure expression.

These restrictions ensure that the evaluation of the mixin constructor
of a trait has no effect. Therefore, traits may appear several times 
in the base class sequence of a template, whereas other classes cannot.
%\item Packagings may add interface classes as new base classes to an
%existing class or modulee.

\example\label{ex:comparable}
The following trait class defines the property of being
ordered, i.e. comparable to objects of some type. It contains an abstract method
\verb@<@ and default implementations of the other comparison operators
\verb@<=@, \verb@>@, and \verb@>=@.

\begin{verbatim}
trait Ord[t <: Ord[t]]: t {
  def < (that: t): Boolean;
  def <=(that: t): Boolean = this < that || this == that;
  def > (that: t): Boolean = that < this;
  def >=(that: t): Boolean = that <= this;
}
\end{verbatim}

\section{Object Definitions}
\label{sec:modules}
\label{sec:object-defs}

\syntax\begin{verbatim}
  ObjectDef       \=::=\= Id [`:' SimpleType] ClassTemplate
\end{verbatim}

An object definition defines a single object of a new class. Its 
most general is
\verb@object m: s extends t@. Here,
\begin{itemize}
\item[]
\verb@m@ is the name of the object to be defined.
\item[] \verb@s@ is the {\em self type} of the object. References to
\verb@m@ are assumed to have type \verb@s@. Furthermore, inside the
template \verb@t@, the type of \verb@this@ is also assumed to be \verb@s@.
The self type must conform to the self types of all classes which are
inherited by the template \verb@t@. The self type declaration
\verb@:s@ may be omitted, in which case the self type of the class is
assumed to be equal to the anonymous class defined by \verb@t@.
\item[] 
\verb@t@ is a
template (\sref{sec:templates}) of the form
\verb@sc with mc$_1$ with ... with mc$_n$ { stats }@
which defines the base classes, behavior and initial state of \verb@m@.
The extends clause \verb@extends sc with ... with mc$_n$@
can be omitted, in which case
\verb@extends scala.Object@ is assumed.  The class body
\verb@{stats}@ may also be omitted, in which case the empty body
\verb@{}@ is assumed.
\end{itemize}
The object definition defines a single object (or: {\em module} 
conforming to the template \verb@t@.  It is almost equivalent to a class
definition and a value definition that creates an object of the class:
\begin{verbatim}
final class m$\Dollar$class: s extends t;
final val m: s = new m$\Dollar$class;
\end{verbatim}
(The \verb@final@ modifiers are omitted if the definition occurs as
part of a block. The class name \verb@m$\Dollar$class@ is not
accessible for user programs.)

There are however two differences between a module definition and a
pair of class and value definition as the one given above.  First,
object definitions may appear as top-level definitions in a
compilation unit, whereas value definitions cannot.  Second, modules
are instantiated lazily.  The \verb@new m$\Dollar$class@ constructor
is evaluated not at the point of the value definition, but is instead
evaluated the first time $m$ is dereferenced during execution of the
program (which might be never at all). An attempt to dereference $m$
again in the course of evaluation of the constructor leads to a
infinite loop or run-time error.  Other threads trying to dereference
$m$ while the constructor is being evaluated block until evaluation is
complete. 

\example
Classes in Scala do not have static members; however, an equivalent
effect can be achieved by an accompanying object definition
E.g.
\begin{verbatim}
abstract class Point {
  val x: Double;
  val y: Double;
  def isOrigin = (x == 0.0 && y == 0.0);
}
object Point {
  val origin = new Point() with { val x = 0.0, y = 0.0 }
}
\end{verbatim}
This defines a class \verb@Point@ and an object \verb@Point@ which
contains \verb@origin@ as a member.  Note that the double use of the
name \verb@Point@ is lgel, since the class definition defines the name
\verb@Point@ in the type name space, whereas the object definition
defines the name in the term namespace.

\comment{
\example Here's an outline of a module definition for a file system.

\begin{verbatim}
module FileSystem with {
  private type FileDirectory;
  private val dir: FileDirectory

  interface File with {
    def read(xs: Array[Byte])
    def close: Unit
  }

  private class FileHandle extends File with { ... }

  def open(name: String): File = ...
}
\end{verbatim}
}

\chapter{Expressions}
\label{sec:exprs}

\syntax\begin{verbatim}
  Expr            \=::=\= [Bindings `=>'] Expr
		  \>  |\> if `(' Expr `)' Expr [[`;'] else Expr]
		  \>  |\> try Expr [`;'] (except Expr | finally Expr)
		  \>  |\> while '(' Expr ')' Expr
		  \>  |\> do Expr [`;'] while `(' Expr ')'
		  \>  |\> for `(' Enumerators `)' (do | yield) Expr
                  \>  |\> [SimpleExpr `.'] Id `=' Expr
		  \>  |\> SimpleExpr ArgumentExpr `=' Expr
                  \>  |\> PostfixExpr [`:' Type1]
  PostfixExpr     \>::=\> InfixExpr [Id]
  InfixExpr       \>::=\> PrefixExpr
		  \>  |\> InfixExpr Id InfixExpr
  PrefixExpr      \>::=\> [`-' | `+' | `~' | `!'] SimpleExpr 
  SimpleExpr      \>::=\> literal
		  \>  |\> Path
		  \>  |\> `(' [Expr] `)'
		  \>  |\> BlockExpr
		  \>  |\> new Template 
		  \>  |\> SimpleExpr `.' Id 
		  \>  |\> Id `#' Id 
		  \>  |\> SimpleExpr TypeArgs
		  \>  |\> SimpleExpr ArgumentExpr
  ArgumentExpr    \>::=\> `(' Expr ')'
                  \>  |\> BlockExpr
  BlockExpr       \>::=\> `{' CaseClause {CaseClause} `}'
		  \>  |\> `{' Block `}'
  Block           \>::=\> {BlockStat `;'} [Expr]
  Exprs           \>::= \> Expr {`,' Expr}
\end{verbatim}

Expressions are composed of operators and operands. Expression forms are
discussed subsequently in decreasing order of precedence. 

\section{Literals}

\syntax\begin{verbatim}
  SimpleExpr    \=::= \= literal
  literal         \>::= \> intLit
                  \>  |\> floatLit
                  \>  |\> charLit
                  \>  |\> stringLit
		  \>  |\> symbolLit
\end{verbatim}

Typing and evaluation of literals are generally as in Java.
An integer literal denotes an integer number. Its type is normally
\verb@int@. However, if the expected type \verb@pt@ of the expression
is either \verb@byte@, \verb@short@, or \verb@char@ and the integer
number fits in the numeric range defined by the type, then the number
is converted to \verb@pt@ and the expression's type is \verb@pt@.  A
floating point literal denotes a single-precision or double precision
IEEE floating point number. A character literal denotes a Unicode
character. A string literal denotes a member of
\verb@java.lang.String@. 

A symbol literal \verb@'id@ is a shorthand for the expression
\verb@scala.Symbol("id")@. If the symbol literal is followed by an
actual parameters, as in \verb@'id(args)@, then the whole expression
is taken to be a shorthand for
\verb@scala.Labelled(scala.Symbol("id"), args)@.

\section{Boolean constants}

\begin{verbatim}
  SimpleExpr \=::= \= true | false
\end{verbatim}

The boolean truth values are denoted by the reserved words \verb@true@
and \verb@false@. The type of these expressions is \verb@boolean@, and
their evaluation is immediate.

\section{The $\NULL$ Reference}

\syntax\begin{verbatim}
  SimpleExpr    \=::= \= null
\end{verbatim}

The \verb@null@ expression is of type \verb@scala.AllRef@. It
denotes a reference value which refers to a special ``null' object,
which implements methods in class \verb@scala.AnyRef@ as follows:
\begin{itemize}
\item[]
\verb@eq(x)@, \verb@==(x)@, \verb@equals(x)@ return \verb@true@ iff their
argument \verb@x@ is also the ``null'' object.
\item[]
\verb@isInstance[T]@ always returns \verb@false@.
\item[]
\verb@asInstance[T]@ always returns the ``null'' object itself.
\item[]
\verb@toString@ returns the string ``\verb@null@''.
\end{itemize}
A reference to any other member of the ``null'' object causes a
\verb@NullPointerException@ to be thrown. 

\section{Designators}
\label{sec:designators}

\syntax\begin{verbatim}
  Designator    \=::= \= Path
	\>  | \> SimpleExpr `.' Id
\end{verbatim}

A designator refers to a named term. It can be a {\em simple name} or
a {\em selection}. If $r$ is a stable identifier of type $T$, the
selection $r.x$ refers to the term member of $r$ that is identified in
$T$ by the name $x$.  For other expressions $e$, $e.x$ is typed as if
it was $(\VAL;y=e\semi y.x)$ for some fresh name $y$. The typing rules
for blocks implies that in that case $x$'s type may not refer to any
abstract type member of \verb@e@.

The type of a designator is normally the type of the entity it refers
to. However, if the designator is a path \sref{sec:paths} \verb@p@
it's type is \verb@p.type@, provided the expression's expected type is
a singleton type, or \verb@p@ occurs as the prefix of a selection,
type selection, or mixin super expression.

The selection $e.x$ is evaluated by first evaluating the qualifier
expression $e$. The selection's result is then the value to which the
selector identifier is bound in the selected object designated by $e$.

\section{This and Super}
\label{sec:this-super}

\syntax\begin{verbatim}
  SimpleExpr    \=::= \= [Ident `.'] $\This$
                \>  | \> [Ident `.'] super `.' Id
\end{verbatim}

The expression \verb@this@ can appear in the statement part of a
template or compound type. It stands for the object being defined by
the innermost template or compound type enclosing the reference. If
this is a compound type, the type of \verb@this@ is the compount type.
If it is a template of an instance creation expression, the type of
\verb@this@ is the type of that template. If it is a template of a
class or object definition with simple name \verb@C@, the type of this
is the same as the type of \verb@C.this@.

The expression \verb@C.this@ is legal in the statement part of an
enclosing class or object definition with simple name \verb@C@. It
stands for the object being defined by the innermost such definition.
Its type is the self type of that definition if one is given;
otherwise its type is \verb@C.this.type@.

A reference \verb@super.$m$@ in a template refers to the definition of
\verb@m@ in the actual superclass (\sref{sec:base-classes}) of the
template.  A reference \verb@C.super.m@ refers to the definition of
\verb@m@ in the actual superclass of the innermost enclosing class or
object definition enclosing the reference. The definition referred to
by \verb@super@ or \verb@C.super@ must be concrete, or the template
containing the reference must contain a definition which has an
\verb@override@ modifier and which overrides \verb@m@.

\example\label{ex:super}
Consider the following class definitions

\begin{verbatim}
class Root { val x = "Root" }
class A extends Root { override val x = "A" ; val superA = super.x }
class B extends Root { override val x = "B" ; val superB = super.x }
class C extends A with B with { override val x = "C" ; val superC = super.x }
class D extends A with { val superD = super.x }
class E extends C with D with { val superE = super.x }
\end{verbatim}
Then we have:
\begin{verbatim}
new A.superA = "Root", new B.superB = "Root"
new C.superA = "Root", new C.superB = "A", new C.superC = "A"
new D.superA = "Root", new D.superD = "A"
new E.superA = "Root", new E.superB = "A", new E.superC = "A",
   new E.superD = "C", new E.superE = "C"
\end{verbatim}
Note that the \verb@superB@ function returns different results
depending on whether \verb@B@ is used as defining class or as a mixin class.

\section{Mixin Super}
\label{sec:this-mxin-super}

\syntax\begin{verbatim}
  SimpleExpr    \=::= \= Id `#' Id
\end{verbatim}

In the statement part of a template
the reference \verb@M#m@ refers to the definition of the member \verb@m@ in
the mixin parent class of the template with simple name \verb@M@.

\example Consider the following class definitions:
\begin{verbatim}
class Shape {
  def equals(other: Shape);
  ...
}
trait Bordered extends Shape {
  val thinkness: int
  def equals(other: Any) = other match {
    case that: Bordered => this.thickness == that.thickness;
    case _ => false
  }
  ...
}
trait Colored extends Shape {
  val color: Color
  def equals(other: Any) = other match {
    case that: Colored => this.color == that.color;
    case _ => false
  }
  ...
}
\end{verbatim}

All three definitions of \verb@equals@ are combined in the class
below, which makes use of \verb@super@ as well as mixin
super references.
\begin{verbatim}
class BorderedColoredShape extends Shape with Bordered with Colored {
  def equals(other: Any) = other match {
    case that: BorderedColoredShape => 
      super.equals(that) && Bordered # equals(that) && Colored # equals(that)
    case _ => false
  }
}
\end{verbatim}

\section{Function Applications}
\label{sec:apply}

\syntax\begin{verbatim}
  SimpleExpr    \=::= \= SimpleExpr ArgumentExpr
\end{verbatim}

An application \verb@f(e$_1$ \commadots e$_n$)@ applies the function \verb@f@ to the
argument expressions \verb@e$_1$ \commadots e$_n$@. If \verb@f@ has a method type
\verb@(x$_1$: T$_1$ \commadots x$_n$: T$_n$)U@, the type of each argument
expression \verb@e$_i$@ must conform to the corresponding parameter type
\verb@T$_i$@. If \verb@f@ has some value type, the application is taken to be
equivalent to \verb@f.apply(e$_1$ \commadots e$_n$)@, i.e.\ the
application of an \verb@apply@ function defined by \verb@f@.

%Class constructor functions
%(\sref{sec:classes}) can only be applied in constructor invocations
%(\sref{sec:constr-invoke}), never in expressions.

Evaluation of \verb@f(e$_1$ \commadots e$_n$)@ usually entails evaluation of
\verb@f@ and \verb@e$_1$ \commadots e$_n$@ in that order. Each argument expression
is converted to the type of its corresponding formal parameter.  After
that, the application is rewritten to the function's right hand side,
with actual arguments substituted for formal parameters.  The result
of evaluating the rewritten right-hand side is finally converted to
the function's declared result type, if one is given.

The case where of a formal \verb@def@-parameter with a parameterless
method type \verb@[]T@ is treated specially. In this case, the
corresponding actual argument expression is not evaluated before the
application. Instead, every use of the formal parameter on the
right-hand side of the rewrite rule entails a re-evaluation of the
actual argument expression. In other words, the evaluation order for
\verb@def@-parameters is {\em call-by-name} whereas the evaluation
order for normal parameters is {\em call-by-value}.

\example A function equivalent to a \verb@while@-loop could be defined as:

\begin{verbatim}
  def whileLoop(def c: boolean)(def s: unit): unit  =
    if (c) { s ; while(c)(s) } else {}
\end{verbatim}
Therefore the call
\begin{verbatim}
  whileLoop (x != 0) { y = y + 1/x ; x = x - 1 }
\end{verbatim}
will never produce a division-by-zero error at run-time, since the
expression \verb@(y = 1/x)@ will be evaluated in the body of
\verb@while@ only if the condition parameter is false.

\section{Type Applications}
\label{sec:type-app}
\syntax\begin{verbatim}
  SimpleExpr    \=::= \= SimpleExpr `[' Types `]'
\end{verbatim}

A type application \verb@f[T$_1$, ..., T$_n$]@ instantiates a
polymorphic value \verb@f@ of type
\verb@[a$_1$ >: L$_1$ <: U$_1$, ..., a$_n$ >: L$_n$ <: U$_n$]S@ with
argument types \verb@T$_1$, ..., T$_n$@.  Every argument type
\verb@T$_i$@ must obey corresponding bounds \verb@L$_i$@ and
\verb@U$_i$@.  That is, for each \verb@i = 1 \commadots n@, we must
have \verb@L$_i \sigma$ <: T$_i$ <: U$_i \sigma$@, where $\sigma$ is the
substitution \verb@[a$_1$ := T$_1$, ..., a$_n$ := T$_n$]@.  The type
of the application is \verb@S$\sigma$@.  

The function part \verb@f@ may also have some value type. In this case
the type application is taken to be equivalent to
\verb@f.apply[\verb@T$_1$, ..., T$_n$]@, i.e.\ the
application of an \verb@apply@ function defined by \verb@f@.

Type applications can be omitted if local type inference
(\sref{sec:local-type-inf}) can infer best type parameters for a
polymorphic functions from the types of the actual function arguments
and the expected result type.

\section{References to Overloaded Bindings}
\label{sec:overloaded-refs}

If a name \verb@f@ referenced in an identifier or selection is
overloaded (\sref{sec:overloaded-types}), the context of the reference
has to identify a unique alternative of the overloaded binding. The
way this is done depends on whether or not \verb@f@ is used as a
function.  Let $\AA$ be the set of all type alternatives of
\verb@f@.

Assume first that \verb@f@ appears as a function in an application, as
in \verb@f(args)@.  If there is precisely one alternative in
\verb@\AA@ which is a (possibly polymorphic) method type whose arity
matches the number of arguments given, that alternative is chosen.

Otherwise, let \verb@argtypes@ be the vector of types obtained by
typing each argument with an undefined prototype. One determines first
the set of applicable alternatives. A method type alternative is {\em
applicable} if each type in \verb@argtypes@ is compatible with the
corresponding formal parameter type in the alternative, and, if a
prototype is given, the method's result type is compatible to it.  A
polymorphic method type is applicable if local type inference can
determine type arguments so that the instantiated method type is
applicable.

Here, a type \verb@T@ is {\em compatible} to a type \verb@U@ if one of the
following three clauses applies:
\begin{enumerate}
\item
\verb@T@ conforms to \verb@U@.
\item
\verb@T@ is a parameterless method type \verb@[]T'@ and \verb@T'@
conforms to \verb@U@.
\item 
\verb@T@ is a monomorphic method type \verb@(ps$_1$) ... (ps$_n$)S@ which
can be converted to a function type \verb@T'@ by using the rules for
implicit conversions \sref{sec:impl-conv} and \verb@T'@ conforms to
\verb@U@.
\end{enumerate}

Let $\BB$ be the set of applicable alternatives. It is an error if
$\BB$ is empty. Otherwise, one chooses the {\em most specific}
alternative among the alternatives in $\BB$, according to the
following definition of being ``more specific''.
\begin{itemize} 
\item
A method type \verb@(ps)T@ is more specific than some other
type \verb@S@ if \verb@S@ is applicable to arguments \verb@(ps)@.
\item
A polymorphic method type
\verb@[a$_1$ >: L$_1$ <: U$_1$, ..., a$_n$ >: L$_n$ <: U$_n$](ps)T@ is
more specific than some other type \verb@S@ if \verb@(ps)T@ is more
specific than \verb@S@ under the assumption that for
\verb@i = 1 \commadots n@ each \verb@a$_i$@ is an abstract type name
bounded from below by \verb@L$_i$@ and from above by \verb@U$_i$@.
\item
Any other type is always more specific than a parameterized method
type or a polymorphic type.
\end{itemize}
It is an error if there is no unique alternative in $\BB$ which is
more specific than all other alternatives in $\BB$.

Assume next that \verb@f@ appears as a function in a type
application, as in \verb@f[targs]@. Then we choose an alternative in
$\AA$ which takes the same number of type parameters as there are
type arguments in \verb@targs@. It is an error if no such alternative
exists, or if it is not unique.

Assume finally that \verb@f@ does not appear as a function in either
an application or a type application. If an expected type is given,
let \verb@BB@ be the set of those alternatives in \verb@AA@ which are
compatible to it. Otherwise, let \verb@BB@ be the same as \verb@AA@.
We choose in this case the most specific alternative among all
alternatives in \verb@BB@. It is an error if there is no unique
alternative in $\BB$ which is more specific than all other
alternatives in $\BB$.

\example Consider the following definitions:

\begin{verbatim}
  class A extends B {}
  def f(x: B, y: B) = ...
  def f(x: A, y: B) = ...
  val a: A, b: B
\end{verbatim}
Then the application \verb@f(b, b)@ refers to the first
definition of \verb@f@ whereas the application \verb@f(a, a)@
refers to the second.  Assume now we add a third overloaded definition
\begin{verbatim}
  def f(x: B, y: A) = ...
\end{verbatim}
Then the application \verb@f(a, a)@ is rejected for being ambiguous, since
no most specific applicable signature exists.

\section{Instance Creation Expressions}
\label{sec:inst-creation}

\syntax\begin{verbatim}
  Expr4     \=::= \= new Template
\end{verbatim}

A simple instance creation expression is \verb@new c@ where \verb@c@
is a constructor invocation (\sref{sec:constr-invoke}).  Let \verb@T@
be the type of \verb@c@. Then \verb@T@ must denote a (a type instance
of) a reference class which is non-abstract, and which conforms to its self
type. The expression is evaluated by creating a fresh object of the
type \verb@T@, which is is initialized by evaluating \verb@c@. The
type of the expression is \verb@T@'s self type (which might be less
specific than \verb@T@).

A general instance creation expression is
\verb@new sc with mc$_1$ with ... with mc$_n$ {stats}@ where
\verb@n $\geq$ 0@, \verb@sc@ as well as \verb@mc$_1$, ..., mc$_n$@ are
constructor invocations (of types \verb@S, T$_1$, ...,T$_n$@, say) and
\verb@stats@ is a statement sequence containing initializer statements
and member definitions (\sref{sec:members}). The type of such an
instance creation expression is then the compound type
\verb@S with T$_1$ with ... with T$_n$ {R}@, where \verb@{R}@ is a
refinement (\sref{sec:compound-types}) which declares exactly those
members of \verb@stats@ that override a member of \verb@S@ or
\verb@T$_1$, ..., T$_n$@. For this type to be well-formed, \verb@R@
may not reference types defined in \verb@stats@ which do not
themselves form part of \verb@R@.

The instance creation expression is evaluated by creating a fresh
object, which is initialized by evaluating the expression template.

\example Consider the class
\begin{verbatim}
abstract class C {
  type T; val x: T; def f(x: T): Object
}
\end{verbatim}
and the instance creation expression
\begin{verbatim}
C { type T = Int; val x: T = 1; def f(x: T): T = y; val y: T = 2 }
\end{verbatim}
Then the created object's type is:
\begin{verbatim}
C { type T = Int; val x: T; def f(x: T): T }
\end{verbatim}
The value \verb@y@ is missing from the type, since \verb@y@ does not
override a member of \verb@C@.

\section{Blocks}
\label{sec:blocks}

\syntax\begin{verbatim}
  BlockExpr   \=::= \= `{' Block `}'
  Block       \>::= \> [{BlockStat `;'} Expr]
\end{verbatim}

A block expression
\verb@{s$_1$; ...; s$_n$; e} is constructed from a sequence of block statements \verb@s$_1$
, ..., s$_n$@ and a final expression \verb@e@. The final expression
can be omitted, in which case the unit value \verb@()@ is assumed.

%Whether or not the scope includes the statement itself
%depends on the kind of definition.

The expected type of the final expression \verb@e@ is the expected
type of the block. The expected type of all preceding statements is
undefined.

The type of a block \verb@s$_1$; ...; s$_n$; e@ is usually the type of
\verb@e@.  That type must be equivalent to a type which does not refer
to an entity defined locally in the block. If this condition is
violated, but a fully defined expected type is given, the type of the
block is instead assumed to be the expected type.

Evaluation of the block entails evaluation of its statement sequence,
followed by an evaluation of the final expression \verb@e@, which
defines the result of the block.

\example
Written in isolation, 
the block \verb@{ class C extends B {...} ; new C }@ is illegal, since its type
refers to class \verb@C@, which is defined locally in the block.

However, when used in a definition such as 
\begin{verbatim}
val x: B = { class C extends B {...} ; new C }
\end{verbatim}
the block is well-formed, since the problematic type \verb@C@ can be
replaced by the expected type \verb@B@.

\section{Prefix, Infix, and Postfix Operations}
\label{sec:infix-operations}

\syntax\begin{verbatim}
  PostfixExpr     \=::=\= InfixExpr [Id]
  InfixExpr       \>::=\> PrefixExpr
		  \>  |\> InfixExpr Id InfixExpr
  PrefixExpr      \>::=\> [`-' | `+' | `!' | `~'] SimpleExpr 
\end{verbatim}

Expressions can be constructed from operands and operators.  A prefix
operation \verb@op e@ consists of a prefix operator \verb@op@, 
which must be one of \verb@+@, \verb@-@, \verb@!@, or \verb@~@,
and a simple expression \verb@e@.  The
expression is equivalent to the postfix method application \verb@e.op@.

Prefix operators are different from normal function applications in
that their operand expression need not be atomic. For instance, the
input sequence \verb@-sin(x)@ is read as \verb@-(sin(x))@, whereas the
function application \verb@negate sin(x)@ would be parsed as the
application of the infix operator \verb@sin@ to the operands
\verb@negate@ and \verb@(x)@.

An infix or postfix operator can be an arbitrary identifier. Binary
operators have precedence and associativity defined as follows:

The {\em precedence} of an operator is determined by the operator's first
character. Characters are listed below in increasing order of
precedence, with characters on the same line having the same precedence.
\begin{verbatim}
	(all letters)
	|
	^
	&
	< >
	= !
        :
	+ -
	* / %
	(all other special characters)
\end{verbatim}
That is, operators starting with a letter have lowest precedence,
followed by operators starting with `\verb@|@', etc.

The {\em associativity} of an operator is determined by the operator's
last character.  Operators ending with a colon `\verb@:@' are
right-associative. All other operators are left-associative.

Precedence and associativity of operators determine the grouping of
parts of an expression as follows.
\begin{itemize}
\item If there are several infix operations in an
expression, then operators with higher precedence bind more closely
than operators with lower precedence.
\item If there are consecutive infix
operations \verb@e$_0$ op$_1$ e$_1$ op$_2$ ... op$_n$ e$_n$@ 
with operators \verb@op$_1$, ..., op$_n$@ of the same precedence, 
then all these operators must
have the same associativity. If all operators are left-associative,
the sequence is interpreted as
\verb@(...(e$_0$ op$_1$ e$_1$) op$_2$...) op$_n$ e$_n$@. 
Otherwise, if all operators are right-associative, the
sequence is interpreted as
\verb@e$_0$ op$_1$ (e$_1$ op$_2$ (... op$_n$ e$_n$)...)@.
\item
Postfix operators always have lower precedence than infix
operators. E.g.\ \verb@e$_1$ op$_1$ e$_2$ op$_2$@ is always equivalent to
\verb@(e$_1$ op$_1$ e$_2$) op$_2$@.
\end{itemize}
A postfix operation \verb@e;op@ is interpreted as \verb@e.op@. A
left-associative binary operation \verb@e$_1$ op e$_2$@ is interpreted as
\verb@e$_1$.op(e$_2$)@. If \verb@op@ is right-associative, the same operation is
interpreted as \verb@(val x=e$_1$; e$_2$.op(x))@, 
where \verb@x@ is a fresh name.

\section{Typed Expressions}

\syntax\begin{verbatim}
  Expr	       \=::= \= PostfixExpr [`:' Type1]
\end{verbatim}

The typed expression \verb@e: T@ has type \verb@T@. The type of
expression \verb@e@ is expected to conform to \verb@T@. The result of
the expression is the value of \verb@e@ converted to type \verb@T@.

\example Here are examples of well-typed and illegal typed expressions.

\begin{verbatim}
  1: int                             \=// legal, of type int
  1: long	    	\>// legal, of type long
  // 1: string 	\>// illegal
\end{verbatim}

\section{Assignments}

\syntax\begin{verbatim}
  Expr	       \=::= \= Designator `=' Expr
	       \>  | \> SimpleExpr ArgumentExpr `=' Expr
\end{verbatim}

An assignment to a simple variable \verb@x = e@ is interpreted as the invocation
\verb@x_=(e)@ of the setter function for variable
\verb@x@ (\sref{sec:vardef}).
Analogously, an assignment \verb@f.x = e@ to a field \verb@x@ is interpreted
as the invocation \verb@f.x_=(e)@.

An assignment \verb@f(args) = e@ with a function application to the
left of the ``\verb@=@' operator is interpreted as 
\verb@f.update(args, e)@, i.e.\
the invocation of an \verb@update@ function defined by \verb@f@.

\example Here is the usual imperative code for matrix multiplication.

\begin{verbatim}
def matmul(xs: Array[Array[double]], ys: Array[Array[double]]) = {
  val zs: Array[Array[double]] = new Array(xs.length, ys.length);
  var i = 0;
  while (i < xs.length) {
    var j = 0;
    while (j < ys(0).length) {
      var acc = 0.0;
      var k = 0;
      while (k < ys.length) {
	acc = acc + xs(i)(k) * ys(k)(j);
	k = k + 1
      }
      zs(i)(j) = acc;
      j = j + 1
    }
    i = i + 1
  }
  zs
}
\end{verbatim}
Desugaring the array accesses and assignments yields the following
expanded version:
\begin{verbatim}
def matmul(xs: Array[Array[double]], ys: Array[Array[double]]) = {
  val zs: Array[Array[double]] = new Array(xs.length, ys.length);
  var i = 0;
  while (i < xs.length) {
    var j = 0;
    while (j < ys(0).length) {
      var acc = 0.0;
      var k = 0;
      while (k < ys.length) {
	acc = acc + xsa.apply(i).apply(k) * ys.apply(k).apply(j);
	k = k + 1
      }
      zs.apply(i).update(j, acc);
      j = j + 1
    }
    i = i + 1
  }
  zs
}
\end{verbatim}

\begin{verbatim}
def matmul(xss: Array[Array[double]], yss: Array[Array[double]]) = {
  val ysst = transpose(yss);
  for (val xs <- xs) yield
    for (val ys <- ysst) yield 
      scalprod(xs, ys)
}

def transpose[a](xss: Array[Array[a]]) {
  for (val i <- Array.range(0, xss(0).length)) yield
    Array(for (val xs <- xss) yield xs(i))

def scalprod(xs: Array[double], ys: Array[double]) {
  var acc = 0.0;
  for (val Pair(x, y) <- xs zip ys) do acc = acc + x * y; 
  acc
}
\end{verbatim}


\section{Conditional Expressions}

\syntax\begin{verbatim}
  Expr	       \=::= \= \verb@\IF@ `(' Expr `)' Expr [\verb@\ELSE@ Expr]
\end{verbatim}

The conditional expression \verb@if (\verb@e$_1$@) \verb@e$_2$@ else \verb@e$_3$@@ is treated as
a shorthand for the expression
\begin{verbatim}
\verb@e$_1$@.ifThenElse(\verb@e$_2$@)(\verb@e$_3$@)  .
\end{verbatim}
The conditional expression \verb@if (\verb@e$_1$@) \verb@e$_2$@@ without an
else-part is treated as a shorthand for the expression
\begin{verbatim}
\verb@e$_1$@.ifThen(\verb@e$_2$@)  .
\end{verbatim}
The predefined type \verb@Boolean@ (\sref{sec:cls-Boolean})
contains implementations of
methods \verb@ifThen@ and \verb@ifThenElse@.

\section{Comprehensions}

\todo{change syntax; treat refulatble patterns as filters}

\syntax\begin{verbatim}
  Expr	           \=::= \= for `(' Enumerator `)' [do | yield] Expr
  Enumerator      \>::=\> Generator {`;' Enumerator1}
  Enumerator1     \>::=\> Generator
                  \>  |\> Expr
                  \>  |\>
  Generator       \>::=\> val Pattern `<-' Expr
\end{verbatim}

A comprehension \verb@for (g) e@ evaluates expression \verb@e@ for each
binding generated by the enumerator \verb@g@. An enumerator is a generator,
possibly followed by further generators or filters.  A generator
\verb@val p <- e@ produces bindings from an expression \verb@e@ which is
matched in some way against pattern \verb@p@. Filters are expressions which
restrict enumerated bindings. The precise meaning of generators and
filters is defined by translation to invocations of four methods:
\verb@map@, \verb@filter@, \verb@flatMap@, and \verb@foreach@. These
methods can be implemented in different ways for different carrier
types.  As an example, an implementation of these methods for lists is
given in (\sref{cls-list}).

The translation scheme is by repeated applications of the following
rules, until the comprehension has been eliminated.

\ifnewfor{
First, a definition: A pattern \verb@P@ is a {\em binding} if \verb@P@ is a
variable pattern or a tuple pattern consisting only of pattern
variables. In the following, we let \verb@B@ range over bindings, \verb@P@ over
patterns other than bindings, \verb@E, F@ over expressions, and \verb@G@ over
enumerators.

%\begin{itemize}
%\item
If
\verb@x$_1$ \commadots x$_n$@ are the free variables of \verb@p@, then
the generator \verb@p <- e@ is translated to:
\begin{verbatim}
\verb@(x$_1$ \commadots x$_n$)@ <- e.filter(case p => True case _ => False).map(case p => \verb@(x$_1$ \commadots x$_n$)@)
\end{verbatim}

%\item
A generator \verb@P <- E@ followed by a filter \verb@F@ is translated to
a single generator \verb@P <- E.filter(x$_1$ \commadots x$_n$ => F)@.

%\item
The comprehension \verb@for (B <- E) E'@ is translated to
\verb@E.map(B => E')@.

%\item
The comprehension \verb@for (B <- E, G) E'@ is translated to
\begin{verbatim}
(val x$\Dollar$ = E ; x$\Dollar$.combine(for (B <- E) for (G) E'))
\end{verbatim}

%\end{itemize}
}
\begin{itemize}
\item
A for-comprehension
\verb@for (val p <- e) yield e'@
is translated to
\verb@e.map(x$_1$, ..., x$_n$ => e')@.
where x$_1$, ..., x$_n$ are the free variables of $p$.
A for-comprehension
\verb@for (val p <- e) do e'@
is translated to
\verb@e.foreach(x$_1$, ..., x$_n$ => e')@.
\item
A for-comprehension
\verb@for (val p <- e; f; s) yield e'@
where \verb@f@ is a filter expression and \verb@s@ is a (possibly empty)
sequence of generators or filters
is translated to
\verb@for (val p <- e.filter(x$_1$, ..., x$_n$ => f); s) yield e'@
where x$_1$, ..., x$_n$ are the free variables of $p$. The translation
of \verb@for@-\verb@do@ comprehensions is analogous for this case.
\item
A for-comprehension
\begin{verbatim}
for (val p <- e; val p' <- e'; s) yield e''
\end{verbatim}
where \verb@s@ is a (possibly empty)
sequence of generators or filters
is translated to
\begin{verbatim}
e.flatmap(x$_1$, ..., x$_n$ => for (val p' <- e'; s) yield e'') .
\end{verbatim}
A for-comprehension
\begin{verbatim}
for (val p <- e; val p' <- e'; s) do e'' .
\end{verbatim}
is translated to
\begin{verbatim}
e.foreach(x$_1$, ..., x$_n$ => for (val p' <- e'; s) do e'')
\end{verbatim}
\end{itemize}

\example
the following code produces all pairs of numbers
between \verb@1@ and \verb@n@ whose sums are prime.
\begin{verbatim}
for \= { \= val i <- range(1, n);
    \>   \> val j <- range(1, i-1);
    \>   \> isPrime(i+j)
} yield (i, j)
\end{verbatim}
The for-comprehension is translated to:
\begin{verbatim}
range(1, n)
  .flatMap(
     i => range(1, i-1)
            .filter(j => isPrime(i+j))
            .map(j => (i, j)))
\end{verbatim}

\comment{
\example
\begin{verbatim}
package class List[a] with {
  def map[b](f: (a)b): List[b] = match {
    case <> => <>
    case x :: xs => f(x) :: xs.map(f)
  }
  def filter(p: (a)Boolean) = match {
    case <> => <>
    case x :: xs => if p(x) then x :: xs.filter(p) else xs.filter(p)
  }
  def flatMap[b](f: (a)List[b]): List[b] =
    if (isEmpty) Nil
    else f(head) ::: tail.flatMap(f);
  }
  def foreach(f: (a)Unit): Unit =
    if (isEmpty) ()
    else (f(head); tail.foreach(f));
}
\end{verbatim}

\example
\begin{verbatim}
abstract class Graph[Node] {
  type Edge = (Node, Node)
  val nodes: List[Node]
  val edges: List[Edge]
  def succs(n: Node) = for ((p, s) <- g.edges, p == n) s
  def preds(n: Node) = for ((p, s) <- g.edges, s == n) p
}
def topsort[Node](g: Graph[Node]): List[Node] = {
  val sources = for (n <- g.nodes, g.preds(n) == <>) n
  if (g.nodes.isEmpty) <>
  else if (sources.isEmpty) new Error(``topsort of cyclic graph'') throw
  else sources :+: topsort(new Graph[Node] with {
    val nodes = g.nodes diff sources
    val edges = for ((p, s) <- g.edges, !(sources contains p)) (p, s)
  })
}
\end{verbatim}
}

\section{Tuples}

\syntax\begin{verbatim}
  ArgumentExpr   \=::= \= `(' [Expr `,' Exprs] `)'
\end{verbatim}

An expression $(e_1 \commadots e_n)$ where $n \geq 2$
is a shorthand for the constructor invocation
\verb@Tuple$\,n$($e_1 \commadots e_n)$@.
The empty tuple $()$ is a shorthand for the constructor invocation
\verb@Unit@.
There are no tuples of length 1, since $(e)$ is seen as an expression
in parentheses, not as a tuple.
See \sref{sec:cls-tuple}
for a definition of the \verb@Unit@ and \verb@Tuple$\,n$@ classes.
\comment{
An $n$-tuple type implicitly defines parameterless functions $\_1 \commadots
\_n$ that select the tuple's components.

\example Here is an example of how tuples are formed and accessed.

\begin{verbatim}
  def helloWorld = {
    val pair = ("hello", "world")
    println (pair._1 + " " + pair._2)
  }
\end{verbatim}
}

\section{List Expressions}
\label{sec:list-exprs}

\syntax\begin{verbatim}
  ListExpr	  \=::= \= `[' [Exprs] `]'
\end{verbatim}

The list expression $[ e_1 \commadots e_n ]$ with $n \geq 0$ is a
shorthand for
\begin{verbatim}
 cons(e$_1$, ..., cons(e$_n$, nil) ... ) .
\end{verbatim}
\verb@cons@ and \verb@nil@ are defined in module \verb@Predef@ as:
\begin{verbatim}
    def cons[a](x: a, xs: List[a]): List[a] = new ::_class(x)(xs);
    def nil[a]: List[a] = Nil[a];
\end{verbatim}
Like all other members of module \verb@Predef@, these definitions
are implicitly imported in all programs.

\example Here is how the basic type of lists can be defined, so that
list expressions ``do the right thing''.
\begin{verbatim}
final class List[a] with {
  def ::(x: a) = new ::_class(x)(this);
}
case class Nil[a] extends List[a];
case class ::_class[a](head: a)(tail: List[a]) extends List[a];
\end{verbatim}
Class \verb@List@ is declared \verb@final@ which has as
consequence that no further case classes for \verb@List@ can be
defined except for the two case classes \verb@Nil@ and \verb@::_class@ that are
defined together with the class.

Note the definition of ``\verb@::@'' as a method in class \verb@List@
and of \verb@::_class@ as a case class outside.  This lets clients of
\verb@List@ compose and decompose lists with the infix operator
``\verb@::@''. E.g.\ \verb@x :: y :: zs@ is a valid list expression
because ``\verb@::@'' is a unary method of \verb@List@ and it is a
valid infix operation pattern (\sref{sec:patterns}) because
``\verb@::_class@'' is a case class with two parameter sections.

Note that list expressions are not argument expressions, hence it is
necessary to enclose them in parentheses when used as parameters.
This is necessary to distinguish list expression parameters from type arguments.

\example\

\begin{verbatim}
  f([x, y])	       \=// application of function `f' to the list `[x,y]'
  g[x, y]	       \>// instantiation of polymorphic `g' with types `x, y'
\end{verbatim}


\section{Anonymous Functions}
\label{sec:closures}

\syntax\begin{verbatim}
  ArgumentExpr    \=::= \= `(' Closure `)'
  Closure         \>::= \> [Bindings] `=>' Closure
                  \>  | \> Block
  Bindings	  \>::= \> Binding {`,' Binding}
  Binding \>::= \> Id [`:' Type]
\end{verbatim}

An anonymous function is a closure $x_1: T_1 \commadots x_n: T_n
\Arrow c$ in parentheses, where $n \geq 0$ and $c$ is a block or another closure.
The types of the parameters can be omitted if they can be inferred
from the context by local type inference.  The scope of every
parameter name is the closure $c$. Let $T_1
\commadots T_n$ be the types of the parameters and let $U$ be the type
of $c$. The type $U$ may not depend on any of the parameter
names $x_1 \commadots x_n$. The type of the whole expression is then
$(T_1 \commadots T_n) U$. The expression is interpreted as
\begin{verbatim}
  scala.Function$\,n$[$T_1 \commadots T_n, U$] with ( def apply($x_1 \commadots x_n$) = $c$ )  .
\end{verbatim}

\example Examples of anonymous functions:

\begin{verbatim}
  (x => x)                                         \=// The identity function

  (f => g => x => f(g(x)))	\>// Curried function composition

  (x: Int,y: String => (y,x))   \>// A function which swaps its arguments

  ( => count = count + 1; count)   \>// The function which takes an empty
                                   \>// parameter list $()$, increments a global
                                   \>// variable count and returns the new value.
\end{verbatim}

\section{Statements}
\label{sec:statements}

\syntax\begin{verbatim}
  BlockStat        \=::= \=  Import
            \>  | \> Def
	    \>  | \> $\VAL$ SimplePattern Id Expr
	    \>  | \> [Expr]
  TemplateStat    \>::= \>  Import
	\>  | \> {Modifier} Def
	\>  | \> {Modifier} Dcl
        \>  | \> [Expr]
  PureStat        \>::=\> Import
                  \>  |\> {Modifier} PureDef
		  \>  |\> `;'
\end{verbatim}

A statement can be a definition or an expression, or it can be empty.
Statements used in the template of a class definition can also be
declarations.  An expression that is used as a statement can have an
arbitrary value type. An expression statement $e$ is evaluated by
evaluating $e$ and discarding the result of the evaluation.

Block statements may be definitions which bind local names in the
block. No modifiers are allowed in block-local definitions.

With the exception of overloaded definitions
(\sref{sec:overloaded-defs}), a statement sequence making up a block
or template may not contain two definitions or declarations that bind
the same name in the same namespace.  Evaluation of a statement
sequence entails evaluation of the statements in the order they are
written.

{\em Pure statements} are statements that are guaranteed to be
evaluated without side effect. A pure statement is either the empty
statement, or an import clause, or a pure definition (\sref{sec:defs}).

\chapter{Pattern Matching}

\section{Patterns}

% 2003 July - changed to new pattern syntax + semantic Burak

\label{sec:patterns}

\syntax\begin{verbatim}
Pattern        \=::=  \= TreePattern { `|' TreePattern }

TreePattern    \>::=  \> varid `:' Type
               \> |   \> `_' `:' Type
               \> |   \> SimplePattern [ '*' | '?' | '+' ]
               \> |   \> SimplePattern { Id SimplePattern }

SimplePattern  \=::= \> varid [ '@' SimplePattern ]
               \> |  \> `_'
               \> |  \> literal
               \> |  \> StableId [ `(' [Patterns] `)' ]
               \> |  \> `(' Patterns `)'
               \> |  \> 

 Patterns      \>::= \> Pattern {`,' Pattern}
\end{verbatim}

A pattern is built from constants, constructors, variables and regular
operators. Pattern matching tests whether a given value (sequence
of values) has the shape defined by a pattern, and, if it does, binds
the variables in the pattern to the corresponding components of the
value (sequence of values).  The same variable name may not be
bound more than once in a pattern.

The type of a pattern and the expected types of variables within the
pattern are mainly determined by the context, except for patterns
that employ regular operators. In the latter case  the missing 
information is provided by the structure of the pattern. 
We distinguish the following kinds of patterns.

A {\em wild-card pattern} \_ matches any value. 

A {\em variable-binding pattern} $x @ p$ is a simple identifier $x$
which starts with a lower case letter, together with a pattern $p$. It
matches a value or a sequence of values whenever $p$ does, and in
addition binds the variable name to that value or to that sequence of
values. The type of $x$ is either $T$ as determined from the context, or
\verb@List[ T ]@, if $p$ matches sequences of values. A
special case is a {\em variable pattern} $x$ which is treated as $x @
\_$. 

A {\em typed pattern} $x: T$ consists of a pattern variable $x$ and a
simple type $T$. The type $T$ may be a class type or a compound type;
it may not contain a refinement (\sref{sec:refinements}).  This
pattern matches any value of type $T$ and binds the variable name to
that value.  $T$ must conform to the pattern's expected type. The
type of $x$ is $T$.

A {\em pattern literal} $l$ matches any value that is equal (in terms
of $==$) to it. It's type must conform to the expected type of the
pattern.

A {\em named pattern constant} $r$ is a stable identifier
(\sref{sec:stableids}). To resolve the syntactic overlap with a
variable pattern, a named pattern constant may not be a simple name
starting with a lower-case letter. The type of $r$ must conform
to the expected type of the pattern. The pattern matches any value $v$
such that \verb@$r$ == $v$@ (\sref{sec:cls-object}).

A {\em sequence pattern} $p_1 \commadots p_n$ where $n \geq 0$ is a
sequence of patterns separated by commata and matching the sequence of
values that are matched by the components. Sequence pattern may only
appear under constructor applications. Note that empty sequence
patterns are allowed. The type of the value patterns that appear in
the pattern is the expected type as determined from the context.

A {\em choice pattern} $p_1 | \ldots | p_n$ is a choice among several
alternatives, which may not contain variable-binding patterns. It
matches every value matched by at least one of its alternatives. Note
that the empty sequence may appear as an alternative.  An {\em option
pattern} $p?$ is an abbreviation for $(p| )$. If the alternatives
are value patterns, then the whole choice pattern is a value pattern,
whose type is the least upper bound of the types of the alternatives.

An {\em iterated pattern} $p*$ matches the sequence of values
consisting of zero, one or more occurrences of values matched by $p$,
where $p$ may not contain a variable-binding pattern. A {\em non-empty
iterated pattern} $p+$ is an abbreviation for $(p,p*)$. 

A non-regular sequence pattern is a sequence pattern $p_1 \commadots p_n$ 
where $n \geq 1$ with no component pattern containing iterated or nested
sequence patterns.

A {\em constructor pattern} $c ( p )$ consists of an identifier $c$,
followed by a pattern $p$.  The constructor $c$ is either a simple
name or a qualified name $r.id$ where $r$ is a stable identifier. It
refers to a (possibly overloaded) function which has one alternative
of result type \verb@class C@, and which may not have other overloaded
alternatives with a class constructor type as result type.
Furthermore, the respective type parameters and value parameters of
(said alternative of) $c$ and of the primary constructor function of
class $C$ must be the same, after renaming corresponding type
parameter names.  If $C$ is monomorphic and not a subclass of
\verb@Seq[ T ]@ then \verb@C@ must conform to the expected type of the
pattern, the pattern must be a non-regular sequence pattern $p_1
\commadots p_n$ whose length corresponds to the number of arguments of
$C$'s primary constructor. The expected types of the component
patterns are then taken from the formal parameter types of (said)
constructor.  If $C$ is polymorphic and not a subclass of
\verb@Seq[ T ]@, then there must be a unique type application instance
of it such that the instantiation of $C$ conforms to the expected type
of the pattern. The instantiated formal parameter types of $C$'s
primary constructor are then taken as the expected types of the
component patterns $p_1\commadots p_n$.  In both cases, the pattern
matches all objects created from constructor invocations
$c(v_1 \commadots v_n)$ where each component pattern $p_i$ matches the
corresponding value $v_i$. If $C$ is a subclass of \verb@Seq[ T ]@,
then $p$ may be an arbitrary pattern. Value patterns in $p$ are
expected to conform to type $T$, and the pattern matches all objects
whose \verb@elements()@ method returns a sequence that matches $p$.

The pattern $(p)$ is regarded as equivalent to the pattern $p$, if $p$
is a nonempty sequence pattern. The empty tuple $()$ is a shorthand
for the constructor pattern \verb@Unit@.

An {\em infix operation pattern} \verb@p id p'@ is a shorthand for the
constructor pattern \verb@id_class(p)(p')@.  The precedence and
associativity of operators in patterns is the same as in expressions
(\sref{sec:infix-operations}). The operands may not be empty sequence
patterns.

Regular expressions that contain variable bindings may be ambiguous,
i.e. there might be several ways to match a sequence against the
pattern. In these cases, the \emph{shortest-match policy} applies:
patterns that appear before other, overlapping patterns match
as little as possible.

\example Some examples of patterns are:
\begin{enumerate}
\item
The pattern \verb@ex: IOException@ matches all instances of class
\verb@IOException@, binding variable \verb@ex@ to the instance.
\item
The pattern \verb@(x, _)@ matches pairs of values, binding \verb@x@ to
the first component of the pair. The second component is matched
with a wildcard pattern.
\item
The pattern \verb+List( x, y, xs @ _ * )+ matches lists of length $\geq 2$,
binding \verb@x@ to the lists's first element, \verb@y@ to the list's
second element, and \verb@xs@ to the remainder, which may be empty.
\item
The pattern \verb=List( 1, x@(( 'a' | 'b' )+),y,_ )= matches a list that
contains 1 as its first element, continues with a non-empty sequence of 
\verb@'a'@s and \verb@'b'@s, followed by two more elements. The sequence 'a's and 'b's
is bound to \verb@x@, and the next to last element is bound to \verb@y@.
\item
The pattern \verb=List( x@( 'a'* ), 'a'+ )= matches a non-empty list of
\verb@'a'@s. Because of the shortest match policy, \verb@x@ will always be bound to
the empty sequence.
\item
The pattern \verb=List( x@( 'a'+ ), 'a'* )= also matches a non-empty list of
\verb@'a'@s. Here, \verb@x@ will always be bound to
the sequence containing one \verb@'a'@
\end{enumerate}

\subsection{Pattern Matching Expressions}
\label{sec:pattern-match}

\syntax\begin{verbatim}
  ArgumentExpr   \=::= \= `(' Case {Case} `)'
  Case	       \>::= \> $\CASE$ Patterns [$\IF$ Expr] `$=>$' Block
\end{verbatim}

A pattern matching expression $(\CASE;p_1 \Arrow b_1 ;\ldots; \CASE;p_n
\Arrow b_n)$ consists of a number $n \geq 1$ of cases.  Each case
consists of a pattern $p_i$ and a block $b_i$.  The scope
of the pattern variables in $p_i$ is the corresponding block
$b_i$.

The type of a pattern matching expression must in part be given from
the outside. It must be either \verb@Function[T$_p$, T$_r$]@ or
\verb@PartialFunction[T$_p$, T$_r$]@, where the argument type
\verb@T$_p$@ must be fully determined, but the result type \verb@T$_r$@
may be partially undetermined.  All patterns are typed relative to the
expected type $T_p$ (\sref{sec:patterns}).  Let $T_b$ be the least
type in the $(\conforms)$ ordering to which the types of all blocks
conform.  The type of the pattern matching expression is then the
required type with \verb@T$_r$@ replaced by \verb@T$_b$@ (i.e. the
type is either \verb@Function[T$_p$, T$_b$]@ or
\verb@PartialFunction[T$_p$, T$_b$]@ -- see \sref{sec:cls-function} for
a definition of these classes).

When applying a pattern matching expression to a selector value,
patterns are tried in sequence until one is found which matches the
selector value (\sref{sec:patterns}). Say this case is $\CASE;p_i
\Arrow b_i$.  The result of the whole expression is then the result of
evaluating $b_i$, where all pattern variables of $p_i$ are bound to
the corresponding parts of the selector value.  If no matching pattern
is found, a \verb@MatchError@ exception is thrown.

The pattern in a case may also be followed by a guard suffix $\IF;e$
with a Boolean expression $e$.  The guard expression is evaluated if
the preceding pattern in the case matches. If the guard expression
evaluates to \verb@True@, the pattern match succeeds as normal. If the
guard expression evaluates to \verb@False@, the patterns in the case
are considered not to match and the search for a matching pattern
continues.

A case with several patterns $\CASE;p_1 \commadots p_n ;\IF; e \Arrow b$  is a
shorthand for a sequence of single-pattern cases $\CASE;p_1;\IF;e \Arrow b
;\ldots; \CASE;p_n ;\IF;e\Arrow b$. In this case none of the patterns
$p_i$ may contain a named pattern variable (but the patterns may contain
wild-cards).

In the interest of efficiency the evaluation of a pattern matching
expression may try patterns in some other order than textual
sequence. This might affect evaluation through side effects in
let-bound (\sref{sec:letdef}) selector expressions or through side
effects in guards. However, it is guaranteed that a guard expression
is evaluated only if the pattern it guards matches.


\example
Often, pattern matching expressions are used as arguments
of the \verb@match@ method, which is predefined in class \verb@Any@
(\sref{sec:cls-object}) and is implemented there by postfix function
application. Here is an example:
\begin{verbatim}
def length [a] (xs: List[a]) = xs match {
  case Nil =>  0
  case x :: xs1  =>  1 + length (xs1)
}
\end{verbatim}

\chapter{Top-Level Definitions}
\label{sec:topdefs}

\syntax\begin{verbatim}
  ClsDef         \=::= \= Packaging
                 \>  | \> ClassDef
		 \>  | \> ModuleDef
  Packaging     \=::= \= package QualId with `(' {ClsDef `;'} `)'
\end{verbatim}

There are two kinds of compilation units: Main programs and library
units. A main program is a sequence of arbitrary definitions. A
library unit is a sequence of top-level definitions, i.e. class
definitions, module definitions, or packagings.

Implicitly imported into every compilation unit is the package
\verb@scala@ and the module \verb@scala.Predef@ (\sref{cls-predef}).

\section{Packagings}

\syntax\begin{verbatim}
  Packaging     \=::= \= package QualId with `(' {ClsDef `;'} `)'
\end{verbatim}

A packaging \verb@package P with ( D )@ declares all definitions in
\verb@D@ as members of the package whose qualified name is
\verb@P@. Packages are as in Java (this is likely to change in the
future).
\ifpackaging{
Packagings augment top-level modules and classes. A simple packaging
$$\PACKAGE;id;\WITH;mi_1;\ldots;\WITH;mi_n;\WITH;(stats)$$ augments the
template of the top-level module or class named $id$ with new mixin
classes and with new member definitions.

The static effect of such a packaging can be expressed as a
source-to-source tranformation which adds $mi_1 \commadots mi_n$ to
the mixin classes of $id$, and which adds the definitions in $stats$
to the statement part of $id$'s template.  Each type $mi_j$ must refer
to an interface type and $stats$ must consists only of pure and local
definitions.  The augmented template and any class that extends it
must be well-formed.  The aditional definitions may not overwrite
definitions of the augmented template, and they may not access private
members of it.

Several packagings can be applied to the same top-level definition,
and those packagings may reside in different compilation units.

A qualified packaging $\PACKAGE;Q.id;\WITH;t$ is equivalent to the
nested packagings
\begin{verbatim}
package $Q$ with {
  package $id$ with $t$
}
\end{verbatim}

A packaging with type parameters $\PACKAGE;c[tps];\WITH;...$ applies to
a parameterized class $c$. The number of type parameters must equal
the number of type parameters of $c$, and every bound in $tps$ must
conform to the corresponding bound in the original definition of $c$.

The augmented class has the type parameters given in its original
definition. If a parameter $a$ of an augmented class has a bound $T$
which is a strict subtype of the corresponding bound in the original
class, $a \conforms T$ is taken as an {\em application condition} for
the packaging. That is, every time a member defined in the packaging
is accessed or a conformance between class $c$ and a mixin base class
of the packaging needs to be established, an (instantiation of) the
application condition is checked. An unvalidated application
condition constitutes a type error. \todo{Need to specify more
precisely when application conditions are checked}

\example The following example will create a hello world program as
function \verb@main@ of module \verb@test.HelloWorld@.
\begin{verbatim}
package test with {
  module HelloWord with {
    def main(args: Array[String]) = out.println("hello world")
  }
}
\end{verbatim}
This assumes there exists a top-level definition that defines a
\verb@test@ module, e.g.:
\begin{verbatim}
module test
\end{verbatim}

\example The following packaging adds class \verb@Comparable@
(\ref{ex:comparable}) as a mixin to class
\verb@scala.List@, provided the list elements are also comparable.
Every instance of \verb@List[$T$]@ will then implement
\verb@Comparable[List[$T$]]@ in the way it is defined in the
packaging. Each use of the added functionality for an instance type
\verb@List[$T$]@ requires that the application condition
\verb@T $<:$ Comparable@ is satisfied.
\begin{verbatim}
package scala.List[a extends Comparable[a]] with Comparable[List[a]] with {
  def < (that: List[a]) = (this, that) match {
    case (_, Nil) => False
    case (Nil, _) => True
    case (x :: xs, y :: ys) => (x < y) || (x == y && xs < ys)
  }
}
\end{verbatim}
}
\chapter{Local Type Inference}
\label{sec:local-type-inf}

This needs to be specified in detail.
Essentially, similar to what is done for GJ.

\comment{
\section{Definitions}

For a possibly recursive definition such as $\LET;x_1 = e_1
;\ldots; \LET x_n = e_n$, local type inference proceeds as
follows.
A first phase assigns {\em a-priori types} to the $x_i$. The a-priori
type of $x$ is the declared type of $x$ if a declared type is
given. Otherwise, it is the inherited type, if one is
given. Otherwise, it is undefined.

A second phase assigns completely defined types to the $x_i$, in some
order.  The type of $x$ is the a-priori type, if it is completely
defined. Otherwise, it is the a-priori type of $x$'s right hand side.
The a-priori type of an expression $e$ depends on the form of $e$.
\begin{enumerate}
\item
The a-priori type of a
typed expression $e:T$ is $T$.
\item
The a-priori type of a class instance
creation expression $c;\WITH;(b)$ is $C;\WITH;R$ where $C$ is the
type of the class given in $c$ and $R$ is the a-priori type of block
$b$.
\item
The a-priori type of a block is a record consisting the a-priori
types of each non-private identifier which is declared in the block
and which is visible at in last statement of the block.  Here, it is
required that every import clause $\IMPORT;e_1 \commadots e_n$ refers
to expressions whose type can be computed with the type information
determined so far. Otherwise, a compile time error results.
\item
The a-priori type of any other expression is the expression's type, if
that type can be computed with the type information determined so far.
Otherwise, a compile time error results.
\end{enumerate}
The compiler will find an ordering in which types are assigned without
compiler errors to all variables $x_1 \commadots x_n$, if such an
ordering exists. This can be achieved by lazy evaluation.
}

\chapter{The Scala Standard Library}

The Scala standard library consists of the package \verb@scala@ with a
number of classes and modules.

\section{Root Classes}
\label{sec:cls-root}
\label{sec:cls-any}
\label{sec:cls-object}

The root of the Scala class hierarchy is formed by class \verb@Any@.
Every class in a Scala execution environment inherits directly or
indirectly from this class.  Class \verb@Any@ has exactly two direct
subclasses: \verb@AnyRef@ and\verb@AnyVal@.

The subclass \verb@AnyRef@ represents all values which are represented
as objects in the underlying host system. The type of the \verb@null@
value copnforms to every subclass of \verb@AnyRef@.  A direct subclass
of
\verb@AnyRef@ is class \verb@Object@. Every user-defined Scala
class inherits directly or indirectly from this class. Classes written
in other languages still inherit from \verb@scala.AnyRef@, but not
necessarily from \verb@scala.Object@.

The class \verb@AnyVal@ has a fixed number subclasses, which describe
values which are not implemented as objects in the underlying host
system.

Classes \verb@AnyRef@ and \verb@AnyVal@ are required to provide only
the members declared in class \verb@Any@, but implementations may add
host-specific methods to these classes (for instance, an
implementation may identify class \verb@AnyRef@ with its own root
class for objects).

The standard interfaces of these root classes is described by the
following definitions.

\begin{verbatim}
abstract class Any with {
  /** Get runtime type descriptor */
  def getType: Type = ...

  /** Reference equality */
  def eq (that: Any): Boolean = ...

  /** Hash code */
  def def hashCode: Int = ...
\end{verbatim}
\begin{verbatim}
  /** Type test */
  def is [a]: Boolean = ...

  /** Type cast */
  def as[a]: a = if (is[a]) ... else new CastException() throw

  /** Semantic equality between values of same type */
  def == (that: Any): Boolean  =  this equals that

  /** Semantic inequality between values of same type */
  def != (that: Any): Boolean  =  !(this == that)

  /** Semantic equality between arbitrary values */
  def equals (that: Any): Boolean  =  ...

  /** Representation as string */
  def toString: String  =  getType.toString ++ "@" ++ hashCode

  /** Concatenation of string representations */
  final def + (that: Any)  =  toString.concat(that)

  /** Pattern matching application */
  final def match [a] (f: (Any)a): a  =  f(this)
}
final class AnyVal extends Any
class AnyRef extends Any
class Object extends AnyRef
\end{verbatim}


\section{Value Classes}
\label{sec:cls-value}

Value classes are classes whose instances are not represented as
objects by the underlying host system.  All value classes inherit from
class \verb@AnyVal@. Scala implementations need to provide the
following value classes (but are free to provide others as well).

\begin{verbatim}
final class Unit extends AnyVal with { ... }
final class Boolean extends AnyVal with { ... }
final class Double extends AnyVal with { ... }
final class Float extends Double with { ... }
final class Long extends Float with { ... }
final class Int extends Long with { ... }
final class Char extends Int with { ... }
final class Short extends Int with { ... }
final class Byte extends Short with { ... }
\end{verbatim}

These classes are defined in the following.

\subsection{Class \prog{Double}}

\begin{verbatim}
final class Double extends AnyVal with Ord with {
  def asDouble: Double                          \=// convert to Double
  def asFloat: Float   \>// convert to Float
  def asLong: Long     \>// convert to Long
  def asInt: Int       \>// convert to Int
  def asChar: Char     \>// convert to Char
  def asShort: Short   \>// convert to Short
  def asByte: Byte     \>// convert to Byte

  def + (that: Double): Double \>// double addition
  def - (that: Double): Double \>// double subtraction
  def * (that: Double): Double \>// double multiplication
  def / (that: Double): Double \>// double division
  def % (that: Double): Double \>// double remainder

  def == (that: Double): Boolean \>// double equality
  def != (that: Double): Boolean \>// double inequality
  def < (that: Double): Boolean \>// double less
  def > (that: Double): Boolean \>// double greater
  def <= (that: Double): Boolean \>// double less or equals
  def >= (that: Double): Boolean \>// double greater or equals

  def - : Double = 0.0 - this   \>// double negation
  def + : Double = this
}
\end{verbatim}

\subsection{Class \prog{Float}}

\begin{verbatim}
final class Float extends Double with {
  def asDouble: Double                          \=// convert to Double
  def asFloat: Float   \>// convert to Float
  def asLong: Long     \>// convert to Long
  def asInt: Int       \>// convert to Int
  def asChar: Char     \>// convert to Char
  def asShort: Short   \>// convert to Short
  def asByte: Byte     \>// convert to Byte

  def + (that: Double): Double = asDouble + that
  def + (that: Float): Double \>// float addition
  /* analogous for -, *, /, % */

  def == (that: Double): Boolean = asDouble == that
  def == (that: Float): Boolean    \>// float equality
  /* analogous for !=, <, >, <=, >= */

  def - : Float = 0.0f - this   \>// float negation
  def + : Float = this
}
\end{verbatim}

\subsection{Class \prog{Long}}

\begin{verbatim}
final class Long extends Float with {
  def asDouble: Double                          \=// convert to Double
  def asFloat: Float   \>// convert to Float
  def asLong: Long     \>// convert to Long
  def asInt: Int       \>// convert to Int
  def asChar: Char     \>// convert to Char
  def asShort: Short   \>// convert to Short
  def asByte: Byte     \>// convert to Byte

  def + (that: Double): Double = asDouble + that
  def + (that: Float): Double = asFloat + that
  def + (that: Long): Long =  \>// long addition
  /* analogous for -, *, /, % */

  def << (cnt: Int): Long    \>// long left shift
  def >> (cnt: Int): Long          \>// long signed right shift
  def >>> (cnt: Int): Long          \>// long unsigned right shift
  def & (that: Long): Long          \>// long bitwise and
  def | (that: Long): Long          \>// long bitwise or
  def ^ (that: Long): Long	    \>// long bitwise exclusive or

  def == (that: Double): Boolean = asDouble == that
  def == (that: Float): Boolean = asFloat == that
  def == (that: Long): Boolean   \>// long equality
  /* analogous for !=, <, >, <=, >= */

  def - : Long = 0l - this   \>// long negation
  def + : Long = this
}
\end{verbatim}


\subsection{Class \prog{Int}}

\begin{verbatim}
class Int extends Long with {
  def asDouble: Double                          \=// convert to Double
  def asFloat: Float   \>// convert to Float
  def asLong: Long     \>// convert to Long
  def asInt: Int       \>// convert to Int
  def asChar: Char     \>// convert to Char
  def asShort: Short   \>// convert to Short
  def asByte: Byte     \>// convert to Byte

  def + (that: Double): Double = asDouble + that
  def + (that: Float): Double = asFloat + that
  def + (that: Long): Long =  \>// long addition
  def + (that: Int): Int =  \>// long addition
  /* analogous for -, *, /, % */

  def << (cnt: Int): Int    \>// long left shift
  /* analogous for >>, >>> */

  def & (that: Long): Long = asLong & that
  def & (that: Int): Int      \>// bitwise and
  /* analogous for |, ^ */

  def == (that: Double): Boolean = asDouble == that
  def == (that: Float): Boolean = asFloat == that
  def == (that: Long): Boolean   \>// long equality
  /* analogous for !=, <, >, <=, >= */

  def - : Long = 0l - this   \>// long negation
  def + : Long = this
}
\end{verbatim}

\subsection{Class \prog{Boolean}}
\label{sec:cls-boolean}

\begin{verbatim}
abstract final class Boolean extends AnyVal with Ord with {
  def ifThenElse[a](def t: a)(def e: a): a

  def ifThen(def t: Unit): Unit  =  ifThenElse(t)()

  def && (def x: Boolean): Boolean  =  ifThenElse(x)(False)
  def || (def x: Boolean): Boolean  =  ifThenElse(True)(x)
  def ! (def x: Boolean): Boolean  =  ifThenElse(False)(True)

  def == (x: Boolean): Boolean  =  ifThenElse(x)(x.!)
  def != (x: Boolean): Boolean  =  ifThenElse(x.!)(x)
  def <  (x: Boolean): Boolean  =  ifThenElse(False)(x)
  def >  (x: Boolean): Boolean  =  ifThenElse(x.!)(False)
  def <= (x: Boolean): Boolean  =  ifThenElse(x)(True)
  def >= (x: Boolean): Boolean  =  ifThenElse(True)(x.!)
}
case class True extends Boolean with { def ifThenElse(t)(e) = t }
case class False extends Boolean with { def ifThenElse(t)(e) = e }
\end{verbatim}


\comment{
\section{Reflection}

\subsection{Classes \prog{Type}, \prog{Class}, \prog{CompoundType}}

\begin{verbatim}
class Type[A] with {
  def isSubType [B] (that: Type[B]): Boolean = ...
}
\end{verbatim}

\begin{verbatim}
class Class[A] extends Type[A] with {
  ...
}
\end{verbatim}

\begin{verbatim}
abstract class CompoundType[A] extends Type[A] with {
  def components: List[Class[A]]
  ...
}
\end{verbatim}
}
\section{Other Standard Classes}

\subsection{Class \prog{Unit} and the \prog{Tuple} Classes}
\label{sec:cls-tuple}

\begin{verbatim}
case class Unit with {
  def toString = "()"
}
case class Tuple$\,n$[a$_1$, ..., a$_n$](x$_1$: a$_1$, ..., x$_n$: a$_n$) with {
  def $\_1$: a$_1$ = x$_1$
  ...
  def $\_n$: a$_n$ = x$_n$
  def toString = "(" ++ x$_1$ ++ "," ++ ... ++ x$_n$ ++ ")"
}
\end{verbatim}

\subsection{The \prog{Function} Classes}
\label{sec:cls-function}

\begin{verbatim}
class Function$\,n$[a$_1$, ..., a$_n$,b] with {
  // some methods in Any are overwritten
  def apply(x$_1$: a$_1$, ..., x$_n$: a$_n$): b
}
\end{verbatim}
Class \verb@Function1@ additionally defines the method
\begin{verbatim}
  def o [c] (f: Function1[c,a$_1$]): Function1[c,b] = x: c => apply(f(x))
\end{verbatim}
There is also a module \verb@Function@, defined as follows.
\begin{verbatim}
module Function {
  def compose[a](fs: List[(a)a]): (a)a  = {
    x => fs match {
      case Nil => x
      case f :: fs1 => compose(fs1)(f(x))
    }
  }
}
\end{verbatim}
A subclass of \verb@Function$\,n$@ describes partial functions, which
are undefined on some points in their domain.

\begin{verbatim}
class PartialFunction$\,n$[a$_1$, ..., a$_n$,b] extends Function$\,n$[a$_1$, ..., a$_n$,b] with {
  def isDefined(x$_1$: a$_1$, ..., x$_n$: a$_n$): Boolean
}
\end{verbatim}

In addition to the \verb@apply@ method of functions, partial functions
also have a \verb@isDefined@ method, which tells whether the function
is defined at the given argument.

Classes \verb@Function@ and \verb@PartialFunction@ are defined to be aliases for
\verb@Function1@ and \verb@PartialFunction1@:
\begin{verbatim}
  type Function[a,b] = Function1[a,b]
  type PartialFunction[a,b] = PartialFunction1[a,b]
  def Function[a,b]: class Function1[a,b] = Function1[a,b]
  def PartialFunction[a,b]: class PartialFunction1[a,b] = PartialFunction1[a,b]
\end{verbatim}

\subsection{Class \prog{List}}\label{cls-list}

\begin{verbatim}
abstract class List[a] with {
  abstract def isEmpty: Boolean;
  abstract def head: a;
  abstract def tail: List[a];

  def ::(x: a): List[a] =
    new ::_class(x)(this);

  def :::(prefix: List[a]): List[a] =
    if (prefix.isEmpty) this
    else prefix.head :: (prefix.tail ::: this);

  def length: Int = {
    this match {
	case [] => 0
    case _ :: xs => xs.length + 1}
  }
\end{verbatim}
\begin{verbatim}
  def init: List[a] =
    if (isEmpty) error("Nil.init")
    else if (tail.isEmpty) Nil
    else head :: tail.init;

  def last: a =
    if (isEmpty) error("Nil.last")
    else if (tail.isEmpty) head
    else tail.last;

  def take(n: Int): List[a] =
    if (n == 0) Nil
    else head :: tail.take(n-1);

  def drop(n: Int): List[a] =
    if (n == 0) this
    else tail.drop(n-1);

  def takeWhile(p: (a)Boolean): List[a] =
    if (isEmpty || !p(head)) Nil
    else head :: tail.takeWhile(p);

  def dropWhile(p: (a)Boolean): List[a] =
    if (isEmpty || !p(head)) this
    else tail.dropWhile(p);

  def at(n: Int) = drop(n).head;
\end{verbatim}
\begin{verbatim}
  def map[b](f: (a)b): List[b] =
    if (isEmpty) Nil
    else f(head) :: tail.map(f);

  def foreach(f: (a)Unit): Unit =
    if (isEmpty) ()
    else (f(head); tail.foreach(f));

  def filter(p: (a)Boolean): List[a] =
    if (isEmpty) this
    else if (p(head)) head :: tail.filter(p)
    else tail.filter(p);

  def forall(p: (a)Boolean): Boolean =
    isEmpty || (p(head) && tail.forall(p));

  def exists(p: (a)Boolean): Boolean =
    !isEmpty && (p(head) || tail.exists(p));
\end{verbatim}
\begin{verbatim}
  def :_foldl[b](z: b)(f: (b, a)b) = match {
    case [] => z
    case x :: xs => (f(z, x) :_foldl xs)(f)
  }

  def foldr[b](z: b)(f: (a, b)b) = match {
    case [] => z
    case x :: xs => f(x, (xs foldr z)(f))
  }

  def redl(f: (a, a)a) = match {
    case [] => error("redl of empty list")
    case x :: xs => (x :_foldl xs)(f)
  }

  def redr(f: (a, a)a): a = match {
	case [] => error("redr of empty list")
	case [x] => x
	case x :: xs => f(x, xs redr f)
  }
\end{verbatim}
\begin{verbatim}
  def flatMap[b](f: (a)List[b]): List[b] =
    if (isEmpty) Nil
    else f(head) ::: tail.flatMap(f);

  def reverse: List[a] = {
    def snoc(xs: List[a], x: a): List[a] = x :: xs;
    fold(snoc)(Nil)
  }

  def print: Unit =
    if (isEmpty) System.out.println("[]")
    else {
      System.out.print(head.as[java.lang.Object]);
      System.out.print(" :: ");
      tail.print
    }

  def toArray: Array[a] = {
    val xs = new Array[a](length);
    copyToArray(xs, 0);
    xs
  }

  def copyToArray(xs: Array[a], start: Int): Int = {
    xs(start) = head;
    tail.copyToArray(xs, start + 1)
  }
\end{verbatim}
\begin{verbatim}
  def mkString(start: String, sep: String, end: String): String =
    start +
    (if (isEmpty) end
     else if (tail.isEmpty) head.toString() + end
     else head.toString().concat(sep).concat(tail.mkString("", sep, end)));

  def zip[b](that: List[b]): List[(a,b)] =
    if (this.isEmpty || that.isEmpty) Nil
    else (this.head, that.head) :: this.tail.zip(that.tail);
\end{verbatim}
\begin{verbatim}
  def contains(elem: a) = exists(x => x == elem);

  def union(that: List[a]): List[a] =
    if (this.isEmpty) that
    else {
      val result = this.tail union that;
      if (that contains this.head) result else this.head :: result;
    }

  def diff(that: List[a]): List[a] =
    if (that.isEmpty) this
    else {
      val result = this.tail diff that;
      if (that contains this.head) result else this.head :: result;
    }

  def intersect(that: List[a]): List[a] = filter(x => that contains x);

  def removeDuplicates: List[a] =
    if (isEmpty) this
    else {
      val rest = tail.removeDuplicates;
	  if (rest contains head) rest else head :: rest
    }
}
\end{verbatim}
\begin{verbatim}
final case class ::_class[b](hd: b)(tl: List[b]) extends List[b] with {
  def isEmpty = False;
  def head = hd;
  def tail = tl;
  override def toString(): String = mkString("[", ",", "]");
}
\end{verbatim}
\begin{verbatim}
final case class Nil[c] extends List[c] with {
  def isEmpty = True;
  def head: c = error("head of empty list");
  def tail: List[c] = error("tail of empty list");
  override def toString(): String = "[]";
}
\end{verbatim}

\subsection{Class \prog{Array}}

The class of generic arrays is defined as follows.

\begin{verbatim}
class Array[a](l: int) extends Function[Int, a] with {
  def length: int = l
  def apply(i: Int): a = ...
  def update(i: Int)(x: a): Unit = ...
}
\end{verbatim}
\comment{
\begin{verbatim}
module Array {
  def create[a](i1: Int): Array[a] = Array[a](i1)
  def create[a](i1: Int, i2: Int): Array[Array[a]] = {
    val x: Array[Array[a]] = create(i1)
    0 to (i1 - 1) do { i => x(i) = create(i2) }
    x
  }
  ...
  def create[a](i1: Int, i2: Int, i3: Int, i4: Int, i5: Int,
                i6: Int, i7: Int, i8: Int, i9: Int, i10: Int)
    : Array[Array[Array[Array[Array[Array[Array[Array[Array[Array[a]]]]]]]]]] = {
    val x: Array[Array[Array[Array[Array[Array[Array[Array[Array[a]]]]]]]]] = create(i1)
    0 to (i1 - 1) do { i => x(i) = create(i2, i3, i4, i5, i6, i7, i8, i9, i10) }
    x
  }
}
\end{verbatim}
}
\section{Exceptions}
\label{sec:exceptions}

There is a predefined type \verb@Throwable@, as well as functions to
throw and handle values of type \verb@Throwable@. These are declared
as follows.

\begin{verbatim}
  class Throwable with {
    def throw[a]: a
  }
  class ExceptOrFinally[a] with {
    def except (handler: PartialFunction[Throwable,a]): a
    def finally (def handler: Unit): a
  }
  def try [a] (def body: a): ExceptOrFinally[a]
\end{verbatim}

The type \verb@Throwable@ represents exceptions and error objects; it
may be identified with an analogous type of the underlying
implementation such as \verb@java.lang.Throwable@. We will in the
following loosely call values of type \verb@Throwable@ exceptions.

The \verb@throw@ method in \verb@Throwable@ aborts execution of the
thread executing it and passes the thrown exception to the handler
that was most recently installed by a
\verb@try@ function in the current thread. If no \verb@try@ method is
active, the thread terminates.

The \verb@try@ function executes its body with the given exception
handler. A \verb@try@ expression comes in two forms. The first form is

\begin{verbatim}
try $body$ except $handler$  .
\end{verbatim}

If $body$ executes without an exception being thrown, then executing
the try expression is equivalent to just executing $body$. If some
exception is thrown from within $body$ for which \verb@handler@ is defined,
the handler is invoked with the thrown exception as argument.

The second form of a try expression is

\begin{verbatim}
try $body$ finally $handler$  .
\end{verbatim}

This expression will execute $body$. A normal execution of $body$ is
followed by an invocation of the $handler$ expression.  The $handler$
expression does not take arguments and has \verb@Unit@ as result type.
If execution of the handler expression throws an exception, this
exception is propagated out of the \verb@try@ statement.  Otherwise,
if an exception was thrown in $body$ prior to invocation of $handler$,
that exception is re-thrown after the invocation. Finally, if both
$body$ and $handler$ terminate normally, the original result of
$body$ is the result of the \verb@try@ expression.

\example An example of a try-except expression:

\begin{verbatim}
try {
  System.in.readString()
} except {
  case ex: EndOfFile => ""
}
\end{verbatim}

\example An example of a try-finally expression:

\begin{verbatim}
file = open (fileName)
if (file != null) {
  try {
    process (file)
  } finally {
    file.close
  }
}
\end{verbatim}

\section{Concurrency}
\label{sec:concurrency}

\subsection{Basic Concurrency Constructs}

Scala programs may be executed by several threads that operate
concurrently.  The thread model used is based on the model of the
underlying run-time system. We postulate a predefined
class \verb@Thread@ for run-time threads,
\verb@fork@ function to spawn off a new thread,
as well as \verb@Monitor@ and \verb@Signal@ classes. These are
specified as follows\nyi{Concurrentcy constructs are}.


\begin{verbatim}
class Thread with { ... }
def fork (def p: Unit): Thread
\end{verbatim}

The \verb@fork@ function runs its argument computation \verb@p@ in a
separate thread.  It returns the thread object immediately to its
caller.  Unhandled exceptions (\sref{sec:exceptions}) thrown during
evaluation of \verb@p@ abort execution of the forked thread and are
otherwise ignored.

\begin{verbatim}
class Monitor with {
  def synchronized [a] (def e: a): a
}
\end{verbatim}

Monitors define a \verb@synchronized@ method which provides mutual
exclusion between threads.  It executes its argument computation
\verb@e@ while asserting exclusive ownership of the monitor
object whose method is invoked. If some other thread has ownership of
the same monitor object, the computation is delayed until the other
process has relinquished its ownership. Ownership of a monitor is
relinquished at the end of the argument computation, and while the
computation is waiting for a signal.

\begin{verbatim}
class Signal with {
  def wait: Unit
  def wait(msec: Long): Unit
  def notify: Unit
  def notifyAll: Unit
}
\end{verbatim}

The \verb@Signal@ class provides the basic means for process
synchronization.  The \verb@wait@ method of a signal suspends the
calling thread until it is woken up by some future invocation of the
signal's \verb@notify@ or \verb@notifyAll@ method. The \verb@notify@
method wakes up one thread that is waiting for the signal. The
\verb@notifyAll@ method wakes up all threads that are waiting for the
signal. A second version of the \verb@wait@ method takes a time-out
parameter (given in milliseconds). A thread calling \verb@wait(msec)@
will suspend until unblocked by a \verb@notify@ or \verb@notifyAll@
method, or until the \verb@msec@ millseconds have passed.

\subsection{Channels}

\begin{verbatim}
class Channel[a] with {
  def write(x: a): Unit
  def read: a
}
\end{verbatim}

An object of type \verb@Channel[a]@ Channels offer a write-operation
which writes data of type \verb@a@ to the channel, and a read
operation, which returns written data as a result. The write operation
is non-blocking; that is it returns immediately without waiting for
the written data to be read.

\subsection{Message Spaces}

The Scala library also provides message spaces as a higher-level,
flexible construct for process synchronization and communication. A
{\em message} is an arbitrary object that inherits from the
\verb@Message@ class.
There is a special message \verb@TIMEOUT@ which is used to signal a time-out.
\begin{verbatim}
class Message
case class TIMEOUT extends Message
\end{verbatim}
Message spaces implement the following class.
\begin{verbatim}
class MessageSpace with {
  def send(msg: Message): Unit
  def receive[a](f: PartialFunction1[Message, a]): a
  def receiveWithin[a](msec: Long)(f: PartialFunction1[Message, a]): a
}
\end{verbatim}
The state of a message space consists of a multi-set of messages.
Messages are added to the space using the \verb@send@ method. Messages
are removed using the \verb@receive@ method, which is passed a message
processor \verb@f@ as argument, which is a partial function from
messages to some arbitrary result type. Typically, this function is
implemented as a pattern matching expression. The \verb@receive@
method blocks until there is a message in the space for which its
message processor is defined.  The matching message is then removed
from the space and the blocked thread is restarted by applying the
message processor to the message. Both sent messages and receivers are
ordered in time. A receiver $r$ is applied to a matching message $m$
only if there is no other (message, receiver) pair which precedes $(m,
r)$ in the partial ordering on pairs that orders each component in
time.

The message space class also offers a method \verb@receiveWithin@
which blocks for only a specified maximal amount of time.  If no
message is received within the specified time interval (given in
milliseconds), the message processor argument $f$ will be unblocked
with the special \verb@TIMEOUT@ message.

\appendix
\chapter{Scala Syntax Summary}

The lexical syntax of Scala is given by the following grammar in EBNF
form.

\begin{verbatim}
  upper               \=::= \=`A' | ... | `Z' | `$\Dollar$' | `_'
  lower           \>::= \>`a' | ... | `z'
  letter          \>::= \>upper | lower
  digit           \>::= \>`0' | ... | `9'
  special         \>::= \>``everything else except parentheses ([{}]) and period''

  op              \>::= \>special {special} [`_' [id]]
  varid           \>::= \>lower {letter | digit} [`_' [id]]
  id              \>::= \>upper {letter | digit} [`_' [id]]
                  \>   |\>varid
                  \>   |\>op

  intLit          \>::= \>``as in Java''
  floatLit        \>::= \>``as in Java''
  charLit         \>::= \>``as in Java''
  stringLit       \>::= \>``as in Java''

  comment         \>::= \>`/*' ``any sequence of characters'' `*/'
                  \>   |\>`//' `any sequence of characters up to end of line''
\end{verbatim}

The context-free syntax of Scala is given by the following EBNF
grammar.

\begin{verbatim}
  literal               \=::= \= intLit
                  \>  |\> floatLit
                  \>  |\> charLit
                  \>  |\> stringLit
		  \>  |\> symbolLit

  Id              \>::=\> id  |  `+'  |  `-'  |  `!'
  QualId          \>::=\> Id {`.' Id}
  Ids             \>::=\> Id {`,' Id}
\end{verbatim}

\begin{verbatim}
  StableId                   \=::= \= Id
                  \>  |\> Path `.' Id
                  \>  |\> [Ident '.'] super `.' Id
  Path            \>::=\> StableId
                  \>  |\> [Ident `.'] this

  Type          \>::= \> Type1 `=>' Type
                \>   |\> `(' [Types] `)' `=>' Type
	        \>   |\> Type1
  Type1         \>::= \> SimpleType {with SimpleType} [Refinement]
  SimpleType   	\>::= \> SimpleType TypeArgs
	        \>   |\> SimpleType `#' Id
	        \>   |\> StableId
                \>   |\> Path `.' type
		\>   |\> `(' Type ')'
  TypeArgs     	\>::= \> `[' Types `]'
  Types         \>::= \> Type {`,' Type}
  Refinement      \>::=\> `{' [RefineStat {`;' RefineStat}] `}'
  RefineStat      \>::=\> Dcl
                  \>  |\> type TypeDef {`,' TypeDef}
                  \>  |\>

  Exprs           \>::=\> Expr {`,' Expr}
  Expr            \>::=\> [Bindings `=>'] Expr
		  \>  |\> if `(' Expr `)' Expr [[`;'] else Expr]
		  \>  |\> try Expr [`;'] (except Expr | finally Expr)
		  \>  |\> while '(' Expr ')' Expr
		  \>  |\> do Expr [`;'] while `(' Expr ')'
		  \>  |\> for `(' Enumerators `)' (do | yield) Expr
                  \>  |\> [SimpleExpr `.'] Id `=' Expr
		  \>  |\> SimpleExpr ArgumentExpr `=' Expr
                  \>  |\> PostfixExpr [`:' Type1]
  PostfixExpr     \>::=\> InfixExpr [Id]
  InfixExpr       \>::=\> PrefixExpr
		  \>  |\> InfixExpr Id InfixExpr
  PrefixExpr      \>::=\> [`-' | `+' | `~' | `!'] SimpleExpr 
  SimpleExpr      \>::=\> literal
		  \>  |\> true
		  \>  |\> false
		  \>  |\> null
		  \>  |\> Path
		  \>  |\> `(' [Expr] `)'
		  \>  |\> BlockExpr
		  \>  |\> new Template 
		  \>  |\> SimpleExpr `.' Id 
		  \>  |\> Id `#' Id 
		  \>  |\> SimpleExpr TypeArgs
		  \>  |\> SimpleExpr ArgumentExpr
  ArgumentExpr    \>::=\> `(' Expr ')'
                  \>  |\> BlockExpr
  BlockExpr       \>::=\> `{' CaseClause {CaseClause} `}'
		  \>  |\> `{' Block `}'
  Block           \>::=\> {BlockStat `;'} [Expr]
\end{verbatim}

\begin{verbatim}
  Enumerators              \=::= \= Generator {`;' Enumerator}
  Enumerator      \>::=\> Generator
                  \>  |\> Expr
  Generator       \>::=\> val Pattern `<-' Expr
  Block           \>::=\> {BlockStat `;'} [Expr]
  BlockStat       \>::=\> Import
                  \>  |\> Def
		  \>  |\> {LocalModifier} ClsDef
		  \>  |\> Expr
                  \>  |\>

  CaseClause      \>::=\> case Pattern [`if' PostfixExpr] `=>' Block 

  Constr          \>::=\> StableId [TypeArgs] [`(' [Exprs] `)']  

  Pattern         \>::=  \= TreePattern { `|' TreePattern }

  TreePattern     \>::=  \> varid `:' Type
                  \> |   \> `_' `:' Type
                  \> |   \> SimplePattern [ '*' | '?' | '+' ]
                  \> |   \> SimplePattern { Id SimplePattern }

  SimplePattern   \>::= \> varid [ '@' SimplePattern ]
                  \> |  \> `_'
                  \> |  \> literal
                  \> |  \> StableId [ `(' [Patterns] `)' ]
                  \> |  \> `(' Patterns `)'
                  \> |  \> 

  Patterns        \>::= \> Pattern {`,' Pattern}

  TypeParamClause    \>::=\> `[' TypeParam {`,' TypeParam} `]'
  FunTypeParamClause \>::=\> `[' TypeDcl {`,' TypeDcl} `]'
  TypeParam       \>::=\>  [`+' | `-'] TypeDcl
  ParamClause     \>::=\> `(' [Param {`,' Param}] `)'
  Param           \>::=\> [def] Id `:' Type [`*']
  Bindings        \>::=\> Id [`:' Type1]
                  \>  |\> `(' Binding {`,' Binding `)'
                  \>  |\>
  Binding         \>::=\> Id [`:' Type]

  Modifier        \>::=\> LocalModifier
		  \>  |\> private
                  \>  |\> protected
                  \>  |\> override 
  LocalModifier   \>::=\> abstract
                  \>  |\> final
		  \>  |\> sealed

  Template        \>::=\> Constr {`with' Constr} [TemplateBody]
  TemplateBody    \>::=\> `{' [TemplateStat {`;' TemplateStat}] `}'
\end{verbatim}
\begin{verbatim}
  TemplateStat         \=::= \= Import
                  \>  |\> {Modifier} Def
		  \>  |\> {Modifier} Dcl
		  \>  |\> Expr
                  \>  |\>

  Import          \>::=\> import ImportExpr {`,' ImportExpr}
  ImportExpr      \>::=\> StableId `.' (Id | `_' | ImportSelectors)
  ImportSelectors \>::=\> `{' {ImportSelector `,'} (ImportSelector | `_') `}'
  ImportSelector  \>::=\> Id [`=>' Id | `=>' `_']

  Dcl             \>::=\> val ValDcl {`,' ValDcl}
                  \>  |\> var VarDcl {`,' VarDcl}
                  \>  |\> def FunDcl {`,' FunDcl}
                  \>  |\> type TypeDcl {`,' TypeDcl}
  ValDcl          \>::=\> Id `:' Type
  VarDcl          \>::=\> Id `:' Type
  FunDcl          \>::=\> Id [FunTypeParamClause] {ParamClause} `:' Type
  TypeDcl         \>::=\> Id [`>:' Type] [`<:' Type]

  Def             \>::=\> val PatDef {`,' PatDef}
		  \>  |\> var VarDef {`,' VarDef}
  	          \>  |\> def FunDef {`,' FunDef}
                  \>  |\> type TypeDef {`,' TypeDef}
	          \>  |\> ClsDef
  PatDef          \>::=\> Pattern `=' Expr
  VarDef          \>::=\> Id [`:' Type] `=' Expr
                  \>  |\> Id `:' Type `=' `_'
  FunDef          \>::=\> Id [FunTypeParamClause] {ParamClause} [`:' Type] `=' Expr
                      |   this ParamClause `=' ConstrExpr
  TypeDef         \>::=\> Id `=' Type
  ClsDef          \>::=\> ([case] class | trait) ClassDef {`,' ClassDef}
                  \>  |\> [case] object ObjectDef {`,' ObjectDef}
  ClassDef        \>::=\> Id [TypeParamClause] [ParamClause] [`:' SimpleType] ClassTemplate
  ObjectDef       \>::=\> Id [`:' SimpleType] ClassTemplate
  ClassTemplate   \>::=\> extends Template
	          \>  |\> TemplateBody
		  \>  |\>
  ConstrExpr      \>::=\> this ArgumentExpr
                  \>  |\>  `{' {BlockStat `;'} this ArgumentExpr {`;' BlockStat} `}'

  CompilationUnit \>::=\> [package QualId `;'] [{TopStat `;'} TopStat]
  TopStat         \>::=\> {Modifier} ClsDef
	          \>  |\> Packaging
	          \>  |\> Import
		  \>  |\>
  Packaging       \>::=\> package QualId `{' [{TopStat `;'} TopStat] `}'
\end{verbatim}

case class extends { ... }

trait List { }
class Nil
class Cons

\comment{changes:
  Type                    \=::= \= SimpleType {with SimpleType} [with Refinement]
                  \>  |\> class SimpleType
  SimpleType      \>::=\> SimpleType [TypeArgs]
		  \>  |\> `(' [Types] `)'
		  \>  |\>
                  \>  |\> this
}
\end{document}

\comment{changes:

  Type                    \=::= \= SimpleType {with SimpleType} [with Refinement]
                  \>  |\> class SimpleType
  SimpleType      \>::=\> TypeDesignator [TypeArgs]
		  \>  |\> `(' Type `,' Types `)'
		  \>  |\> `(' [Types] `)' Type
                  \>  |\> this

  PureDef         \>::=\> module ModuleDef {`,' ModuleDef}
                  \>::=\> def FunDef {`,' FunDef}
                  \>  |\> type TypeDef {`,' TypeDef}
                  \>  |\> [case] class ClassDef {`,' ClassDef}
		  \>  |\> case CaseDef {`,' CaseDef}
  CaseDef         \>::=\> Ids ClassTemplate

  Modifier        \>::=\> final
		  \>  |\> private
                  \>  |\> protected
                  \>  |\> override [QualId]
		  \>  |\> qualified
		  \>  |\> abstract

\section{Class Aliases}
\label{sec:class-alias}

\syntax\begin{verbatim}
  ClassDef         \=::= \= ClassAlias
  InterfaceDef     \>::= \> ClassAlias
  ClassAlias   \>::= \> Id [TypeParamClause] `=' SimpleType
\end{verbatim}

Classes may also be defined to be aliases for other classes.  A class
alias is of the form $\CLASS;c[tps] = d[targs]$ where $d[targs]$ is a
class type. Both $tps$ and $targs$ may be empty.
This introduces the type $c[tps]$ as an alias for type
$d[targs]$, in the same way the following type alias definition would:
\begin{verbatim}
type c[tps] = d[targs]
\end{verbatim}
The class alias definition is legal if the type alias definition would be legal.

Assuming $d$ defines a class with type parameters $tps'$ and
parameters $(ps_1) \ldots (ps_n)$, the newly defined type is also
introduced as a class with a constructor which takes type parameters
$[tps]$, and which takes value parameters
$([targs/tps']ps_1)\ldots([targs/tps']ps_n)$.

The modifiers \verb@private@ and
\verb@protected@ apply to a class alias independently of the class it represents.
The class $c$ is regarded as final if $d$ is final, or if a
\verb@final@ modifier is given for the alias definition.
$c$ is regarded as a case class iff $d$ is one.  In this
case,
\begin{itemize}
\item the alias definition may also be prefixed with \verb@case@, and
\item the case constructor is also aliased, as if it was
defined such:
\begin{verbatim}
def c[tps](ps$_1$)\ldots(ps$_n$):D = d[targs]$([targs/tps']ps$_1$)\ldots([targs/tps']ps$_n$)$  .
\end{verbatim}
The new function $c$ is again classified as a case constructor, so
it may appear in constructor patterns (\sref{sec:patterns}).
\end{itemize}
Aliases for interfaces follow the same rules as class aliases, but
start with \verb@interface@ instead of \verb@class@.
}

type T extends { ... }

class C extends { ... }

new C { ... }

type C
class C < { ... }

A & B & C &
\ifqualified{
Parameter clauses (\sref{sec:funsigs}),
definitions that are local to a block (\sref{sec:blocks}), and import
clauses always introduce {\em simple names} $x$, which consist of a
single identifier.  On the other hand, definitions and declarations
that form part of a module (\sref{sec:modules}) or a class
(\sref{sec:classes}) conceptually always introduce {\em qualified
names}\nyi{Qualified names are}
$Q\qex x$ where a simple name $x$ comes with a qualified
identifier $Q$. $Q$ is either the fully qualified name of a module or
class which is labelled
\verb@qualified@, or it is the empty name $\epsilon$.

The {\em fully qualified name} of a module or class $M[targs]$ with
simple name $M$ and type arguments $[targs]$ is
\begin{itemize}
\item $Q.M$, if the definition of $M$ appears in the template defining
a module or class with fully qualified name $Q$.
\item
$M$ if the definition of $M$ appears on the top-level or as a definition
in a block.
\end{itemize}
}

\ifqualified{
It is possible that a definition in some class or module $M$
introduces several qualified names $Q_1\qex x \commadots Q_n\qex x$ in a name
space that have the same simple name suffix but different qualifiers
$Q_1 \commadots Q_n$. This happens for instance if a module \verb@M@
implements two qualified classes \verb@C@, \verb@D@ that each define a
function \verb@f@:
\begin{verbatim}
qualified abstract class B { def f: Unit = ""}
qualified abstract class C extends B { def f: Unit }
qualified abstract class D extends B { def f: Unit }

module M extends C with D with {
  override C def f = println("C::f")
  override D def f = println("D::f")
                        \=
  // f                  \>// error: ambiguous
  (this:D).f 		\>// prints ``D::f''
}

def main() = (M:C).f	\>// prints ``C::f''
\end{verbatim}
Members of modules or classes are accessed using simple names,
not qualified names.

The {\em qualified expansion} of a simple name $x$ in some type $T$ is
determined as follows: Let $Q_1\qex x \commadots Q_n\qex x$ be all the
qualified names of members of $T$ that have a simple name suffix $x$.
If one of the qualifiers $Q_i$ is the empty name $\epsilon$, then the
qualified expansion of $x$ in $T$ is $\epsilon\qex x$. Otherwise, let
$C_1
\commadots C_n$ be the base classes (\sref{sec:base-classes})
of $T$ that have fully qualified
names $Q_1
\commadots Q_n$, respectively. If there exists a least class $C_j$
among the $C_i$ in the subclass ordering, then the qualified expansion
of $x$ in $T$ is $Q_j\qex x$. Otherwise the qualified expansion does not
exist.

Conversely, if $Q\qex x$ is the qualified expansion of some simple
name $x$ in $M$, we say that the entity named $Q\qex x$ in $M$ is {\em
identified in $M$ by the simple name} $x$. We leave out the
qualification ``in $M$'' if it is clear from the context.
In the example above, the qualified expansion of \verb@f@ in \verb@C@
is \verb@C::f@, because \verb@C@ is a subclass of \verb@B@. On the
other hand, the qualified expansion of \verb@f@ in \verb@M@ does not
exist, since among the two choices \verb@C::f@ and \verb@D::f@ neither
class is a subclass of the other.

A member access $e.x$ of some type term $e$ of type $T$ references the
member identified in $T$ by the simple name $x$ (i.e.\ the member
which is named by the qualified expansion of $x$ in $T$).

In the example above, the simple name \verb@f@ in \verb@M@ would be
ambiguous since the qualified expansion of \verb@f@ in \verb@M@ does
not exist.  To reference one of the two functions with simple name
\verb@f@, one can use an explicit typing. For instance, the name
\verb@(this:D).f@ references the implementation of \verb@D::f@ in
\verb@M@.
}

\comment{
\example The following example illustrates the difference between
virtual and non-virtual members with respect to overriding.

\begin{verbatim}
class C with {
  virtual def f = "f in C"
  def g = "g in C"
  def both1 = this.f ++ ", " ++ this.g
  def both2 = f ++ ", " ++ g
}

class D extends C with {
  override def f = "f in D"
  override def g = "redefined g in D"
  new def g = "new g in D"
}

val d = D
println(d.f)                \=// prints ``f in D''
println(d.g)		    \>// prints ``new g in D''
println(d.both1)	    \>// prints ``f in D, redefined g in D''
println(d.both2)	    \>// prints ``f in D, g in C''

val c: C = d
println(c.f)                \=// prints ``f in D''
println(c.g)		    \>// prints ``redefined g in D''
println(c.both1)	    \>// prints ``f in D, redefined g in D''
println(c.both2)	    \>// prints ``f in D, g in C''
\end{verbatim}
}

\comment{
\section{The Self Type}
\label{sec:self-type}

\syntax\begin{verbatim}
SimpleType     \=::= \= $\This$
\end{verbatim}

The self type \verb@this@ may be used in the statement part of a
template, where it refers to the type of the object being defined by
the template.  It is the type of the self reference \verb@this@.

For every leaf class (\sref{sec:modifiers}) $C$, \verb@this@ is
treated as an alias for the class itself, as if it was declared such:
\begin{verbatim}
final class C ... with {
  type this = C
  ...
}
\end{verbatim}
For non-leaf classes $C$, \verb@this@ is treated as an abstract type
bounded by the class itself, as if it was declared such:
\begin{verbatim}
abstract class C ... with {
  type this extends C
  ...
}
\end{verbatim}

Analogously, for every compound type \verb@$T_1$ with ... with $T_n$@,
\verb@this@ is treated as an abstract type conforming to the whole compound
type, as if it was bound in the refinement
\begin{verbatim}
type this extends $T_1$ with ... with $T_n$ .
\end{verbatim}
Finally, for every declaration of a parameter or abstract type
\mbox{$a \extends T$}, \verb@this@ is treated as an an abstract type
conforming to $a$, as if the bound type $T$ was augmented to
\verb@$T$ with { abstract type this extends $a$ }@.
On the other hand, if the parameter or abstract type is declared
\verb@final@, as in $\FINAL;a \extends T$, then \verb@this@ is treated as an alias
for $a$, as if the bound type $T$ was augmented to
\verb@$T$ with { type this = $a$ }@.

\example
Consider the following classes for one- and two-dimensional
points with a \verb@distance@ method that computes the distance
between two points of the same type.
\begin{verbatim}
class Point1D(x: Float) with {
  def xCoord = x
  def distance (that: this)  =  abs(this.xCoord - that.xCoord)
  def self: this = this
}
final class FinalPoint1D(x: Float) extends Point1D(x)

class Point2D(x: Float, y: Float) extends Point1D(x) with {
  def yCoord = y
  override def distance(that: this) =
    sqrt (square(this.xCoord - that.xCoord) + square(this.yCoord - that.yCoord))
}
\end{verbatim}
Assume the following definitions:
\begin{verbatim}
val p1f: FinalPoint1D = FinalPoint1D(0.0)
val p1a: Point1D = p1f
val p1b: Point1D = Point2D(3.0, 4.0)
\end{verbatim}
Of the following expressions, three are well-formed, the other three
are ill-formed.
\begin{verbatim}
p1f distance p1f            \=// OK, yields 0,0
p1f distance p1b	    \>// OK, yields 3.0
p1a distance p1a 	    \>// OK, yields 0.0
p1a distance p1f		\>// ERROR, required: p1a.this, found: FinalPoint1D
p1a distance p1b		\>// ERROR, required: p1a.this, found: p1b.this
p1b distance p1a		\>// ERROR, required: p1b.this, found: p1a.this
\end{verbatim}
The last of these expressions would cause an illegal access to a
non-existing class \verb@yCoord@ of an object of type \verb@Point1D@,
if it were permitted to execute in spite of being not well-typed.
}

