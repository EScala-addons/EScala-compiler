%% $Id$

\documentclass[11pt]{book}

\usepackage{fleqn,a4wide,vquote,modefs,math,prooftree,scaladefs}
\newcommand{\exercise}{\paragraph{Exercise:}}
\newcommand{\rewriteby}[1]{\mbox{\tab\tab\rm(#1)}}

\title{Scala By Examples}

\author{
Martin Odersky \\ LAMP/EPFL
}

\sloppy
\begin{document}
\maketitle
\bibliographystyle{alpha}

\chapter{\label{chap:intro}Introduction}

\input{rationale-chapter.tex}

The rest of this document is structured as
follows. Chapters~\ref{chap:example-one} and
\ref{chap:example-auction} highlight some of the features that make
Scala interesting. The following chapters introduce the language
constructs of Scala in a more thorough
way. Chapter~\ref{chap:simple-funs} introduces basic expressions and
simple functions. Chapter~\ref{chap:first-class-funs} introduces
higher-order functions. (to be continued).

This document ows a great dept to Sussman and Abelson's wonderful book
``Structure and Interpretation of Computer
Programs''\cite{abelson-sussman:structure}. Many of their examples and
exercises are also present here. Of course, the working language has
in each case been changed from Scheme to Scala. Furthermore, the
examples make use of Scala's object-oriented constructs where
appropriate.


\chapter{\label{chap:example-one}A First Example}

As a first example, here is an implementation of Quicksort in Scala.
\begin{verbatim}
def sort(xs: Array[int]): unit = {

  def swap(i: int, j: int): unit = {
    val t = xs(i); xs(i) = xs(j); xs(j) = t;
  }

  def sort1(l: int, r: int): unit = {
    val pivot = xs((l + r) / 2);
    var i = l, j = r;
    while (i <= j) {
      while (xs(i) < pivot) { i = i + 1 }
      while (xs(j) > pivot) { j = j - 1 }
      if (i <= j) { 
        swap(i, j);
        i = i + 1;
        j = j - 1;
      }
    } 
    if (l < j) sort1(l, j);
    if (j < r) sort1(i, r);
  }

  sort1(0, xs.length - 1);
}
\end{verbatim}
The implementation looks quite similar to what one would write in Java
or C.  We use the same operators and similar control structures.
There are also some minor syntactical differences. In particular:
\begin{itemize}
\item
Definitions start with a reserved word. Function definitions start
with \verb@def@, variable definitions start with \verb@var@ and
definitions of values (i.e. read only variables) start with \verb@val@.
\item
The declared type of a symbol is given after the symbol and a colon.
The declared type can often be omitted, because the compiler can infer
it from the context.
\item
We use \verb@unit@ instead of \verb@void@ to define the result type of
a procedure.
\item
Array types are written \verb@Array[T]@ rather than \verb@T[]@, 
and array selections are written \verb@a(i)@ rather than \verb@a[i]@.
\item
Functions can be nested inside other functions. Nested functions can
access parameters and local variables of enclosing functions. For
instance, the name of the array \verb@a@ is visible in functions
\verb@swap@ and \verb@sort1@, and therefore need not be passed as a
parameter to them.
\end{itemize}
So far, Scala looks like a fairly conventional language with some
syntactic pecularities. In fact it is possible to write programs in a
conventional imperative or object-oriented style. This is important
because it is one of the things that makes it easy to combine Scala
components with components written in mainstream languages such as
Java, C\# or Visual Basic.

However, it is also possible to write programs in a style which looks
completely different. Here is Quicksort again, this time written in
functional style.

\begin{verbatim}
def sort(xs: List[int]): List[int] = {
  val pivot = a(a.length / 2);
  sort(a.filter(x => x < pivot))
    :::  a.filter(x => x == pivot)
    :::  sort(a.filter(x => x > pivot))
}
\end{verbatim}

The functional program works with lists instead of arrays.\footnote{In
a future complete implemenetation of Scala, we could also have used arrays
instead of lists, but at the moment arrays do not yet support
\verb@filter@ and \verb@:::@.}
It captures the essence of the quicksort algorithm in a concise way:
\begin{itemize}
\item Pick an element in the middle of the list as a pivot.
\item Partition the lists into two sub-lists containing elements that
are less than, respectively greater than the pivot element, and a
third list which contains elements equal to privot.
\item Sort the first two sub-lists by a recursive invocation of
the sort function.\footnote{This is not quite what the imperative algorithm does;
the latter partitions the array into two sub-arrays containing elements
less than or greater or equal to pivot.}
\item The result is obtained by appending the three sub-lists together.
\end{itemize}
Both the imperative and the functional implementation have the same
asymptotic complexity -- $O(N;log(N))$ in the average case and
$O(N^2)$ in the worst case. But where the imperative implementation
operates in place by modifying the argument array, the functional
implementation returns a new sorted list and leaves the argument
list unchanged. The functional implementation thus requires more
transient memory than the imperative one.

The functional implementation makes it look like Scala is a language
that's specialized for functional operations on lists. In fact, it
is not; all of the operations used in the example are simple library
methods of a class \verb@List[t]@ which is part of the standard
Scala library, and which itself is implemented in Scala.

In particular, there is the method \verb@filter@ which takes as
argument a {\em predicate function} that maps list elements to
boolean values. The result of \verb@filter@ is a list consisting of
all the elements of the original list for which the given predicate
function is true.  The \verb@filter@ method of an object of type
\verb@List[t]@ thus has the signature
\begin{verbatim}
  def filter(p: t => boolean): List[t]  .
\end{verbatim}
Here, \verb@t => boolean@ is the type of functions that take an element
of type \verb@t@ and return a \verb@boolean@.  Functions like
\verb@filter@ that take another function as argument or return one as
result are called {\em higher-order} functions.

In the quicksort program, \verb@filter@ is applied three times to an
anonymous function argument.  The first argument,
\verb@x => x <= pivot@ represents the function that maps its parameter
\verb@x@ to the boolean value \verb@x <= pivot@. That is, it yields
true if \verb@x@ is smaller or equal than \verb@pivot@, false
otherwise. The function is anonymous, i.e.\ it is not defined with a
name. The type of the \verb@x@ parameter is omitted because a Scala
compiler can infer it automatically from the context where the
function is used. To summarize, \verb@xs.filter(x => x <= pivot)@
returns a list consisting of all elements of the list \verb@xs@ that are
smaller than \verb@pivot@.

\comment{
It is also possible to apply higher-order functions such as
\verb@filter@ to named function arguments. Here is functional
quicksort again, where the two anonymous functions are replaced by
named auxiliary functions that compare the argument to the
\verb@pivot@ value.

\begin{verbatim}
def sort (xs: List[int]): List[int] = {
  val pivot = xs(xs.length / 2);
  def leqPivot(x: int) = x <= pivot;
  def gtPivot(x: int) = x > pivot;
  def eqPivot(x: int) = x == pivot;
  sort(xs filter leqPivot)  
    ::: sort(xs filter eqPivot)  
    ::: sort(xs filter gtPivot)
}
\end{verbatim}
}

An object of type \verb@List[t]@ also has a method ``\verb@:::@''
which takes an another list and which returns the result of appending this
list to itself. This method has the signature
\begin{verbatim}
  def :::(that: List[t]): List[t] .
\end{verbatim}
Scala does not distinguish between identifiers and operator names. An
identifier can be either a sequence of letters and digits which begins
with a letter, or it can be a sequence of special characters, such as
``\verb@+@'', ``\verb@*@'', or ``\verb@:@''.  The last definition thus
introduced a new method identifier ``\verb@:::@''.  This identifier is
used in the Quicksort example as a binary infix operator that connects
the two sub-lists resulting from the partition. In fact, any method
can be used as an operator in Scala.  The binary operation $E;op;E'$
is always interpreted as the method call $E.op(E')$. This holds also
for binary infix operators which start with a letter. The recursive call
to \verb@sort@ in the last quicksort example is thus equivalent to
\begin{verbatim}
  sort(a.filter(x => x < pivot))
    .:::(sort(a.filter(x => x == pivot)))
    .:::(sort(a.filter(x => x > pivot))) .
\end{verbatim}

Looking again in detail at the first, imperative implementation of
Quicksort, we find that many of the language constructs used in the
second solution are also present, albeit in a disguised form.

For instance, ``standard'' binary operators such as \verb@+@,
\verb@-@, or \verb@<@ are not treated in any special way. Like
\verb@append@, they are methods of their left operand. Consequently,
the expression \verb@i + 1@ is regarded as the invocation
\verb@i.+(1)@ of the \verb@+@ method of the integer value \verb@x@.
Of course, a compiler is free (if it is moderately smart, even expected)
to recognize the special case of calling the \verb@+@ method over
integer arguments and to generate efficient inline code for it.

Control constructs such as \verb@while@ are also not primitive but are
predefined functions in the standard Scala library. Here is the
definition of \verb@while@ in Scala.
\begin{verbatim}
def while (def p: boolean) (def s: unit): unit = if (p) { s ; while(p)(s) }
\end{verbatim}
The \verb@while@ function takes as first parameter a test function,
which takes no parameters and yields a boolean value. As second
parameter it takes a command function which also takes no parameters
and yields a trivial result. \verb@while@ invokes the command function
as long as the test function yields true. Again, compilers are free to
pick specialized implementations of \verb@while@ that have the same
behavior as the invocation of the function given above.

\chapter{\label{chap:example-auction}Programming with Actors and Messages}

Here's an example that shows an application area for which Scala is
particularly well suited. Consider the task of implementing an
electronic auction service. We use an Erlang-style actor process
model to implement the participants of the auction. Actors are
objects to which messages are sent. Every process has a ``mailbox'' of
its incoming messages which is represented as a queue. It can work
sequentially through the messages in its mailbox, or search for
messages matching some pattern.

For every traded item there is an auctioneer process that publishes
information about the traded item, that accepts offers from clients
and that communicates with the seller and winning bidder to close the
transaction. We present an overview of a simple implementation
here.

As a first step, we define the messages that are exchanged during an
auction. There are two abstract base classes (called {\em traits}):
\verb@AuctionMessage@ for messages from clients to the auction
service, and \verb@AuctionReply@ for replies from the service to the
clients.  These are defined as follows.
\begin{verbatim}
trait AuctionMessage;
case class 
  Offer(bid: int, client: Actor),                            \=// make a bid
  Inquire(client: Actor) extends AuctionMessage;      \>// inquire status

trait AuctionReply;
case class
  Status(asked: int, expiration: Date),               \>// asked sum, expiration date
  BestOffer,                                        \>// yours is the best offer
  BeatenOffer(maxBid: int),                           \>// offer beaten by maxBid
  AuctionConcluded(seller: Actor, client: Actor),     \>// auction concluded
  AuctionFailed,                                    \>// failed with no bids
  AuctionOver extends AuctionReply;                 \>// bidding is closed
\end{verbatim}

\begin{figure}[htb]
\begin{verbatim}
class Auction(seller: Actor, minBid: int, closing: Date) extends Actor {
  val timeToShutdown = 36000000; // msec
  val bidIncrement = 10;
  def execute {
    var maxBid = minBid - bidIncrement;
    var maxBidder: Actor = _;
    var running = true;
    while (running) {
      receiveWithin ((closing.getTime() - new Date().getTime())) {
	case Offer(bid, client) =>
	  if (bid >= maxBid + bidIncrement) { 
            if (maxBid >= minBid) maxBidder send BeatenOffer(bid);
            maxBid = bid; maxBidder = client; client send BestOffer;
          } else {
            client send BeatenOffer(maxBid);
          }

	case Inquire(client) =>
	  client send Status(maxBid, closing);

	case TIMEOUT =>
	  if (maxBid >= minBid) {
	    val reply = AuctionConcluded(seller, maxBidder);
	    maxBidder send reply; seller send reply;
	  } else {
	    seller send AuctionFailed;
          }
          receiveWithin(timeToShutdown) {
            case Offer(_, client) => client send AuctionOver
            case TIMEOUT => running = false;
          }}}}}
\end{verbatim}
\caption{\label{fig:simple-auction}Implementation of an Auction Service}
\end{figure}

For each base class, there are a number of {\em case classes} which
define the format of particular messages in the class. These messages
might well be ultimately mapped to small XML documents. We expect
automatic tools to exist that convert between XML documents and
internal data structures like the ones defined above.

Figure~\ref{fig:simple-auction} presents a Scala implementation of a
class \verb@Auction@ for auction processes that coordinate the bidding
on one item. Objects of this class are created by indicating
\begin{itemize}
\item
a seller process which needs to be notified when the auction is over,
\item
a minimal bid,
\item
the date when the auction is to be closed.
\end{itemize}
The process behavior is defined by its \verb@run@ method. That method
repeatedly selects (using \verb@receiveWithin@) a message and reacts to it,
until the auction is closed, which is signalled by a \verb@TIMEOUT@
message. Before finally stopping, it stays active for another period
determined by the \verb@timeToShutdown@ constant and replies to
further offers that the auction is closed.  

Here are some further explanations of the constructs used in this
program:
\begin{itemize}
\item
The \verb@receiveWithin@ method of class \verb@Actor@ takes as
parameters a time span given in milliseconds and a function that
processes messages in the mailbox. The function is given by a sequence
of cases that each specify a pattern and an action to perform for
messages matching the pattern. The \verb@receiveWithin@ method selects
the first message in the mailbox which matches one of these patterns
and applies the corresponding action to it.
\item
The last case of \verb@receiveWithin@ is guarded by a
\verb@TIMEOUT@ pattern. If no other messages are received in the meantime, this
pattern is triggered after the time span which is passed as argument
to the enclosing \verb@receiveWithin@ method. \verb@TIMEOUT@ is a
particular instance of class \verb@Message@, which is triggered by the
\verb@Actor@ implementation itself.
\item
Reply messages are sent using syntax of the form
\verb@destination send SomeMessage@. \verb@send@ is used here as a
binary operator with a process and a message as arguments. This is
equivalent in Scala to the method call
\verb@destination.send(SomeMessage)@, i.e. the invocation of
the \verb@send@ of the destination process with the given message as
parameter.
\end{itemize}
The preceding discussion gave a flavor of distributed programming in
Scala. It might seem that Scala has a rich set of language constructs
that support actor processes, message sending and receiving,
programming with timeouts, etc. In fact, the opposite is true. All the
constructs discussed above are offered as methods in the library class
\verb@Actor@. That class is itself implemented in Scala, based on the underlying 
thread model of the host language (e.g. Java, or .NET).
The implementation of all features of class \verb@Actor@ used here is
given in Section~\ref{sec:actors}.

The advantages of this approach are relative simplicity of the core
language and flexibility for library designers. Because the core
language need not specify details of high-level process communication,
it can be kept simpler and more general. Because the particular model
of messages in a mailbox is a library module, it can be freely
modified if a different model is needed in some applications.  The
approach requires however that the core language is expressive enough
to provide the necessary language abstractions in a convenient
way. Scala has been designed with this in mind; one of its major
design goals was that it should be flexible enough to act as a
convenient host language for domain specific languages implemented by
library modules. For instance, the actor communication constructs
presented above can be regarded as one such domain specific language,
which conceptually extends the Scala core.

\chapter{\label{chap:simple-funs}Expressions and Simple Functions}

The previous examples gave an impression of what can be done with
Scala.  We now introduce its constructs one by one in a more
systematic fashion. We start with the smallest level, expressions and
functions.

\section{Expressions And Simple Functions}

A Scala system comes with an interpreter which can be seen as a
fancy calculator. A user interacts with the calculator by typing in
expressions and obtaining the results of their evaluation. Example:

\begin{verbatim}
? 87 + 145
232

? 1000 - 333
667

? 5 + 2 * 3
11
\end{verbatim}
It is also possible to name a sub-expression and use the name instead
of the expression afterwards:
\begin{verbatim}
? def size = 2
def size: int

? 5 * size
10
\end{verbatim}
\begin{verbatim}
? def pi = 3.14159
def pi: double

? def radius = 10
def radius: int


? 2 * pi * radius
62.8318
\end{verbatim}
Definitions start with the reserved word \verb@def@; they introduce a
name which stands for the expression following the \verb@=@ sign.  The
interpreter will answer with the introduced name and its type.

Executing a definition such as \verb@def x = e@ will not evaluate the
expression \verb@e@.  Instead \verb@e@ is evaluated whenever \verb@x@
is used. Alternatively, Scala offers a value definition 
\verb@val x = e@, which does evaluate the right-hand-side \verb@e@ as part of the
evaluation of the definition. If \verb@x@ is then used subsequently,
it is immediately replaced by the pre-computed value of
\verb@e@, so that the expression need not be evaluated again.
 
How are expressions evaluated? An expression consisting of operators
and operands is evaluated by repeatedly applying the following
simplification steps.
\begin{itemize}
\item pick the left-most operation
\item evaluate its operands
\item apply the operator to the operand values.
\end{itemize}
A name defined by \verb@def@\ is evaluated by replacing the name by the
definition's right hand side. A name defined by \verb@val@ is
evaluated by replacing the name by the value of the definitions's
right-hand side.  The evaluation process stops once we have reached a
value. A value is some data item such as a string, a number, an array,
or a list.

\example
Here is an evaluation of an arithmetic expression.
\begin{verbatim}
    \=(2 * pi) * radius
->  \>(2 * 3.14159) * radius
->  \>6.28318 * radius
->  \>6.28318 * 10
->  \>62.8318
\end{verbatim}
The process of stepwise simplification of expressions to values is
called {\em reduction}.

\section{Parameters}

Using \verb@def@, one can also define functions with parameters. Example:
\begin{verbatim}
? def square(x: double) = x * x
def square(x: double): double

? square(2)
4.0

? square(5 + 4)
81.0

? square(square(4))
256.0

? def sumOfSquares(x: double, y: double) = square(x) + square(y)
def sumOfSquares(x: double, y: double): double
\end{verbatim}

Function parameters follow the function name and are always enclosed
in parentheses.  Every parameter comes with a type, which is indicated
following the parameter name and a colon. At the present time, we only
need basic numeric types such as the type \verb@double@ of double
precision numbers. These are written as in Java.

Functions with parameters are evaluated analogously to operators in
expressions. First, the arguments of the function are evaluated (in
left-to-right order). Then, the function application is replaced by
the function's right hand side, and at the same time all formal
parameters of the function are replaced by their corresponding actual
arguments.

\example\ 
 
\begin{verbatim}
    \=sumOfSquares(3, 2+2)
->  \>sumOfSquares(3, 4)
->  \>square(3) + square(4)
->  \>3 * 3 + square(4)
->  \>9 + square(4)
->  \>9 + 4 * 4
->  \>9 + 16
->  \>25
\end{verbatim}

The example shows that the interpreter reduces function arguments to
values before rewriting the function application.  One could instead
have chosen to apply the function to unreduced arguments. This would
have yielded the following reduction sequence:
\begin{verbatim}
    \= sumOfSquares(3, 2+2)
->  \>square(3) + square(2+2)
->  \>3 * 3 + square(2+2)
->  \>9 + square(2+2)
->  \>9 + (2+2) * (2+2)
->  \>9 + 4 * (2+2)
->  \>9 + 4 * 4
->  \>9 + 16
->  \>25
\end{verbatim}

The second evaluation order is known as \emph{call-by-name},
whereas the first one is known as \emph{call-by-value}.  For
expressions that use only pure functions and that therefore can be
reduced with the substitution model, both schemes yield the same final
values.  

Call-by-value has the advantage that it avoids repeated evaluation of
arguments. Call-by-name has the advantage that it avoids evaluation of
arguments when the parameter is not used at all by the function.
Call-by-value is usually more efficient than call-by-name, but a
call-by-value evaluation might loop where a call-by-name evaluation
would terminate. Consider:
\begin{verbatim}
? def loop: int = loop
def loop: int

? def first(x: int, y: int) = x
def first(x: int, y: int): int
\end{verbatim}
Then \verb@first(1, loop)@ reduces with call-by-name to \verb@1@,
whereas the same term reduces with call-by-value repeatedly to itself,
hence evaluation does not terminate.
\begin{verbatim}
    \=first(1, loop)
->  \>first(1, loop)
->  \>first(1, loop)
->  \>...
\end{verbatim}
Scala uses call-by-value by default, but it switches to call-by-name evaluation
if the parameter is preceded by \verb@def@.

\example\ 
 
\begin{verbatim}
? def constOne(x: int, def y: int) = 1
constOne(x: int, def y: int): int

? constOne(1, loop)
1

? constOne(loop, 2)                    // gives an infinite loop.
^C
\end{verbatim}

\section{Conditional Expressions}

Scala's \verb@if-else@ lets one choose between two alternatives.  Its
syntax is like Java's \verb@if-else@. But where Java's \verb@if-else@
can be used only as an alternative of statements, Scala allows the
same syntax to choose between two expressions. Scala's \verb@if-else@
hence also replaces Java's conditional expression \verb@ ... ? ... :
...@.

\example\ 

\begin{verbatim}
? def abs(x: double) = if (x >= 0) x else -x
abs(x: double): double
\end{verbatim}
Scala's boolean expressions are similar to Java's; they are formed
from the constants
\verb@true@ and
\verb@false@, comparison operators, boolean negation \verb@!@ and the
boolean operators \verb@&&@ and \verb@||@.

\section{\label{sec:sqrt}Example: Square Roots by Newton's Method}

We now illustrate the language elements introduced so far in the
construction of a more interesting program. The task is to write a
function
\begin{verbatim}
def sqrt(x: double): double = ... 
\end{verbatim}
which computes the square root of \verb@x@.

A common way to compute square roots is by Newton's method of
successive approximations. One starts with an initial guess \verb@y@
(say: \verb@y = 1@). One then repeatedly improves the current guess
\verb@y@ by taking the average of \verb@y@ and \verb@x/y@.
As an example, the next three columns indicate the guess \verb@y@, the
quotient \verb@x/y@, and their average for the first approximations of
$\sqrt 2$. 
\begin{verbatim}
1               \=2/1 = 2               \=1.5
1.5		\>2/1.5 = 1.3333        \>1.4167
1.4167		\>2/1.4167 = 1.4118	\>1.4142
1.4142		\>...			\>...

y               \>x/y                   \>(y + x/y)/2
\end{verbatim}
One can implement this algorithm in Scala by a set of small functions,
which each represent one of the elements of the algorithm.  

We first define a function for iterating from a guess to the result:
\begin{verbatim}
def sqrtIter(guess: double, x: double): double = 
  if (isGoodEnough(guess, x)) guess
  else sqrtIter(improve(guess, x), x);
\end{verbatim}
Note that \verb@sqrtIter@ calls itself recursively.  Loops in
imperative programs can always be modelled by recursion in functional
programs. 

Note also that the definition of \verb@sqrtIter@ contains a return
type, which follows the parameter section. Such return types are
mandatory for recursive functions. For a non-recursive function, the
return type is optional; if it is missing the type checker will
compute it from the type of the function's right-hand side. However,
even for non-recursive functions it is often a good idea to include a
return type for better documentation.

As a second step, we define the two functions called by
\verb@sqrtIter@: a function to \verb@improve@ the guess and a
termination test \verb@isGoodEnough@. Here's their definition.
\begin{verbatim}
def improve(guess: double, x: double) = 
  (guess + x / guess) / 2;

def isGoodEnough(guess: double, x: double) = 
  abs(square(guess) - x) < 0.001;
\end{verbatim}

Finally, the \verb@sqrt@ function itself is defined by an aplication
of \verb@sqrtIter@.
\begin{verbatim}
def sqrt(x: double) = sqrtIter(1.0, x);
\end{verbatim}

\exercise The \verb@isGoodEnough@ test is not very precise for small numbers
and might lead to non-termination for very large ones (why?).
Design a different version \verb@isGoodEnough@ which does not have these problems.

\exercise Trace the execution of the \verb@sqrt(4)@ expression.

\section{Nested Functions}

The functional programming style encourages the construction of many
small helper functions. In the last example, the implementation
of \verb@sqrt@ made use of the helper functions
\verb@sqrtIter@, \verb@improve@ and
\verb@isGoodEnough@. The names of these functions 
are relevant only for the implementation of
\verb@sqrt@. We normally do not want users of \verb@sqrt@ to acess these functions
directly.

We can enforce this (and avoid name-space pollution) by including
the helper functions within the calling function itself:
\begin{verbatim}
def sqrt(x: double) = {
  def sqrtIter(guess: double, x: double): double = 
    if (isGoodEnough(guess, x)) guess
    else sqrtIter(improve(guess, x), x);

  def improve(guess: double, x: double) = 
    (guess + x / guess) / 2;

  def isGoodEnough(guess: double, x: double) = 
    abs(square(guess) - x) < 0.001;

  sqrtIter(1.0, x)
}
\end{verbatim}
In this program, the braces \verb@{ ... }@ enclose a {\em block}.
Blocks in Scala are themselves expressions.  Every block ends in a
result expression which defines its value.  The result expression may
be preceded by auxiliary definitions, which are visible only in the
block itself.

Every definition in a block must be followed by a semicolon, which
separates this definition from subsequent definitions or the result
expression. However, a semicolon is inserted implicitly if the
definition ends in a right brace and is followed by a new line.
Therefore, the following are all legal:
\begin{verbatim}
def f(x) = x + 1; /* `;' mandatory */
f(1) + f(2)

def g(x) = {x + 1}
g(1) + g(2)

def h(x) = {x + 1};  /* `;' mandatory */  h(1) + h(2)
\end{verbatim}
Scala uses the usual block-structured scoping rules. A name defined in
some outer block is visible also in some inner block, provided it is
not redefined there. This rule permits us to simplify our
\verb@sqrt@ example. We need not pass \verb@x@ around as an additional parameter of
the nested functions, since it is always visible in them as a
parameter of the outer function \verb@sqrt@. Here is the simplified code:
\begin{verbatim}
def sqrt(x: double) = {
  def sqrtIter(guess: double): double = 
    if (isGoodEnough(guess)) guess
    else sqrtIter(improve(guess));

  def improve(guess: double) = 
    (guess + x / guess) / 2;

  def isGoodEnough(guess: double) = 
    abs(square(guess) - x) < 0.001;

  sqrtIter(1.0)
}
\end{verbatim}

\section{Tail Recursion}

Consider the following function to compute the greatest common divisor
of two given numbers.

\begin{verbatim}
def gcd(a: int, b: int): int = if (b == 0) a else gcd(b, a % b)
\end{verbatim}

Using our substitution model of function evaluation, 
\verb@gcd(14, 21)@ evaluates as follows:

\begin{verbatim}
        \=gcd(14, 21)  
->      \>if (21 == 0) 14 else gcd(21, 14 % 21)
->      \>if (false) 14 else gcd(21, 14 % 21)
->      \>gcd(21, 14 % 21)
->      \>gcd(21, 14)
->      \>if (14 == 0) 21 else gcd(14, 21 % 14)
-> ->   \>gcd(14, 21 % 14)
->      \>gcd(14, 7)
->      \>if (7 == 0) 14 else gcd(7, 14 % 7)
-> ->   \>gcd(7, 14 % 7)
->      \>gcd(7, 0)
->      \>if (0 == 0) 7 else gcd(0, 7 % 0)
-> ->   \>7
\end{verbatim}

Contrast this with the evaluation of another recursive function, 
\verb@factorial@:

\begin{verbatim}
def factorial(n: int): int = if (n == 0) 1 else n * factorial(n - 1)
\end{verbatim}

The application \verb@factorial(5)@ rewrites as follows:
\begin{verbatim}
        \=factorial(5)
->      \>if (5 == 0) 1 else 5 * factorial(5 - 1)
->      \>5 * factorial(5 - 1)
->      \>5 * factorial(4)
-> ... -> \>5 * (4 * factorial(3))
-> ... -> \>5 * (4 * (3 * factorial(2)))
-> ... -> \>5 * (4 * (3 * (2 * factorial(1))))
-> ... -> \>5 * (4 * (3 * (2 * (1 * factorial(0))))
-> ... -> \>5 * (4 * (3 * (2 * (1 * 1))))
-> ... -> \>120

\end{verbatim}
There is an important difference between the two rewrite sequences:
The terms in the rewrite sequence of \verb@gcd@ have again and again
the same form. As evaluation proceeds, their size is bounded by a
constant. By contrast, in the evaluation of factorial we get longer
and longer chains of operands which are then multiplied in the last
part of the evaluation sequence.

Even though actual implementations of Scala do not work by rewriting
terms, they nevertheless should have the same space behavior as in the
rewrite sequences. In the implementation of \verb@gcd@, one notes that
the recursive call to \verb@gcd@ is the last action performed in the
evaluation of its body. One also says that \verb@gcd@ is
``tail-recursive''. The final call in a tail-recursive function can be
implemented by a jump back to the beginning of that function. The
arguments of that call can overwrite the parameters of the current
instantiation of \verb@gcd@, so that no new stack space is needed.
Hence, tail recursive functions are iterative processes, which can be
executed in constant space.

By contrast, the recursive call in \verb@factorial@ is followed by a
multiplication.  Hence, a new stack frame is allocated for the
recursive instance of factorial, and is decallocated after that
instance has finished. The given formulation of the factorial function
is not tail-recursive; it needs space proportional to its input
parameter for its execution.

More generally, if the last action of a function is a call to another
(possibly the same) function, only a single stack frame is needed for
both functions. Such calls are called ``tail calls''. In principle,
tail calls can always re-use the stack frame of the calling function.
However, some run-time environments (such as the Java VM) lack the
primititives to make stack frame re-use for tail calls efficient.  A
production quality Scala implementation is therefore only required to re-use
the stack frame of a directly tail-recursive function whose last
action is a call to itself.  Other tail calls might be optimized also,
but one should not rely on this across
implementations\footnote{The current Scala implementation is not yet
production quality; it never optimizes tail calls, not even directly
recursive ones}.

\exercise Design a tail-recursive version of
\verb@factorial@.

\chapter{\label{chap:first-class-funs}First-Class Functions}

A function in Scala is a ``first-class value''. Like any other value,
it may be passed as a parameter or returned as a result.  Functions
which take other functions as parameters or return them as results are
called {\em higher-order} functions. This chapter introduces
higher-order functions and shows how they provide a flexible mechanism
for program composition.

As a motivating example, consider the following three related tasks:
\begin{enumerate}
\item
Write a function to sum all integers between two given numbers \verb@a@ and \verb@b@:
\begin{verbatim}
def sumInts(a: int, b: int): double =
  if (a > b) 0 else a + sumInts(a + 1, b)
\end{verbatim}
\item 
Write a function to sum the cubes of all integers between two given numbers 
\verb@a@ and \verb@b@:
\begin{verbatim}
def cube(x: int): double = x * x * x
def sumCubes(a: int, b: int): double =
  if (a > b) 0 else cube(a) + sumSqrts(a + 1, b)
\end{verbatim}
\item
Write a function to sum the reciprocals of all integers between two given numbers 
\verb@a@ and \verb@b@:
\begin{verbatim}
def sumReciprocals(a: int, b: int): double =
  if (a > b) 0 else 1.0 / a + sumReciprocals(a + 1, b)
\end{verbatim}
\end{enumerate}
These functions are all instances of
\(\sum^b_a f(n)\) for different values of $f$. 
We can factor out the common pattern by defining a function \verb@sum@:
\begin{verbatim}
def sum(f: int => double, a: int, b: int): double =
  if (a > b) 0 else f(a) + sum(f, a + 1, b)
\end{verbatim}
The type \verb@int => double@ is the type of functions that
take arguments of type \verb@int@ and return results of type
\verb@double@. So \verb@sum@ is a function which takes another function as 
a parameter. In other words, \verb@sum@ is a {\em higher-order}
function.

Using \verb@sum@, we can formulate the three summing functions as
follows.
\begin{verbatim}
def sumInts(a: int, b: int): double = sum(id, a, b);
def sumCubes(a: int, b: int): double = sum(cube, a, b);
def sumReciprocals(a: int, b: int): double = sum(reciprocal, a, b);
\end{verbatim}
where
\begin{verbatim}
def id(x: int): double = x;
def cube(x: int): double = x * x * x;
def reciprocal(x: int): double = 1.0/x;
\end{verbatim}

\section{Anonymous Functions}

Parameterization by functions tends to create many small functions. In
the previous example, we defined \verb@id@, \verb@cube@ and
\verb@reciprocal@ as separate functions, so that they could be 
passed as arguments to \verb@sum@.

Instead of using named function definitions for these small argument
functions, we can formulate them in a shorter way as {\em anonymous
functions}. An anonymous function is an expression that evaluates to a
function; the function is defined without giving it a name. As an
example consider the anonymous reciprocal function:
\begin{verbatim}
  x: int => 1.0/x
\end{verbatim}
The part before the arrow `\verb@=>@' is the parameter of the function,
whereas the part following the `\verb@=>@' is its body. If there are
several parameters, we need to enclose them in parentheses. For
instance, here is an anonymous function which multiples its two arguments.
\begin{verbatim}
  (x: double, y: double) => x * y
\end{verbatim}
Using anonymous functions, we can reformulate the three summation
functions without named auxiliary functions:
\begin{verbatim}
def sumInts(a: int, b: int): double = sum(x: int => x, a, b);
def sumCubes(a: int, b: int): double = sum(x: int => x * x * x, a, b);
def sumReciprocals(a: int, b: int): double = sum(x: int => 1.0/x, a, b);
\end{verbatim}
Often, the Scala compiler can deduce the parameter type(s) from the
context of the anonymous function. In this case, they can be omitted.
For instance, in the case of \verb@sumInts@, \verb@sumCubes@ and
verb@sumReciprocals@, one knows from the type of
\verb@sum@ that the first parameter must be a function of type
\verb@int => double@.  Hence, the parameter type \verb@int@ is
redundant and may be omitted:
\begin{verbatim}
def sumInts(a: int, b: int): double = sum(x => x, a, b);
def sumCubes(a: int, b: int): double = sum(x => x * x * x, a, b);
def sumReciprocals(a: int, b: int): double = sum(x => 1.0/x, a, b);
\end{verbatim}

Generally, the Scala term
\verb@(x$_1$: T$_1$, ..., x$_n$: T$_n$) => E@ 
defines a function which maps its parameters
\verb@x$_1$, ..., x$_n$@ to the result of the expression \verb@E@
(where \verb@E@ may refer to \verb@x$_1$, ..., x$_n$@).  Anonymous
functions are not essential language elements of Scala, as they can
always be expressed in terms of named functions. Indeed, the 
anonymous function
\verb@(x$_1$: T$_1$, ..., x$_n$: T$_n$) => E@ 
is equivalent to the block
\begin{verbatim}
{ def f (x$_1$: T$_1$, ..., x$_n$: T$_n$) = E ; f }
\end{verbatim}
where \verb@f@ is fresh name which is used nowhere else in the program.
We also say, anonymous functions are ``syntactic sugar''.

\section{Currying}

The latest formulation of the three summing function is already quite
compact. But we can do even better. Note that
\verb@a@ and \verb@b@ appear as parameters and arguments of every function
but they do not seem to take part in interesting combinations. Is
there a way to get rid of them?

Let's try to rewrite \verb@sum@ so that it does not take the bounds
\verb@a@ and \verb@b@ as parameters:
\begin{verbatim}
def sum(f: int => double) = {
  def sumF(a: int, b: int): double = 
    if (a > b) 0 else f(a) + sumF(a + 1, b);
  sumF
}
\end{verbatim}
In this formulation, \verb@sum@ is a function which returns another
function, namely the specialized summing function \verb@sumF@. This
latter function does all the work; it takes the bounds \verb@a@ and
\verb@b@ as parameters, applies \verb@sum@'s function parameter \verb@f@ to all
integers between them, and sums up the results. 

Using this new formulation of \verb@sum@, we can now define:
\begin{verbatim}
def sumInts = sum(x => x);
def sumCubes = sum(x => x * x * x);
def sumReciprocals = sum(x => 1.0/x);
\end{verbatim}
Or, equivalently, with value definitions:
\begin{verbatim}
val sumInts = sum(x => x);
val sumCubes = sum(x => x * x * x);
val sumReciprocals = sum(x => 1.0/x);
\end{verbatim}
These functions can be applied like other functions. For instance,
\begin{verbatim}
? sumCubes(1, 10) + sumReciprocals (10, 20)
3025.7687714031754
\end{verbatim}
How are function-returning functions applied? As an example, in the expression
\begin{verbatim}
sum (x => x * x * x) (1, 10) ,
\end{verbatim}
the function \verb@sum@ is applied to the cubing function 
\verb@(x => x * x * x)@. The resulting function is then 
applied to the second argument list, \verb@(1, 10)@.

This notation is possible because function application associates to the left.
That is, if $args_1$ and $args_2$ are argument lists, then 
\bda{lcl}
f(args_1)(args_2) & \ \ \mbox{is equivalent to}\ \ & (f(args_1))(args_2)
\eda
In our example, \verb@sum(x => x * x * x)(1, 10)@ is equivalent to 
\verb@(sum(x => x * x * x))(1, 10)@.

The style of function-returning functions is so useful that Scala has
special syntax for it. For instance, the next definition of \verb@sum@
is equivalent to the previous one, but is shorter:
\begin{verbatim}
def sum(f: int => double)(a: int, b: int): double =
  if (a > b) 0 else f(a) + sum(f)(a + 1, b)
\end{verbatim}
Generally, a curried function definition 
\begin{verbatim}
def f (args$_1$) ... (args$_n$) = E
\end{verbatim}
where $n > 1$ expands to
\begin{verbatim}
def f (args$_1$) ... (args$_{n-1}$) = { def g (args$_n$) = E ; g }
\end{verbatim}
where \verb@g@ is a fresh identifier. Or, shorter, using an anonymous function:
\begin{verbatim}
def f (args$_1$) ... (args$_{n-1}$) = ( args$_n$ ) => E .
\end{verbatim}
Performing this step $n$ times yields that
\begin{verbatim}
def f (args$_1$) ... (args$_n$) = E
\end{verbatim}
is equivalent to
\begin{verbatim}
def f = (args$_1$) => ... => (args$_n$) => E .
\end{verbatim}
Or, equivalently, using a value definition:
\begin{verbatim}
val f = (args$_1$) => ... => (args$_n$) => E .
\end{verbatim}
This style of function definition and application is called {\em
currying} after its promoter, Haskell B.\ Curry, a logician of the
20th century, even though the idea goes back further to Moses
Sch\"onfinkel and Gottlob Frege.

The type of a function-returning function is expressed analogously to
its parameter list. Taking the last formulation of \verb@sum@ as an example,
the type of \verb@sum@ is \verb@(int => double) => (int, int) => double@.
This is possible because function types associate to the right. I.e.
\begin{verbatim}
T$_1$ => T$_2$ => T$_3$       \=$\mbox{is equivalent to}$     \=T$_1$ => (T$_2$ => T$_3$) .
\end{verbatim}

\subsection*{Exercises:}

1. The \verb@sum@ function uses a linear recursion. Can you write a
tail-recursive one by filling in the ??'s?

\begin{verbatim}
def sum(f: int => double)(a: int, b: int): double = {
  def iter (a, result) = {
    if (??) ??
    else iter (??, ??)
  }
  iter (??, ??)
}
\end{verbatim}

2. Write a function \verb@product@ that computes the product of the
values of functions at points over a given range.

3. Write \verb@factorial@ in terms of \verb@product@.

4. Can you write an even more general function which generalizes both
\verb@sum@ and \verb@product@?

\section{Example: Finding Fixed Points of Functions}

A number \verb@x@ is called a {\em fixed point} of a function \verb@f@ if
\begin{verbatim}
f(x) = x .
\end{verbatim}
For some functions \verb@f@ we can locate the fixed point by beginning
with an initial guess and then applying \verb@f@ repeatedly, until the
value does not change anymore (or the change is within a small
tolerance). This is possible if the sequence
\begin{verbatim}
x, f(x), f(f(x)), f(f(f(x))), ...
\end{verbatim}
converges to fixed point of $f$. This idea is captured in
the following ``fixed-point finding function'':
\begin{verbatim}
val tolerance = 0.0001;
def isCloseEnough(x: double, y: double) = abs((x - y) / x) < tolerance;
def fixedPoint(f: double => double)(firstGuess: double) = {
  def iterate(guess: double): double = {
    val next = f(guess);
    if (isCloseEnough(guess, next)) next
    else iterate(next)
  }
  iterate(firstGuess)
}
\end{verbatim}
We now apply this idea in a reformulation of the square root function.
Let's start with a specification of \verb@sqrt@:
\begin{verbatim}
sqrt(x)  \==  $\mbox{the {\sl y} such that}$  y * y = x
         \>=  $\mbox{the {\sl y} such that}$  y = x / y
\end{verbatim}
Hence, \verb@sqrt(x)@ is a fixed point of the function \verb@y => x / y@.
This suggests that \verb@sqrt(x)@ can be computed by fixed point iteration:
\begin{verbatim}
def sqrt(x: double) = fixedPoint(y => x / y)(1.0)
\end{verbatim}
Unfortunately, this does not converge. Let's instrument the fixed point
function with a print statement which keeps track of the current
\verb@guess@ value:
\begin{verbatim}
def fixedPoint(f: double => double)(firstGuess: double) = {
  def iterate(guess: double): double = {
    val next = f(guess);
    System.out.println(next);
    if (isCloseEnough(guess, next)) next
    else iterate(next)
  }
  iterate(firstGuess)
}
\end{verbatim}
Then, \verb@sqrt(2)@ yields:
\begin{verbatim}
  2.0
  1.0
  2.0
  1.0
  2.0
  ...
\end{verbatim}
One way to control such oscillations is to prevent the guess from changing too much. 
This can be achieved by {\em averaging} successive values of the original sequence:
\begin{verbatim}
> def sqrt(x: double) = fixedPoint(y => (y + x/y) / 2)(1.0)
> sqrt(2.0)
  1.5
  1.4166666666666665
  1.4142156862745097
  1.4142135623746899
  1.4142135623746899
\end{verbatim}
In fact, expanding the \verb@fixedPoint@ function yields exactly our 
previous definition of fixed point from Section~\ref{sec:sqrt}.

The previous examples showed that the expressive power of a language
is considerably enhanced if functions can be passed as arguments.  The
next example shows that functions which return functions can also be
very useful.

Consider again fixed point iterations. We started with the observation
that $\sqrt(x)$ is a fixed point of the function \verb@y => x / y@.
Then we made the iteration converge by averaging successive values.
This technique of {\em average dampening} is so general that it
can be wrapped in another function.
\begin{verbatim}
def averageDamp(f: double => double)(x: double) = (x + f(x)) / 2
\end{verbatim}
Using \verb@averageDamp@, we can reformulate the square root function
as follows.
\begin{verbatim}
def sqrt(x: double) = fixedPoint(averageDamp(y => x/y))(1.0)
\end{verbatim}
This expresses the elements of the algorithm as clearly as possible.

\exercise Write a function for cube roots using \verb@fixedPoint@ and 
\verb@averageDamp@.

\section{Summary}

We have seen in the previous chapter that functions are essential
abstractions, because they permit us to introduce general methods of
computing as explicit, named elements in our programming language.
The current chapter has shown that these abstractions can be combined by
higher-order functions to create further abstractions.  As
programmers, we should look out for opportunities to abstract and to
reuse. The highest possible level of abstraction is not always the
best, but it is important to know abstraction techniques, so that one
can use abstractions where appropriate.

\section{Language Elements Seen So Far}

Chapters~\ref{chap:simple-funs} and \ref{chap:first-class-funs} have
covered Scala's language elements to express expressions and types
comprising of primitive data and functions.  The context-free syntax
of these language elements is given below in extended Backus-Naur
form, where `\verb@|@' denotes alternatives, \verb@[...]@ denotes
option (0 or 1 occurrences), and \verb@{...}@ denotes repetition (0 or
more occurrences).

\subsection*{Characters}

Scala programs are sequences of (Unicode) characters. We distinguish the
following character sets:
\begin{itemize}
\item
whitespace, such as `\verb@ @', tabulator, or newline characters,
\item
letters `\verb@a@' to `\verb@z@', `\verb@A@' to `\verb@Z@',
\item
digits \verb@`0'@ to `\verb@9@',
\item
the delimiter characters
\begin{verbatim}
.    ,    ;    (    )    {    }    [    ]    \    "    '
\end{verbatim}
\item
operator characters, such as `\verb@#@' `\verb@+@',
`\verb@:@'. Essentially, these are printable characters which are
in none of the character sets above.
\end{itemize}

\subsection*{Lexemes:}

\begin{verbatim}
ident      ::=   letter {letter | digit} | operator { operator } | ident `_' ident
literal    ::=   $\mbox{``as in Java''}$
\end{verbatim}

Literals are as in Java. They define numbers, characters, strings, or
boolean values.  Examples of literals as \verb@0@, \verb@1.0d10@, \verb@'x'@,
\verb@"he said \"hi!\""@, or \verb@true@.

Identifiers can be of two forms. They either start with a letter,
which is followed by a (possibly empty) sequence of letters or
symbols, or they start with an operator character, which is followed
by a (possibly empty) sequence of operator characters.  Both forms of
identifiers may contain underscore characters `\verb@_@'. Furthermore,
an underscore character may be followed by either sort of
identifier. Hence, the following are all legal identifiers:
\begin{verbatim}
x     Room10a     +     --     foldl_:     +_vector
\end{verbatim}
It follows from this rule that subsequent operator-identifiers need to
be separated by whitespace. For instance, the input
\verb@x+-y@ is parsed as the three token sequence \verb@x@, \verb@+-@,
\verb@y@. If we want to express the sum of \verb@x@ with the
negated value of \verb@y@, we need to add at least one space,
e.g. \verb@x+ -y@.

The `\verb@\$@' character is reserved for compiler-generated
identifiers; it should not be used in source programs. %$

The following are reserved words, they may not be used as identifiers:
\begin{verbatim}
abstract    case    class    def    do    else   
extends    false    final    for    if    import    
new    null    object    override    package    
private    protected    super    this    trait
true    type    val    var    with    yield
\end{verbatim}

\subsection*{Types:}

\begin{verbatim}
Type           \= = SimpleType | FunctionType
FunctionType \> = SimpleType `=>' Type | `(' [Types] `)' `=>' Type
SimpleType   \> = byte | short | char | int | long | double | float | boolean | unit | String
Types        \> = Type {`,' Type}
\end{verbatim}

Types can be:
\begin{itemize}
\item number types \verb@byte@, \verb@short@, \verb@char@, \verb@int@, \verb@long@, \verb@float@ and \verb@double@ (these are as in Java),
\item the type \verb@boolean@ with values \verb@true@ and \verb@false@,
\item the type \verb@unit@ with the only value \verb@{}@,
\item the type \verb@String@,
\item function types such as \verb@(int, int) => int@ or \verb@String => Int => String@. 
\end{itemize}

\subsection*{Expressions:}

\begin{verbatim}
Expr               \= = InfixExpr | FunctionExpr | if `(' Expr `)' Expr else Expr
InfixExpr   \> = PrefixExpr | InfixExpr Operator InfixExpr
Operator    \> = ident
PrefixExpr  \> = [`+' | `-' | `!' | `~' ] SimpleExpr
SimpleExpr  \> = ident | literal | SimpleExpr `.' ident | Block
FunctionExpr\> = Bindings `=> Expr
Bindings    \> = ident [`:' SimpleType] | `(' [Binding {`,' Binding}] `)'
Binding     \> = ident [`:' Type]
Block       \> = `{' {Def `;'} Expr `}'
\end{verbatim}

Expressions can be:
\begin{itemize}
\item
identifiers such as \verb@x@, \verb@isGoodEnough@, \verb@*@, or \verb@+-@,
\item
literals, such as \verb@0@, \verb@1.0@, or \verb@"abc"@,
\item
field and method selections, such as \verb@System.out.println@,
\item
function applications, such as \verb@sqrt(x)@, 
\item
operator applications, such as \verb@-x@ or \verb@y + x@,
\item
conditionals, such as \verb@if (x < 0) -x else x@,
\item
blocks, such as \verb@{ val x = abs(y) ; x * 2 }@,
\item
anonymous functions, such as \verb@x => x + 1@ or \verb@(x: int, y: int) => x + y@.
\end{itemize}

\subsection*{Definitions:}

\begin{verbatim}
Def                \= = \=FunDef  |  ValDef
FunDef       \> = \>def ident {`(' [Parameters] `)'} [`:' Type] `=' Expr
ValDef       \> = \>val ident [`:' Type] `=' Expr
Parameters   \> = \>Parameter {`,' Parameter}
Parameter    \> = \>[def] ident `:' Type
\end{verbatim}
Definitions can be:
\begin{itemize}
\item
function definitions such as \verb@def square(x: int) = x * x@, 
\item
value definitions such as \verb@val y = square(2)@.
\end{itemize}

\chapter{Classes and Objects}
\label{chap:classes}

Scala does not have a built-in type of rational numbers, but it is
easy to define one, using a class. Here's a possible implementation.

\begin{verbatim}
class Rational(n: int, d: int) {
  private def gcd(x: int, y: int): int = {
    if (x == 0) y
    else if (x < 0) gcd(-x, y)
    else if (y < 0) -gcd(x, -y)
    else gcd(y % x, x);
  }
  private val g = gcd(n, d);

  val numer: int = n/g;
  val denom: int = d/g;
  def +(that: Rational) =
    new Rational(numer * that.denom + that.numer * denom, denom * that.denom);
  def -(that: Rational) =
    new Rational(numer * that.denom - that.numer * denom, denom * that.denom);
  def *(that: Rational) =
    new Rational(numer * that.numer, denom * that.denom);
  def /(that: Rational) =
    new Rational(numer * that.denom, denom * that.numer);
}
\end{verbatim}
This defines \verb@Rational@ as a class which takes two constructor
arguments \verb@n@ and \verb@d@, containing the number's numerator and
denominator parts.  The class provides fields which return these parts
as well as methods for arithmetic over rational numbers.  Each
arithmetic method takes as parameter the right operand of the
operation. The left operand of the operation is always the rational
number of which the method is a member.

\paragraph{Private members.}
The implementation of rational numbers defines a private method
\verb@gcd@ which computes the greatest common denominator of two
integers, as well as a private field \verb@g@ which contains the
\verb@gcd@ of the constructor arguments. These members are inaccessible
outside class \verb@Rational@. They are used in the implementation of
the class to eliminate common factors in the constructor arguments in
order to ensure that nominator and denominator are always in
normalized form.

\paragraph{Creating and Accessing Objects.}
As an example of how rational numbers can be used, here's a program
that prints the sum of all numbers $1/i$ where $i$ ranges from 1 to 10.
\begin{verbatim}
var i = 1;
var x = Rational(0, 1);
while (i <= 10) {
  x = x + Rational(1,i);
  i = i + 1;
}
System.out.println(x.numer + "/" + x.denom);
\end{verbatim}
The \verb@+@ operation converts both its operands to strings and returns the
concatenation of the result strings. It thus corresponds to \verb@+@ in Java. 
  
\paragraph{Inheritance and Overriding.}
Every class in Scala has a superclass which it extends.
Excepted is only the root class \verb@Object@, which does not have a
superclass, and which is indirectly extended by every other class.
If a class does not mention a superclass in its definition, the root
class \verb@Object@ is implicitly assumed. For instance, class
\verb@Rational@ could equivalently be defined as
\begin{verbatim}
class Rational(n: int, d: int) extends Object {
  ... // as before
}
\end{verbatim}
A class inherits all members from its superclass. It may also redefine
(or: {\em override}) some inherited members. For instance, class
\verb@Object@ defines
a method
\verb@toString@ which returns a representation of the object as a string:
\begin{verbatim}
class Object {
  ...
  def toString(): String = ...
}
\end{verbatim}
The implementation of \verb@toString@ in \verb@Object@
forms a string consisting of the object's class name and a number. It
makes sense to redefine this method for objects that are rational
numbers:
\begin{verbatim}
class Rational(n: int, d: int) extends Object {
  ... // as before
  override def toString() = numer + "/" + denom;
}
\end{verbatim}
Note that, unlike in Java, redefining definitions need to be preceded
by an \verb@override@ modifier.

If class $A$ extends class $B$, then objects of type $A$ may be used
wherever objects of type $B$ are expected. We say in this case that
type $A$ {\em conforms} to type $B$.  For instance, \verb@Rational@
conforms to \verb@Object@, so it is legal to assign a \verb@Rational@
value to a variable of type \verb@Object@:
\begin{verbatim}
var x: Object = new Rational(1,2);
\end{verbatim}

\paragraph{Parameterless Methods.}
%Also unlike in Java, methods in Scala do not necessarily take a
%parameter list. An example is \verb@toString@; the method is invoked
%by simply mentioning its name. For instance:
%\begin{verbatim}
%val r = new Rational(1,2);
%System.out.println(r.toString());	// prints``1/2''
%\end{verbatim}
Also unlike in Java, methods in Scala do not necessarily take a
parameter list. An example is the \verb@square@ method below. This
method is invoked by simply mentioning its name. 
\begin{verbatim}
class Rational(n: int, d: int) extends Object {
  ... // as before
  def square = Rational(numer*numer, denom*denom);
}
val r = new Rational(3,4);
System.out.println(r.square);		// prints``9/16''
\end{verbatim}
That is, parameterless methods are accessed just as value fields such
as \verb@numer@ are. The difference between values and parameterless
methods lies in their definition. The right-hand side of a value is
evaluated when the object is created, and the value does not change
afterwards. A right-hand side of a parameterless method, on the other
hand, is evaluated each time the method is called.  The uniform access
of fields and parameterless methods gives increased flexibility for
the implementer of a class. Often, a field in one version of a class
becomes a computed value in the next version. Uniform access ensures
that clients do not have to be rewritten because of that change.

\paragraph{Abstract Classes}

Consider the task of writing a class for sets of integer numbers with
two operations, \verb@incl@ and \verb@contains@. \verb@(s incl x)@
should return a new set which contains the element \verb@x@ togther
with all the elements of set \verb@s@. \verb@(s contains x)@ should
return true if the set \verb@s@ contains the element \verb@x@, and
should return \verb@false@ otherwise. The interface of such sets is
given by:  
\begin{verbatim}
abstract class IntSet {
  def incl(x: int): IntSet;
  def contains(x: int): boolean;
}
\end{verbatim}
\verb@IntSet@ is labeled as an \emph{abstract class}. This has two
consequences.  First, abstract classes may have {\em deferred} members
which are declared but which do not have an implementation. In our
case, both \verb@incl@ and \verb@contains@ are such members. Second,
because an abstract class might have unimplemented members, no objects
of that class may be created using \verb@new@. By contrast, an
abstract class may be used as a base class of some other class, which
implements the deferred members.

\paragraph{Traits.}

Instead of ``\verb@abstract class@ one also often uses the keyword
\verb@trait@ in Scala. A trait is an abstract class with no state, no
constructor arguments, and no side effects during object
initialization.  Since \verb@IntSet@'s fall in this category, one can
alternatively define them as traits:
\begin{verbatim}
trait IntSet {
  def incl(x: int): IntSet;
  def contains(x: int): boolean;
}
\end{verbatim}
A trait corresponds to an interface in Java, except
that a trait can also define implemented methods.  

\paragraph{Implementing Abstract Classes}

Let's say, we plan to implement sets as binary trees.  There are two
possible forms of trees. A tree for the empty set, and a tree
consisting of an integer and two subtrees. Here are their
implementations.

\begin{verbatim}
class Empty extends IntSet {
  def contains(x: int): boolean = false;
  def incl(x: int): IntSet = new NonEmpty(x, new Empty, new Empty);
}
\end{verbatim}

\begin{verbatim}
class NonEmpty(elem:int, left:IntSet, right:IntSet) extends IntSet {
  def contains(x: int): boolean = 
    if (x < elem) left contains x
    else if (x > elem) right contains x
    else true;
  def incl(x: int): IntSet = 
    if (x < elem) new NonEmpty(elem, left incl x, right)
    else if (x > elem) new NonEmpty(elem, left, right incl x)
    else this;
}
\end{verbatim}
Both \verb@Empty@ and \verb@NonEmpty@ extend class
\verb@IntSet@.  This implies that types \verb@Empty@ and
\verb@NonEmpty@ conform to type \verb@IntSet@ -- a value of type \verb@Empty@ or \verb@NonEmpty@ may be used wherever a value of type \verb@IntSet@ is required.

\exercise Write methods \verb@union@ and \verb@intersection@ to form
the union and intersection between two sets.

\exercise Add a method 
\begin{verbatim}
def excl(x: int)
\end{verbatim}
to return the given set without the element \verb@x@. To accomplish this,
it is useful to also implement a test method
\begin{verbatim}
def isEmpty: boolean
\end{verbatim}
for sets.

\paragraph{Dynamic Binding}

Object-oriented languages (Scala included) use \emph{dynamic dispatch}
for method invocations.  That is, the code invoked for a method call
depends on the run-time type of the object which contains the method.
For example, consider the expression \verb@s contains 7@ where
\verb@s@ is a value of declared type \verb@s: IntSet@. Which code for
\verb@contains@ is executed depends on the type of value of \verb@s@ at run-time.
If it is an \verb@Empty@ value, it is the implementation of \verb@contains@ in class \verb@Empty@ that is executed, and analogously for \verb@NonEmpty@ values. 
This behavior is a direct consequence of our substitution model of evaluation.
For instance,
\begin{verbatim}
    (new Empty).contains(7) 

->  $\rewriteby{by replacing {\sl contains} by its body in class {\sl Empty}}$

    false
\end{verbatim}
Or,
\begin{verbatim}
    new NonEmpty(7, new Empty, new Empty).contains(1)

->  $\rewriteby{by replacing {\sl contains} by its body in class {\sl NonEmpty}}$

    if (1 < 7) new Empty contains 1
    else if (1 > 7) new Empty contains 1
    else true

->  $\rewriteby{by rewriting the conditional}$

    new Empty contains 1

->  $\rewriteby{by replacing {\sl contains} by its body in class {\sl Empty}}$

    false .
\end{verbatim}

Dynamic method dispatch is analogous to higher-order function
calls. In both cases, the identity of code to be executed is known
only at run-time. This similarity is not just superficial. Indeed,
Scala represents every function value as an object (see
Section~\ref{sec:funs-are-objects}).


\paragraph{Objects}

In the previous implementation of integer sets, empty sets were
expressed with \verb@new Empty@; so a new object was created every time
an empty set value was required. We could have avoided unnecessary
object creations by defining a value \verb@empty@ once and then using
this value instead of every occurrence of \verb@new Empty@. E.g.
\begin{verbatim}
val empty = new Empty;
\end{verbatim}
One problem with this approach is that a value definition such as the
one above is not a legal top-level definition in Scala; it has to be
part of another class or object. Also, the definition of class
\verb@Empty@ now seems a bit of an overkill -- why define a class of objects, 
if we are only interested in a single object of this class? A more
direct approach is to use an {\em object definition}. Here is
a more streamlined alternative definition of the empty set:
\begin{verbatim}
object empty extends IntSet {
  def contains(x: int): boolean = false;

  def incl(x: int): IntSet = new NonEmpty(x, empty, empty);
}
\end{verbatim}
The syntax of an object definition follows the syntax of a class
definition; it has an optional extends clause as well as an optional
body. As is the case for classes, the extends clause defines inherited
members of the object whereas the body defines overriding or new
members.  However, an object definition defines a single object only;
it is not possible to create other objects with the same structure
using \verb@new@.  Therefore, object definitions also lack constructor
parameters, which might be present in class definitions.

Object definitions can appear anywhere in a Scala program; including
at top-level.  Since there is no fixed execution order of top-level
entities in Scala, one might ask exactly when the object defined by an
object definition is created and initialized. The answer is that the
object is created the first time one of its members is accessed. This
strategy is called {\em lazy evaluation}.

\paragraph{Standard Classes}

Scala is a pure object-oriented language. This means that every value
in Scala can be regarded as an object.  In fact, even primitive types
such as \verb@int@ or \verb@boolean@ are not treated specially. They
are defined as type aliases of Scala classes in module \verb@Predef@:
\begin{verbatim}
type boolean = scala.Boolean;
type int = scala.Int;
type long = scala.Long;
...
\end{verbatim}
For efficiency, the compiler usually represents values of type
\verb@scala.Int@ by 32 bit integers, values of type
\verb@scala.Boolean@ by Java's booleans, etc.  But it converts these
specialized representations to objects when required, for instance
when a primitive \verb@int@ value is passed to a function that with a
parameter of type \verb@Object@.  Hence, the special representation of
primitive values is just an optimization, it does not change the
meaning of a program.

Here is a specification of class \verb@Boolean@.
\begin{verbatim}
package scala;
trait Boolean {
  def && (def x: Boolean)\=: Boolean;
  def || (def x: Boolean)\>: Boolean;
  def !                  \>: Boolean;

  def == (x: Boolean)\>: Boolean
  def != (x: Boolean)\>: Boolean
  def <  (x: Boolean)\>: Boolean
  def >  (x: Boolean)\>: Boolean
  def <= (x: Boolean)\>: Boolean
  def >= (x: Boolean)\>: Boolean
}
\end{verbatim}
Booleans can be defined using only classes and objects, without
reference to a built-in type of booleans or numbers. A possible
implementation of class \verb@Boolean@ is given below.  This is not
the actual implementation in the standard Scala library. For
efficiency reasons the standard implementation is built from built-in
booleans.
\begin{verbatim}
package scala;
trait Boolean {
  def ifThenElse(def thenpart: Boolean, def elsepart: Boolean)

  def && (def x: Boolean)\=: Boolean  =  ifThenElse(x, false);
  def || (def x: Boolean)\>: Boolean  =  ifThenElse(true, x);
  def !                  \>: Boolean  =  ifThenElse(false, true);

  def == (x: Boolean)\>: Boolean  =  ifThenElse(x, x.!);
  def != (x: Boolean)\>: Boolean  =  ifThenElse(x.!, x);
  def <  (x: Boolean)\>: Boolean  =  ifThenElse(false, x);
  def >  (x: Boolean)\>: Boolean  =  ifThenElse(x.!, false);
  def <= (x: Boolean)\>: Boolean  =  ifThenElse(x, true);
  def >= (x: Boolean)\>: Boolean  =  ifThenElse(true, x.!);
}
object True extends Boolean \={ def ifThenElse(def t: Boolean, def e: Boolean) = t }
object False extends Boolean \>{ def ifThenElse(def t: Boolean, def e: Boolean) = e }
\end{verbatim}
Here is a partial specification of class \verb@Int@.

\begin{verbatim}
package scala;
trait Int extends Long { 
  def + (that: Double): Double;
  def + (that: Float): Float;
  def + (that: Long): Long;
  def + (that: Int): Int;             \=/* analogous for -, *, /, %  */

  def << (cnt: Int): Int;   \>/* analogous for >>, >>>  */

  def & (that: Long): Long;
  def & (that: Int): Int;   \>/* analogous for |, ^ */

  def == (that: Double): Boolean;
  def == (that: Float): Boolean;
  def == (that: Long): Boolean;  \> /* analogous for !=, <, >, <=, >=  */
}
\end{verbatim}

Class \verb@Int@ can in principle also be implemented using just
objects and classes, without reference to a built in type of
integers. To see how, we consider a slightly simpler problem, namely
how to implement a type \verb@Nat@ of natural (i.e. non-negative)
numbers. Here is the definition of a trait \verb@Nat@:
\begin{verbatim}
trait Nat {
  def isZero: Boolean;
  def predecessor: Nat;
  def successor: Nat;
  def + (that: Nat): Nat;
  def - (that: Nat): Nat;
}
\end{verbatim}
To implement the operations of class \verb@Nat@, we define a subobject
\verb@Zero@ and a subclass \verb@Succ@ (for successor). Each number
\verb@N@ is represented as \verb@N@ applications of the \verb@Succ@
constructor to \verb@Zero@:
\[
\underbrace{\mbox{\sl new Succ( ... new Succ}}_{\mbox{$N$ times}}\mbox{\sl (Zero) ... )}
\]
The implementation of the \verb@Zero@ object is straightforward:
\begin{verbatim}
object Zero extends Nat {
  def isZero: Boolean = true;
  def predecessor: Nat = error("negative number");
  def successor: Nat = new Succ(Zero);
  def + (that: Nat): Nat = that;
  def - (that: Nat): Nat = if (that.isZero) Zero else error("negative number")
}
\end{verbatim}

The implementation of the predecessor and subtraction functions on
\verb@Zero@ contains a call to the predefined \verb@error@
function. This function aborts the program with the given error
message.

Here is the implementation of the successor class:
\begin{verbatim}
class Succ(x: Nat) extends Nat  {
  def isZero: Boolean = false;
  def predecessor: Nat = x;
  def successor: Nat = new Succ(this);
  def + (that: Nat): Nat = x.+(that.successor);
  def - (that: Nat): Nat = x.-(that.predecessor)
}
\end{verbatim}
Note the implementation of method \verb@successor@. To create the
successor of a number, we need to pass the object itself as an
argument to the \verb@Succ@ constructor.  The object itself is
referenced by the reserved name \verb@this@.   

The implementations of \verb@+@ and \verb@-@ each contain a recursive
call with the constructor argument as receiver. The recursion will
terminate once the receiver is the \verb@Zero@ object (which is
guaranteed to happen eventually from the way numbers are formed).

\exercise Write an implementation \verb@Integer@ of integer numbers
The implementation should support all operations of class \verb@Nat@
while adding two methods
\begin{verbatim}
def isPositive: Boolean
def negate: Integer
\end{verbatim}
The first method should return \verb@true@ if the number is positive. The second method should negate the number.
Do not use any of Scala's standard numeric classes in your
implementation. (Hint: There are two possible ways to implement
\verb@Integer@. One can either make use the existing implementation of
\verb@Nat@, representing an integer as a natural number and a sign.
Or one can generalize the given implementation of \verb@Nat@ to
\verb@Integer@, using the three subclasses \verb@Zero@ for 0, 
\verb@Succ@ for positive numbers and \verb@Pred@ for negative numbers.)



\paragraph{Language Elements Introduced In This Chapter}

\paragraph{Types:}
\begin{verbatim}
Type         \= = ...  |  ident
\end{verbatim}

Types can now be arbitrary identifiers which represent classes.

\paragraph{Expressions:}
\begin{verbatim}
Expr         \= = ...  |  Expr `.' ident  |  new Expr  |  this
\end{verbatim}

An expression can now be an object creation, or
a selection \verb@E.m@ of a member \verb@m@
from an object-valued expression \verb@E@, or it can be the reserved name \verb@this@.

\paragraph{Definitions and Declarations:}
\begin{verbatim}
Def                     \= = \=FunDef  |  ValDef  |  ClassDef  |  TraitDef  |  ObjectDef
ClassDef     \> = \>[abstract] class ident [`(' [Parameters] `)'] 
             \>   \>[extends Expr] [`{' {TemplateDef} `}']
TraitDef     \> = \>trait ident [extends Expr] [`{' {TemplateDef} `}']
ObjectDef    \> = \>object ident [extends Expr] [`{' {ObjectDef} `}']
TemplateDef  \> = \>[Modifier] (Def | Dcl)
ObjectDef    \> = \>[Modifier] Def
Modifier     \> = \>private  |  override
Dcl          \> = \>FunDcl | ValDcl
FunDcl       \> = \>def ident {`(' [Parameters] `)'} `:' Type
ValDcl       \> = \>val ident `:' Type
\end{verbatim}

A definition can now be a class, trait or object definition such as
\begin{verbatim}
class C(params) extends B { defs }
trait T extends B { defs }
object O extends B { defs }
\end{verbatim}
The definitions \verb@defs@ in a class, trait or object may be
preceded by modifiers \verb@private@ or \verb@override@.

Abstract classes and traits may also contain declarations. These
introduce {\em deferred} functions or values with their types, but do
not give an implementation. Deferred members have to be implemented in
subclasses before objects of an abstract class or trait can be created.

\chapter{Case Classes and Pattern Matching}

Say, we want to write an interpreter for arithmetic expressions.  To
keep things simple initially, we restrict ourselves to just numbers
and \verb@+@ operations. Such expressions can be represented as a class hierarchy, with an abstract base class \verb@Expr@ as the root, and two subclasses \verb@Number@ and
\verb@Sum@. Then, an expression \verb@1 + (3 + 7)@ would be represented as
\begin{verbatim}
new Sum(new Number(1), new Sum(new Number(3), new Number(7)))
\end{verbatim}
Now, an evaluator of an expression like this needs to know of what
form it is (either \verb@Sum@ or \verb@Number@) and also needs to
access the components of the expression.  The following
implementation provides all necessary methods.
\begin{verbatim}
abstract class Expr {
  def isNumber: boolean;
  def isSum: boolean;
  def numValue: int;
  def leftOp: Expr;
  def rightOp: Expr;
}
\end{verbatim}
\begin{verbatim}
class Number(n: int) extends Expr {
  def isNumber: boolean = true;
  def isSum: boolean = false;
  def numValue: int = n;
  def leftOp: Expr = error("Number.leftOp");
  def rightOp: Expr = error("Number.rightOp");
}
\end{verbatim}
\begin{verbatim}
class Sum(e1: Expr, e2: Expr) extends Expr {
  def isNumber: boolean = false;
  def isSum: boolean = true;
  def numValue: int = error("Sum.numValue");
  def leftOp: Expr = e1;
  def rightOp: Expr = e2;}
\end{verbatim}
With these classification and access methods, writing an evaluator function is simple:
\begin{verbatim}
def eval(e: Expr): int = {
  if (e.isNumber) e.numValue
  else if (e.isSum) eval(e.leftOp) + eval(e.rightOp)
  else error("unrecognized expression kind")
}
\end{verbatim}
However, defining all these methods in classes \verb@Sum@ and
\verb@Number@ is rather tedious. Furthermore, the problem becomes worse 
when we want to add new forms of expressions. For instance, consider
adding a new expression form
\verb@Prod@ for products. Not only do we have to implement a new class \verb@Prod@, with all previous classification and access methods; we also have to introduce a
new abstract method \verb@isProduct@ in class \verb@Expr@ and
implement that method in subclasses \verb@Number@, \verb@Sum@, and
\verb@Prod@. Having to modify existing code when a system grows is always problematic, since it introduces versioning and maintenance problems. 

The promise of object-oriented programming is that such modifications
should be unnecessary, because they can be avoided by re-using
existing, unmodified code through inheritance. Indeed, a more
object-oriented decomposition of our problem solves the problem.  The
idea is to make the ``high-level'' operation \verb@eval@ a method of
each expression class, instead of implementing it as a function
outside the expression class hierarchy, as we have done
before. Because \verb@eval@ is now a member of all expression nodes,
all classification and access methods become superfluous, and the implementation is simplified considerably:
\begin{verbatim}
abstract class Expr {
  def eval: int;
}
class Number(n: int) extends Expr {
  def eval: int = n;
}
class Sum(e1: Expr, e2: Expr) extends Expr {
  def eval: int = e1.eval + e2.eval;
}
\end{verbatim}
Furthermore, adding a new \verb@Prod@ class does not entail any changes to existing code:
\begin{verbatim}
class Prod(e1: Expr, e2: Expr) extends Expr {
  def eval: int = e1.eval * e2.eval;
}
\end{verbatim}

The conclusion we can draw from this example is that object-oriented
decomposition is the technique of choice for constructing systems that
should be extensible with new types of data. But there is also another
possible way we might want to extend the expression example. We might
want to add new {\em operations} on expressions.  For instance, we might
want to add an operation that pretty-prints an expression tree to standard output.

If we have defined all classification and access methods, such an
operation can easily be written as an external function. Here is an
implementation:
\begin{verbatim}
def print(e: Expr): unit = 
  if (e.isNumber) System.out.print(e.numValue)
  else if (e.isSum) {
    System.out.print("("); 
    print(e.leftOp); 
    System.out.print("+");
    print(e.rightOp);
    System.out.print(")");
  } else error("unrecognized expression kind");
\end{verbatim}
However, if we had opted for an object-oriented decomposition of
expressions, we would need to add a new \verb@print@ method
to each class:
\begin{verbatim}
abstract class Expr {
  def eval: int;
  def print: unit;
}
class Number(n: int) extends Expr {
  def eval: int = n;
  def print: unit = System.out.print(n);
}
class Sum(e1: Expr, e2: Expr) extends Expr {
  def eval: int = e1.eval + e2.eval;
  def print: unit = {
    System.out.print("("); 
    print(e1); 
    System.out.print("+");
    print(e2);
    System.out.print(")");
}
\end{verbatim}
Hence, classical object-oriented decomposition requires modification
of all existing classes when a system is extended with new operations.

As yet another way we might want to extend the interpreter, consider
expression simplification. For instance, we might want to write a
function which rewrites expressions of the form
\verb@a * b + a * c@ to \verb@a * (b + c)@. This operation requires inspection of 
more than a single node of the expression tree at the same
time. Hence, it cannot be implemented by a method in each expression
kind, unless that method can also inspect other nodes. So we are
forced to have classification and access methods in this case. This
seems to bring us back to square one, with all the problems of
verbosity and extensibility.

Taking a closer look, one observers that the only purpose of the
classification and access functions is to {\em reverse} the data
construction process.  They let us determine, first, which sub-class
of an abstract base class was used and, second, what were the
constructor arguments. Since this situation is quite common, Scala has
a way to automate it with case classes. 

\paragraph{Case Classes.}
A {\em case class} is defined like a normal class, except that the definition
is prefixed with the modifier \verb@case@.  For instance, the definitions
\begin{verbatim}
abstract class Expr;
case class Number(n: int) extends Expr;
case class Sum(e1: Expr, e2: Expr) extends Expr;
\end{verbatim}
introduce \verb@Number@ and \verb@Sum@ as case classes.
The \verb@case@ modifier in front of a class definition has the following effects.
\begin{enumerate}
\item Case classes implicitly come with a constructor function, with the same name as the class. In our example, the two functions
\begin{verbatim}
def Number(n: int) = new Number(n);
def Sum(e1: Expr, e2: Expr) = new Sum(e1, e2);
\end{verbatim}
would be added. Hence, one can now construct expression trees a bit more concisely, as in
\begin{verbatim}
Sum(Sum(Number(1), Number(2)), Number(3))
\end{verbatim} 
\item Case classes implicity come with implementations of methods
\verb@toString@, \verb@equals@ and \verb@hashCode@, which override the
methods with the same name in class \verb@Object@. The implementation
of these methods takes in each case the structure of a member of a
case class into account. The \verb@toString@ method represents an
expression tree the way it was constructed. So,
\begin{verbatim}
Sum(Sum(Number(1), Number(2)), Number(3))
\end{verbatim} 
would be converted to exactly that string, whereas he default
implementation in class \verb@Object@ would return a string consisting
of the outermost constructor name \verb@Sum@ and a number.  The
\verb@equals@ methods treats two case members of a case class as equal
if they have been constructed with the same constructor and with
arguments which are themselves pairwise equal. This also affects the
implementation of \verb@==@ and \verb@!=@, which are implemented in
terms of \verb@equals@ in Scala. So,
\begin{verbatim}
Sum(Number(1), Number(2)) == Sum(Number(1), Number(2))
\end{verbatim}
will yield \verb@true@. If \verb@Sum@ or \verb@Number@ were not case
classes, the same expression would be \verb@false@, since the standard
implementation of \verb@equals@ in class \verb@Object@ always treats
objects created by different constructor calls as being different.
The \verb@hashCode@ method follows the same principle as other two
methods. It computes a hash code from the case class constructor name
and the hash codes of the constructor arguments, instead of from the object's
address, which is what the as the default implementation of \verb@hashCode@ does.
\item 
Case classes implicity come with nullary accessor methods which
retrieve the constructor arguments.
In our example, \verb@Number@ would obtain an accessor method
\begin{verbatim}
def n: int
\end{verbatim}
which returns the constructor parameter \verb@n@, whereas \verb@Sum@ would obtain two accessor methods
\begin{verbatim}
def e1: Expr, e2: Expr;
\end{verbatim}
Hence, if for a value \verb@s@ of type \verb@Sum@, say, one can now
write \verb@s.e1@, to access the left operand. However, for a value
\verb@e@ of type \verb@Expr@, the term \verb@e.e1@ would be illegal
since \verb@e1@ is defined in \verb@Sum@; it is not a member of the
base class \verb@Expr@. 
So, how do we determine the constructor and access constructor
arguments for values whose static type is the base class \verb@Expr@?
This is solved by the fourth and final particularity of case classes.
\item 
Case classes allow the constructions of {\em patterns} which refer to
the case class constructor.
\end{enumerate}

\paragraph{Pattern Matching.}

Pattern matching is a generalization of C or Java's \verb@switch@
statement to class hierarchies. Instead of a \verb@switch@ statement,
there is a standard method \verb@match@, which is defined in Scala's
root class \verb@Any@, and therefore is available for all objects.
The \verb@match@ method takes as argument a number of cases. 
For instance, here is an implementation of \verb@eval@ using 
pattern matching.
\begin{verbatim}
def eval(e: Expr): int = e match { 
  case Number(x) => x 
  case Sum(l, r) => eval(l) + eval(r) 
}
\end{verbatim}
In this example, there are two cases. Each case associates a pattern
with an expression. Patterns are matched against the selector
values \verb@e@.  The first pattern in our example,
\verb@Number(n)@, matches all values of the form \verb@Number(v)@, 
where \verb@v@ is an arbitrary value.  In that case, the {\em pattern
variable} \verb@n@ is bound to the value \verb@v@. Similarly, the
pattern \verb@Sum(l, r)@ matches all selector values of form
\verb@Sum(v$_1$, v$_2$)@ and binds the pattern variables \verb@l@ and \verb@r@ 
to \verb@v$_1$@ and \verb@v$_2$@, respectively. 

In general, patterns are built from
\begin{itemize}
\item Case class constructors, e.g. \verb@Number@, \verb@Sum@, whose arguments
      are again patterns,
\item pattern variables, e.g. \verb@n@, \verb@e1@, \verb@e2@,
\item the ``wildcard'' pattern \verb@_@,
\item constants, e.g. \verb@1@, \verb@true@, "abc", \verb@MAXINT@.
\end{itemize}
Pattern variables always start with a lower-case letter, so that they
can be distinguished from constant identifiers, which start with an
upper case letter. The only exceptions to that rule are the reserved
words \verb@null@, \verb@true@, \verb@false@, which are treated as constants.
Each variable name may occur only once in a pattern. For instance,
\verb@Sum(x, x)@ would be illegal as a pattern, since \verb@x@ occurs
twice in it.

\paragraph{Meaning of Pattern Matching.}
A pattern matching expression 
\begin{verbatim}
e.match { case p$_1$ => e$_1$ ... case p$_n$ => e$_n$ }
\end{verbatim}
matches the patterns $p_1 \commadots p_n$ in the order they
are written against the selector value \verb@e@.
\begin{itemize}
\item
A constructor pattern $C(p_1 \commadots p_n)$ matches all values that
are of type \verb@C@ (or a subtype thereof) and that have been constructed with 
\verb@C@-arguments matching patterns $p_1 \commadots p_n$.
\item 
A variable pattern \verb@x@ matches any value and binds the variable
name to that value.  
\item 
The wildcard pattern `\verb@_@' matches any value but does not bind a name to that value. 
\item A constant pattern \verb@C@ matches a value which is
equal (in terms of \verb@==@) to \verb@C@.
\end{itemize}
The pattern matching expression rewrites to the right-hand-side of the
first case whose pattern matches the selector value. References to
pattern variables are replaced by corresponding constructor arguments.
If none of the patterns matches, the pattern matching expression is
aborted with a \verb@MatchError@ exception.

\example Our substitution model of program evaluation extends quite naturally to pattern matching, For instance, here is how \verb@eval@ applied to a simple expression is re-written:
\begin{verbatim}
     \= eval(Sum(Number(1), Number(2)))

->   \> $\mbox{\tab\tab\rm(by rewriting the application)}$

     \> Sum(Number(1), Number(2)) match {
     \>     case Number(n) => n
     \>     case Sum(e1, e2) => eval(e1) + eval(e2)
     \> }

->   \> $\mbox{\tab\tab\rm(by rewriting the pattern match)}$

     \> eval(Number(1)) + eval(Number(2))

->   \> $\mbox{\tab\tab\rm(by rewriting the first application)}$

     \> Number(1) match {
     \>     case Number(n) => n
     \>     case Sum(e1, e2) => eval(e1) + eval(e2)
     \> } + eval(Number(2))

->   \> $\mbox{\tab\tab\rm(by rewriting the pattern match)}$

     \> 1 + eval(Number(2))

->$^*$ \> 1 + 2 -> 3
\end{verbatim}

\paragraph{Pattern Matching and Methods.} In the previous example, we have used pattern
matching in a function which was defined outside the class hierarchy
over which it matches.  Of course, it is also possible to define a
pattern matching function in that class hierarchy itself. For
instance, we could have defined
\verb@eval@ is a method of the base class \verb@Expr@, and still have used pattern matching in its implementation:
\begin{verbatim}
abstract class Expr { 
  def eval: int = this match { 
    case Number(n) => n
    case Sum(e1, e2) => e1.eval + e2.eval 
  } 
}
\end{verbatim}

\exercise
Consider the following three classes representing trees of integers. 
These classes can be seen as an alternative representation of \verb@IntSet@:
\begin{verbatim}
trait IntTree;
case class Empty extends IntTree;
case class Node(elem: int, left: IntTree, right: IntTree) extends IntTree;
\end{verbatim}
Complete the following implementations of function \verb@contains@ and \verb@insert@ for 
\verb@IntTree@'s.
\begin{verbatim} 
def contains(t: IntTree, v: int): boolean = t match { ... 
  ...
}
def insert(t: IntTree, v: int): IntTree = t match { ... 
  ...
}
\end{verbatim}

\subsection*{Tuples}

Sometimes, a function needs to return more than one result. For
instance, take the function \verb@divmod@ which returns the quotient
and rest of two given integer arguments.  Of course, one can
define a class to hold the two results of \verb@divmod@, as in:
\begin{verbatim}
case class TwoInts(first: int, second: int);

def divmod(x: int, y: int): TwoInts = new TwoInts(x / y, x % y)
\end{verbatim}
However, having to define a new class for every possible pair of
result types is very tedious. It should also be unneccessary because
all such classes have exactly the same structure. In Scala, the
repetition can be avoided by defining a {\em generic class}:
\begin{verbatim}
case class Pair[+a, +b](first: a, second: b);

def divmod(x: int, y: int): Pair[int, int] = new Pair[Int, Int](x / y, x % y)
\end{verbatim}
In this example, \verb@[a, b]@ are {\em type parameters} of class
\verb@Pair@. In a \verb@Pair@ type, these parameters are replaced by
concrete types. For instance, \verb@Pair[int, String]@ represents the
type of pairs with \verb@int@ and \verb@String@ elements.

Type arguments can be omitted in constructors, if the correct type can
be inferred from the other constructor arguments or the constructor's
expected result type. In our example, we could have omitted the type
arguments in the body of \verb@divmod@, because they can be deduced
from the two value parameters of type \verb@int@:
\begin{verbatim}
def divmod(x: int, y: int): Pair[int, int] = new Pair(x / y, x % y)
\end{verbatim}
Type parameters are never used in patterns. For instance, here is an
expression in which \verb@divmod@'s result is decomposed:
\begin{verbatim}
divmod(x, y) match {
  case Pair(n, d) => System.out.println("quotient: " + n + ", rest: " + d);
}
\end{verbatim}
The type parameters in class \verb@Pair@ are each prefixed by a
\verb@+@ sign.  This indicates that \verb@Pair@s are {\em
covariant}. That is, if types \verb@T$_1$@ and \verb@T$_2$@ are
subtypes of types \verb@S$_1$@ and \verb@S$_2$@, then
\verb@Pair[T$_1$, T$_2$]@ is a subtype of
\verb@Pair[S$_1$, S$_2$]@. For instance, \verb@Pair[String, int]@ is a
subtype of \verb@Pair[Object, long]@. If the \verb@+@-annotation was
missing, the type constructor would be treated as being
non-variant. That is, pairs with different element types would never
be in a subtype relation. 
Besides, \verb@+@, there is also a prefix
\verb@-@ for contra-variant type constructors. 
The precise rules that
for variance annotations are given in Chapter~\ref{sec:variance}.

The idea of pairs is generalized in Scala to tuples of greater arity.
There is a predefined case class \verb@Tuple$_n$@ for every \verb@n@
from \verb@2@ to \verb@9@ in Scala's standard library. The
definitions all follow the template
\begin{verbatim}
case class Tuple$_n$[+a$_1$, ..., +a$_n$](_1: a$_1$, ..., _n: a$_n$);
\end{verbatim}
Class \verb@Pair@ (as well as class \verb@Triple@) are also
predefined, but not as classes of their own. Instead 
\verb@Pair@ is an alias of \verb@Tuple2@ and \verb@Triple@ is an
alias of \verb@Tuple3@. 

\chapter{Lists}

The list is an important data structure in many Scala programs.  
A list with elements \verb@x$_1$, ..., x$_n$@ is written
\verb@List(x$_1$, ..., x$_n$)@. Examples are:
\begin{verbatim}
val fruit  \= = List("apples", "oranges", "pears");
val nums   \> = List(1, 2, 3, 4);
val diag3  \> = List(List(1, 0, 0), List(0, 1, 0), List(0, 0, 1));
val empty  \> = List();
\end{verbatim}
Lists are similar to arrays in languages such as C or Java, but there
are also three important differences. First, lists are immutable. That
is, elements of a list can not be changed by assignment. Second, 
lists have a recursive structure, whereas arrays are flat. Third,
lists support a much richer set of operations than arrays usually do.

\paragraph{The list type.}
Like arrays, lists are {\em homogeneous}. That is, the elements of a
list all have the same type.  The type of a list with elements of type
\verb@T@ is written \verb@List[T]@. (Compare to \verb@[]T@ for the
type of arrays of type \verb@T@ in C or Java.). Therefore, the 
definitions of list values above can be annotated with types as
follows.
\begin{verbatim}
val fruit \= : List[String]       \= = List("apples", "oranges", "pears");
val nums  \> : List[int]          \> = List(1, 2, 3, 4);
val diag3 \> : List[List[int]]    \> = List(List(1, 0, 0), List(0, 1, 0), List(0, 0, 1));
val empty \> : List[int]          \> = List();
\end{verbatim}

\paragraph{List constructors.}
All lists are built from two more fundamental constructors, \verb@Nil@
and \verb@::@ (pronounced ``cons''). \verb@Nil@ represents an empty
list. The infix operator \verb@::@ expresses list extension. That is,
\verb@x :: xs@ represents a list whose first element is \verb@x@,
which is followed by (the elements of) list \verb@xs@.  Hence, the
list values above could also have been defined as follows (in fact
their previous definition is simply syntactic sugar for the definitions below).
\begin{verbatim}
val fruit  \= =  "apples" :: ("oranges" :: ("pears" :: Nil));
val nums   \> =  1 :: (2 :: (3 :: (4 :: Nil)));
val diag3  \> = \= (1 :: (0 :: (0 :: Nil))) ::
           \>   \> (0 :: (1 :: (0 :: Nil))) ::
           \>   \> (0 :: (0 :: (1 :: Nil))) :: Nil;
val empty  \> =  Nil;
\end{verbatim}
The `\verb@::@' operation associates to the right: \verb@A :: B :: C@ is
interpreted as \verb@A :: (B :: C)@.  Therefore, we can drop the
parentheses in the definitions above. For instance, we can write
shorter
\begin{verbatim}
val nums  =  1 :: 2 :: 3 :: 4 :: Nil;
\end{verbatim}

\paragraph{Basic operations on lists.}
All operations on lists can be expressed in terms of the following three:

\begin{tabular}{ll}
\verb@head@  &  returns the first element of a list,\\
\verb@tail@  &  returns the list consisting of all elements except the\\
first element,
\verb@isEmpty@ & returns \verb@true@ iff the list is empty
\end{tabular}

These operations are defined as methods of list objects. So we invoke
them by selecting from the list that's operated on. Examples:
\begin{verbatim}
empty.isEmpty     \= = true
fruit.isEmpty   \> = false
fruit.head      \> = "apples"
fruit.tail.head \> = "oranges"
diag3.head      \> = List(1, 0, 0)
\end{verbatim}
Both \verb@head@ and \verb@tail@ are only defined for non-empty lists.
When selected from an empty list, they cause an error instead.

As an example of how lists can be processed, consider sorting the
elements of a list of numbers into ascending order. One simple way to
do so is {\em insertion sort}, which works as follows: To sort a
non-empty list with first element \verb@x@ and rest \verb@xs@, sort
the remainder \verb@xs@ and insert the element \verb@x@ at the right
position in the result. Sorting an empty list will of course yield the
empty list. Expressed as Scala code:
\begin{verbatim}
def isort(xs: List[int]): List[int] =
  if (xs.isEmpty) Nil
  else insert(xs.head, isort(xs.tail))
\end{verbatim}

\exercise Provide an implementation of the missing function
\verb@insert@.

\paragraph{List patterns.} In fact, \verb@::@ is
defined defined as a case class in Scala's standard library. Hence, it
is possible to decompose lists by pattern matching, using patterns
composed from the \verb@Nil@ and \verb@::@ constructors. For instance,
\verb@isort@ can be written alternatively as follows.
\begin{verbatim}
def isort(xs: List[int]): List[int] = xs match {
  case List() => List()
  case x :: xs1 => insert(x, isort(xs1))
}
\end{verbatim}
where
\begin{verbatim}
def insert(x: int, xs: List[int]): List[int] = xs match {
  case List() => List(x)
  case y :: ys => if (x <= y) x :: xs else y :: insert(x, ys)
}
\end{verbatim}

\paragraph{Polymorphic functions.} Consider the problem of writing a
  function \verb@concat@, which takes a list of element lists as
  arguments. The result of \verb@concat@ should be the concatenation of all
  element lists into a single list. 

When trying to define such a function, we observe that we need to give
a type for the list elements:
\begin{verbatim}
def concat(xss: List[List[ ?? ]]): List[ ?? ] = ...
\end{verbatim}
Clearly, one could replace \verb@??@ by \verb@int@, say, to obtain a
function \verb@concat@ that works on lists of lists of integers. But then the
same function could not be applied to other kinds of lists. This is a
pity, since clearly the same algorithm of list concatenation can work
for lists of any element type. Parameterization lets us generalize
from a specific instance of a problem to a more general one. So far,
we have used parameterization only for values, but it is available
also for types. To arrive at a general version of \verb@concat@, we 
equip it with a type parameter.
\begin{verbatim}
def concat[a](xs: List[List[a]]): List[a] = xs match {
  case List() => xs
  case List() :: yss => concat[a](yss)
  case (x :: xs) :: yss => x :: concat[a](xs :: yss)
}
\end{verbatim}
Type parameters are arbitrary names; they are enclosed in brackets
instead of parentheses, so that they can be easily distinguished from
value parameters. Functions like \verb@concat@ that take type
parameters are called {\em polymorphic}. The term comes from the
Greek, where it means ``having many forms''.

To apply \verb@concat@, we pass type parameters as well as value
parameters to it. For instance,
\begin{verbatim}
val diag3 = List(List(1, 0, 0), List(0, 1, 0), List(0, 0, 1));
concat[int](diag3)
\end{verbatim}
yields \verb@List(1, 0, 0, 0, 1, 0, 0, 0, 1)@.

\paragraph{Local Type Inference.}
Passing type parameters such as \verb@[int]@ all the time can become
tedious in applications where polymorphic functions are used a
lot. Quite often, the information in a type parameter is redundant,
because the correct parameter type can also be determined by
inspecting the function's value parameters or expected result type.
Taking \verb@concat[int](diag3)@ function as an example, we know that
its value parameter is of type \verb@List[List[int]]@, so we can
deduce that the type parameter must be \verb@int@. Scala has a
fairly powerful type inferencer which allows one to omit type
parameters to polymorphic functions and constructors in situations
like these.  In the example above, the \verb@int@ type parameter would
have been inferred if it was not given explicitly. In fact, the same
principle applies in the definition of the value \verb@diag3@. 
Here, type parameters have been inferred for the four calls of
\verb@List@. 

\paragraph{Definition of class \verb@List@}

Lists are not built in in Scala; they are defined by an abstract class
\verb@List@, which comes with two subclasses for \verb@::@ and \verb@Nil@.
In the following we present a tour through class \verb@List@.
\begin{verbatim}
package scala;
abstract class List[+a] {
\end{verbatim}
\verb@List@ is an abstract class, so one cannot define elements by
calling the empty \verb@List@ constructor (e.g. by
\verb@new List).  The class has a type parameter \verb@a@. It is
co-variant in this parameter, which means that
\verb@List[S] <: List[T]@ for all types \verb@S@ and \verb@T@ such that
\verb@S <: T@.  The class is situated in the package
\verb@scala@. This is a package containing the most important standard
classes of Scala. \verb@List@ defines a number of methods, which are
explained in the following.

First, there are the three basic functions \verb@isEmpty@, 
\verb@head@, \verb@tail@. Their implementation in terms of pattern
matching is straightforward:
\begin{verbatim}
def isEmpty: boolean = match {
  case Nil => true
  case x :: xs => false 
}   
def head: a = match { 
  case Nil => error("Nil.head") 
  case x :: xs => x 
}
def tail: List[a] = match { 
  case Nil => error("Nil.tail") 
  case x :: xs => x 
}
\end{verbatim}

The next function computes the length of a list.
\begin{verbatim}
def length = match {
  case Nil => 0
  case x :: xs => 1 + xs.length
}
\end{verbatim}

\exercise Design a tail-recursive version of \verb@length@.

The next two functions are the complements of \verb@head@ and
\verb@tail@.
\begin{verbatim}
def last: a;
def init: List[a];
\end{verbatim}
\verb@xs.last@ returns the last element of list \verb@xs@, whereas
\verb@xs.init@ returns all elements of \verb@xs@ except the last.
Both functions have to traverse the entire list, and are thus less
efficient than their \verb@head@ and \verb@tail@ analogues.
Here is the implementation of \verb@last@.
\begin{verbatim}
def last: a = match {
  case Nil      \==> error("Nil.last")
  case x :: Nil \>=> x
  case x :: xs  \>=> xs.last
}
\end{verbatim}
The implementation of \verb@init@ is analogous.

The next three functions return a prefix of the list, or a suffix, or
both.
\begin{verbatim}
def take(n: int): List[a] = 
  if (n == 0 || isEmpty) Nil else head :: tail.take(n-1);

def drop(n: int): List[a] = 
  if (n == 0 || isEmpty) this else tail.drop(n-1);

def split(n: int): Pair[List[a], List[a]] = 
  if (n == 0 || isEmpty) Pair(Nil, this)
  else tail.split(n - 1) match { case Pair(xs, ys) => (head :: xs, ys) }
\end{verbatim}
\verb@(xs take n)@ returns the first \verb@n@ elements of list
\verb@xs@, or the whole list, if its length is smaller than \verb@n@.
\verb@(xs drop n)@ returns all elements of \verb@xs@ except the
\verb@n@ first ones. Finally, \verb@(xs split n)@ returns a pair
consisting of the lists resulting from \verb@xs take n@ and
\verb@xs drop n@, but the call is more efficient than performing the
two calls separately. 

The next function returns an element at a given index in a list.
It is thus analogous to array subscripting. Indices start at 0.
\begin{verbatim}   
def at(n: int): a = drop(n).head;
\end{verbatim}

With \verb@take@ and \verb@drop@, we can extract sublists consisting
of consecutive elements of the original list.  To extract the sublist
$xs_m \commadots xs_{n-1}$ of a list \verb@xs@, use:

\begin{verbatim}
xs.drop(m).take(n - m)
\end{verbatim}

The next function combines two lists into a list of pairs.
Given two lists 
\begin{verbatim}
xs = List(x$_1$, ..., x$_n$)   $\mbox{\rm, and}$
ys = List(y$_1$, ..., y$_n$)   ,
\end{verbatim}
\verb@xs zip ys@ constructs the list \verb@Pair(x$_1$, y$_1$), ..., Pair(x$_n$, y$_n$)@.
If the two lists have different lengths, the longer one of the two is
truncated. Here is the definition of \verb@zip@ -- note that it is a
polymorphic method.
\begin{verbatim}
def zip[b](that: List[b]): List[Pair[a,b]] = 
  if (this.isEmpty || that.isEmpty) Nil
  else Pair(this.head, that.head) :: (this.tail zip that.tail);
\end{verbatim}

Like any infix operator, \verb@::@
is also implemented as a method of an object. In this case, the object
is the list that is extended. This is possible, because operators
ending with a `\verb@:@' character are treated specially in Scala.  
All such operators are treated as methods of their right operand. E.g.,
\begin{verbatim}
    x :: y = y.::(x)       \=$\mbox{\rm whereas}$       x + y = x.+(y)                  
\end{verbatim}
Note, however, that operands of a binary operation are in each case
evaluated from left to right.  So, if \verb@D@ and \verb@E@ are
expressions with possible side-effects, \verb@D :: E@ is translated to
\verb@{val x = D; E.::(x)}@ in order to maintain the left-to-right
order of operand evaluation.

Another difference between operators ending in a `\verb@:@' and other
operators concerns their associativity.  Operators ending in
`\verb@:@' are right-associative, whereas other operators are
left-associative.  E.g.,
\begin{verbatim}
    x :: y :: z = x :: (y :: z)      \=$\mbox{\rm whereas}$       x + y + z = (x + y) + z
\end{verbatim}
The definition of \verb@::@ as a method in
class \verb@List@ is as follows:
\begin{verbatim}
def ::[b >: a](x: b): List[b] = new scala.::(x, this);
\end{verbatim}
Note that \verb@::@ is defined for all elements \verb@x@ of type
\verb@B@ and lists of type \verb@List[A]@ such that the type \verb@B@
of \verb@x@ is a supertype of the list's element type \verb@A@. The result
is in this case a list of \verb@B@'s. This
is expressed by the type parameter \verb@b@ with lower bound \verb@a@
in the signature of \verb@::@. 

An operation similar to \verb@::@ is list concatenation, written
`\verb@:::@'. The result of \verb@(xs ::: ys)@ is a list consisting of
all elements of \verb@xs@, followed by all elements of \verb@ys@.
Because it ends in a colon, \verb@:::@ is right-associative and is
considered as a method of its right-hand operand. Therefore,
\begin{verbatim}
xs ::: ys ::: zs  \= =   xs ::: (ys ::: zs)
                  \> =   zs.:::(ys).:::(xs)
\end{verbatim}
Here is the implementation of the \verb@:::@ method:
\begin{verbatim}
  def :::[b >: a](prefix: List[b]): List[b] = prefix match {
    case Nil => this
    case p :: ps => this.:::(ps).::(p)
  }
\end{verbatim}

\paragraph{Example: Reverse.} As another example of how to program with
lists consider a list reversal. There is a method \verb@reverse@ in
\verb@List@ to that effect, but let's implement it as a function
outside the class. Here is a possible implementation of
\verb@reverse@:
\begin{verbatim}
def reverse[a](xs: List[a]): List[a] = xs match {
  case List() => List()
  case x :: xs => reverse(xs) ::: List(x)
}
\end{verbatim}
The implementation is quite simple. However, it is not very efficient.
Indeed, one concatenation is executed for every element in the
list. List concatenation takes time proportional to the length
of its first operand. Therefore, the complexity of \verb@reverse(xs)@ is 
\[
n + (n - 1) + ... + 1 = n(n+1)/2
\]
where $n$ is the length of \verb@xs@. Can \verb@reverse@ be
implemented more efficiently? We will see later that there is exists
another implementation which has only linear complexity.

\paragraph{Example: Merge sort.}
The insertion sort presented earlier in this chapter is simple to
formulate, but also not very efficient. It's average complexity is
proportional to the square of the length of the input list. We now
design a program to sort the elements of a list which is more
efficient than insertion sort. A good algorithm for this is {\em merge
sort}, which works as follows.

First, if the list has zero or one elements, it is already sorted, so
one returns the list unchanged. Longer lists are split into two
sub-lists, each containing about half the elements of the original
list. Each sub-list is sorted by a recursive call to the sort
function, and the resulting two sorted lists are then combined in a
merge operation.

For a general implementation of merge sort, we still have to specify
the type of list elements to be sorted, as well as the function to be
used for the comparison of elements. We obtain a function of maximal
generality by passing these two items as parameters. This leads to the
following implementation.
\begin{verbatim}
def msort[a](less: (a, a) => boolean)(xs: List[a]): List[a] = {
  val n = xs.length/2;
  if (n == 0) xs
  else {
    def merge(xs1: List[a], xs2: List[a]): List[a] = 
      if (xs1.isEmpty) xs2
      else if (xs2.isEmpty) xs1
      else if (less(xs1.head, xs2.head)) xs1.head :: merge(xs1.tail, xs2)
      else xs2.head :: merge(xs1, xs2.tail);

    merge(msort(less)(xs take n), msort(less)(xs drop n))
  }
}
\end{verbatim}
The complexity of \verb@msort@ is $O(N;log(N))$, where $N$ is the
length of the input list. To see why, note that splitting a list in
two and merging two sorted lists each take time proportional to the
length of the argument list(s). Each recursive call of \verb@msort@
halves the number of elements in its input, so there are $O(log(N))$
consecutive recursive calls until the base case of lists of length 1
is reached.  However, for longer lists each call spawns off two
further calls. Adding everything up we obtain that at each of the
$O(log(N))$ call levels, every element of the original lists takes
part in one split operation and in one merge operation. Hence, every
call level has a total cost proportional to $O(N)$. Since there are
$O(log(N))$ call levels, we obtain an overall cost of
$O(N;log(N))$. That cost does not depend on the initial distribution
of elements in the list, so the worst case cost is the same as the
average case cost. This makes merge sort an attractive algorithm for
sorting lists.

Here is an example how \verb@msort@ is used.
\begin{verbatim}
def iless(x: int, y: int) = x < y
msort(iless)(List(5, 7, 1, 3))
\end{verbatim}
The definition of \verb@msort@ is curried, to make it easy to specialize it with particular
comparison functions. For instance,
\begin{verbatim}

val intSort = msort(iless)
val reverseSort = msort(x: int, y: int => x > y)
\end{verbatim}

\section*{Higher-Order Methods}

\chapter{Computing with Streams}

The previous chapters have introduced variables, assignment and
stateful objects.  We have seen how real-world objects that change
with time can be modelled by changing the state of variables in a
computation.  Time changes in the real world thus are modelled by time
changes in program execution. Of course, such time changes are usually
stretched out or compressed, but their relative order is the same.
This seems quite natural, but there is a also price to pay: Our simple
and powerful substitution model for functional computation is no
longer applicable once we introduce variables and assignment.

Is there another way? Can we model state change in the real world
using only immutable functions? Taking mathematics as a guide, the
answer is clearly yes: A time-changing quantity is simply modelled by
a function \verb@f(t)@ with a time parameter \verb@t@. The same can be
done in computation. Instead of overwriting a variable with successive
values, we represent all these values as successive elements in a
list. So, a mutabel variable \verb@var x: T@ gets replaced by an
immutable value \verb@val x: List[T]@. In a sense, we trade space for
time -- the different values of the variable now all exit concurrently
as different elements of the list.  One advantage of the list-based
view is that we can ``time-travel'', i.e. view several successive
values of the variable at the same time. Another advantage is that we
can make use of the powerful library of list processing functions,
which often simplifies computation. For instance, consider the way
imperative way to compute the sum of all prime numbers in an interval:
\begin{verbatim}
def sumPrimes(start: int, end: int): int = {
  var i = start;
  var acc = 0;
  while (i < end) {
    if (isPrime(i)) acc = acc + i;
    i = i + 1;
  }
  acc
}
\end{verbatim}
Note that the variable \verb@i@ ``steps through'' all values of the interval
\verb@[start .. end-1]@.
%\es\bs
A more functional way is to represent the list of values of variable \verb@i@ directly as \verb@range(start, end)@. Then the function can be rewritten as follows.
\begin{verbatim}
def sumPrimes(start: int, end: int) =
  sum(range(start, end) filter isPrime);
\end{verbatim}

No contest which program is shorter and clearer!  However, the
functional program is also considerably less efficient since it
constructs a list of all numbers in the interval, and then another one
for the prime numbers. Even worse from an efficiency point of view is
the following example:

To find the second prime number between \verb@1000@ and \verb@10000@:
\begin{verbatim}
  range(1000, 10000) filter isPrime at 1
\end{verbatim}
Here, the list of all numbers between \verb@1000@ and \verb@10000@ is
constructed.  But most of that list is never inspected!

However, we can obtain efficient execution for examples like these by
a trick:
\begin{quote}
%\red
 Avoid computing the tail of a sequence unless that tail is actually
     necessary for the computation.
\end{quote}
We define a new class for such sequences, which is called \verb@Stream@.

Streams are created using the constant \verb@empty@ and the constructor \verb@cons@,
which are both defined in module \verb@scala.Stream@. For instance, the following
expression constructs a stream with elements \verb@1@ and \verb@2@:
\begin{verbatim}
Stream.cons(1, Stream.cons(2, Stream.empty))
\end{verbatim}
As another example, here is the analogue of \verb@List.range@,
but returning a stream instead of a list:
\begin{verbatim}
def range(start: Int, end: Int): Stream[Int] = 
  if (start >= end) Stream.empty
  else Stream.cons(start, range(start + 1, end));
\end{verbatim}
(This function is also defined as given above in module
\verb@Stream@).  Even though \verb@Stream.range@ and \verb@List.range@
look similar, their execution behavior is completely different: 

\verb@Stream.range@ immediately returns with a \verb@Stream@ object
whose first element is \verb@start@.  All other elements are computed
only when they are \emph{demanded} by calling the \verb@tail@ method
(which might be never at all).  

Streams are accessed just as lists. as for lists, the basic access
methods are \verb@isEmpty@, \verb@head@ and \verb@tail@. For instance,
we can print all elements of a stream as follows.
\begin{verbatim}
def print(xs: Stream[a]): unit = 
  if (!xs.isEmpty) { System.out.println(xs.head); print(xs.tail) }
\end{verbatim}
Streams also support almost all other methods defined on lists (see
below for where their methods sets differ). For instance, we can find
the second prime number between \verb@1000@ and \verb@10000@ by applying methods
\verb@filter@ and \verb@at@ on an interval stream:
\begin{verbatim}
  Stream.range(1000, 10000) filter isPrime at 1
\end{verbatim}
The difference to the previous list-based implementation is that now
we do not needlessly construct and test for primality any numbers
beyond 3.

\paragraph{Consing and appending streams.} Two methods in class \verb@List@
which are not supported by class \verb@Stream@ are \verb@::@ and
\verb@:::@.  The reason is that these methods are dispatched on their
right-hand side argument, which means that this argument needs to be
evaluated before the method is called. For instance, in the case of
\verb@x :: xs@ on lists, the tail \verb@xs@ needs to be evaluated
before \verb@::@ can be called and the new list can be constructed.
This does not work for streams, where we require that the tail of a
stream should not be evaluated until it is demanded by a \verb@tail@ operation.
The argument why list-append \verb@:::@ cannot be adapted to streams is analogous.

Intstead of \verb@x :: xs@, one uses \verb@Stream.cons(x, xs)@ for
constructing a stream with first element \verb@x@ and (unevaluated)
rest \verb@xs@.  Instead of \verb@xs ::: ys@, one uses the operation
\verb@xs append ys@.  

%\redtext
{Is there another way?}



\bibliography{examples}
\end{document}


 
\paragrph{Higher Order Functions
\bsh{Patterns of Computation over Lists}

\bi
\item The examples show that functions over lists often have similar
      structures
\item We can identify several patterns of computation like
      \bi
      \item Transform every element of a list in some way.
      \item Extract from a list all elements satisfying a criterion.
      \item Combine the elements of a list using some operator.
      \ei
\item Functional programming languages enable programmers to write
      general functions which implement patterns like this
\item These functions are \redtext{\em higher-order functions} which get
      a transformation or an operator as one argument
\ei
\es

Pairs, and tuples or greater arity are useful enough to 

 



\chapter{Generic Types and Methods}

Classes in Scala can have type parameters. We demonstrate the use of
type parameters with iterators as an example. An iterator is an object
which traverses a sequence of values, using two abstract methods.
\begin{verbatim}
abstract class Iterator[a] {
  def hasNext: boolean;
  def next: a;
\end{verbatim}
Method \verb@next@ returns successive elements.  Method \verb@hasNext@
indicates whether there are still more elements to be returned by
\verb@next@. The type of the elements returned by an iterator is
arbitrary. We express this by giving the class \verb@Iterator@ the
type parameter \verb@a@. Type parameters are written in square
brackets, in contrast to normal value parameters, which are written in
parentheses.  Iterators also support other methods, which are
explained later.

Here's an iterator which traverses an interval of integer values.
\begin{verbatim}
class RangeIterator(start: int, end: int) extends Iterator[int] {
  private var current = 0;
  def hasNext = current < end;
  def next = {
    val r = current;
    if (current < end) current = current + 1
    else error("end of iterator");
    r
  }
}
\end{verbatim}
The superclass of \verb@RangeIterator@ is \verb@Iterator[int]@,
i.e. an iterator returning integer numbers. 

Note that, unlike the classes we have seen so far,
\verb@RangeIterator@ has internal state


Here is a function that takes an iterator of arbitrary element type
\verb@a@ and a procedure that maps \verb@a@-values to the trivial type \verb@unit@.
It applies the given function to every value returned by the iterator.
\begin{verbatim}
  def forall[a](i: Iterator[a])(f: a => boolean): boolean =
    !hasNext || { val x = next; f(x) && forall(i, f) }
\end{verbatim}
\verb@forEach@ can work with any type of iterator, 
since the iterator's element type is passed as a type parameter \verb@a@.
Functions that take type parameters are called {\em polymorphic}. The term
comes from Greek, where it means ``having many forms''.

Finally, here is an application which uses \verb@RangeIterator@ and
\verb@foreach@ to test whether a given number is prime, i.e. whether it can be 
divided only by 1 and itself.
\begin{verbatim}
def isPrime(x: int) =
  forall[int](new RangeIterator(2, n)) { x => x % n != 0 }
\end{verbatim}
As always, the actual parameters of \verb@forEach@ correspond to its
formal parameters.  First comes the type parameter \verb@int@, which
determines the element type of the iterator which is passed next.

\paragraph{Local Type Inference.}
Passing type parameters such as \verb@[int]@ all the time can become
tedious in applications where polymorphic functions are used a
lot. Quite often, the information in a type parameter is redundant,
because the correct parameter type can also be determined by
inspecting the function's value parameters or expected result type.
Taking the \verb@isPrime@ function as an example, we know that its
first value parameter is of type \verb@Iterator[int]@, so we can
determine the type parameter \verb@int@ from it. Scala contains a
fairly powerful local type inferencer which allows one to omit type
parameters to polymorphic functions and constructors in situations
like these.  In the example above, the \verb@int@ type parameter would
have been inferred  if it was not given explicitly.

Here is another
application which prints all prime numbers between 1 and 10000. 
\begin{verbatim}
forall(new RangeIterator(1, 10001)){ x => if (isPrime(x)) System.out.println(x) }
\end{verbatim}
This time, the type parameter for \verb@forEach@ was omitted (and was
inferred to be \verb@int@).

Method \verb@append@ constructs an iterator which resumes with the
given iterator \verb@it@ after the current iterator has finished.
\begin{verbatim}
  def append(that: Iterator[a]): Iterator[a] = new Iterator[a] {
    def hasNext = outer.hasNext || that.hasNext;
    def next = if (outer.hasNext) outer.next else that.next;
  }
\end{verbatim}
The terms \verb@outer.next@ and \verb@outer.hasNext@ in the definition
of \verb@append@ call the corresponding methods as they are defined in
the enclosing \verb@Iterator@ class. Generally, an
\verb@outer@ prefix in a selection indicates an identifier that is
visible immediately outside the current class or template. If the
\verb@outer@ prefix would have been missing,
\verb@hasNext@ and \verb@next@ would have
called recursively the methods being defined in the iterator
constructed by \verb@append@, which is not what we want.

Method \verb@filter@ constructs an iterator which returns all elements
of the original iterator that satisfy a criterion \verb@p@.
\begin{verbatim}
  def filter(p: a => boolean) = new Iterator[a] {
    private class Cell[T](elem_: T) { def elem = elem_; }
    private var head: Cell[a] = null;
    private var isAhead = false;
    def hasNext: boolean =
      if (isAhead) true
      else if (outer.hasNext) {
        head = Cell(outer.next); isAhead = p(head.elem); hasNext }
      else false;
    def next: a =
      if (hasNext) { isAhead = false; head.elem }
      else error("next on empty iterator");
  }
\end{verbatim}
Method \verb@map@ constructs an iterator which returns all elements of
the original iterator transformed by a given function \verb@f@.
\begin{verbatim}
  def map[b](f: a => b) = new Iterator[b] {
    def hasNext: boolean = outer.hasNext;
    def next: b = f(outer.next);
  }
\end{verbatim}
The return type of the transformation function \verb@f@ is
arbitrary. This is expressed by a type parameter \verb@b@ in the
signature of method \verb@map@, which represents the return type.
We also say, \verb@map@ is a {\em polymorphic} function.

Method \verb@flatMap@ is like method \verb@map@, except that the
transformation function \verb@f@ now returns an iterator.
The result of \verb@flatMap@ is the iterator resulting from appending
together all iterators returned from successive calls of \verb@f@.
\begin{verbatim}
    private var cur: Iterator[b] = new EmptyIterator[b];
    def hasNext: boolean =
      if (cur.hasNext) true
      else if (outer.hasNext) { cur = f(outer.next); hasNext }
      else false;
    def next: b =
      if (cur.hasNext) cur.next
      else if (outer.hasNext) { cur = f(outer.next); next }
      else error("next on empty iterator");
  }
\end{verbatim}
Finally, method \verb@zip@ takes another iterator and
returns an iterator consisting of pairs of corresponding elements
returned by the two iterators.
\begin{verbatim}
  def zip[b](that: Iterator[b]) = new Iterator[(a, b)] {
    def hasNext = outer.hasNext && that.hasNext;
    def next = (outer.next, that.next);
  }
} //end iterator;
\end{verbatim}
Concrete iterators need to provide implementations for the two
abstract methods \verb@next@ and \verb@hasNext@ in class
\verb@Iterator@. The simplest iterator is \verb@EmptyIterator@
which always returns an empty sequence:
\begin{verbatim}
class EmptyIterator[a] extends Iterator[a] {
  def hasNext = false;
  def next: a = error("next on empty iterator");
}
\end{verbatim}
A more interesting iterator enumerates all elements of an array.
This iterator is formulated here as a polymorphic function. It could
have also been written as a class, like \verb@EmptyIterator@. The
difference between the two formulation is that classes also define new
types, whereas functions do not.
\begin{verbatim}
def arrayIterator[a](xs: Array[a]) = new Iterator[a] {
  private var i = 0;
  def hasNext: boolean =
    i < xs.length;
  def next: a =
    if (i < xs.length) { val x = xs(i) ; i = i + 1 ; x }
    else error("next on empty iterator");
}
\end{verbatim}
Another iterator enumerates an integer interval:
\begin{verbatim}
def range(lo: int, hi: int) = new Iterator[int] {
  private var i = lo;
  def hasNext: boolean =
    i <= hi;
  def next: int =
    if (i <= hi) { i = i + 1 ; i - 1 }
    else error("next on empty iterator");
}
\end{verbatim}
%In fact, enumerating integer intervals is so common that it is
%supported by a method
%\begin{verbatim}
%def to(hi: int): Iterator[int]
%\end{verbatim}
%in class \verb@int@. Hence, one could also write \verb@l to h@ instead of
%\verb@range(l, h)@.
All iterators seen so far terminate eventually. It is also possible to
define iterators that go on forever. For instance, the following
iterator returns successive integers from some start
value\footnote{Due to the finite representation of type \prog{int},
numbers will wrap around at $2^31$.}.
\begin{verbatim}
def from(start: int) = new Iterator[int] {
  private var last = start - 1;
  def hasNext = true;
  def next = { last = last + 1; last }
}
\end{verbatim}
Here are two examples how iterators are used. First, to print all
elements of an array \verb@xs: Array[int]@, one can write:
\begin{verbatim}
  arrayIterator[int](xs) foreach (x => System.out.println(x))
\end{verbatim}
Here, \verb@[int]@ is a type argument clause, which matches the type
parameter clause \verb@[a]@ of function \verb@arrayIterator@. It
substitutes the formal argument \verb@int@ for the formal argument
\verb@a@ in the type of the method that follows. Hence,
\verb@arrayIterator[a]@ is a function that takes an \verb@Array[int]@
and that returns an \verb@Iterator[int]@.

In this example, the formal type argument \verb@int@ is redundant
since it could also have been inferred from the value \verb@xs@, which
is, after all, an array of \verb@int@. The Scala compiler contains a
fairly powerful type inferencer which infers type arguments for
methods and constructors from the types of value arguments and the
expected return type. In our example, the \verb@[int]@ clause can be
inferred, so that one can abbreviate to:
\begin{verbatim}
  arrayIterator(xs) foreach (x => System.out.println(x))
\end{verbatim}
%As a second example, consider the problem of finding the indices of
%all the elements in an array of \verb@double@s greater than some
%\verb@limit@. The indices should be returned as an iterator.
%This is achieved by the following expression.
%\begin{verbatim}
%arrayIterator(xs)
%  .zip(from(0))
%  .filter(x, i => x > limit)
%  .map(x, i => i)
%\end{verbatim}
%The first line in this expression iterates through all array elements,
%the second lines pairs elements with their indices, the third line
%selects all value/index pairs where the value is greater than
%\verb@limit@, and the fourth line returns the index part of all
%selected pairs.

%Note that we have omitted the type arguments for the calls of
%\verb@arrayIterator@, \verb@zip@ and \verb@map@. These are all
%implicitly inserted by the type inferencer.



\es
\paragraph{Abstract Methods.}
Classes can also omit some of the definitions of their members.  As an
example, consider the following class \verb@Ord@ which provides the
comparison operators \verb@<, >, <=, >=@.
%\begin{verbatim}
%abstract class Ord {
%  abstract def <(that: this);
%  def <=(that: this)  =  this < that || this == that;
%  def >(that: this)  =  that < this;
%  def >=(that: this)  =  that <= this;
%}
%\end{verbatim}
\begin{verbatim}
abstract class Ord {
  def <(that: this): boolean;
  def <=(that: this)  =  this < that || this == that;
  def >(that: this)  =  that < this;
  def >=(that: this)  =  that <= this;
}
\end{verbatim}
Since we want to leave open which objects are compared, we are unable
to give an implementation for the \verb@<@ method. However, once
\verb@<@ is given, we can define the other three comparison operators
in terms of \verb@<@ and the equality test \verb@==@ (which is defined
in class \verb@Object@).  This is expressed by having in \verb@Ord@ an
{\em abstract} method \verb@<@ to which the implementations of the
other methods refer.

\paragraph{Self References.} The name \verb@this@ refers in this class
to the current object. The type of \verb@this@ is also called
\verb@this@ (generally, every name in Scala can have a definition as a
term and another one as a type).  When used as a type, \verb@this@
refers to the type of the current object. This type is always
compatible with the class being defined (\verb@Ord@ in this case).

\paragraph{Mixin Composition.}
We can now define a class of \verb@Rational@ numbers that
support comparison operators.
\begin{verbatim}
final class OrderedRational(n: int, d: int)
 extends Rational(n, d) with Ord {
  override def ==(that: OrderedRational) =
    numer == that.numer && denom == that.denom;
  def <(that: OrderedRational): boolean =
    numer * that.denom < that.numer * denom;
}
\end{verbatim}
Class \verb@OrderedRational@ redefines method \verb@==@, which is
defined as reference equality in class \verb@Object@. It also
implements the abstract method \verb@<@ from class \verb@Ord@.  In
addition, it inherits all members of class \verb@Rational@ and all
non-abstract members of class \verb@Ord@. The implementations of
\verb@==@ and \verb@<@ replace the definition of \verb@==@ in class
\verb@Object@ and the abstract declaration of \verb@<@ in class
\verb@Ord@. The other inherited comparison methods then refer to this
implementation in their body.

The clause ``\verb@Rational(d, d) with Ord@'' is an instance of {\em
mixin composition}. It describes a template for an object that is
compatible with both \verb@Rational@ and \verb@Ord@ and that contains
all members of either class. \verb@Rational@ is called the {\em
superclass} of \verb@OrderedRational@ while \verb@Ord@ is called a
{\em mixin class}. The type of this template is the {\em compound
type} ``\verb@Rational with Ord@''.

On the surface, mixin composition looks much like multiple
inheritance. The difference between the two becomes apparent if we
look at superclasses of inherited classes. With multiple inheritance,
both \verb@Rational@ and \verb@Ord@ would contribute a superclass
\verb@Object@ to the template. We therefore have to answer some
tricky questions, such as whether members of \verb@Object@ are present
once or twice and whether the initializer of \verb@Object@ is called
once or twice. Mixin composition avoids these complications.  In the
mixin composition \verb@Rational with Ord@, class
\verb@Rational@ is treated as actual superclass of class \verb@Ord@.
A mixin composition \verb@C with M@ is well-formed as long as the
first operand \verb@C@ conforms to the declared superclass of the
second operand \verb@M@. This holds in our example, because
\verb@Rational@ conforms to \verb@Object@. In a sense, mixin composition
amounts to overriding the superclass of a class.

\paragraph{Final Classes.}
Note that class \verb@OrderedRational@ was defined
\verb@final@. This means that no classes extending \verb@OrderedRational@
may be defined in other parts of the program.
%Within final classes the
%type \verb@this@ is an alias of the defined class itself. Therefore,
%we could define our \verb@<@ method with an argument of type
%\verb@OrderedRational@ as a well-formed implementation of the abstract class
%\verb@less(that: this)@ in class \verb@Ord@.


\chapter{Generic Types and Methods}

Classes in Scala can have type parameters. We demonstrate the use of
type parameters with iterators as an example. An iterator is an object
which traverses a sequence of values, using two abstract methods.
\begin{verbatim}
abstract class Iterator[a] {
  def hasNext: boolean;
  def next: a;
\end{verbatim}
Method \verb@next@ returns successive elements.  Method \verb@hasNext@
indicates whether there are still more elements to be returned by
\verb@next@. The type of elements returned by an iterator is
arbitrary. We express that by giving the class \verb@Iterator@ the
type parameter \verb@a@. Type parameters are written in square
brackets, in contrast to normal value parameters, which are written in
parentheses.  Iterators also support other methods, which are
explained in the following.

Method \verb@foreach@ applies a procedure (i.e. a function
returning \verb@unit@ to each element returned by the iterator:
\begin{verbatim}
  def foreach(f: a => unit): unit =
    while (hasNext) { f(next) }
\end{verbatim}

Method \verb@append@ constructs an iterator which resumes with the
given iterator \verb@it@ after the current iterator has finished.
\begin{verbatim}
  def append(that: Iterator[a]): Iterator[a] = new Iterator[a] {
    def hasNext = outer.hasNext || that.hasNext;
    def next = if (outer.hasNext) outer.next else that.next;
  }
\end{verbatim}
The terms \verb@outer.next@ and \verb@outer.hasNext@ in the definition
of \verb@append@ call the corresponding methods as they are defined in
the enclosing \verb@Iterator@ class. Generally, an
\verb@outer@ prefix in a selection indicates an identifier that is
visible immediately outside the current class or template. If the
\verb@outer@ prefix would have been missing,
\verb@hasNext@ and \verb@next@ would have
called recursively the methods being defined in the iterator
constructed by \verb@append@, which is not what we want.

Method \verb@filter@ constructs an iterator which returns all elements
of the original iterator that satisfy a criterion \verb@p@.
\begin{verbatim}
  def filter(p: a => boolean) = new Iterator[a] {
    private class Cell[T](elem_: T) { def elem = elem_; }
    private var head: Cell[a] = null;
    private var isAhead = false;
    def hasNext: boolean =
      if (isAhead) true
      else if (outer.hasNext) {
        head = Cell(outer.next); isAhead = p(head.elem); hasNext }
      else false;
    def next: a =
      if (hasNext) { isAhead = false; head.elem }
      else error("next on empty iterator");
  }
\end{verbatim}
Method \verb@map@ constructs an iterator which returns all elements of
the original iterator transformed by a given function \verb@f@.
\begin{verbatim}
  def map[b](f: a => b) = new Iterator[b] {
    def hasNext: boolean = outer.hasNext;
    def next: b = f(outer.next);
  }
\end{verbatim}
The return type of the transformation function \verb@f@ is
arbitrary. This is expressed by a type parameter \verb@b@ in the
signature of method \verb@map@, which represents the return type.
We also say, \verb@map@ is a {\em polymorphic} function.

Method \verb@flatMap@ is like method \verb@map@, except that the
transformation function \verb@f@ now returns an iterator.
The result of \verb@flatMap@ is the iterator resulting from appending
together all iterators returned from successive calls of \verb@f@.
\begin{verbatim}
    private var cur: Iterator[b] = new EmptyIterator[b];
    def hasNext: boolean =
      if (cur.hasNext) true
      else if (outer.hasNext) { cur = f(outer.next); hasNext }
      else false;
    def next: b =
      if (cur.hasNext) cur.next
      else if (outer.hasNext) { cur = f(outer.next); next }
      else error("next on empty iterator");
  }
\end{verbatim}
Finally, method \verb@zip@ takes another iterator and
returns an iterator consisting of pairs of corresponding elements
returned by the two iterators.
\begin{verbatim}
  def zip[b](that: Iterator[b]) = new Iterator[(a, b)] {
    def hasNext = outer.hasNext && that.hasNext;
    def next = (outer.next, that.next);
  }
} //end iterator;
\end{verbatim}
Concrete iterators need to provide implementations for the two
abstract methods \verb@next@ and \verb@hasNext@ in class
\verb@Iterator@. The simplest iterator is \verb@EmptyIterator@
which always returns an empty sequence:
\begin{verbatim}
class EmptyIterator[a] extends Iterator[a] {
  def hasNext = false;
  def next: a = error("next on empty iterator");
}
\end{verbatim}
A more interesting iterator enumerates all elements of an array.
This iterator is formulated here as a polymorphic function. It could
have also been written as a class, like \verb@EmptyIterator@. The
difference between the two formulation is that classes also define new
types, whereas functions do not.
\begin{verbatim}
def arrayIterator[a](xs: Array[a]) = new Iterator[a] {
  private var i = 0;
  def hasNext: boolean =
    i < xs.length;
  def next: a =
    if (i < xs.length) { val x = xs(i) ; i = i + 1 ; x }
    else error("next on empty iterator");
}
\end{verbatim}
Another iterator enumerates an integer interval:
\begin{verbatim}
def range(lo: int, hi: int) = new Iterator[int] {
  private var i = lo;
  def hasNext: boolean =
    i <= hi;
  def next: int =
    if (i <= hi) { i = i + 1 ; i - 1 }
    else error("next on empty iterator");
}
\end{verbatim}
%In fact, enumerating integer intervals is so common that it is
%supported by a method
%\begin{verbatim}
%def to(hi: int): Iterator[int]
%\end{verbatim}
%in class \verb@int@. Hence, one could also write \verb@l to h@ instead of
%\verb@range(l, h)@.
All iterators seen so far terminate eventually. It is also possible to
define iterators that go on forever. For instance, the following
iterator returns successive integers from some start
value\footnote{Due to the finite representation of type \prog{int},
numbers will wrap around at $2^31$.}.
\begin{verbatim}
def from(start: int) = new Iterator[int] {
  private var last = start - 1;
  def hasNext = true;
  def next = { last = last + 1; last }
}
\end{verbatim}
Here are two examples how iterators are used. First, to print all
elements of an array \verb@xs: Array[int]@, one can write:
\begin{verbatim}
  arrayIterator[int](xs) foreach (x => System.out.println(x))
\end{verbatim}
Here, \verb@[int]@ is a type argument clause, which matches the type
parameter clause \verb@[a]@ of function \verb@arrayIterator@. It
substitutes the formal argument \verb@int@ for the formal argument
\verb@a@ in the type of the method that follows. Hence,
\verb@arrayIterator[a]@ is a function that takes an \verb@Array[int]@
and that returns an \verb@Iterator[int]@.

In this example, the formal type argument \verb@int@ is redundant
since it could also have been inferred from the value \verb@xs@, which
is, after all, an array of \verb@int@. The Scala compiler contains a
fairly powerful type inferencer which infers type arguments for
methods and constructors from the types of value arguments and the
expected return type. In our example, the \verb@[int]@ clause can be
inferred, so that one can abbreviate to:
\begin{verbatim}
  arrayIterator(xs) foreach (x => System.out.println(x))
\end{verbatim}
%As a second example, consider the problem of finding the indices of
%all the elements in an array of \verb@double@s greater than some
%\verb@limit@. The indices should be returned as an iterator.
%This is achieved by the following expression.
%\begin{verbatim}
%arrayIterator(xs)
%  .zip(from(0))
%  .filter(x, i => x > limit)
%  .map(x, i => i)
%\end{verbatim}
%The first line in this expression iterates through all array elements,
%the second lines pairs elements with their indices, the third line
%selects all value/index pairs where the value is greater than
%\verb@limit@, and the fourth line returns the index part of all
%selected pairs.

%Note that we have omitted the type arguments for the calls of
%\verb@arrayIterator@, \verb@zip@ and \verb@map@. These are all
%implicitly inserted by the type inferencer.

\chapter{\label{sec:for-notation}For-Comprehensions}

The last chapter has demonstrated that the use of higher-order
functions over sequences can lead to very concise programs. But
sometimes the level of abstraction required by these functions makes a
program hard to understand.

Here, Scala's \verb@for@ notation can help. For instance, say we are
given a sequence \verb@persons@ of persons with \verb@name@ and
\verb@age@ fields.  That sequence could be an array, or a list, or an
iterator, or some other type implementing the sequence abstraction
(this will be made more precise below). To print the names of all
persons in the sequence which are aged over 20, one writes:
\begin{verbatim}
for { val p <- persons; p.age > 20 } yield p.name
\end{verbatim}
This is equivalent to the following expression , which uses
higher-order functions \verb@filter@ and \verb@map@:
\begin{verbatim}
persons filter (p => p.age > 20) map (p => p.name)
\end{verbatim}
The for-expression looks a bit like a for-loop in imperative languages,
except that it constructs a list of the results of all iterations.

Generally, a for-comprehension is of the form
\begin{verbatim}
for ( s ) yield e
\end{verbatim}
(Instead of parentheses, braces may also be used.)
Here, \verb@s@ is a sequence of {\em generators} and {\em filters}.
\begin{itemize}
\item A {\em generator} is of the form \verb@val x <- e@,
where \verb@e@ is a list-valued expression. It binds \verb@x@ to
successive values in the list.
\item A {\em filter} is an expression \verb@f@ of type \verb@boolean@.
It omits from consideration all bindings for which \verb@f@ is \verb@false@.
\end{itemize}
The sequence must start with a generator.
If there are several generators in a sequence, later generators vary
more rapidly than earlier ones.

Here are two examples that show how for-comprehensions are used.

First, given a positive integer \verb@n@, find all pairs of positive
integers
\verb@i@, \verb@j@, where \verb@1 <= j < i <= n@ such that \verb@i + j@ is prime.
\begin{verbatim}
for \={ \=val i <- range(1, n);
    \>  \>val j <- range(1, i-1);
    \>  \>isPrime(i+j)
} yield (i, j)
\end{verbatim}

As second example, the scalar product of two vectors \verb@xs@ and
\verb@ys@ can now be written as
follows.
\begin{verbatim}
  sum (for { val (x, y) <- xs zip ys } yield x * y)
\end{verbatim}
The for-notation is essentially equivalent to common operations of
database query languages.  For instance, say we are given a book
database \verb@books@, represented as a list of books, where
\verb@Book@ is defined as follows.
\begin{verbatim}
abstract class Book {
  val title: String;
  val authors: List[String]
}
\end{verbatim}
\begin{verbatim}
val books: List[Book] = [
  new Book {
    val title = "Structure and Interpretation of Computer Programs";
    val authors = ["Abelson, Harald", "Sussman, Gerald J."];
  },
  new Book {
    val title = "Principles of Compiler Design";
    val authors = ["Aho, Alfred", "Ullman, Jeffrey"];
  },
  new Book {
    val title = "Programming in Modula-2";
    val authors = ["Wirth, Niklaus"];
  }
];
\end{verbatim}
Then, to find the titles of all books whose author's last name is ``Ullman'':
\begin{verbatim}
for { val b <- books; val a <- b.authors; a startsWith "Ullman"
} yield b.title
\end{verbatim}
(Here, \verb@startsWith@ is a method in \verb@java.lang.String@).  Or,
to find the titles of all books that have the string ``Program'' in
their title:
\begin{verbatim}
for { val b <- books; (b.title indexOf "Program") >= 0
} yield b.title
\end{verbatim}
Or, to find the names of all authors that have written at least two
books in the database.
\begin{verbatim}
for { \=val b1 <- books;
      \>val b2 <- books;
      \>b1 != b2;
      \>val a1 <- b1.authors;
      \>val a2 <- b2.authors;
      \>a1 == a2 } yield a1
\end{verbatim}
The last solution is not yet perfect, because authors will appear
several times in the list of results.  We still need to remove
duplicate authors from result lists.  This can be achieved with the
following function.
\begin{verbatim}
def removeDuplicates[a](xs: List[a]): List[a] =
  if (xs.isEmpty) xs
  else xs.head :: removeDuplicates(xs.tail filter (x => x != xs.head));
\end{verbatim}
The last expression can be equivalently expressed as follows.
\begin{verbatim}
xs.head :: removeDuplicates(for (val x <- xs.tail; x != xs.head) yield x)
\end{verbatim}

\subsection*{Translation of \prog{for}}

Every for-comprehensions can be expressed in terms of the three
higher-order functions \verb@map@, \verb@flatMap@ and \verb@filter@.
Here is the translation scheme, which is also used by the Scala compiler.
\begin{itemize}
\item
A simple for-comprehension
\begin{verbatim}
for (val x <- e) yield e'
\end{verbatim}
is translated to
\begin{verbatim}
e.map(x => e')
\end{verbatim}
\item
A for-comprehension
\begin{verbatim}
for (val x <- e; f; s) yield e'
\end{verbatim}
where \verb@f@ is a filter and \verb@s@ is a (possibly empty)
sequence of generators or filters
is translated to
\begin{verbatim}
for (val x <- e.filter(x => f); s) yield e'
\end{verbatim}
and then translation continues with the latter expression.
\item
A for-comprehension
\begin{verbatim}
for (val x <- e; y <- e'; s) yield e''
\end{verbatim}
where \verb@s@ is a (possibly empty)
sequence of generators or filters
is translated to
\begin{verbatim}
e.flatMap(x => for (y <- e'; s) yield e'')
\end{verbatim}
and then translation continues with the latter expression.
\end{itemize}
For instance, taking our "pairs of integers whose sum is prime" example:
\begin{verbatim}
for \= { \= val i <- range(1, n);
    \>   \> val j <- range(1, i-1);
    \>   \> isPrime(i+j)
} yield (i, j)
\end{verbatim}
Here is what we get when we translate this expression:
\begin{verbatim}
range(1, n)
  .flatMap(i =>
    range(1, i-1)
      .filter(j => isPrime(i+j))
      .map(j => (i, j)))
\end{verbatim}

\exercise
Define the following function in terms of \verb@for@.
\begin{verbatim}
def concat(xss: List[List[a]]): List[a] =
  (xss foldr []) { xs, ys => xs ::: ys }
\end{verbatim}
\exercise
Translate
\begin{verbatim}
for { val b <- books; val a <- b.authors; a startsWith "Bird" } yield b.title
for { val b <- books; (b.title indexOf "Program") >= 0 } yield b.title
\end{verbatim}
to higher-order functions.

We have seen that the for-translation only relies on the presence of
methods \verb@map@,
\verb@flatMap@, and \verb@filter@.
This gives programmers the possibility to have for-syntax for
other types as well -- one only needs to define \verb@map@,
\verb@flatMap@, and \verb@filter@ for these types.
That's also why we were able to define \verb@for@ at the same time for
arrays, iterators, and lists -- all these types have the required
three methods \verb@map@,\verb@flatMap@, and \verb@filter@ as members.
Of course, it is also possible for users and library designers to
define other types with these methods. There are many examples where
this is useful: Databases, XML trees, optional values. We will see in
Chapter~\ref{sec:parsers-results} how for-comprehensions can be used in the
definition of parsers for context-free grammars that construct
abstract syntax trees.

\chapter{\label{sec:simple-examples}Pattern Matching}

\todo{Complete}

Consider binary trees whose leafs contain integer arguments. This can
be described by a class for trees, with subclasses for leafs and
branch nodes:
\begin{verbatim}
abstract class Tree;
case class Branch(left: Tree, right: Tree) extends Tree;
case class Leaf(x: int) extends Tree;
\end{verbatim}
Note that the class \verb@Tree@ is not followed by an extends
clause or a body. This defines \verb@Tree@ to be an empty
subclass of \verb@Object@, as if we had written
\begin{verbatim}
class Tree extends Object {}
\end{verbatim}
Note also that the two subclasses of \verb@Tree@ have a \verb@case@
modifier.  That modifier has two effects. First, it lets us construct
values of a case class by simply calling the constructor, without
needing a preceding \verb@new@. Example:
\begin{verbatim}
val tree1 = Branch(Branch(Leaf(1), Leaf(2)), Branch(Leaf(3), Leaf(4)))
\end{verbatim}
Second, it lets us use constructors for these classes in patterns, as
is illustrated in the following example.
\begin{verbatim}
def sumLeaves(t: Tree): int = t match {
  case Branch(l, r) => sumLeaves(l) + sumLeaves(r)
  case Leaf(x) => x
}
\end{verbatim}
The function \verb@sumLeaves@ sums up all the integer values in the
leaves of a given tree \verb@t@. It is is implemented by calling the
\verb@match@ method of \verb@t@ with a {\em choice expression} as
argument (\verb@match@ is a predefined method in class \verb@Object@).
The choice expression consists of two cases which both
relate a pattern with an expression. The pattern of the first case,
\verb@Branch(l, r)@ matches all instances of class \verb@Branch@
and binds the {\em pattern variables} \verb@l@ and \verb@r@ to the
constructor arguments, i.e.\ the left and right subtrees of the
branch.  Pattern variables always start with a lower case letter; to
avoid ambiguities, constructors in patterns should start with an upper
case letter.

The effect of the choice expression is to select the first alternative
whose pattern matches the given select value, and to evaluate the body
of this alternative in a context where pattern variables are bound to
corresponding parts of the selector. For instance, the application
\verb@sumLeaves(tree1)@ would select the first alternative with the
\verb@Branch(l,r)@ pattern, and would evaluate the expression
\verb@sumLeaves(l) + sumLeaves(r)@ with bindings
\begin{verbatim}
l = Branch(Leaf(1), Leaf(2)), r = Branch(Leaf(3), Leaf(4)).
\end{verbatim}
As another example, consider the following class
\begin{verbatim}
abstract final class Option[+a];
case object None extends Option[All];
case class Some[a](item: a) extends Option[a];
\end{verbatim}
...

%\todo{Several simple and intermediate examples needed}.

\begin{verbatim}
def find[a,b](it: Iterator[(a, b)], x: a): Option[b] = {
  var result: Option[b] = None;
  while (it.hasNext && result == None) {
    val (x1, y) = it.next;
    if (x == x1) result = Some(y)
  }
  result
}
find(xs, x) match {
  case Some(y) => System.out.println(y)
  case None => System.out.println("no match")
}
\end{verbatim}

\comment{


class MaxCounter {
  var maxVal: Option[int] = None;
  def set(x: int) = maxVal match {
    case None => maxVal = Some(x)
    case Some(y) => maxVal = Some(Math.max(x, y))
  }
}
\end{verbatim}
}
\comment{
\begin{verbatim}
class Stream[a] = List[a]

module Stream {
  def concat(xss: Stream[Stream[a]]): Stream[a] = {
    let result: Stream[a] = xss match {
      case [] => []
      case [] :: xss1 => concat(xss1)
      case (x :: xs) :: xss1 => x :: concat(xs :: xss1)
    }
    result
  }
}
\end{verbatim}
}
\comment{
\chapter{Implementing Abstract Types: Search Trees}

This chapter presents unbalanced binary search trees, implemented in
three different styles: algebraic, object-oriented, and imperative.
In each case, a search tree package is seen as an implementation
of a class {\em MapStruct}.
\begin{verbatim}
abstract class MapStruct[kt, vt] {
  abstract type Map extends kt => vt {
    def apply(key: kt): vt;
    def extend(key: kt, value: vt): Map;
    def remove(key: kt): Map;
    def domain: Stream[kt];
    def range: Stream[vt];
  }
  def empty: Map;
}
\end{verbatim}
The \verb@MapStruct@ class is parameterized with a type of keys
\verb@kt@ and a type of values \verb@vt@. It
specifies an abstract type \verb@Map@ and an abstract value
\verb@empty@, which represents empty maps.  Every implementation
\verb@Map@ needs to conform to that abstract type, which
extends the function type \verb@kt => vt@
with four new
methods. The method \verb@domain@ yields a stream that enumerates the
map's domain, i.e. the set of keys that are mapped to non-null values.
The method \verb@range@ yields a stream that enumerates the function's
range, i.e.\ the values obtained by applying the function to arguments
in its domain.  The method
\verb@extend@ extends the map with a given key/value binding, whereas
\verb@remove@ removes a given key from the map's domain. Both
methods yield a new map value as result, which has the same
representation as the receiver object.

\begin{figure}[t]
\begin{verbatim}
class AlgBinTree[kt extends Ord, vt] extends MapStruct[kt, vt] {
  private case
    Empty extends Map,
    Node(key: kt, value: vt, l: Map, r: Map) extends Map

  final class Map extends kt => vt {
    def apply(key: kt): vt = this match {
      case Empty => null
      case Node(k, v, l, r) =>
	if (key < k) l.apply(key)
	else if (key > k) r.apply(key)
	else v
    }

    def extend(key: kt, value: vt): Map = this match {
      case Empty => Node(k, v, Empty, Empty)
      case Node(k, v, l, r) =>
	if (key < k) Node(k, v, l.extend(key, value), r)
	else if (key > k) Node(k, v, l, r.extend(key, value))
	else Node(k, value, l, r)
    }

    def remove(key: kt): Map = this match {
      case Empty => Empty
      case Node(k, v, l, r) =>
	if (key < k) Node(k, v, l.remove(key), r)
	else if (key > k) Node(k, v, l, r.remove(key))
	else if (l == Empty) r
	else if (r == Empty) l
	else {
	  val midKey = r.domain.head
	  Node(midKey, r.apply(midKey), l, r.remove(midKey))
	}
    }

    def domain: Stream[kt] = this match {
      case Empty => []
      case Node(k, v, l, r) => Stream.concat([l.domain, [k], r.domain])
    }
    def range: Stream[vt] = this match {
      case Empty => []
      case Node(k, v, l, r) => Stream.concat([l.range, [v], r.range])
    }
  }
  def empty: Map = Empty
}
\end{verbatim}
\caption{\label{fig:algbintree}Algebraic implementation of binary
search trees}
\end{figure}
We now present three implementations of \verb@Map@, which are all
based on binary search trees. The \verb@apply@ method of a map is
implemented in each case by the usual search function over binary
trees, which compares a given key with the key stored in the topmost
tree node, and depending on the result of the comparison, searches the
left or the right hand sub-tree. The type of keys must implement the
\verb@Ord@ class, which contains comparison methods
(see Chapter~\ref{chap:classes} for a definition of class \verb@Ord@).

The first implementation, \verb@AlgBinTree@, is given in
Figure~\ref{fig:algbintree}. It represents a map with a
data type \verb@Map@ with two cases, \verb@Empty@ and \verb@Node@.

Every method of \verb@AlgBinTree[kt, vt].Map@ performs a pattern
match on the value of
\verb@this@ using the \verb@match@ method which is defined as postfix
function application in class \verb@Object@ (\sref{sec:class-object}).

The functions \verb@domain@ and \verb@range@ return their results as
lazily constructed lists. The \verb@Stream@ class is an alias of
\verb@List@ which should be used to indicate the fact that its values
are constructed lazily.

\begin{figure}[thb]
\begin{verbatim}
class OOBinTree[kt extends Ord, vt] extends MapStruct[kt, vt] {
  abstract class Map extends kt => vt {
    def apply(key: kt): v
    def extend(key: kt, value: vt): Map
    def remove(key: kt): Map
    def domain: Stream[kt]
    def range: Stream[vt]
  }
  module empty extends Map {
    def apply(key: kt) = null
    def extend(key: kt, value: vt) = Node(key, value, empty, empty)
    def remove(key: kt) = empty
    def domain = []
    def range = []
  }
  private class Node(k: kt, v: vt, l: Map, r: Map) extends Map {
    def apply(key: kt): vt =
      if (key < k) l.apply(key)
      else if (key > k) r.apply(key)
      else v
    def extend(key: kt, value: vt): Map =
      if (key < k) Node(k, v, l.extend(key, value), r)
      else if (key > k) Node(k, v, l, r.extend(key, value))
      else Node(k, value, l, r)
    def remove(key: kt): Map =
      if (key < k) Node(k, v, l.remove(key), r)
      else if (key > k) Node(k, v, l, r.remove(key))
      else if (l == empty) r
      else if (r == empty) l
      else {
	val midKey = r.domain.head
	Node(midKey, r(midKey), l, r.remove(midKey))
      }
    def domain: Stream[kt] = Stream.concat([l.domain, [k], r.domain]	)
    def range: Stream[vt] = Stream.concat([l.range, [v], r.range])
  }
}
\end{verbatim}
\caption{\label{fig:oobintree}Object-oriented implementation of binary
search trees}
\end{figure}

The second implementation of maps is given in
Figure~\ref{fig:oobintree}.  Class \verb@OOBinTree@ implements the
type \verb@Map@ with a module \verb@empty@ and a class
\verb@Node@, which define the behavior of empty and non-empty trees,
respectively.

Note the different nesting structure of \verb@AlgBinTree@ and
\verb@OOBinTree@. In the former, all methods form part of the base
class \verb@Map@. The different behavior of empty and non-empty trees
is expressed using a pattern match on the tree itself. In the
latter, each subclass of \verb@Map@ defines its own set of
methods, which override the methods in the base class. The pattern
matches of the algebraic implementation have been replaced by the
dynamic binding that comes with method overriding.

Which of the two schemes is preferable depends to a large degree on
which extensions of the type are anticipated. If the type is later
extended with a new alternative, it is best to keep methods in each
alternative, the way it was done in \verb@OOBinTree@.  On the other
hand, if the type is extended with additional methods, then it is
preferable to keep only one implementation of methods and to rely on
pattern matching, since this way existing subclasses need not be
modified.

\begin{figure}
\begin{verbatim}
class MutBinTree[kt extends Ord, vt] extends MapStruct[kt, vt] {
  class Map(key: kt, value: vt) extends kt => vt {
    val k = key
    var v = value
    var l = empty, r = empty

    def apply(key: kt): vt =
      if (this eq empty) null
      else if (key < k) l.apply(key)
      else if (key > k) r.apply(key)
      else v

    def extend(key: kt, value: vt): Map =
      if (this eq empty) Map(key, value)
      else {
	if (key < k) l = l.extend(key, value)
	else if (key > k) r = r.extend(key, value)
	else v = value
	this
      }

    def remove(key: kt): Map =
      if (this eq empty) this
      else if (key < k) { l = l.remove(key) ; this }
      else if (key > k) { r = r.remove(key) ; this }
      else if (l eq empty) r
      else if (r eq empty) l
      else {
	var mid = r
	while (!(mid.l eq empty)) { mid = mid.l }
	mid.r = r.remove(mid.k)
	mid.l = l
	mid
      }

    def domain: Stream[kt] = Stream.concat([l.domain, [k], r.domain])
    def range: Stream[vt] = Stream.concat([l.range, [v], r.range])
  }
  let empty = new Map(null, null)
}
\end{verbatim}
\caption{\label{fig:impbintree}Side-effecting implementation of binary
search trees}
\end{figure}

The two versions of binary trees presented so far are {\em
persistent}, in the sense that maps are values that cannot be changed
by side effects. By contrast, in the next implementation of binary
trees, the implementations of \verb@extend@ and
\verb@remove@ do have an effect on the state of their receiver
object. This corresponds to the way binary trees are usually
implemented in imperative languages. The new implementation can lead
to some savings in computing time and memory allocation, but care is
required not to use the original tree after it has been modified by a
side-effecting operation.

In this implementation, \verb@value@, \verb@l@ and \verb@r@ are
variables that can be affected by method calls.  The
class \verb@MutBinTree[kt, vt].Map@ takes two instance parameters
which define the \verb@key@ value and the initial value of the
\verb@value@ variable. Empty trees are represented by a
value \verb@empty@, which has \verb@null@ (signifying undefined) in
both its key and value fields. Note that this value needs to be
defined lazily using \verb@let@ since its definition involves the
creation of a
\verb@Map@ object,
which accesses \verb@empty@ recursively as part of its initialization.
All methods test first whether the current tree is empty using the
reference equality operator \verb@eq@ (\sref{sec:class-object}).

As a program using the \verb@MapStruct@ abstraction, consider a function
which creates a map from strings to integers and then applies it to a
key string:
\begin{verbatim}
def mapTest(def mapImpl: MapStruct[String, int]): int = {
  val map: mapImpl.Map = mapImpl.empty.extend("ab", 1).extend("bx", 3)
  val x = map("ab")             // returns 1
}
\end{verbatim}
The function is parameterized with the particular implementation of
\verb@MapStruct@. It can be applied to any one of the three implementations
described above. E.g.:
\begin{verbatim}
mapTest(AlgBinTree[String, int])
mapTest(OOBinTree[String, int])
mapTest(MutBinTree[String, int])
\end{verbatim}
}
\chapter{Programming with Higher-Order Functions: Combinator Parsing}

In this chapter we describe how to write combinator parsers in
Scala. Such parsers are constructed from predefined higher-order
functions, so called parser combinators, that closely model the
constructions of an EBNF grammar \cite{ebnf}.

As running example, we consider parsers for arithmetic expressions
described by the following context-free grammar.
\bda{p{3cm}cp{10cm}}
letter &::=& /* all letters */ \\
digit  &::=& /* all digits */ \\[0.5em]
ident  &::=& letter \{letter $|$ digit \}\\
number &::=& digit \{digit\}\\[0.5em]

expr &::=& expr1 \{`+' expr1 $|$ `$-$' expr1\}\\
expr1 &::=& expr2 \{`*' expr2 $|$ `/' expr2\}\\
expr2 &::=& ident $|$ number $|$ `(' expr `)'
\eda

\section{Simple Combinator Parsing}

In this section we will only be concerned with the task of recognizing
input strings, not with processing them. So we can describe parsers
by the sets of input strings they accept.  There are two
fundamental operators over parsers:
\verb@&&&@ expresses the sequential composition of a parser with
another, while \verb@|||@ expresses an alternative. These operations
will both be defined as methods of a \verb@Parser@ class.  We will
also define constructors for the following primitive parsers:

\begin{quote}\begin{tabular}{ll}
\verb@empty@	& The parser that accepts the empty string
\\
\verb@fail@     & The parser that accepts no string
\\
\verb@chr@      & The parser that accepts any character.
\\
\verb@chr(c: char)@
		& The parser that accepts the single-character string ``$c$''.
\\
\verb@chrWith(p: char => boolean)@
		& The parser that accepts single-character strings
                  ``$c$'' \\
	        & for which $p(c)$ is true.
\end{tabular}\end{quote}

There are also the two higher-order parser combinators \verb@opt@,
expressing optionality and \verb@rep@, expressing repetition.
For any parser $p$, \verb@opt($p$)@ yields a parser that
accepts the strings accepted by $p$ or else the empty string, while
\verb@rep($p$)@ accepts arbitrary sequences of the strings accepted by
$p$. In EBNF, \verb@opt($p$)@ corresponds to $[p]$ and \verb@rep($p$)@
corresponds to $\{p\}$.

The central idea of parser combinators is that parsers can be produced
by a straightforward rewrite of the grammar, replacing \verb@::=@ with
\verb@=@, sequencing with
\verb@&&&@, choice
\verb@|@ with \verb@|||@, repetition \verb@{...}@ with
\verb@rep(...)@ and optional occurrence with \verb@[...]@.
Applying this process to the grammar of arithmetic
expressions yields:
\begin{verbatim}
module ExprParser {
  import Parse;

  def letter   \= =  \= chrWith(c => c.isLetter);
  def digit    \= =  \> chrWith(c => c.isDigit);

  def ident    \> =  \> letter &&& rep(letter ||| digit);
  def number   \> =  \> digit &&& rep(digit);

  def expr:Parser\> =  expr1 &&& rep((chr('+') &&& expr1) ||| (chr('-') &&& expr1));
  def expr1    \> =  expr2 &&& rep((chr('*') &&& expr2) ||| (chr('/') &&& expr2));
  def expr2    \> =  ident ||| number ||| (chr('(') &&& expr &&& chr(')'));
}
\end{verbatim}
It remains to explain how to implement a library with the combinators
described above. We will pack combinators and their underlying
implementation in a module \verb@Parse@.  The first question to decide
is which underlying representation type to use for a parser. We treat
parsers here as functions that take a list of characters as input
parameter and that yield a parse result.
\begin{verbatim}
module Parse {

  type Result = Option[List[char]];

  abstract class Parser extends Function1[List[char],Result] {
\end{verbatim}
\comment{
The \verb@Option@ type is predefined as follows.
\begin{verbatim}
abstract final class Option[a];
case class None[a] extends Option[a];
case class Some[a](x: a) extends Option[a];
\end{verbatim}
}
A parser returns either the constant \verb@None@, which
signifies that the parser did not recognize a legal input string, or
it returns a value \verb@Some(in1)@ where \verb@in1@ represents that
part of the input list that the parser did not consume.

Parsers are instances of functions from \verb@List[char]@ to
\verb@Parse.Result@, which also implement the combinators
for sequence and alternative. This is modeled by
defining \verb@Parser@ as a class that extends type
\verb@Function1[List[char],Result]@ and that defines an \verb@apply@
method, as well as methods \verb@&&&@ and \verb@|||@.
\begin{verbatim}
    abstract def apply(in: List[char]): Result;
\end{verbatim}
\begin{verbatim}
    def &&& (def p: Parser) = new Parser {
      def apply(in: List[char]) = outer.apply(in) match {
        case Some(in1) => p(in1)
        case n => n
      }
    }

    def ||| (def p: Parser) = new Parser {
      def apply(in: List[char]) = outer.apply(in) match {
        case None => p(in)
        case s => s
      }
    }
  }
\end{verbatim}
The implementations of the primitive parsers \verb@empty@, \verb@fail@,
\verb@chrWith@ and \verb@chr@ are as follows.
\begin{verbatim}

  def empty = new Parser { def apply(in: List[char]) = Some(in) }

  def fail = new Parser { def apply(in: List[char]) = None[List[char]] }

  def chrWith(p: char => boolean) = new Parser {
    def apply(in: List[char]) = in match {
      case [] => None[List[char]]
      case (c :: in1) => if (p(c)) Some(in1) else None[List[char]]
    }
  }

  def chr(c: char): Parser = chrWith(d => d == c);
\end{verbatim}
The higher-order parser combinators \verb@opt@ and \verb@rep@ can be
defined in terms of the combinators for sequence and alternative:
\begin{verbatim}
  def opt(p: Parser): Parser = p ||| empty;
  def rep(p: Parser): Parser = opt(rep1(p));
  def rep1(p: Parser): Parser = p &&& rep(p);
} // end Parser
\end{verbatim}
This is all that's needed. Parsers such as the one for arithmetic
expressions given above can now be composed from these building
blocks.  These parsers need not refer to the underlying implementation of
parsers as functions from input lists to parse results.

The presented combinator parsers use backtracking to change from one
alternative to another.  If one restricts the focus to LL(1) grammars,
a non-backtracking implementation of parsers is also possible. This
implementation can then be based on iterators instead of lists.

\section{\label{sec:parsers-results}Parsers that Return Results}

The combinator library of the previous section does not support the
generation of output from parsing. But usually one does not just want
to check whether a given string belongs to the defined language, one
also wants to convert the input string into some internal
representation such as an abstract syntax tree.

In this section, we modify our parser library to build parsers that
produce results. We will make use of the for-comprehensions introduced
in Chapter~\ref{sec:for-notation}.  The basic combinator of sequential
composition, formerly \verb@p &&& q@, now becomes
\begin{verbatim}
for (val x <- p; val y <- q) yield e
\end{verbatim}.
Here, the names \verb@x@ and \verb@y@ are bound to the results of
executing the parsers \verb@p@ and \verb@q@. \verb@e@ is an expression
that uses these results to build the tree returned by the composed
parser.

Before describing the implementation of the new parser combinators, we
explain how the new building blocks are used. Say we want to modify
our arithmetic expression parser so that it returns an abstract syntax
tree of the parsed expression. The class of syntax trees is given by:
\begin{verbatim}
abstract class Tree;
case class Var(n: String) extends Tree;
case class Num(n: int) extends Tree;
case class Binop(op: char, l: Tree, r: Tree) extends Tree;
\end{verbatim}
That is, a syntax tree is a named variable, an integer number, or a
binary operation with two operand trees and a character indicating the
operation.

As a first step towards parsers that produce syntax trees, we need to
modify the ``micro-syntax'' parsers \verb@letter@, \verb@digit@,
\verb@ident@ and \verb@number@ so that they return representations of
the parsed input:
\begin{verbatim}
def letter: Parser[char] = chrWith(c => c.isLetter);
def digit : Parser[char] = chrWith(c => c.isDigit);

def ident: Parser[String] =
  for (val c <- letter; val cs <- rep(letter ||| digit))
  yield ((c :: cs) foldr "") {c, s => c+ s};

def number: Parser[int] =
  for (val d <- digit; val ds <- rep(digit))
  yield ((d - '0') :_foldl ds) {x, y => x * 10 + (y - '0')};
\end{verbatim}
The \verb@letter@ and \verb@digit@ parsers simply return the letter
that was parsed. The \verb@ident@ and \verb@number@ parsers return the
string, respectively integer number that was parsed.  In both cases,
sub-parsers are applied in a for-comprehension and their results are
embedded in the result of the calling parser.  The remainder of the
parser for arithmetic expressions follows the same scheme.
\begin{verbatim}
def expr: Parser[Tree] =
  for {
    val e1 <- expr1;
    val es <- rep (
      for {
        val op <- chr('+') ||| chr('-');
	val e <- expr1
      } yield (x => Binop(op, x, e)) : Tree => Tree
    )
  } yield applyAll(es, e1);
\end{verbatim}
\begin{verbatim}
def expr1: Parser[Tree] =
  for {
    val e1 <- expr2;
    val es <- rep (
      for {
        val op <- chr('*') ||| chr('/');
        val e <- expr2
      } yield (x => Binop(op, x, e)) : Tree => Tree
    )
  } yield applyAll(es, e1);
\end{verbatim}
\begin{verbatim}
def expr2: Parser[Tree] = {
    \= ( for { val n <- ident } yield Var(n) : Tree )
  |||\> ( for { val n <- number } yield Num(n) : Tree )
  |||\> ( for { val _ <- chr('('); val e <- expr; val _ <- chr(')') } yield e );
}
\end{verbatim}
Note the treatment of the repetitions in \verb@expr@ and
\verb@expr1@. The parser for an expression suffix $op;e$ consisting of an
operator $op$ and an expression $e$ returns a function, which, given a
left operand expression $d$, constructs a \verb@Binop@ node that
represents $d;op;e$. The \verb@rep@ parser combinator forms a list of
all these functions. The final \verb@yield@ part applies all functions
to the first operand in the sequence, which is represented by
\verb@e1@. Here \verb@applyAll@ applies the list of functions passed as its first
argument to its second argument. It is defined as follows.
\begin{verbatim}
def applyAll[a](fs: List[a => a], e: a): a =
  (e :_foldl fs) { x, f => f(x) }
\end{verbatim}
We now present the parser combinators that support the new
scheme. Parsers that succeed now return a parse result besides the
un-consumed input.
\begin{verbatim}
module Parse {

  type Result[a] = Option[(a, List[char])]
\end{verbatim}
Parsers are parameterized with the type of their result. The class
\verb@Parser[a]@ now defines new methods \verb@map@, \verb@flatMap@
and \verb@filter@. The \verb@for@ expressions are mapped by the
compiler to calls of these functions using the scheme described in
Chapter~\ref{sec:for-notation}.

Here is the complete definition of the new \verb@Parser@ class.
\begin{verbatim}
  abstract class Parser[a] extends Function1[List[char],Result[a]] {

    def apply(in: List[char]): Result[a];

    def filter(p: a => boolean) = new Parser[a] {
      def apply(in: List[char]): Result[a] = outer.apply(in) match {
        case Some((x, in1)) => if (p(x)) Some((x, in1)) else None
        case None => None
      }
    }

    def map[b](f: a => b) = new Parser[b] {
      def apply(in: List[char]): Result[b] = outer.apply(in) match {
	case Some((x, in1)) => Some((f(x), in1))
        case None => None
      }
    }

    def flatMap[b](f: a => Parser[b]) = new Parser[b] {
      def apply(in: List[char]): Result[b] = outer.apply(in) match {
	case Some((x, in1)) => f(x)(in1)
        case None => None
      }
    }

    def ||| (def p: Parser[a]) = new Parser[a] {
      def apply(in: List[char]): Result[a] = outer.apply(in) match {
	case None => p(in)
	case s => s
      }
    }

    def &&& [b](def p: Parser[b]): Parser[b] =
      for (val _ <- this; val result <- p) yield result;
  }
\end{verbatim}

The \verb@filter@ method takes as parameter a predicate $p$ which it
applies to the results of the current parser. If the predicate is
false, the parser fails by returning \verb@None@; otherwise it returns
the result of the current parser.  The \verb@map@ method takes as
parameter a function $f$ which it applies to the results of the
current parser. The \verb@flatMap@ tales as parameter a function
\verb@f@ which returns a parser.  It applies \verb@f@ to the result of
the current parser and then continues with the resulting parser.  The
\verb@|||@ method is essentially defined as before.  The
\verb@&&&@ method can now be defined in terms of \verb@for@.

% Here is the code for fail, chrWith and chr
%
%\begin{verbatim}
%  def fail[a] = new Parser[a] { def apply(in: List[char]) = None[(a,List[char])] }
%
%  def chrWith(p: char => boolean) = new Parser[char] {
%    def apply(in: List[char]) = in match {
%      case [] => None[(char,List[char])]
%      case (c :: in1) => if (p(c)) Some((c,in1)) else None[(char,List[char])]
%    }
%  }
%
%  def chr(c: char): Parser[char] = chrWith(d => d == c);
%\end{verbatim}
The primitive parser \verb@succeed@ replaces \verb@empty@. It consumes
no input and returns its parameter as result.
\begin{verbatim}
  def succeed[a](x: a) = new Parser[a] {
    def apply(in: List[char]) = Some((x, in))
  }
\end{verbatim}
The \verb@fail@ parser is as before.  The parser combinators
\verb@rep@ and \verb@opt@ now also return results. \verb@rep@ returns
a list which contains as elements the results of each iteration of its
sub-parser. \verb@opt@ returns an
\verb@Option@ type which indicates whether something was recognized by
its sub-parser.
\begin{verbatim}
  def rep[a](p: Parser[a]): Parser[List[a]] =
    rep1(p) ||| succeed([]);

  def rep1[a](p: Parser[a]): Parser[List[a]] =
    for (val x <- p; val xs <- rep(p)) yield x :: xs;

  def opt[a](p: Parser[a]): Parser[Option [a]] =
    { for (val x <- p) yield (Some(x): Option[a]) } ||| succeed((None: Option[a]));
} // end Parse
\end{verbatim}

\chapter{\label{sec:hm}Programming with Patterns: Hindley/Milner Type Inference}

This chapter demonstrates Scala's data types and pattern matching by
developing a type inference system in the Hindley/Milner style. The
source language for the type inferencer is lambda calculus with a let
construct. Abstract syntax trees for the source language are
represented by the following data type of \verb@Terms@.
\begin{verbatim}
abstract class Term;
case class Var(x: String) extends Term;
case class Lam(x: String, e: Term) extends Term;
case class App(f: Term, e: Term) extends Term;
case class Let(x: String, e: Term, f: Term) extends Term;
\end{verbatim}
There are four tree constructors: \verb@Var@ for variables, \verb@Lam@
for function abstractions, \verb@App@ for function applications, and
\verb@Let@ for let expressions. Note that these tree constructors are
defined outside the \verb@Term@ class. It would also be possible
to define further constructors for this type in other parts of the
program.

The next data type describes the form of types that are
computed by the inference system.
\begin{verbatim}
module Types {
  abstract final class Type;
  case class Tyvar(a: String) extends Type;
  case class Arrow(t1: Type, t2: Type) extends Type;
  case class Tycon(k: String, ts: List[Type]) extends Type;
  private var n: int = 0;
  def newTyvar: Type = { n = n + 1 ; Tyvar("a" + n) }
}
import Types;
\end{verbatim}
There are three type constructors: \verb@Tyvar@ for type variables,
\verb@Arrow@ for function types and \verb@Tycon@ for type
constructors such as \verb@boolean@ or \verb@List@. Type constructors
have as component a list of their type parameters. This list is empty
for type constants such as \verb@boolean@. The data type is packaged
in a module \verb@Types@. Also contained in that module is a function
\verb@newTyvar@ which creates a fresh type variable each time it is
called. The module definition is followed by an import clause
\verb@import Types@, which makes the non-private members of
this module available without qualification in the code that follows.

Note that \verb@Type@ is a \verb@final@ class. This means that no
subclasses or data constructors that extend \verb@Type@ can be formed
except for the three constructors that follow the class.  This makes
\verb@Type@ into a {\em closed} algebraic data type with a fixed
number of alternatives. By contrast, type \verb@Term@ is an {\em open}
algebraic type for which further alternatives can be defined.

The next data type describes type schemes, which consist of a type and
a list of names of type variables which appear universally quantified
in the type scheme. For instance, the type scheme $\forall a\forall
b.a \arrow b$ would be represented in the type checker as:
\begin{verbatim}
TypeScheme(["a", "b"], Arrow(Tyvar("a"), Tyvar("b"))) .
\end{verbatim}
The data type definition of type schemes does not carry an extends
clause; this means that type schemes extend directly class
\verb@Object@.
Even though there is only one possible way to construct a type scheme,
a \verb@case class@ representation was chosen since it offers a convenient
way to decompose a type scheme into its parts using pattern matching.
\begin{verbatim}
case class TypeScheme(ls: List[String], t: Type) {
  def newInstance: Type = {
    val instSubst =
      ((EmptySubst: Subst) :_foldl ls) { s, a => s.extend(Tyvar(a), newTyvar) }
    instSubst(t)
  }
}
\end{verbatim}
Type scheme objects come with a method \verb@newInstance@, which
returns the type contained in the scheme after all universally type
variables have been renamed to fresh variables.

The next class describes substitutions. A substitution is an
idempotent function from type variables to types. It maps a finite
number of given type variables to given types, and leaves all other
type variables unchanged. The meaning of a substitution is extended
point-wise to a mapping from types to types.

\begin{verbatim}
abstract class Subst extends Function1[Type,Type] {
  def lookup(x: Tyvar): Type;
  def apply(t: Type): Type = t match {
    case Tyvar(a) => val u = lookup(Tyvar(a)); if (t == u) t else apply(u);
    case Arrow(t1, t2) => Arrow(apply(t1), apply(t2))
    case Tycon(k, ts) => Tycon(k, ts map apply)
  }
  def extend(x: Tyvar, t: Type) = new Subst {
    def lookup(y: Tyvar): Type = if (x == y) t else outer.lookup(y);
  }
}
case class EmptySubst extends Subst { def lookup(t: Tyvar): Type = t }
\end{verbatim}
We represent substitutions as functions, of type
\verb@Type => Type@. To be an instance of this type, a
substitution \verb@s@ has to implement an \verb@apply@ method that takes a
\verb@Type@ as argument and yields another \verb@Type@ as result. A function
application \verb@s(t)@ is then interpreted as \verb@s.apply(t)@.

The \verb@lookup@ method is abstract in class \verb@Subst@.  Concrete
substitutions are defined by the case class \verb@EmptySubst@ and the
method \verb@extend@ in class \verb@Subst@.

The next class gives a naive implementation of sets using lists as the
implementation type. It implements methods \verb@contains@ for
membership tests as well as \verb@union@ and \verb@diff@ for set union
and difference. Alternatively, one could have used a more efficient
implementation of sets in some standard library.
\begin{verbatim}
class ListSet[a](xs: List[a]) {
  val elems: List[a] = xs;

  def contains(y: a): boolean = xs match {
    case [] => false
    case x :: xs1 => (x == y) || (xs1 contains y)
  }

  def union(ys: ListSet[a]): ListSet[a] = xs match {
    case [] => ys
    case x :: xs1 =>
      if (ys contains x) ListSet(xs1) union ys
      else ListSet(x :: (ListSet(xs1) union ys).elems)
  }

  def diff(ys: ListSet[a]): ListSet[a] = xs match {
    case [] => ListSet([])
    case x :: xs1 =>
      if (ys contains x) ListSet(xs1) diff ys
      else ListSet(x :: (ListSet(xs1) diff ys).elems)
  }
}
\end{verbatim}

We now present the type checker module. The type checker
computes a type for a given term in a given environment. Environments
associate variable names with type schemes. They are represented by a
type alias \verb@Env@ in module \verb@TypeChecker@:
\begin{verbatim}
module TypeChecker {

  /** Type environments are lists of bindings that associate a
   * name with a type scheme.
   */
  type Env = List[(String, TypeScheme)];
\end{verbatim}
There is also an exception \verb@TypeError@, which is thrown when type
checking fails. Exceptions are modeled as case classes that inherit
from the predefined \verb@Exception@ class.
\begin{verbatim}
  case class TypeError(msg: String) extends Exception(msg);
\end{verbatim}
The \verb@Exception@ class defines a \verb@throw@ method which causes
the exception to be thrown.

The \verb@TypeChecker@ module contains several utility
functions. Function
\verb@tyvars@ yields the set of free type variables of a type,
of a type scheme, of a list of types, or of an environment. Its
implementation is as four overloaded functions, one for each type of
argument.
\begin{verbatim}
  def tyvars(t: Type): ListSet[String] = t match {
    case Tyvar(a) => new ListSet([a])
    case Arrow(t1, t2) => tyvars(t1) union tyvars(t2)
    case Tycon(k, ts) => tyvars(ts)
  }
  def tyvars(ts: TypeScheme): ListSet[String] = ts match {
    case TypeScheme(as, t) => tyvars(t) diff new ListSet(as)
  }
  def tyvars(ts: List[Type]): ListSet[String] = ts match {
    case [] => new ListSet[String]([])
    case t :: ts1 => tyvars(t) union tyvars(ts1)
  }
  def tyvars(env: Env): ListSet[String] = env match {
    case [] => new ListSet[String]([])
    case (x, t) :: env1 => tyvars(t) union tyvars(env1)
  }
\end{verbatim}
The next utility function, \verb@lookup@, returns the type scheme
associated with a given variable name in the given environment, or
returns \verb@null@ if no binding for the variable exists in the environment.
\begin{verbatim}
  def lookup(env: Env, x: String): TypeScheme = env match {
    case [] => null
    case (y, t) :: env1 => if (x == y) t else lookup(env1, x)
  }
\end{verbatim}
The next utility function, \verb@gen@, returns the type scheme that
results from generalizing a given type in a given environment. This
means that all type variables that occur in the type but not in the
environment are universally quantified.
\begin{verbatim}
  def gen(env: Env, t: Type): TypeScheme =
    TypeScheme((tyvars(t) diff tyvars(env)).elems, t);
\end{verbatim}
The next utility function, \verb@mgu@, computes the most general
unifier of two given types $t$ and $u$ under a pre-existing
substitution $s$.  That is, it returns the most general
substitution $s'$ which extends $s$, and which makes $s'(t)$ and
$s'(u)$ equal types. The function throws a \verb@TypeError@ exception
if no such substitution exists. This can happen because the two types
have different type constructors at corresponding places, or because
a type variable is unified with a type that contains the type variable
itself.
\begin{verbatim}
  def mgu(t: Type, u: Type)(s: Subst): Subst = (s(t), s(u)) match {
    case (Tyvar( a), Tyvar(b)) if a == b =>
      s
    case (Tyvar(a), _) =>
      if (tyvars(u) contains a)
         TypeError("unification failure: occurs check").throw
      else s.extend(Tyvar(a), u)
    case (_, Tyvar(a)) =>
      mgu(u, t)(s)
    case (Arrow(t1, t2), Arrow(u1, u2)) =>
      mgu(t1, u1)(mgu(t2, u2)(s))
    case (Tycon(k1, ts), Tycon(k2, us)) if k1 == k2 =>
      (s :_foldl ((ts zip us) map (case (t,u) => mgu(t,u)))) { s, f => f(s) }
    case _ => TypeError("unification failure").throw
  }
\end{verbatim}
The main task of the type checker is implemented by function
\verb@tp@. This function takes as first parameters an environment $env$, a
term $e$ and a proto-type $t$. As a second parameter it takes a
pre-existing substitution $s$.  The function yields a substitution
$s'$ that extends $s$ and that
turns $s'(env) \ts e: s'(t)$ into a derivable type judgment according
to the derivation rules of the Hindley/Milner type system \cite{hindley-milner}.  A
\verb@TypeError@ exception is thrown if no such substitution exists.
\begin{verbatim}
  def tp(env: Env, e: Term, t: Type)(s: Subst): Subst = e match {
    case Var(x) => {
      val u = lookup(env, x);
      if (u == null) TypeError("undefined: x").throw
      else mgu(u.newInstance, t)(s)
    }
    case Lam(x, e1) => {
      val a = newTyvar, b = newTyvar;
      val s1 = mgu(t, Arrow(a, b))(s);
      val env1 = (x, TypeScheme([], a)) :: env;
      tp(env1, e1, b)(s1)
    }
    case App(e1, e2) => {
      val a = newTyvar;
      val s1 = tp(env, e1, Arrow(a, t))(s);
      tp(env, e2, a)(s1)
    }
    case Let(x, e1, e2) => {
      val a = newTyvar;
      val s1 = tp(env, e1, a)(s);
      tp((x, gen(env, s1(a))) :: env, e2, t)(s1)
    }
  }
\end{verbatim}
The next function, \verb@typeOf@ is a simplified facade for
\verb@tp@. It computes the type of a given term $e$ in a given
environment $env$. It does so by creating a fresh type variable \verb$a$,
computing a typing substitution that makes \verb@env $\ts$ e: a@ into
a derivable type judgment, and finally by returning the result of
applying the substitution to $a$.
\begin{verbatim}
  def typeOf(env: Env, e: Term): Type = {
    val a = newTyvar;
    tp(env, e, a)(EmptySubst)(a)
  }
}
\end{verbatim}
This concludes the presentation of the type inference system.
To apply the system, it is convenient to have a predefined environment
that contains bindings for commonly used constants. The module
\verb@Predefined@ defines an environment \verb@env@ that contains
bindings for booleans, numbers and lists together with some primitive
operations over these types. It also defines a fixed point operator
\verb@fix@, which can be used to represent recursion.
\begin{verbatim}
module Predefined {
  val booleanType = Tycon("Boolean", []);
  val intType = Tycon("Int", []);
  def listType(t: Type) = Tycon("List", [t]);

  private def gen(t: Type): TypeScheme = TypeChecker.gen([], t);
  private val a = newTyvar;
  val env = [
    ("true", gen(booleanType)),
    ("false", gen(booleanType)),
    ("$\mbox{\prog{if}}$", gen(Arrow(booleanType, Arrow(a, Arrow(a, a))))),
    ("zero", gen(intType)),
    ("succ", gen(Arrow(intType, intType))),
    ("$\mbox{\prog{nil}}$", gen(listType(a))),
    ("cons", gen(Arrow(a, Arrow(listType(a), listType(a))))),
    ("isEmpty", gen(Arrow(listType(a), booleanType))),
    ("head", gen(Arrow(listType(a), a))),
    ("tail", gen(Arrow(listType(a), listType(a)))),
    ("fix", gen(Arrow(Arrow(a, a), a)))
  ];
}
\end{verbatim}
Here's an example how the type inferencer is used.
Let's define a function \verb@showType@ which returns the type of
a given term computed in the predefined environment
\verb@Predefined.env@:
\begin{verbatim}
> def showType(e: Term) = TypeChecker.typeOf(Predefined.env, e);
\end{verbatim}
Then the application
\begin{verbatim}
> showType(Lam("x", App(App(Var("cons"), Var("x")), Var("$\mbox{\prog{nil}}$"))));
\end{verbatim}
would give the response
\begin{verbatim}
> TypeScheme([a0], Arrow(Tyvar(a0), Tycon("List", [Tyvar(a0)])));
\end{verbatim}

\exercise
Add \verb@toString@ methods to the data constructors of class
\verb@Type@ and \verb@TypeScheme@ which represent types in a more
natural way.

\chapter{Abstractions for Concurrency}\label{sec:ex-concurrency}

This section reviews common concurrent programming patterns and shows
how they can be implemented in Scala.

\section{Signals and Monitors}

\example
The {\em monitor} provides the basic means for mutual exclusion
of processes in Scala. It is defined as follows.
\begin{verbatim}
class Monitor {
  def synchronized [a] (def e: a): a;
}
\end{verbatim}
The \verb@synchronized@ method in class \verb@Monitor@ executes its
argument computation \verb@e@ in mutual exclusive mode -- at any one
time, only one thread can execute a \verb@synchronized@ argument of a
given monitor.

Threads can suspend inside a monitor by waiting on a signal.  The
\verb@Signal@ class offers two methods \verb@send@ and
\verb@wait@.  Threads that call the \verb@wait@ method wait until a
\verb@send@ method of the same signal is called subsequently by some
other thread. Calls to \verb@send@ with no threads waiting for the
signal are ignored. Here is the specification of the \verb@Signal@
class.
\begin{verbatim}
class Signal {
  def wait: unit;
  def wait(msec: long): unit;
  def notify: unit;
  def notifyAll: unit;
}
\end{verbatim}
A signal also implements a timed form of \verb@wait@, which blocks
only as long as no signal was received or the specified amount of time
(given in milliseconds) has elapsed. Furthermore, there is a
\verb@notifyAll@ method which unblocks all threads which wait for the
signal. \verb@Signal@ and \verb@Monitor@ are primitive classes in
Scala which are implemented in terms of the underlying runtime system.

As an example of how monitors and signals are used, here is is an
implementation of a bounded buffer class.
\begin{verbatim}
class BoundedBuffer[a](N: int) extends Monitor {
  var in = 0, out = 0, n = 0;
  val elems = new Array[a](N);
  val nonEmpty = new Signal;
  val nonFull = new Signal;
\end{verbatim}
\begin{verbatim}
  def put(x: a) = synchronized {
    if (n == N) nonFull.wait;
    elems(in) = x ; in = (in + 1) % N ; n = n + 1;
    if (n == 1) nonEmpty.send;
  }
\end{verbatim}
\begin{verbatim}
  def get: a = synchronized {
    if (n == 0) nonEmpty.wait
    val x = elems(out) ; out = (out + 1) % N ; n = n - 1;
    if (n == N - 1) nonFull.send;
    x
  }
}
\end{verbatim}
And here is a program using a bounded buffer to communicate between a
producer and a consumer process.
\begin{verbatim}
val buf = new BoundedBuffer[String](10)
fork { while (true) { val s = produceString ; buf.put(s) } }
fork { while (true) { val s = buf.get ; consumeString(s) } }
\end{verbatim}
The \verb@fork@ method spawns a new thread which executes the
expression given in the parameter. It can be defined as follows.
\begin{verbatim}
def fork(def e: unit) = {
  val p = new Thread { def run = e; }
  p.run
}
\end{verbatim}

\comment{
\section{Logic Variable}

A logic variable (or lvar for short) offers operations \verb@:=@
and \verb@value@ to define the variable and to retrieve its value.
Variables can be \verb@define@d only once. A call to \verb@value@
blocks until the variable has been defined.

Logic variables can be implemented as follows.

\begin{verbatim}
class LVar[a] extends Monitor {
  private val defined = new Signal
  private var isDefined: boolean = false
  private var v: a
  def value = synchronized {
    if (!isDefined) defined.wait
    v
  }
  def :=(x: a) = synchronized {
    v = x ; isDefined = true ; defined.send
  }
}
\end{verbatim}
}

\section{SyncVars}

A synchronized variable (or syncvar for short) offers \verb@get@ and
\verb@put@ operations to read and set the variable. \verb@get@ operations
block until the variable has been defined. An \verb@unset@ operation
resets the variable to undefined state.

Synchronized variables can be implemented as follows.
\begin{verbatim}
class SyncVar[a] extends Monitor {
  private val defined = new Signal;
  private var isDefined: boolean = false;
  private var value: a;
  def get = synchronized {
    if (!isDefined) defined.wait;
    value
  }
  def set(x: a) = synchronized {
    value = x ; isDefined = true ; defined.send;
  }
  def isSet: boolean =
    isDefined;
  def unset = synchronized {
    isDefined = false;
  }
}
\end{verbatim}

\section{Futures}
\label{sec:futures}

A {\em future} is a value which is computed in parallel to some other
client thread, to be used by the client thread at some future time.
Futures are used in order to make good use of parallel processing
resources.  A typical usage is:

\begin{verbatim}
val x = future(someLengthyComputation);
anotherLengthyComputation;
val y = f(x()) + g(x());
\end{verbatim}

Futures can be implemented in Scala as follows.

\begin{verbatim}
def future[a](def p: a): unit => a = {
  val result = new SyncVar[a];
  fork { result.set(p) }
  (=> result.get)
}
\end{verbatim}

The \verb@future@ method gets as parameter a computation \verb@p@ to
be performed. The type of the computation is arbitrary; it is
represented by \verb@future@'s type parameter \verb@a@.  The
\verb@future@ method defines a guard \verb@result@, which takes a
parameter representing the result of the computation. It then forks
off a new thread that computes the result and invokes the
\verb@result@ guard when it is finished. In parallel to this thread,
the function returns an anonymous function of type \verb@a@.
When called, this functions waits on the result guard to be
invoked, and, once this happens returns the result argument.
At the same time, the function reinvokes the \verb@result@ guard with
the same argument, so that future invocations of the function can
return the result immediately.

\section{Parallel Computations}

The next example presents a function \verb@par@ which takes a pair of
computations as parameters and which returns the results of the computations
in another pair. The two computations are performed in parallel.

\begin{verbatim}
def par[a, b](def xp: a, def yp: b): (a, b) = {
  val y = new SyncVar[a];
  fork { y.set(yp) }
  (xp, y)
}
\end{verbatim}

The next example presents a function \verb@replicate@ which performs a
number of replicates of a computation in parallel. Each
replication instance is passed an integer number which identifies it.

\begin{verbatim}
def replicate(start: int, end: int)(def p: int => unit): unit = {
  if (start == end) {
  } else if (start + 1 == end) {
    p(start)
  } else {
    val mid = (start + end) / 2;
    par ( replicate(start, mid)(p), replicate(mid, end)(p) )
  }
}
\end{verbatim}

The next example shows how to use \verb@replicate@ to perform parallel
computations on all elements of an array.

\begin{verbatim}
def parMap[a,b](f: a => b, xs: Array[a]): Array[b] = {
  val results = new Array[b](xs.length);
  replicate(0, xs.length) { i => results(i) = f(xs(i)) }
  results
}
\end{verbatim}

\section{Semaphores}

A common mechanism for process synchronization is a {\em lock} (or:
{\em semaphore}). A lock offers two atomic actions: \prog{acquire} and
\prog{release}. Here's the implementation of a lock in Scala:

\begin{verbatim}
class Lock extends Monitor with Signal {
  var available = true;
  def acquire = {
    if (!available) wait;
    available = false
  }
  def release = {
    available = true;
    notify
  }
}
\end{verbatim}

\section{Readers/Writers}

A more complex form of synchronization distinguishes between {\em
readers} which access a common resource without modifying it and {\em
writers} which can both access and modify it. To synchronize readers
and writers we need to implement operations \prog{startRead}, \prog{startWrite},
\prog{endRead}, \prog{endWrite}, such that:
\begin{itemize}
\item there can be multiple concurrent readers,
\item there can only be one writer at one time,
\item pending write requests have priority over pending read requests,
but don't preempt ongoing read operations.
\end{itemize}
The following implementation of a readers/writers lock is based on the
{\em message space} concept (see Section~\ref{sec:messagespace}).

\begin{verbatim}
class ReadersWriters {
  val m = new MessageSpace;
  private case class Writers(n: int), Readers(n: int);
  Writers(0); Readers(0);
  def startRead = m receive {
    case Writers(n) if n == 0 => m receive {
      case Readers(n) => Writers(0) ; Readers(n+1);
    }
  }
  def startWrite = m receive {
    case Writers(n) =>
      Writers(n+1);
      m receive { case Readers(n) if n == 0 => }
  }
\end{verbatim}
\begin{verbatim}
  def endRead = receive {
    case Readers(n) => Readers(n-1)
  }
  def endWrite = receive {
    case Writers(n) => Writers(n-1) ; if (n == 0) Readers(0)
  }
}
\end{verbatim}

\section{Asynchronous Channels}

A fundamental way of interprocess communication is the asynchronous
channel. Its implementation makes use the following class for linked
lists:
\begin{verbatim}
class LinkedList[a](x: a) {
  val elem: a = x;
  var next: LinkedList[a] = null;
}
\end{verbatim}
To facilitate insertion and deletion of elements into linked lists,
every reference into a linked list points to the node which precedes
the node which conceptually forms the top of the list.
Empty linked lists start with a dummy node, whose successor is \verb@null@.

The channel class uses a linked list to store data that has been sent
but not read yet. In the opposite direction, a signal \verb@moreData@ is
used to wake up reader threads that wait for data.
\begin{verbatim}
class Channel[a] {
  private val written = new LinkedList[a](null);
  private var lastWritten = written;
  private val moreData = new Signal;

  def write(x: a) = {
    lastWritten.next = new LinkedList(x);
    lastWritten = lastWritten.next;
    moreData.notify;
  }

  def read: a = {
    if (written.next == null) moreData.wait;
    written = written.next;
    written.elem;
  }
}
\end{verbatim}

\section{Synchronous Channels}

Here's an implementation of synchronous channels, where the sender of
a message blocks until that message has been received. Synchronous
channels only need a single variable to store messages in transit, but
three signals are used to coordinate reader and writer processes.
\begin{verbatim}
class SyncChannel[a] {
  val data = new SyncVar[a];

  def write(x: a): unit = synchronized {
    val empty = new Signal, full = new Signal, idle = new Signal;
    if (data.isSet) idle.wait;
    data.put(x);
    full.send;
    empty.wait;
    data.unset;
    idle.send;
  }

  def read: a = synchronized {
    if (!(data.isSet)) full.wait;
    x = data.get;
    empty.send;
    x
  }
}
\end{verbatim}

\section{Workers}

Here's an implementation of a {\em compute server} in Scala. The
server implements a \verb@future@ method which evaluates a given
expression in parallel with its caller. Unlike the implementation in
Section~\ref{sec:futures} the server computes futures only with a
predefined number of threads. A possible implementation of the server
could run each thread on a separate processor, and could hence avoid
the overhead inherent in context-switching several threads on a single
processor.

\begin{verbatim}
class ComputeServer(n: int) {
  private abstract class Job {
    abstract type t;
    def task: t;
    def return(x: t): unit;
  }

  private val openJobs = new Channel[Job]

  private def processor: unit = {
    while (true) {
      val job = openJobs.read;
      job.return(job.task)
    }
  }
\end{verbatim}
\begin{verbatim}
  def future[a](def p: a): () => a = {
    val reply = new SyncVar[a];
    openJobs.write(
      new Job {
	type t = a;
	def task = p;
	def return(x: a) = reply.set(x);
      }
    )
    (=> reply.get)
  }

  replicate(n){processor};
}
\end{verbatim}

Expressions to be computed (i.e. arguments
to calls of \verb@future@) are written to the \verb@openJobs@
channel. A {\em job} is an object with
\begin{itemize}
\item
An abstract type \verb@t@ which describes the result of the compute
job.
\item
A parameterless \verb@task@ method of type \verb@t@ which denotes
the expression to be computed.
\item
A \verb@return@ method which consumes the result once it is
computed.
\end{itemize}
The compute server creates $n$ \verb@processor@ processes as part of
its initialization.  Every such process repeatedly consumes an open
job, evaluates the job's \verb@task@ method and passes the result on
to the job's
\verb@return@ method. The polymorphic \verb@future@ method creates
a new job where the \verb@return@ method is implemented by a guard
named \verb@reply@ and inserts this job into the set of open jobs by
calling the \verb@isOpen@ guard. It then waits until the corresponding
\verb@reply@ guard is called.

The example demonstrates the use of abstract types. The abstract type
\verb@t@ keeps track of the result type of a job, which can vary
between different jobs. Without abstract types it would be impossible
to implement the same class to the user in a statically type-safe
way, without relying on dynamic type tests and type casts.

\section{Message Spaces}
\label{sec:messagespace}

Message spaces are high-level, flexible constructs for process
synchronization and communication. A {\em message} in this context is
an arbitrary object.  There is a special message \verb@TIMEOUT@ which
is used to signal a time-out.
\begin{verbatim}
case class TIMEOUT;
\end{verbatim}
Message spaces implement the following signature.
\begin{verbatim}
class MessageSpace {
  def send(msg: Any): unit;
  def receive[a](f: PartialFunction[Any, a]): a;
  def receiveWithin[a](msec: long)(f: PartialFunction[Any, a]): a;
}
\end{verbatim}
The state of a message space consists of a multi-set of messages.
Messages are added to the space using the \verb@send@ method. Messages
are removed using the \verb@receive@ method, which is passed a message
processor \verb@f@ as argument, which is a partial function from
messages to some arbitrary result type. Typically, this function is
implemented as a pattern matching expression. The \verb@receive@
method blocks until there is a message in the space for which its
message processor is defined.  The matching message is then removed
from the space and the blocked thread is restarted by applying the
message processor to the message. Both sent messages and receivers are
ordered in time. A receiver $r$ is applied to a matching message $m$
only if there is no other (message, receiver) pair which precedes $(m,
r)$ in the partial ordering on pairs that orders each component in
time.

As a simple example of how message spaces are used, consider a
one-place buffer:
\begin{verbatim}
class OnePlaceBuffer {
  private val m = new MessageSpace;        \=// An internal message space
  private case class Empty, Full(x: int);    \>// Types of messages we deal with

  m send Empty;                           \>// Initialization

  def write(x: int): unit =
    m receive { case Empty => m send Full(x) }

  def read: int =
    m receive { case Full(x) => m send Empty ; x }
}
\end{verbatim}
Here's how the message space class can be implemented:
\begin{verbatim}
class MessageSpace {

  private abstract class Receiver extends Signal {
    def isDefined(msg: Any): boolean;
    var msg = null;
  }
\end{verbatim}
We define an internal class for receivers with a test method
\verb@isDefined@, which indicates whether the receiver is
defined for a given message.  The receiver inherits from class
\verb@Signal@ a \verb@notify@ method which is used to wake up a
receiver thread. When the receiver thread is woken up, the message it
needs to be applied to is stored in the \verb@msg@ variable of
\verb@Receiver@.
\begin{verbatim}
  private val sent = new LinkedList[Any](null) ;
  private var lastSent = sent;
  private var receivers = new LinkedList[Receiver](null);
  private var lastReceiver = receivers;
\end{verbatim}
The message space class maintains two linked lists,
one for sent but unconsumed messages, the other for waiting receivers.
\begin{verbatim}
  def send(msg: Any): unit = synchronized {
    var r = receivers, r1 = r.next;
    while (r1 != null && !r1.elem.isDefined(msg)) {
      r = r1; r1 = r1.next;
    }
    if (r1 != null) {
      r.next = r1.next; r1.elem.msg = msg; r1.elem.notify;
    } else {
      l = new LinkedList(msg); lastSent.next = l; lastSent = l;
    }
  }
\end{verbatim}
The \verb@send@ method first checks whether a waiting receiver is

applicable to the sent message. If yes, the receiver is notified.
Otherwise, the message is appended to the linked list of sent messages.
\begin{verbatim}
  def receive[a](f: PartialFunction[Any, a]): a = {
    val msg: Any = synchronized {
      var s = sent, s1 = s.next;
      while (s1 != null && !f.isDefined(s1.elem)) {
	s = s1; s1 = s1.next
      }
      if (s1 != null) {
        s.next = s1.next; s1.elem
      } else {
	val r = new LinkedList(
          new Receiver {
            def isDefined(msg: Any) = f.isDefined(msg);
          });
	lastReceiver.next = r; lastReceiver = r;
	r.elem.wait;
	r.elem.msg
      }
    }
    f(msg)
  }
\end{verbatim}
The \verb@receive@ method first checks whether the message processor function
\verb@f@ can be applied to a message that has already been sent but that
was not yet consumed. If yes, the thread continues immediately by
applying \verb@f@ to the message. Otherwise, a new receiver is created
and linked into the \verb@receivers@ list, and the thread waits for a
notification on this receiver. Once the thread is woken up again, it
continues by applying \verb@f@ to the message that was stored in the receiver.

The message space class also offers a method \verb@receiveWithin@
which blocks for only a specified maximal amount of time.  If no
message is received within the specified time interval (given in
milliseconds), the message processor argument $f$ will be unblocked
with the special \verb@TIMEOUT@ message.  The implementation of
\verb@receiveWithin@ is quite similar to \verb@receive@:
\begin{verbatim}
  def receiveWithin[a](msec: long)(f: PartialFunction[Any, a]): a = {
    val msg: Any = synchronized {
      var s = sent, s1 = s.next;
      while (s1 != null && !f.isDefined(s1.elem)) {
	s = s1; s1 = s1.next ;
      }
      if (s1 != null) {
        s.next = s1.next; s1.elem
      } else {
	val r = new LinkedList(
          new Receiver {
            def isDefined(msg: Any) = f.isDefined(msg);
          }
        )
	lastReceiver.next = r; lastReceiver = r;
	r.elem.wait(msec);
        if (r.elem.msg == null) r.elem.msg = TIMEOUT;
	r.elem.msg
      }
    }
    f(msg)
  }
} // end MessageSpace
\end{verbatim}
The only differences are the timed call to \verb@wait@, and the
statement following it.

\section{Actors}
\label{sec:actors}

Chapter~\ref{sec:ex-auction} sketched as a program example the
implementation of an electronic auction service. This service was
based on high-level actor processes, that work by inspecting messages
in their mailbox using pattern matching. An actor is simply a thread
whose communication primitives are those of a message space.
Actors are therefore defined by a mixin composition of threads and message spaces.
\begin{verbatim}
abstract class Actor extends Thread with MessageSpace;
\end{verbatim}

\comment{
As an extended example of an application that uses actors, we come
back to the auction server example of Section~\ref{sec:ex-auction}.
The following code implements:

\begin{figure}[thb]
\begin{verbatim}
class AuctionMessage;
case class
  \=Offer(bid: int, client: Process),                             \=// make a bid
     \>Inquire(client: Process) extends AuctionMessage  \>// inquire status

class AuctionReply;
case class
  \=Status(asked; int, expiration: Date),   \>// asked sum, expiration date
     \>BestOffer,                                       \>// yours is the best offer
     \>BeatenOffer(maxBid: int),                        \>// offer beaten by maxBid
     \>AuctionConcluded(seller: Process, client: Process), \>// auction concluded
     \>AuctionFailed                                     \>// failed with no bids
     \>AuctionOver extends AuctionReply                  \>// bidding is closed
\end{verbatim}
\end{figure}

\begin{verbatim}
class Auction(seller: Process, minBid: int, closing: Date)
 extends Process {

  val timeToShutdown = 36000000 // msec
  val delta = 10                // bid increment
\end{verbatim}
\begin{verbatim}
  def run = {
    var askedBid = minBid
    var maxBidder: Process = null
    while (true) {
      receiveWithin ((closing - Date.currentDate).msec) {
	case Offer(bid, client) => {
	  if (bid >= askedBid) {
            if (maxBidder != null && maxBidder != client) {
              maxBidder send BeatenOffer(bid)
            }
            maxBidder = client
            askedBid = bid + delta
            client send BestOffer
          } else client send BeatenOffer(maxBid)
        }
\end{verbatim}
\begin{verbatim}
	case Inquire(client) => {
	  client send Status(askedBid, closing)
        }
\end{verbatim}
\begin{verbatim}
	case TIMEOUT => {
	  if (maxBidder != null) {
	    val reply = AuctionConcluded(seller, maxBidder)
	    maxBidder send reply
	    seller send reply
	  } else seller send AuctionFailed
          receiveWithin (timeToShutdown) {
            case Offer(_, client) => client send AuctionOver ; discardAndContinue
            case _ => discardAndContinue
            case TIMEOUT => stop
          }
        }
\end{verbatim}
\begin{verbatim}
        case _ => discardAndContinue
      }
    }
  }
\end{verbatim}
\begin{verbatim}
  def houseKeeping: int = {
    val Limit = 100
    var nWaiting: int = 0
    receiveWithin(0) {
      case _ =>
        nWaiting = nWaiting + 1
        if (nWaiting > Limit) {
	  receiveWithin(0) {
            case Offer(_, _) => continue
            case TIMEOUT =>
            case _ => discardAndContinue
          }
        } else continue
      case TIMEOUT =>
    }
  }
}
\end{verbatim}
\begin{verbatim}
class Bidder (auction: Process, minBid: int, maxBid: int)
 extends Process {
  val MaxTries = 3
  val Unknown = -1

  var nextBid = Unknown
\end{verbatim}
\begin{verbatim}
  def getAuctionStatus = {
    var nTries = 0
    while (nextBid == Unknown && nTries < MaxTries) {
      auction send Inquiry(this)
      nTries = nTries + 1
      receiveWithin(waitTime) {
        case Status(bid, _) => bid match {
          case None => nextBid = minBid
          case Some(curBid) => nextBid = curBid + Delta
        }
        case TIMEOUT =>
        case _ => continue
      }
    }
    status
  }
\end{verbatim}
\begin{verbatim}
  def bid: unit = {
    if (nextBid < maxBid) {
      auction send Offer(nextBid, this)
      receive {
        case BestOffer =>
          receive {
            case BeatenOffer(bestBid) =>
              nextBid = bestBid + Delta
              bid
            case AuctionConcluded(seller, client) =>
     	      transferPayment(seller, nextBid)
            case _ => continue
	  }

        case BeatenOffer(bestBid) =>
          nextBid = nextBid + Delta
          bid

        case AuctionOver =>

        case _ => continue
      }
    }
  }
\end{verbatim}
\begin{verbatim}
  def run = {
    getAuctionStatus
    if (nextBid != Unknown) bid
  }

  def transferPayment(seller: Process, amount: int)
}
\end{verbatim}
}
%\todo{We also need some XML examples.}
\end{document}



  case ([], _) => ys
  case (_, []) => xs
  case (x :: xs1, y :: ys1) =>
    if (x < y) x :: merge(xs1, ys) else y :: merge(xs, ys1)
}

def split (xs: List[a]): (List[a], List[a]) = xs match {
  case [] => ([], [])
  case [x] => (x, [])
  case y :: z :: xs1 => val (ys, zs) = split(xs1) ; (y :: ys, z :: zs)
}

def sort(xs: List[a]): List[a] = {
  val (ys, zs) = split(xs)
  merge(sort(ys), sort(zs))
}


def sort(a:Array[String]): Array[String] = {
  val pivot = a(a.length / 2)
  sort(a.filter(x => x < pivot)) ++
       a.filter(x => x == pivot) ++
  sort(a.filter(x => x > pivot))
}

def sort(a:Array[String]): Array[String] = {

  def swap (i: int, j: int): unit = {
    val t = a(i) ; a(i) = a(j) ; a(j) = t
  }

  def sort1(l: int, r: int): unit = {
    val pivot = a((l + r) / 2)
    var i = l, j = r
    while (i <= r) {
      while (i < r && a(i) < pivot) { i = i + 1 }
      while (j > l && a(j) > pivot) { j = j - 1 }
      if (i <= j) {
        swap(i, j)
        i = i + 1
        j = j - 1
      }
    }
    if (l < j) sort1(l, j)
    if (j < r) sort1(i, r)
  }

  sort1(0, a.length - 1)
}

class Array[a] {

  def copy(to: Array[a], src: int, dst: int, len: int): unit
  val length: int
  val apply(i: int): a
  val update(i: int, x: a): unit

  def filter (p: a => boolean): Array[a] = {
    val temp = new Array[a](a.length)
    var i = 0, j = 0
    for (i < a.length, i = i + 1) {
      val x = a(i)
      if (p(x)) { temp(j) = x; j = j + 1 }
    }
    val res = new Array[a](j)
    temp.copy(res, 0, 0, j)
  }

  def ++ (that: Array[a]): Array[a] = {
    val a = new Array[a](this.length + that.length)
    this.copy(a, 0, 0, this.length)
    that.copy(a, 0, this.length, that.length)
  }

static

  def concat [a] (as: List[Array[a]]) = {
    val l = (as map (a => a.length)).sum
    val dst = new Array[a](l)
    var j = 0
    as forall {a => { a.copy(dst, j, a.length) ; j = j + a.length }}
    dst
  }

}

module ABT extends AlgBinTree[kt, vt]
ABT.Map
