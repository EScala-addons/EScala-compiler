\renewcommand{\todo}[1]{}
\newcommand{\notyet}[1]{\footnote{#1 not yet implemented.}}
\newcommand{\Ts}{\mbox{\sl Ts}}
\newcommand{\tps}{\mbox{\sl tps}}
\newcommand{\psig}{\mbox{\sl psig}}
\newcommand{\fsig}{\mbox{\sl fsig}}
\newcommand{\csig}{\mbox{\sl csig}}
\newcommand{\args}{\mbox{\sl args}}
\newcommand{\targs}{\mbox{\sl targs}}
\newcommand{\enums}{\mbox{\sl enums}}
\newcommand{\proto}{\mbox{\sl pt}}
\newcommand{\argtypes}{\mbox{\sl Ts}}
\newcommand{\stats}{\mbox{\sl stats}}
\newcommand{\overload}{\la\mbox{\sf and}\ra}
\newcommand{\op}{\mbox{\sl op}}

\newcommand{\ifqualified}[1]{}
\newcommand{\iflet}[1]{}
\newcommand{\ifundefvar}[1]{}
\newcommand{\iffinaltype}[1]{}
\newcommand{\ifpackaging}[1]{}
\newcommand{\ifnewfor}[1]{}

\newcommand{\U}[1]{\mbox{$\backslash$u{#1}}}
\newcommand{\Urange}[2]{\mbox{$\backslash$u{#1}-$\backslash$u{#2}}}
%\newcommand{\U}[1]{\mbox{U+{#1}}}

\chapter{Lexical Syntax}

Scala programs are written using the Unicode character set.
This chapter defines the two modes of Scala's lexical syntax, the
Scala mode and the \textsc{Xml} mode. If not otherwise mentioned, the following 
descriptions of Scala tokens refer to Scala mode, and literal characters `c' refer 
to the ASCII fragment \Urange{0000}{007F}. 

In Scala mode, \textit{Unicode escapes} are replaced by the corresponding
Unicode character with the given hexadecimal code.
\begin{lstlisting}
UnicodeEscape ::= \\{\\\\}u{u} HexDigit HexDigit HexDigit HexDigit
HexDigit      ::= '0' | $\ldots$ | `9' | `A' | $\ldots$ | `F' | `a' | $\ldots$ | `f' |
\end{lstlisting}
To construct tokens, characters are distinguished according to the following classes 
(Unicode general category given in parentheses):
\begin{enumerate}
\item Whitespace characters. \U{0020} | \U{0009} | \U{000D} | \U{000A}
\item Letters, which include lower case letters(Ll), upper case letters(Lu), titlecase letters(Lt), other letters(Lo), letter numerals(Nl) and the 
two characters \U{0024} ~\lstinline@`$\Dollar$'@ and \U{005F} ~\lstinline@`_'@, which
both count as upper case letters
\item Digits ~\lstinline@`0' | $\ldots$ | `9'@.
\item Parentheses ~\lstinline@`(' | `)' | `[' | `]' | `{' | `}'@.
\item Delimiter characters ~\lstinline@``' | `'' | `"' | `.' | `;' | `,'@.
\item Operator characters. These consist of all printable ASCII characters \Urange{0020}{007F}. 
which are in none of the sets above, mathematical symbols(Sm) and other symbols(So).
\end{enumerate}
\newpage
\section{Identifiers}\label{sec:idents}

\syntax\begin{lstlisting}
op        ::=  special {special}
varid     ::=  lower idrest
id        ::=  upper idrest
            |  varid
            |  op
            |  ```string chars`''
idrest    ::= {letter $|$ digit} {'_' (op | idrest)}
\end{lstlisting}

There are three ways to form an identifier. First, an identifier can
start with a letter which can be followed by an arbitrary sequence of
letters and digits. This may be followed by underscore `\lstinline@_@'
characters and other string composed of either letters and digits or
of special characeters.  Second, an identifier can start with a
special character followed by an arbitrary sequence of special
characters.  Finally, an identifier may also be formed by an arbitrary
string between back-quotes (host systems may impose some restrictions
on which strings are legal for identifiers).  As usual, a longest
match rule applies. For instance, the string

\begin{lstlisting}
big_bob++=z3
\end{lstlisting}

decomposes into the three identifiers \lstinline@big_bob@, \lstinline@++=@, and
\code{z3}.  The rules for pattern matching further distinguish between
{\em variable identifiers}, which start with a lower case letter, and
{\em constant identifiers}, which do not.


The `\lstinline[mathescape=false]@$@'\comment{$} character is reserved for compiler-synthesized identifiers.
User programs are not allowed to define identifiers which contain `\lstinline[mathescape=false]@$@'\comment{$}
characters. 

The following names are reserved words instead of being members of the
syntactic class \code{id} of lexical identifiers.

\begin{lstlisting}
abstract    case    catch    class    def    
do    else    extends    false    final    
finally    for    if    import    new    
null    object    override    package    private    
protected    return    sealed    super    this    
throw    trait    try    true    type    
val    var    while    with   yield
_    :    =    =>    <-    <:    >:    #    @
\end{lstlisting}

The Unicode operator \U{21D2} `$\Rightarrow$' has the ASCII equivalent
`$=>$', which is also reserved.

\example
Here are examples of identifiers:
\begin{lstlisting}
    x    Object        maxIndex        p2p      empty_?
    +    +_field       $\alpha\rho\epsilon\tau\eta$
\end{lstlisting}

\section{Braces and Semicolons}

A semicolon `\lstinline@;@' is implicitly inserted after every closing brace
if there is a new line character between closing brace and the next
regular token after it, except if that token cannot legally start a
statement.

The tokens which cannot legally start a statement
are the following delimiters and reserved words:
\begin{lstlisting}
catch    else    extends    finally    with    yield
,    .    ;    :    =    =>    <-    <:    <%    >:    #    @    )    ]    }
\end{lstlisting}

\section{Literals}

There are literals for integer numbers (of types \code{Int} and \code{Long}),
floating point numbers (of types \code{Float} and \code{Double}), characters, and
strings.  The syntax of these literals is in each case as in Java.

\syntax\begin{lstlisting}
intLit       ::=  $\mbox{\rm\em ``as in Java''}$
floatLit     ::=  $\mbox{\rm\em ``as in Java''}$
charLit      ::=  $\mbox{\rm\em ``as in Java''}$
stringLit    ::=  $\mbox{\rm\em ``as in Java''}$
\end{lstlisting}

\section{Whitespace and Comments}

Tokens may be separated by whitespace characters
and/or comments. Comments come in two forms:

A single-line comment is a sequence of characters which starts with
\lstinline@//@ and extends to the end of the line.

A multi-line comment is a sequence of characters between \lstinline@/*@ and
\lstinline@*/@. Multi-line comments may be nested.

\section{XML mode\label{sec::xmlMode}}

In order to allow literal inclusion of XML fragments, lexical analysis
switches from Scala mode to XML mode when encountering an opening
angle bracket '<' in the following circumstance: The '<' must be
preceded either by whitespace, an opening parenthesis or an opening
brace and immediately followed by a character starting an XML name.

\syntax\begin{lstlisting}
 ( whitespace | '(' | '{' ) '<' XNameStart

  XNameStart ::= `_' | BaseChar | Ideographic $\mbox{\rm\em (as in W3C XML, but without }$ `:'
\end{lstlisting}

The scanner switches from XML mode to Scala mode if either
\begin{itemize}
\item the XML expression or the XML pattern started by the initial '<' has been 
successfully parsed, or if

\item the parser encounters an embedded Scala expression or pattern and 
forces the Scanner 
back to normal mode, until the Scala expression or pattern is
successully parsed. In this case, since code and XML fragments can be
nested, the parser has to maintain a stack that reflects the nesting
of XML and Scala expressions adequately.
\end{itemize}

Note that no Scala tokens are constructed in XML mode, and that comments are interpreted
as text.

\chapter{\label{sec:names}Identifiers, Names and Scopes}

Names in Scala identify types, values, methods, and classes which
are collectively called {\em entities}.  Names are introduced by
definitions, declarations (\sref{sec:defs}) or import clauses
(\sref{sec:import}), which are collectively called {\em binders}.

There are two different name spaces, one for types (\sref{sec:types})
and one for terms (\sref{sec:exprs}).  The same name may designate a
type and a term, depending on the context where the name is used.  

A definition or declaration has a {\em scope} in which the entity
defined by a single name can be accessed using a simple name. Scopes
are nested, and a definition or declaration in some inner scope {\em
shadows} a definition in an outer scope that contributes to the same
name space. Furthermore, a definition or declaration shadows bindings
introduced by a preceding import clause, even if the import clause is
in the same block. Import clauses, on the other hand, only shadow
bindings introduced by other import clauses in outer blocks.

A reference to an unqualified (type- or term-) identifier $x$ is bound
by the unique binder, which
\begin{itemize}
\item defines an entity with name $x$ in the same namespace as the
identifier, and
\item shadows all other binders that define entities with name $x$ in that namespace.
\end{itemize}
It is an error if no such binder exists.  If $x$ is bound by an import
clause, then the simple name $x$ is taken to be equivalent to the
qualified name to which $x$ is mapped by the import clause. If $x$ is bound by a definition or declaration,
then $x$ refers to the entity introduced by that
binder. In that case, the type of $x$ is the type of the referenced
entity.

\example Consider the following nested definitions and imports:

\begin{lstlisting}
object m1 {
  object m2 { val x: int = 1; val y: int = 2 }
  object m3 { val x: boolean = true; val y: String = "" }
  val x: int = 3;              
  { import m2._;            // shadows nothing
                            //   reference to `x' is ambiguous here
    val x: String = "abc";  // shadows preceding import
                            //   name `x' refers to latest val definition
    { import m3._           // shadows only preceding import m2
                            // reference to `x' is ambiguous here
                            //   name `y' refers to latest import clause
    }
  }
} 
\end{lstlisting}

A reference to a qualified (type- or term-) identifier $e.x$ refers to
the member of the type $T$ of $e$ which has the name $x$ in the same
namespace as the identifier. It is an error if $T$ is not a value type
(\sref{sec:value-types}). The type of $e.x$ is the member type of the
referenced entity in $T$.

\chapter{\label{sec:types}Types}

\syntax\begin{lstlisting}
  Type          ::=  Type1 `=>' Type
                  |  `(' [Types] `)' `=>' Type
                  |  Type1
  Type1         ::=  SimpleType {with SimpleType} [Refinement]
  SimpleType    ::=  StableId
                  |  SimpleType `#' id
                  |  Path `.' type
                  |  SimpleType TypeArgs
                  |  `(' Type ')'
  Types         ::=  Type {`,' Type}
\end{lstlisting}

We distinguish between first-order types and type constructors, which
take type parameters and yield types. A subset of first-order types
called {\em value types} represents sets of (first-class) values.
Value types are either {\em concrete} or {\em abstract}. Every
concrete value type can be represented as a {\em class type}, i.e.\ a
type designator (\sref{sec:type-desig}) that refers to a 
class\footnote{We assume that objects and packages also
implicitly define a class (of the same name as the object or package,
but inaccessible to user programs).} (\sref{sec:classes}), 
or as a {\em compound type} (\sref{sec:compound-types}) 
consisting of class types and possibly
also a refinement (\sref{sec:refinements}) that further constrains the
types of its members.

A shorthand exists for denoting function types
(\sref{sec:function-types}).  Abstract value types are introduced by
type parameters and abstract type bindings (\sref{sec:typedcl}).
Parentheses in types are used for grouping.

Non-value types capture properties of
identifiers that are not values
(\sref{sec:synthetic-types}).  There is no syntax to express these
types directly in Scala.

\section{Paths}\label{sec:paths}\label{sec:stable-ids}

\syntax\begin{lstlisting}
  StableId        ::=  id
                    |  Path `.' id 
                    |  [id '.'] super [`[' id `]'] `.' id
  Path            ::=  StableId
                    |  [id `.'] this
\end{lstlisting}

Paths are not types themselves, but they can be a part of named types
and in that way form a central role in Scala's type system.

A path is one of the following.
\begin{itemize}
\item
The empty path $\epsilon$ (which cannot be written explicitly in user programs).
\item
\lstinline@$C$.this@, where $C$ references a class. 
The path \code{this} is taken as a shorthand for \lstinline@$C$.this@ where 
$C$ is the name of the class directly enclosing the reference. 
\item
\lstinline@$p$.$x$@ where $p$ is a path and $x$ is a stable member of $p$.
{\em Stable members} are members introduced by value or object
definitions, as well as packages.
\item
\lstinline@$C$.super.$x$@ or \lstinline@$C$.super[$M\,$].$x$@
where $C$ references a class and $x$ references a 
stable member of the super class or designated mixin class $M$ of $C$. 
The prefix \code{super} is taken as a shorthand for \lstinline@$C$.super@ where 
$C$ is the name of the class directly enclosing the reference. 
\end{itemize}
A {\em stable identifier} is a path which ends in an identifier.

\section{Value Types}\label{sec:value-types}

\subsection{Singleton Types}
\label{sec:singleton-type}

\syntax\begin{lstlisting}
  SimpleType  ::=  Path `.' type
\end{lstlisting}

A singleton type is of the form \lstinline@$p$.type@, where $p$ is a
path pointing to a value expected to conform to
\lstinline@scala.AnyRef@.  The type denotes the set of values
consisting of the value denoted by $p$ and \lstinline@null@.

\subsection{Type Projection}
\label{sec:type-project}

\syntax\begin{lstlisting} 
SimpleType  ::=  SimpleType `#' id
\end{lstlisting}

A type projection \lstinline@$T$#$x$@ references the type member named 
$x$ of type $T$. $T$ must be either a singleton type,
or a non-abstract class type, or a Java class type (in either of the
last two cases, it is guaranteed that $T$ has no abstract type
members).

\subsection{Type Designators}
\label{sec:type-desig}

\syntax\begin{lstlisting}
  SimpleType  ::=  StableId
\end{lstlisting}

A type designator refers to a named value type. It can be simple or
qualified. All such type designators are shorthands for type projections.

Specifically, the unqualified type name $t$ where $t$ is bound in some
class, object, or package $C$ is taken as a shorthand for
\lstinline@$C$.this.type#$t$@.  If $t$ is not bound in a class, object, or
package, then $t$ is taken as a shorthand for
\lstinline@$\epsilon$.type#$t$@.

A qualified type designator has the form \lstinline@$p$.$t$@ where $p$ is
a path (\sref{sec:paths}) and $t$ is a type name. Such a type designator is
equivalent to the type projection \lstinline@$p$.type#$x$@.

\example 
Some type designators and their expansions are listed below. We assume
a local type parameter $t$, a value \code{maintable}
with a type member \code{Node} and the standard class \lstinline@scala.Int@, 
\begin{lstlisting}
  t                     $\epsilon$.type#t
  Int                   scala.type#Int
  scala.Int             scala.type#Int
  data.maintable.Node   data.maintable.type#Node
\end{lstlisting}

\subsection{Parameterized Types}
\label{sec:param-types}

\syntax\begin{lstlisting}
  SimpleType      ::=  SimpleType TypeArgs
  TypeArgs        ::=  `[' Types `]'
\end{lstlisting}

A parameterized type $T[U_1 \commadots U_n]$ consists of a type designator
$T$ and type parameters $U_1 \commadots U_n$ where $n \geq 1$.  $T$
must refer to a type constructor which takes $n$ type parameters $a_1 \commadots a_n$ 
with lower bounds $L_1 \commadots L_n$ and upper bounds $U_1 \commadots U_n$.

The parameterized type is well-formed if each actual type parameter
{\em conforms to its bounds}, i.e.\ $L_i\sigma <: T_i <: U_i\sigma$ where $\sigma$
is the substitution $[a_1 := T_1 \commadots a_n := T_n]$.

\example\label{ex:param-types}
Given the partial type definitions:

\begin{lstlisting}
  class TreeMap[a <: Ord[a], b] { $\ldots$ }
  class List[a] { $\ldots$ }
  class I extends Ord[I] { $\ldots$ }
\end{lstlisting}

the following parameterized types are well formed:

\begin{lstlisting}
  TreeMap[I, String]
  List[I]
  List[List[Boolean]]
\end{lstlisting}

\example Given the type definitions of \ref{ex:param-types},
the following types are ill-formed:

\begin{lstlisting}
  TreeMap[I]                // illegal: wrong number of parameters
  TreeMap[List[I], Boolean] // illegal: type parameter not within bound
\end{lstlisting}

\subsection{Compound Types}
\label{sec:compound-types}
\label{sec:refinements}

\syntax\begin{lstlisting}
  Type            ::=  SimpleType {with SimpleType} [Refinement]
  Refinement      ::=  `{' [RefineStat {`;' RefineStat}] `}'
  RefineStat      ::=  Dcl
                    |  type TypeDef
                    |
\end{lstlisting}

A compound type ~\lstinline@$T_1$ with $\ldots$ with $T_n$ {$R\,$}@~  represents
objects with members as given in the component types $T_1 \commadots
T_n$ and the refinement \lstinline@{$R\,$}@. Each component type $T_i$ must be a
class type \todo{Relax for first?}. A
refinement \lstinline@{$R\,$}@ contains declarations and type
definitions. Each declaration or definition in a refinement must
override a declaration or definition in one of the component types
$T_1 \commadots T_n$. The usual rules for overriding (\sref{sec:overriding})
apply. If no refinement is given, the empty refinement is implicitly
added, i.e. ~\lstinline@$T_1$ with $\ldots$ with $T_n$@~ is a shorthand for
~\lstinline@$T_1$ with $\ldots$ with $T_n$ {}@.
 
\subsection{Function Types}
\label{sec:function-types}

\syntax\begin{lstlisting}
  SimpleType     ::=  Type1 `=>' Type
                   |  `(' [Types] `)' `=>' Type
\end{lstlisting}
The type ~\lstinline@($T_1 \commadots T_n$) => $U$@~ represents the set of function
values that take arguments of types $T_1 \commadots T_n$ and yield
results of type $U$.  In the case of exactly one argument type
~\lstinline@$T$ => $U$@~ is a shorthand for ~\lstinline@($T\,$) => $U$@.  Function types
associate to the right, e.g.~\lstinline@($S\,$) => ($T\,$) => $U$@~ is the same as
~\lstinline@($S\,$) => (($T\,$) => $U\,$)@.

Function types are shorthands for class types that define \code{apply}
functions.  Specifically, the $n$-ary function type 
~\lstinline@($T_1 \commadots T_n$) => U@~ is a shorthand for the class type
\lstinline@Function$n$[$T_1 \commadots T_n$,$U\,$]@. Such class
types are defined in the Scala library for $n$ between 0 and 9 as follows.
\begin{lstlisting}
package scala;
trait Function$n$[-$T_1 \commadots$ -$T_n$, +$R$] {
  def apply($x_1$: $T_1 \commadots x_n$: $T_n$): $R$;
  override def toString() = "<function>";
}
\end{lstlisting}
Hence, function types are covariant in their result type, and
contravariant in their argument types.

\section{Non-Value Types}
\label{sec:synthetic-types}

The types explained in the following do not denote sets of values, nor
do they appear explicitly in programs. They are introduced in this
report as the internal types of defined identifiers.

\subsection{Method Types}
\label{sec:method-types}

A method type is denoted internally as $(\Ts)U$, where $(\Ts)$ is a
sequence of types $(T_1 \commadots T_n)$ for some $n \geq 0$
and $U$ is a (value or method) type.  This type represents named
methods that take arguments of types $T_1 \commadots T_n$ 
and that return a result of type $U$.

Method types associate to the right: $(\Ts_1)(\Ts_2)U$ is treated as
$(\Ts_1)((\Ts_2)U)$.

A special case are types of methods without any parameters. They are
written here \lstinline@=> T@. Parameterless methods name expressions
that are re-evaluated each time the parameterless method name is
referenced.

Method types do not exist as types of values. If a method name is used
as a value, its type is implicitly converted to a corresponding
function type (\sref{sec:impl-conv}).

\example The declarations
\begin{lstlisting}
def a: Int
def b (x: Int): Boolean
def c (x: Int) (y: String, z: String): String
\end{lstlisting}
produce the typings
\begin{lstlisting}
a: => Int
b: (Int) Boolean
c: (Int) (String, String) String
\end{lstlisting}

\subsection{Polymorphic Method Types}
\label{sec:poly-types}

A polymorphic method type is denoted internally as ~\lstinline@[$\tps\,$]$T$@~ where
\lstinline@[$\tps\,$]@ is a type parameter section 
~\lstinline@[$a_1$ >: $L_1$ <: $U_1 \commadots a_n$ >: $L_n$ <: $U_n$]@~ 
for some $n \geq 0$ and $T$ is a
(value or method) type.  This type represents named methods that
take type arguments ~\lstinline@$S_1 \commadots S_n$@~ which
conform (\sref{sec:param-types}) to the lower bounds
~\lstinline@$L_1 \commadots L_n$@~ and the upper bounds
~\lstinline@$U_1 \commadots U_n$@~ and that yield results of type $T$.

\example The declarations
\begin{lstlisting}
def empty[a]: List[a]
def union[a <: Comparable[a]] (x: Set[a], xs: Set[a]): Set[a]
\end{lstlisting}
produce the typings
\begin{lstlisting}
empty : [a >: All <: Any] List[a]
union : [a >: All <: Comparable[a]] (x: Set[a], xs: Set[a]) Set[a]  .
\end{lstlisting}

\comment{
\subsection{Overloaded Types}
\label{sec:overloaded-types}

More than one values or methods are defined in the same scope with the
same name, we model

An overloaded type consisting of type alternatives $T_1 \commadots
T_n (n \geq 2)$ is denoted internally $T_1 \overload \ldots \overload T_n$.

\example The definitions
\begin{lstlisting}
def println: unit;
def println(s: string): unit = $\ldots$;
def println(x: float): unit = $\ldots$;
def println(x: float, width: int): unit = $\ldots$;
def println[a](x: a)(tostring: a => String): unit = $\ldots$
\end{lstlisting}
define a single function \code{println} which has an overloaded
type.
\begin{lstlisting}
println:  => unit $\overload$
          (String) unit $\overload$
          (float) unit $\overload$
          (float, int) unit $\overload$
          [a] (a) (a => String) unit
\end{lstlisting}

\example The definitions
\begin{lstlisting}
def f(x: T): T = $\ldots$;
val f = 0
\end{lstlisting}
define a function \code{f} which has type ~\lstinline@(x: T)T $\overload$ Int@.
}

\section{Base Classes and Member Definitions}
\label{sec:base-classes-member-defs}

Types, bounds and base classes of class members depend on the way the
members are referenced.  Central here are three notions, namely:
\begin{enumerate}
\item the notion of the set of base classes of a type $T$,
\item the notion of a type $T$ in some class $C$ seen from some 
      prefix type $S$,
\item the notion of a member binding of some type $T$.
\end{enumerate}
These notions are defined mutually recursively as follows.

1. The set of {\em base classes} of a type is a set of class types, 
given as follows.
\begin{itemize}
\item
The base classes of a class type $C$ are the base classes of class
$C$.
\item
The base classes of an aliased type are the base classes of its alias.
\item
The base classes of an abstract type are the base classes of its upper bound.
\item
The base classes of a parameterized type 
~\lstinline@$C$[$T_1 \commadots T_n$]@~ are the base classes
of type $C$, where every occurrence of a type parameter $a_i$ 
of $C$ has been replaced by the corresponding parameter type $T_i$.
\item
The base classes of a singleton type \lstinline@$p$.type@ are the base classes of
the type of $p$.
\item
The base classes of a compound type 
~\lstinline@$T_1$ with $\ldots$ with $T_n$ {$R\,$}@~ 
are the {\em reduced union} of the base
classes of all $T_i$'s. This means: 
Let the multi-set $\SS$ be the multi-set-union of the
base classes of all $T_i$'s.
If $\SS$ contains several type instances of the same class, say
~\lstinline@$S^i$#$C$[$T^i_1 \commadots T^i_n$]@~ $(i \in I)$, then
all those instances 
are replaced by one of them which conforms to all
others. It is an error if no such instance exists, or if $C$ is not a trait 
(\sref{sec:traits}). It follows that the reduced union, if it exists,
produces a set of class types, where different types are instances of different classes.
\item
The base classes of a type selection \lstinline@$S$#$T$@ are
determined as follows. If $T$ is an alias or abstract type, the
previous clauses apply. Otherwise, $T$ must be a (possibly
parameterized) class type, which is defined in some class $B$.  Then
the base classes of \lstinline@$S$#$T$@ are the base classes of $T$
in $B$ seen from the prefix type $S$.
\end{itemize}

2. The notion of a type $T$
{\em in class $C$ seen from some prefix type
$S\,$} makes sense only if the prefix type $S$
has a type instance of class $C$ as a base class, say
~\lstinline@$S'$#$C$[$T_1 \commadots T_n$]@. Then we define as follows.
\begin{itemize}
 \item 
  If \lstinline@$S$ = $\epsilon$.type@, then $T$ in $C$ seen from $S$ is $T$ itself.
 \item Otherwise, if $T$ is the $i$'th type parameter of some class $D$, then
   \begin{itemize}
   \item
   If $S$ has a base class ~\lstinline@$D$[$U_1 \commadots U_n$]@, for some type parameters
   ~\lstinline@[$U_1 \commadots U_n$]@, then $T$ in $C$ seen from $S$ is $U_i$.
   \item
   Otherwise, if $C$ is defined in a class $C'$, then
   $T$ in $C$ seen from $S$ is the same as $T$ in $C'$ seen from $S'$.
   \item
   Otherwise, if $C$ is not defined in another class, then  
   $T$ in $C$ seen from $S$ is $T$ itself.
  \end{itemize}
\item
   Otherwise, 
   if $T$ is the singleton type \lstinline@$D$.this.type@ for some class $D$
   then
   \begin{itemize}
   \item
   If $D$ is a subclass of $C$ and 
   $S$ has a type instance of class $D$ among its base classes,
   then $T$ in $C$ seen from $S$ is $S$.
   \item
   Otherwise, if $C$ is defined in a class $C'$, then
   $T$ in $C$ seen from $S$ is the same as $T$ in $C'$ seen from $S'$.
   \item
   Otherwise, if $C$ is not defined in another class, then  
   $T$ in $C$ seen from $S$ is $T$ itself.
   \end{itemize}
\item
  If $T$ is some other type, then the described mapping is performed
  to all its type components.
\end{itemize}

If $T$ is a possibly parameterized class type, where $T$'s class
is defined in some other class $D$, and $S$ is some prefix type,
then we use ``$T$ seen from $S$'' as a shorthand for
``$T$ in $D$ seen from $S$''.

3. The {\em member bindings} of a type $T$ are all bindings $d$ such that
there exists a type instance of some class $C$ among the base classes of $T$
and there exists a definition or declaration $d'$ in $C$
such that $d$ results from $d'$ by replacing every
type $T'$ in $d'$ by $T'$ in $C$ seen from $T$.

The {\em definition} of a type projection \lstinline@$S$#$t$@ is the member
binding $d$ of the type $t$ in $S$. In that case, we also say
that \lstinline@$S$#$t$@ {\em is defined by} $d$.

\section{Relations between types}

We define two relations between types.
\begin{quote}\begin{tabular}{l@{\gap}l@{\gap}l}
\em Type equivalence & $T \equiv U$ & $T$ and $U$ are interchangeable
in all contexts.
\\
\em Conformance & $T \conforms U$ & Type $T$ conforms to type $U$.
\end{tabular}\end{quote}

\subsection{Type Equivalence}
\label{sec:type-equiv}

Equivalence $(\equiv)$ between types is the smallest congruence\footnote{ A
congruence is an equivalence relation which is closed under formation
of contexts} such that the following holds:
\begin{itemize}
\item 
If $t$ is defined by a type alias ~\lstinline@type $t$ = $T$@, then $t$ is
equivalent to $T$.
\item
If a path $p$ has a singleton type ~\lstinline@$q$.type@, then
~\lstinline@$p$.type $\equiv q$.type@.
\item
If $O$ is defined by an object definition, and $p$ is a path
consisting only of package or object selectors and ending in $O$, then
~\lstinline@$O$.this.type $\equiv p$.type@.
\item
Two compound types are equivalent if their component types are
pairwise equivalent and their refinements are equivalent. Two
refinements are equivalent if they bind the same names and the
modifiers, types and bounds of every declared entity are equivalent in
both refinements.
\item
Two method types are equivalent if they have equivalent result
types, both have the same number of parameters, and corresponding
parameters have equivalent types as well as the same \code{def} or
\lstinline@*@ modifiers.  Note that the names of parameters do not matter
for method type equivalence.
\item
Two polymorphic types are equivalent if they have the same number of
type parameters, and, after renaming one set of type parameters by
another, the result types as well as lower and upper bounds of
corresponding type parameters are equivalent.
\item
Two overloaded types are equivalent if for every alternative type in
either type there exists an equivalent alternative type in the other.
\end{itemize}

\subsection{Conformance}
\label{sec:subtyping}

The conformance relation $(\conforms)$ is the smallest 
transitive relation that satisfies the following conditions.
\begin{itemize}
\item Conformance includes equivalence. If $T \equiv U$ then $T \conforms U$.
\item For every value type $T$, 
      $\mbox{\code{scala.All}} \conforms T \conforms \mbox{\code{scala.Any}}$. 
\item For every value type $T \conforms \mbox{\code{scala.AnyRef}}$ 
      one has $\mbox{\code{scala.AllRef}} \conforms T$.
\item A type variable or abstract type $t$ conforms to its upper bound and
      its lower bound conforms to $t$. 
\item A class type or parameterized type $c$ conforms to any of its base-types, $b$.
\item A type projection \lstinline@$T$#$t$@ conforms to \lstinline@$U$#$t$@ if 
      $T$ conforms to $U$.
\item A parameterized type ~\lstinline@$T$[$T_1 \commadots T_n$]@~ conforms to 
      ~\lstinline@$T$[$U_1 \commadots U_n$]@~ if
      the following three conditions hold for $i = 1 \commadots n$. 
      \begin{itemize}
      \item
      If the $i$'th type parameter of $T$ is declared covariant, then $T_i \conforms U_i$.
      \item
      If the $i$'th type parameter of $T$ is declared contravariant, then $U_i \conforms T_i$.
      \item
      If the $i$'th type parameter of $T$ is declared neither covariant 
      nor contravariant, then $U_i \equiv T_i$.
      \end{itemize}
\item A compound type ~\lstinline@$T_1$ with $\ldots$ with $T_n$ {$R\,$}@~ conforms to
      each of its component types $T_i$.
\item If $T \conforms U_i$ for $i = 1 \commadots n$ and for every
      binding of a type or value $x$ in $R$ there exists a member
      binding of $x$ in $T$ subsuming it, then $T$ conforms to the
      compound type ~\lstinline@$T_1$ with $\ldots$ with $T_n$ {$R\,$}@.
\item If
        $T_i \equiv T'_i$ for $i = 1 \commadots n$ and $U$ conforms to $U'$ 
        then the method type $(T_1 \commadots T_n) U$ conforms to
        $(T'_1 \commadots T'_n) U'$.
\item If, assuming 
$L'_1 \conforms a_1 \conforms U'_1 \commadots L'_n \conforms a_n \conforms U'_n$ 
one has $L_i \conforms L'_i$ and $U'_i \conforms U_i$
for $i = 1 \commadots n$, as well as $T \conforms T'$, then the polymorphic type
$[a_1 >: L_1 <: U_1 \commadots a_n >: L_n <: U_n] T$ conforms to the polymorphic type
$[a_1 >: L'_1 <: U'_1 \commadots a_n >: L'_n <: U'_n] T'$.
\item 
An overloaded type $T_1 \overload \ldots \overload T_n$ conforms to each of its alternative types $T_i$.
\item
A type $S$ conforms to the overloaded type $T_1 \overload \ldots \overload T_n$
if $S$ conforms to each alternative type $T_i$.  \todo{Really?}
\end{itemize}

A declaration or definition in some compound type of class type $C$
is {\em subsumes} another
declaration of the same name in some compound type or class type $C'$, if one of the following holds.
\begin{itemize}
\item
A value declaration ~\lstinline@val $x$: $T$@~ or value definition
~\lstinline@val $x$: $T$ = $e$@~ subsumes a value declaration
~\lstinline@val $x$: $T'$@~ if $T \conforms T'$.
\item
A type alias
$\TYPE;t=T$ subsumes a type alias $\TYPE;t=T'$ if
$T \equiv T'$.
\item 
A type declaration ~\lstinline@type $t$ >: $L$ <: $U$@~ subsumes
a type declaration ~\lstinline@type $t$ >: $L'$ <: $U'$@~ if $L' \conforms L$ and 
$U \conforms U'$.
\item
A type or class definition of some type $t$ subsumes an abstract
type declaration ~\lstinline@type t >: L <: U@~ if
$L \conforms t \conforms U$.
\end{itemize}

The $(\conforms)$ relation forms a partial order between types. The {\em
least upper bound} or the {\em greatest lower bound} of a set of types
is understood to be relative to that order.

\paragraph{Note} The least upper bound of a set of types does not always exist. For instance, consider
the class definitions
\begin{lstlisting}
class A[+t] {}
class B extends A[B];
class C extends A[C];
\end{lstlisting}
Then the types ~\lstinline@A[Any], A[A[Any]], A[A[A[Any]]], ...@~ form
a descending sequence of upper bounds for \code{B} and \code{C}. The
least upper bound would be the infinite limit of that sequence, which
does not exist as a Scala type. Since cases like this are in general
impossible to detect, a Scala compiler is free to reject a term
which has a type specified as a least upper or greatest lower bound,
and that bound would be more complex than some compiler-set
limit\footnote{The current Scala compiler limits the nesting level
of parameterization in a such bounds to 10.}.

\section{Type Erasure}
\label{sec:erasure}

A type is called {\em generic} if it contains type arguments or type variables.
{\em Type erasure} is a mapping from (possibly generic) types to
non-generic types. We write $|T|$ for the erasure of type $T$.
The erasure mapping is defined as follows.
\begin{itemize}
\item The erasure of a type variable is the erasure of its upper bound.
\item The erasure of a parameterized type $T[T_1 \commadots T_n]$ is $|T|$.
\item The erasure of a singleton type \lstinline@$p$.type@ is the 
      erasure of the type of $p$.
\item The erasure of a type projection \lstinline@$T$#$x$@ is \lstinline@|$T$|#$x$@.
\item The erasure of a compound type ~\lstinline@$T_1$ with $\ldots$ with $T_n$ {$R\,$}@ 
      is $|T_1|$.
\item The erasure of every other type is the type itself.
\end{itemize}

\section{Implicit Conversions}
\label{sec:impl-conv}

\todo{Include Anything to unit?}

The following implicit conversions are applied to expressions of
method type that are used as values, rather than being applied to some
arguments.
\begin{itemize}
\item
A parameterless method $m$ of type \lstinline@=> $T$@
is converted to type $T$ by evaluating the expression to which $m$ is bound.
\item
An expression $e$ of polymorphic type 
\begin{lstlisting}
[$a_1$ >: $L_1$ <: $U_1 \commadots a_n$ >: $L_n$ <: $U_n$]$T$
\end{lstlisting}
which does not appear as the function part of
a type application is converted to type $T$
by determining with local type inference
(\sref{sec:local-type-inf}) instance types ~\lstinline@$T_1 \commadots T_n$@~ 
for the type variables ~\lstinline@$a_1 \commadots a_n$@~ and
implicitly embedding $e$ in the type application
~\lstinline@$e$[$U_1 \commadots U_n$]@~ (\sref{sec:type-app}).
\item
An expression $e$ of monomorphic method type
$(\Ts_1) \ldots (\Ts_n) U$ of arity $n > 0$
which does not appear as the function part of an application is
converted to a function type by implicitly embedding $e$ in
the following term, where $x$ is a fresh variable and each $ps_i$ is a
parameter section consisting of parameters with fresh names of types $\Ts_i$:
\begin{lstlisting}
(val $x$ = $e$ ; $(ps_1) \ldots \Arrow \ldots \Arrow (ps_n) \Arrow x(ps_1)\ldots(ps_n)$)
\end{lstlisting}
This conversion is not applicable to functions with call-by-name
parameters \lstinline@$x$: => $T$@ or repeated parameters
\lstinline@x: T*@, (\sref{sec:parameters}), because its result would
violate the well-formedness rules for anonymous functions
(\sref{sec:closures}). Hence, methods with such parameters
always need to be applied to arguments immediately.
\end{itemize}

When used in an expression, a value of type \code{byte}, \code{char},
or \code{short} is always implicitly converted to a value of type
\code{int}.

%If an expression $e$ has type $T$ where $T$ does not conform to the
%expected type $pt$ and $T$ has a member named \lstinline@coerce@ of type
%$[]U$ where $U$ does conform to $pt$, then the expression is typed and evaluated is if it was
%\lstinline@$e$.coerce@.

Implicit conversions can also be user-defined. This is expained in
Chapter~\ref{sec:views}.

\chapter{Basic Declarations and Definitions}
\label{sec:defs}

\syntax\begin{lstlisting}
  Dcl             ::=  val ValDcl
                    |  var VarDcl
                    |  def FunDcl
                    |  type TypeDcl
  Def             ::=  val PatDef 
                    |  var VarDef 
                    |  def FunDef 
                    |  type TypeDef 
                    |  TmplDef
\end{lstlisting}

A {\em declaration} introduces names and assigns them types. It can
appear as one of the statements of a class definition
(\sref{sec:templates}) or as part of a refinement in a compound
type (\ref{sec:refinements}).

A {\em definition} introduces names that denote terms or types. It can
form part of an object or class definition or it can be local to a
block.  Both declarations and definitions produce {\em bindings} that
associate type names with type definitions or bounds, and that
associate term names with types.

The scope of a name introduced by a declaration or definition is the
whole statement sequence containing the binding.  However, there is a
restriction on forward references: In a statement sequence $s_1 \ldots
s_n$, if a simple name in $s_i$ refers to an entity defined by $s_j$
where $j \geq i$, then every non-empty statement between and including
$s_i$ and $s_j$ must be an import clause,
or a function, type, class, or object definition. It may not be 
a value definition, a variable definition, or an expression.

\comment{
Every basic definition may introduce several defined names, separated
by commas. These are expanded according to the following scheme:
\bda{lcl}
\VAL;x, y: T = e && \VAL; x: T = e \\
                 && \VAL; y: T = x \\[0.5em]

\LET;x, y: T = e && \LET; x: T = e \\
                 && \VAL; y: T = x \\[0.5em]

\DEF;x, y (ps): T = e &\tab\mbox{expands to}\tab& \DEF; x(ps): T = e \\
                      && \DEF; y(ps): T = x(ps)\\[0.5em]

\VAR;x, y: T := e && \VAR;x: T := e\\
                  && \VAR;y: T := x\\[0.5em]

\TYPE;t,u = T && \TYPE; t = T\\
              && \TYPE; u = t\\[0.5em]
\eda

All definitions have a ``repeated form'' where the initial
definition keyword is followed by several constituent definitions
which are separated by commas.  A repeated definition is
always interpreted as a sequence formed from the
constituent definitions. E.g.\ the function definition
~\lstinline@def f(x) = x, g(y) = y@~ expands to
~\lstinline@def f(x) = x; def g(y) = y@~ and
the type definition
~\lstinline@type T, U <: B@~ expands to
~\lstinline@type T; type U <: B@.
}
\comment{
If an element in such a sequence introduces only the defined name,
possibly with some type or value parameters, but leaves out any
additional parts in the definition, then those parts are implicitly
copied from the next subsequent sequence element which consists of
more than just a defined name and parameters. Examples:
\begin{itemize}
\item[]
The variable declaration ~\lstinline@var x, y: int@~ 
expands to ~\lstinline@var x: int; var y: int@.
\item[]
The value definition ~\lstinline@val x, y: int = 1@~ 
expands to ~\lstinline@val x: int = 1; val y: int = 1@.
\item[]
The class definition ~\lstinline@case class X(), Y(n: int) extends Z@~ expands to
~\lstinline@case class X extends Z; case class Y(n: int) extends Z@.
\item
The object definition ~\lstinline@case object Red, Green, Blue extends Color@~
expands to  
\begin{lstlisting}
case object Red extends Color; 
case object Green extends Color;
case object Blue extends Color .
\end{lstlisting}
\end{itemize}
}
\section{Value Declarations and Definitions}
\label{sec:valdef}

\syntax\begin{lstlisting}
  Dcl          ::=  val ValDcl
  ValDcl       ::=  id {`,' id} `:' Type
  Def          ::=  val PatDef 
  PatDef       ::=  Pattern2 {`,' Pattern2} [`:' Type] `=' Expr
\end{lstlisting}

A value declaration ~\lstinline@val $x$: $T$@~ introduces $x$ as a name of a value of
type $T$.  

A value definition ~\lstinline@val $x$: $T$ = $e$@~ defines $x$ as a name of the
value that results from the evaluation of $e$. The type $T$ may be
omitted, in which case the type of expression $e$ is assumed.
If a type $T$ is given, then $e$ is expected to conform to it.

Evaluation of the value definition implies evaluation of its
right-hand side $e$.  The effect of the value definition is to bind
$x$ to the value of $e$ converted to type $T$.

Value definitions can alternatively have a pattern
(\sref{sec:patterns}) as left-hand side.  If $p$ is some pattern other
than a simple name or a name followed by a colon and a type, then the
value definition ~\lstinline@val $p$ = $e$@~ is expanded as follows:

1. If the pattern $p$ has bound variables $x_1 \commadots x_n$, where $n > 1$:
\begin{lstlisting}
val $\Dollar x$ = $e$.match {case $p$ => scala.Tuple$n$($x_1 \commadots x_n$)}
val $x_1$ = $\Dollar x$._1
$\ldots$
val $x_n$ = $\Dollar x$._n  .
\end{lstlisting}
Here, $\Dollar x$ is a fresh name.  The class
\lstinline@Tuple$n$@ is defined for $n = 2 \commadots 9$ in package
\code{scala}.

2. If $p$ has a unique bound variable $x$:
\begin{lstlisting}
val $x$ = $e$.match { case $p$ => $x$ }
\end{lstlisting}

3. If $p$ has no bound variables:
\begin{lstlisting}
$e$.match { case $p$ => ()}
\end{lstlisting}

\example
The following are examples of value definitions
\begin{lstlisting}
val pi = 3.1415;
val pi: double = 3.1415;  // equivalent to first definition
val Some(x) = f();        // a pattern definition
val x :: xs = mylist;     // an infix pattern definition
\end{lstlisting}

The last two definitions have the following expansions.
\begin{lstlisting}
val x = f().match { case Some(x) => x }

val x$\Dollar$ = mylist.match { case x :: xs => scala.Tuple2(x, xs) }
val x = x$\Dollar$._1;
val xs = x$\Dollar$._2;
\end{lstlisting}

A value declaration ~\lstinline@val $x_1 \commadots x_n$: $T$@~
is a
shorthand for the sequence of value declarations
~\lstinline@val $x_1$: $T$; ...; val $x_n$: $T$@.
A value definition ~\lstinline@val $p_1 \commadots p_n$ = $e$@~
is a
shorthand for the sequence of value definitions
~\lstinline@val $p_1$ = $e$; ...; val $p_n$ = $e$@.
A value definition ~\lstinline@val $p_1 \commadots p_n: T$ = $e$@~
is a
shorthand for the sequence of value definitions
~\lstinline@val $p_1: T$ = $e$; ...; val $p_n: T$ = $e$@.

\section{Variable Declarations and Definitions}
\label{sec:vardef}

\syntax\begin{lstlisting}
  Dcl            ::=  var VarDcl
  Def            ::=  var VarDef
  VarDcl         ::=  id {`,' id} `:' Type
  VarDef         ::=  id {`,' id} [`:' Type] `=' Expr
                   |  id {`,' id} `:' Type `=' `_'
\end{lstlisting}

A variable declaration ~\lstinline@var $x$: $T$@~ is equivalent to declarations
of a {\em getter function} $x$ and a {\em setter function}
\lstinline@$x$_=@, defined as follows:

\begin{lstlisting}
  def $x$: $T$;
  def $x$_= ($y$: $T$): unit
\end{lstlisting}

An implementation of a class containing variable declarations
may define these variables using variable definitions, or it may
define setter and getter functions directly.

A variable definition ~\lstinline@var $x$: $T$ = $e$@~ introduces a mutable
variable with type $T$ and initial value as given by the
expression $e$. The type $T$ can be omitted, 
in which case the type of $e$ is assumed. If $T$ is given, then $e$ 
is expected to conform to it.

A variable definition ~\lstinline@var $x$: $T$ = _@~ introduces a mutable
variable with type \ $T$ and a default initial value. 
The default value depends on the type $T$ as follows:
\begin{quote}\begin{tabular}{ll}
\code{0} & if $T$ is \code{int} or one of its subrange types, \\
\code{0L} & if $T$ is \code{long},\\
\lstinline@0.0f@ & if $T$ is \code{float},\\
\lstinline@0.0d@ & if $T$ is \code{double},\\
\code{false} & if $T$ is \code{boolean},\\
\lstinline@()@ & if $T$ is \code{unit}, \\
\code{null} & for all other types $T$.
\end{tabular}\end{quote}

When they occur as members of a template, both forms of variable
definition also introduce a getter function $x$ which returns the
value currently assigned to the variable, as well as a setter function
\lstinline@$x$_=@ which changes the value currently assigned to the variable.
The functions have the same signatures as for a variable declaration.
The getter and setter functions are then members of the template
instead of the variable accessed by them.

\example The following example shows how {\em properties} can be
simulated in Scala. It defines a class \code{TimeOfDayVar} of time
values with updatable integer fields representing hours, minutes, and
seconds. Its implementation contains tests that allow only legal
values to be assigned to these fields. The user code, on the other
hand, accesses these fields just like normal variables.

\begin{lstlisting}
class TimeOfDayVar {
  private var h: int = 0, m: int = 0, s: int = 0;

  def hours              =  h;
  def hours_= (h: int)   =  if (0 <= h && h < 24) this.h = h 
                            else throw new DateError();

  def minutes            =  m
  def minutes_= (m: int) =  if (0 <= m && m < 60) this.m = m
                            else throw new DateError();

  def seconds            =  s
  def seconds_= (s: int) =  if (0 <= s && s < 60) this.s = s
                            else throw new DateError();
}
val t = new TimeOfDayVar;
d.hours = 8; d.minutes = 30; d.seconds = 0;
d.hours = 25;                 // throws a DateError exception
\end{lstlisting}

A variable declaration ~\lstinline@var $x_1 \commadots x_n$: $T$@~
is a
shorthand for the sequence of variable declarations
~\lstinline@var $x_1$: $T$; ...; var $x_n$: $T$@.
A variable definition ~\lstinline@var $x_1 \commadots x_n$ = $e$@~
is a
shorthand for the sequence of variable definitions
~\lstinline@var $x_1$ = $e$; ...; var $x_n$ = $e$@.
A variable definition ~\lstinline@var $x_1 \commadots x_n: T$ = $e$@~
is a
shorthand for the sequence of variable definitions
~\lstinline@var $x_1: T$ = $e$; ...; var $x_n: T$ = $e$@.

\section{Type Declarations and Type Aliases}
\label{sec:typedcl}
\label{sec:typealias}

\syntax\begin{lstlisting}
  Dcl             ::=  type TypeDcl
  TypeDcl         ::=  id [>: Type] [<: Type]
  Def             ::=  type TypeDef
  TypeDef         ::=  id [TypeParamClause] `=' Type
\end{lstlisting}

A {\em type declaration} ~\lstinline@type $t$ >: $L$ <: $U$@~ declares $t$ to
be an abstract type with lower bound type $L$ and upper bound
type $U$.  If such a declaration appears as a member declaration
of a type, implementations of the type may implement $t$ with any
type $T$ for which $L \conforms T \conforms U$. Either or both bounds may
be omitted.  If the lower bound $L$ is missing, the bottom type
\lstinline@scala.All@ is assumed.  If the upper bound $U$ is missing,
the top type \lstinline@scala.Any@ is assumed.

A {\em type alias} ~\lstinline@type $t$ = $T$@~ defines $t$ to be an alias
name for the type $T$.  The left hand side of a type alias may
have a type parameter clause, e.g. ~\lstinline@type $t$[$\tps\,$] = $T$@.  The scope
of a type parameter extends over the right hand side $T$ and the
type parameter clause $\tps$ itself.  

The scope rules for definitions (\sref{sec:defs}) and type parameters
(\sref{sec:funsigs}) make it possible that a type name appears in its
own bound or in its right-hand side.  However, it is a static error if
a type alias refers recursively to the defined type constructor itself.  
That is, the type $T$ in a type alias ~\lstinline@type $t$[$\tps\,$] = $T$@~ may not refer
directly or indirectly to the name $t$.  It is also an error if
an abstract type is directly or indirectly its own upper or lower bound.

\example The following are legal type declarations and definitions:
\begin{lstlisting}
type IntList = List[Integer];
type T <: Comparable[T];
type Two[a] = Tuple2[a, a];
\end{lstlisting}

The following are illegal:
\begin{lstlisting}
type Abs = Comparable[Abs];       // recursive type alias

type S <: T;                      // S, T are bounded by themselves.
type T <: S;

type T <: AnyRef with T;          // T is abstract, may not be part of
                                  // compound type

type T >: Comparable[T.That];     // Cannot select from T.
                                  // T is a type, not a value
\end{lstlisting}

If a type alias ~\lstinline@type $t$[$\tps\,$] = $S$@~ refers to a class type
$S$, the name $t$ can also be used as a constructor for
objects of type $S$.

\example The \code{Predef} module contains a definition which establishes \code{Pair} 
as an alias of the parameterized class \code{Tuple2}:
\begin{lstlisting}
type Pair[+a, +b] = Tuple2[a, b];
\end{lstlisting}
As a consequence, for any two types $S$ and $T$, the type
~\lstinline@Pair[$S$, $T\,$]@~ is equivalent to the type ~\lstinline@Tuple2[$S$, $T\,$]@.
\code{Pair} can also be used as a constructor instead of \code{Tuple2}, as in
\begin{lstlisting}
new Pair[Int, Int](1, 2) .
\end{lstlisting}

\section{Type Parameters}\label{sec:type-params}

\syntax\begin{lstlisting}
  TypeParamClause  ::=  `[' VarTypeParam {`,' VarTypeParam} `]'
  VarTypeParam     ::=  [`+' | `-'] TypeParam
  TypeParam        ::=  id [>: Type] [<: Type | <% Type]
\end{lstlisting}


Type parameters appear in type definitions, class definitions, and
function definitions.  The most general form of a type parameter is
~\lstinline@$\pm t$ >: $L$ <: $U$@.  Here, $L$, and $U$ are lower
and upper bounds that constrain possible type arguments for the
parameter, and $\pm$ is a {\em variance}, i.e.\ an optional prefix 
of either \lstinline@+@, or \lstinline@-@.

\comment{
The upper bound $U$ in a type parameter clauses may not be a final
class. The lower bound may not denote a value type.\todo{Why}
}
The names of all type parameters in a type parameter clause must be
pairwise different.  The scope of a type parameter includes in each
case the whole type parameter clause. Therefore it is possible that a
type parameter appears as part of its own bounds or the bounds of
other type parameters in the same clause.  However, a type parameter
may not be bounded directly or indirectly by itself.

\example Here are some well-formed type parameter clauses:
\begin{lstlisting}
[s, t]
[ex <: Throwable]
[a <: Ord[b], b <: a]
[a, b, c >: a <: b]
\end{lstlisting}
The following type parameter clauses are illegal
since type parameter are bounded by themselves.
\begin{lstlisting}
[a >: a]                 
[a <: b, b <: c, c <: a]
\end{lstlisting}

Variance annotations indicate how type instances with the given type
parameters vary with respect to subtyping (\sref{sec:subtyping}).  A
`\lstinline@+@' variance indicates a covariant dependency, a `\lstinline@-@'
variance indicates a contravariant dependency, and a missing variance
indication indicates an invariant dependency.

A variance annotation constrains the way the annotated type variable
may appear in the type or class which binds the type parameter.  In a
type definition ~\lstinline@type $t$[$\tps\,$] = $S$@, type parameters labeled
`\lstinline@+@' must only appear in covariant position in $S$ whereas
type parameters labeled `\lstinline@-@' must only appear in contravariant
position. Analogously, for a class definition
~\lstinline@class $c$[$\tps\,$]($ps\,$): $s$ extends $t$@, type parameters labeled
`\lstinline@+@' must only appear in covariant position in the self type
$s$ and the template $t$, whereas type
parameters labeled `\lstinline@-@' must only appear in contravariant
position. 

The variance position of a type parameter in a type or template is
defined as follows.  Let the opposite of covariance be contravariance,
and the opposite of invariance be itself.  The top-level of the type
or template is always in covariant position. The variance position
changes at the following constructs.
\begin{itemize}
\item
The variance position of a method parameter is the opposite of the 
variance position of the enclosing parameter clause.
\item
The variance position of a type parameter is the opposite of the
variance position of the enclosing type parameter clause.
\item
The variance position of the lower bound of a type declaration or type parameter 
is the opposite of the variance position of the type declaration or parameter.  
\item
The right hand side $S$ of a type alias ~\lstinline@type $t$[$\tps\,$] = $S$@~ 
is always in invariant position.
\item
The type of a mutable variable is always in invariant position.
\item 
The prefix $S$ of a type selection \lstinline@$S$#$T$@ is always in invariant position.
\item
For a type argument $T$ of a type ~\lstinline@$S$[$\ldots T \ldots$ ]@: If the
corresponding type parameter is invariant, then $T$ is in
invariant position.  If the corresponding type parameter is
contravariant, the variance position of $T$ is the opposite of
the variance position of the enclosing type ~\lstinline@$S$[$\ldots T \ldots$ ]@.
\end{itemize}

\example The following variance annotation is legal. 
\begin{lstlisting}
abstract class P[+a, +b] {
  def fst: a; def snd: b
}
\end{lstlisting}
With this variance annotation, elements
of type $P$ subtype covariantly with respect to their arguments. 
For instance, 
\begin{lstlisting}
P[IOException, String] <: P[Throwable, AnyRef] .
\end{lstlisting}

If we make the elements of $P$ mutable, 
the variance annotation becomes illegal. 
\begin{lstlisting}
abstract class Q[+a, +b] { 
  var fst: a;           // **** error: illegal variance:
  var snd: b            // `a', `b' occur in invariant position.
}
\end{lstlisting}

\example The following variance annotation is illegal, since $a$ appears
in contravariant position in the parameter of \code{append}:

\begin{lstlisting}
trait Vector[+a] {
  def append(x: Vector[a]): Vector[a]; 
                        // **** error: illegal variance: 
                        // `a' occurs in contravariant position.
}
\end{lstlisting} 
The problem can be avoided by generalizing the type of \code{append}
by means of a lower bound:

\begin{lstlisting}
trait Vector[+a] {
  def append[b >: a](x: Vector[b]): Vector[b];
}
\end{lstlisting}

\example Here is a case where a contravariant type parameter is useful.

\begin{lstlisting}
trait OutputChannel[-a] {
  def write(x: a): unit
}
\end{lstlisting}
With that annotation, we have that
\lstinline@OutputChannel[AnyRef]@ conforms to \lstinline@OutputChannel[String]@.  
That is, a
channel on which one can write any object can substitute for a channel
on which one can write only strings.

\section{Function Declarations and Definitions}
\label{sec:defdef}
\label{sec:funsigs}
\label{sec:parameters}

\syntax\begin{lstlisting} 
Dcl                ::=  def FunDcl
FunDcl             ::=  FunSig {`,' FunSig} `:' Type 
Def                ::=  def FunDef
FunDef             ::=  FunSig {`,' FunSig} [`:' Type] `=' Expr 
FunSig             ::=  id [FunTypeParamClause] {ParamClause} 
FunTypeParamClause ::= `[' TypeParam {`,' TypeParam} `]' 
ParamClause        ::= `(' [Param {`,' Param}] `)' 
Param              ::=  id `:' [`=>'] Type [`*']
\end{lstlisting}

A function declaration has the form ~\lstinline@def $f \psig$: $T$@, where
$f$ is the function's name, $\psig$ is its parameter
signature and $T$ is its result type. A function definition
~\lstinline@$f \psig$: $T$ = $e$@~ also includes a {\em function body} $e$,
i.e.\ an expression which defines the function's result.  A parameter
signature consists of an optional type parameter clause \lstinline@[$\tps\,$]@,
followed by zero or more value parameter clauses
~\lstinline@($ps_1$)$\ldots$($ps_n$)@.  Such a declaration or definition
introduces a value with a (possibly polymorphic) method type whose
parameter types and result type are as given.

A type parameter clause $\tps$ consists of one or more type
declarations (\sref{sec:typedcl}), which introduce type parameters,
possibly with bounds.  The scope of a type parameter includes
the whole signature, including any of the type parameter bounds as
well as the function body, if it is present.  

A value parameter clause $ps$ consists of zero or more formal
parameter bindings such as \lstinline@$x$: $T$@, which bind value
parameters and associate them with their types.  The scope of a formal
value parameter name $x$ is the function body, if one is
given. Both type parameter names and value parameter names must be
pairwise distinct.

The type of a value parameter may be prefixed by \code{=>}, e.g.\
~\lstinline@$x$: => $T$@. The type of such a parameter is then the
parameterless method type ~\lstinline@=> $T$@. This indicates that the
corresponding argument is not evaluated at the point of function
application, but instead is evaluated at each use within the
function. That is, the argument is evaluated using {\em call-by-name}.

\example The declaration
\begin{lstlisting}
def whileLoop (cond: => Boolean) (stat: => Unit): Unit
\end{lstlisting}
indicates that both parameters of \code{while} are evaluated using
call-by-name.

The last value parameter of a parameter section may be suffixed by
``\code{*}'', e.g.\ ~\lstinline@(..., $x$:$T$*)@.  The type of such a
{\em repeated} parameter inside the method is then the sequence type
\lstinline@scala.Seq[$T$]@.  Methods with repeated parameters
\lstinline@$T$*@ take a variable number of arguments of type $T$.  That is,
if a method $m$ with type ~\lstinline@($T_1 \commadots T_n, S$*)$U$@~
is applied to arguments $(e_1 \commadots e_k)$ where $k \geq n$, then
$m$ is taken in that application to have type $(T_1 \commadots T_n, S
\commadots S)U$, with $k - n$ occurrences of type $S$.  
\todo{Change to ???: If the method
is converted to a function type instead of being applied immediately,
a repeated parameter \lstinline@$T$*@ is taken to be ~\lstinline@scala.Seq[$T$]@~
instead.}

\example The following method definition computes the sum of a variable number
of integer arguments.
\begin{lstlisting}
def sum(args: int*) {
  var result = 0;
  for (val arg <- args.elements) result = result + arg;
  result
}
\end{lstlisting}
The following applications of this method yield \code{0}, \code{1},
\code{6}, in that order.
\begin{lstlisting}
sum()
sum(1)
sum(1, 2, 3, 4, 5)
\end{lstlisting}


The type of the function body must conform to the function's declared
result type, if one is given. If the function definition is not
recursive, the result type may be omitted, in which case it is
determined from the type of the function body.

For any index $i$ let $\fsig_i$ be a function signature consisting of a function
name, an optional type parameter section, and zero or more parameter
sections.  Then a function declaration 
~\lstinline@def $\fsig_1 \commadots \fsig_n$: $T$@~ 
is a shorthand for the sequence of function
declarations ~\lstinline@def $\fsig_1$: $T$; ...; def $\fsig_n$: $T$@.  
A function definition ~\lstinline@def $\fsig_1 \commadots \fsig_n$ = $e$@~ is a
shorthand for the sequence of function definitions 
~\lstinline@def $\fsig_1$ = $e$; ...; def $\fsig_n$ = $e$@.  
A function definition
~\lstinline@def $\fsig_1 \commadots \fsig_n: T$ = $e$@~ is a shorthand for the
sequence of function definitions 
~\lstinline@def $\fsig_1: T$ = $e$; ...; def $\fsig_n: T$ = $e$@.

\section{Overloaded Definitions}
\label{sec:overloaded-defs}
\todo{change}

An overloaded definition is a set of $n > 1$ value or function
definitions in the same statement sequence that define the same name,
binding it to types ~\lstinline@$T_1 \commadots T_n$@, respectively.
The individual definitions are called {\em alternatives}.  Overloaded
definitions may only appear in the statement sequence of a template.
Alternatives always need to specify the type of the defined entity
completely.  It is an error if the types of two alternatives $T_i$ and
$T_j$ have the same erasure (\sref{sec:erasure}).

\todo{Say something about bridge methods.}
%This must be a well-formed
%overloaded type

\section{Import Clauses}
\label{sec:import}

\syntax\begin{lstlisting}
  Import          ::= import ImportExpr {`,' ImportExpr}
  ImportExpr      ::= StableId `.' (id | `_' | ImportSelectors)
  ImportSelectors ::= `{' {ImportSelector `,'} 
                      (ImportSelector | `_') `}'
  ImportSelector  ::= id [`=>' id | `=>' `_']
\end{lstlisting}

An import clause has the form ~\lstinline@import $p$.$I$@~ where $p$ is a stable
identifier (\sref{sec:paths}) and $I$ is an import expression.
The import expression determines a set of names of members of $p$
which are made available without qualification. The most general form
of an import expression is a list of {\em import selectors}
\begin{lstlisting}
{ $x_1$ => $y_1 \commadots x_n$ => $y_n$, _ }
\end{lstlisting}
for $n \geq 0$, where the final wildcard `\lstinline@_@' may be absent.  It
makes available each member \lstinline@$p$.$x_i$@ under the unqualified name
$y_i$. I.e.\ every import selector ~\lstinline@$x_i$ => $y_i$@~ renames
\lstinline@$p$.$x_i$@ to
$y_i$.  If a final wildcard is present, all members $z$ of
$p$ other than ~\lstinline@$x_1 \commadots x_n$@~ are also made available
under their own unqualified names.

Import selectors work in the same way for type and term members. For
instance, an import clause ~\lstinline@import $p$.{$x$ => $y\,$}@~ renames the term
name \lstinline@$p$.$x$@ to the term name $y$ and the type name \lstinline@$p$.$x$@
to the type name $y$. At least one of these two names must
reference a member of $p$.

If the target in an import selector is a wildcard, the import selector
hides access to the source member. For instance, the import selector
~\lstinline@$x$ => _@~ ``renames'' $x$ to the wildcard symbol (which is
unaccessible as a name in user programs), and thereby effectively
prevents unqualified access to $x$. This is useful if there is a
final wildcard in the same import selector list, which imports all
members not mentioned in previous import selectors.

Several shorthands exist. An import selector may be just a simple name
$x$. In this case, $x$ is imported without renaming, so the
import selector is equivalent to ~\lstinline@$x$ => $x$@. Furthermore, it is
possible to replace the whole import selector list by a single
identifier or wildcard. The import clause ~\lstinline@import $p$.$x$@~ is
equivalent to ~\lstinline@import $p$.{$x\,$}@~, i.e.\ it makes available without
qualification the member $x$ of $p$. The import clause
~\lstinline@import $p$._@~ is equivalent to
~\lstinline@import $p$.{_}@, 
i.e.\ it makes available without qualification all members of $p$
(this is analogous to ~\lstinline@import $p$.*@~ in Java).

An import clause with multiple import expressions
~\lstinline@import $p_1$.$I_1 \commadots p_n$.$I_n$@~ is interpreted as a
sequence of import clauses 
~\lstinline@import $p_1$.$I_1$; $\ldots$; import $p_n$.$I_n$@.

\example Consider the object definition:
\begin{lstlisting}
object M { 
  def z = 0, one = 1; 
  def add(x: Int, y: Int): Int = x + y 
}
\end{lstlisting}
Then the block
\begin{lstlisting}
{ import M.{one, z => zero, _}; add(zero, one) }
\end{lstlisting}
is equivalent to the block 
\begin{lstlisting}
{ M.add(M.z, M.one) } .
\end{lstlisting}

\chapter{Classes and Objects}
\label{sec:globaldefs}

\syntax\begin{lstlisting}
  TmplDef          ::=  ([case] class | trait) ClassDef
                    |  [case] object ObjectDef
\end{lstlisting}

Classes (\sref{sec:classes}) and objects
(\sref{sec:modules}) are both defined in terms of {\em templates}.

\section{Templates}
\label{sec:templates}

\syntax\begin{lstlisting}
  Template        ::=  Constr {`with' Constr} [TemplateBody]
  TemplateBody    ::=  `{' [TemplateStat {`;' TemplateStat}] `}'
\end{lstlisting}

A template defines the type signature, behavior and initial state of a
class of objects or of a single object. Templates form part of
instance creation expressions, class definitions, and object
definitions.  A template
~\lstinline@$sc$ with $mc_1$ with $\ldots$ with $mc_n$ {$\stats\,$}@~
consists of a constructor invocation $sc$
which defines the template's {\em superclass}, constructor invocations
~\lstinline@$mc_1 \commadots mc_n$@~ $(n \geq 0)$, which define the
template's {\em mixin classes}, and a statement sequence $\stats$ which
contains additional member definitions for the template.  Superclass
and mixin classes together are called the {\em parent classes} of a
template.  They must be pairwise different.  The superclass of a
template must be a subtype of the superclass of each mixin class.  The
{\em least proper supertype} of a template is the class type or
compound type (\sref{sec:compound-types}) consisting of the its parent
classes.

\todo{introduce ScalaObject}

Member definitions define new members or overwrite members in the
parent classes.  If the template forms part of a class definition,
the statement part $\stats$ may also contain declarations of abstract members.
%The type of each non-private definition or declaration of a
%template must be equivalent to a type which does not refer to any
%private members of that template.

\todo{Make all references to Java generic}

\paragraph{Inheriting from Java Types} A template may have a Java class as
its superclass and Java interfaces as its mixin classes. On the other
hand, it is not permitted to have a Java class as a mixin class, or a
Java interface as a superclass.

\subsection{Constructor Invocations}
\label{sec:constr-invoke}
\syntax\begin{lstlisting}
  Constr  ::=  StableId [TypeArgs] [`(' [Exprs] `)']  
\end{lstlisting}

Constructor invocations define the type, members, and initial state of
objects created by an instance creation expression, or of parts of an
object's definition which are inherited by a class or object
definition. A constructor invocation is a function application
\lstinline@$x$.$c$($\args\,$)@, where $x$ is a stable identifier
(\sref{sec:stable-ids}), $c$ is a type name which either
designates a class or defines an alias type for one, and $\args$
is an argument list, which matches one of the constructors of that
class. The prefix `\lstinline@$x$.@' can be omitted. 
%The class $c$ must conform to \lstinline@scala.AnyRef@, 
%i.e.\ it may not be a value type.  
The argument list \lstinline@($\args\,$)@ can also be omitted, in which case an
empty argument list \lstinline@()@ is implicitly added.

\subsection{Base Classes}
\label{sec:base-classes}

For every template, class type and constructor invocation we define
two sets of class types: the {\em base classes} and {\em mixin base
classes}. Their definitions are as follows.

The {\em mixin base classes} of a template 
~\lstinline@$sc$ with $mc_1$ with $\ldots$ with $mc_n$ {$\stats\,$}@~ 
are 
the reduced union (\sref{sec:base-classes-member-defs}) of the base classes of all
mixins $mc_i$. The mixin base classes of a class type $C$ are the
mixin base classes of the template augmented by $C$ itself. The
mixin base classes of a constructor invocation of type $T$ are the
mixin base classes of class $T$.

The {\em base classes} of a template consist are the reduced union of
the base classes of its superclass and the template's mixin base
classes.  The base classes of class \lstinline@scala.Any@ consist of
just the class itself. The base classes of some other class type $C$
are the base classes of the template represented by $C$ augmented by
$C$ itself.  The base classes of a constructor invocation of type $T$
are the base classes of $T$.

The notions of mixin base classes and base classes are extended from
classes to arbitrary types following the definitions of
\sref{sec:base-classes-member-defs}.

\comment{
If two types in the base class sequence of a template refer to the
same class definition, then that definition must define a trait
(\sref{sec:traits}), and the type that comes later in the sequence must
conform to the type that comes first. 
(\sref{sec:base-classes-member-defs}).
}

\example 
Consider the following class definitions:
\begin{lstlisting}
class A;
class B extends A;
trait C extends A;
class D extends A;
class E extends B with C with D;
class F extends B with D with E;
\end{lstlisting} 
The mixin base classes and base classes of classes \code{A-F} are given in
the following table:
\begin{quote}\begin{tabular}{|l|l|l|} \hline
 \ & Mixin base classes & Base classes \\  \hline
A & A & A, ScalaObject, AnyRef, Any \\
B & B & B, A, ScalaObject, AnyRef, Any \\
C & C & C, A, ScalaObject, AnyRef, Any \\
D & D & D, A, ScalaObject, AnyRef, Any \\
E & C, D, E & E, B, C, D, A, ScalaObject, AnyRef, Any \\
F & C, D, E, F & F, B, D, E, C, A, ScalaObject, AnyRef, Any \\ \hline
\end{tabular}\end{quote}
Note that \code{D} is inherited twice by \code{F}, once directly, the
other time indirectly through \code{E}. This is permitted, since
\code{D} is a trait.


\subsection{Evaluation}

The evaluation of a template or constructor invocation depends on
whether the template defines an object or is a superclass of a
constructed object, or whether it is used as a mixin for a defined
object.  In the second case, the evaluation of a template used as a
mixin depends on an {\em actual superclass}, which is known at the
point where the template is used in a definition of an object, but not
at the point where it is defined. The actual superclass is used in the
determination of the meaning of \code{super} (\sref{sec:this-super}).

We therefore define two notions of template evaluation: (Plain)
evaluation (as a defining template or superclass) and mixin evaluation
with a given superclass $sc$. These notions are defined for templates
and constructor invocations as follows.

A {\em mixin evaluation with superclass $sc$} of a template
~\lstinline@$sc'$ with $mc_1$ with $mc_n$ {$\stats\,$}@~ consists of mixin
evaluations with superclass $sc$ of the mixin constructor invocations
~\lstinline@$mc_1 \commadots mc_n$@~ in the order they are given, followed by an
evaluation of the statement sequence $\stats$.  Within $\stats$ the
actual superclass refers to $sc$.  A mixin evaluation with superclass
$sc$ of a class constructor invocation \code{ci} consists of an evaluation
of the constructor function and its arguments in the order they are
given, followed by a mixin evaluation with superclass $sc$ of the
template represented by the constructor invocation.

An {\em evaluation} of a template
~\lstinline@$sc$ with $mc_1$ with $mc_n$ with ($\stats\,$)@~ consists of an evaluation of
the superclass constructor invocation $sc$,
followed by a mixin evaluation with superclass $sc$ of the template. An
evaluation of a class constructor invocation \code{ci} consists of an
evaluation of the constructor function and its arguments in
the order they are given, followed by an evaluation of the template
represented by the constructor invocation.

\subsection{Template Members}

\label{sec:members}

The object resulting from evaluation of a template has directly bound
members and inherited members. Members can be abstract or concrete.
For a template $T$ these categories are defined as follows. 
\begin{enumerate}
\item
A {\em directly bound} member of $T$ is an entity introduced by a member
definition or declaration in $T$'s statement sequence. The
member is called {\em abstract} if it is introduced by a declaration,
{\em concrete} otherwise.
\item
A {\em concrete inherited} member of $T$ is a non-private, concrete member of
one of $T$'s parent classes, except if a member with the same name is
already directly bound in $T$ or the member is mixin-overridden in
$T$. A member $m$ of $T$'s superclass is {\em mixin-overridden} in $T$
if there is a concrete member of a mixin base class of $T$ which
either overrides $m$ itself or overrides a member named $m$ of a base
class of $T$'s superclass.
\item
An {\em abstract inherited} member of $T$ is a non-private, abstract member
of one of $T$'s parent classes $P_i$, except if the template has a
directly bound or concrete inherited member with the same name, or the
template has an abstract member inherited from a parent class $P_j$ where
$j > i$\todo{OK to leave out?: , and which has the same modifiers and type as the member
inherited from $P_j$ would have in $T$}.
\end{enumerate}
It is an error if a template has more than one member with
the same name. 



\comment{
The type of a member $m$ is determined as follows: If $m$ is defined
in $\stats$, then its type is the type as given in the member's
declaration or definition. Otherwise, if $m$ is inherited from the
base class ~\lstinline@$B$[$T_1$, $\ldots$. $T_n$]@, $B$'s class declaration has formal
parameters ~\lstinline@[$a_1 \commadots a_n$]@, and $M$'s type in $B$ is $U$, then
$M$'s type in $C$ is ~\lstinline@$U$[$a_1$ := $T_1 \commadots a_n$ := $T_n$]@.

\ifqualified{
Members of templates have internally qualified names $Q\qex x$ where
$x$ is a simple name and $Q$ is either the empty name $\epsilon$, or
is a qualified name referencing the module or class that first
introduces the member. A basic declaration or definition of $x$ in a
module or class $M$ introduces a member with the following qualified
name:
\begin{enumerate}
\item
If the binding is labeled with an ~\lstinline@override $Q$@\notyet{Override
  with qualifier} modifier,
where $Q$ is a fully qualified name of a base class of $M$, then the
qualified name is the qualified expansion (\sref{sec:names}) of $x$ in
$Q$.
\item
If the binding is labeled with an \code{override} modifier without a
base class name, then the qualified name is the qualified expansion
of $x$ in $M$'s least proper supertype (\sref{sec:templates}).
\item
An implicit \code{override} modifier is added and case (2) also
applies if $M$'s least proper supertype contains an abstract member
with simple name $x$.
\item
If no \code{override} modifier is given or implied, then if $M$ is
labeled \code{qualified}, the qualified name is $M\qex x$. If $M$ is
not labeled \code{qualified}, the qualified name is $\epsilon\qex x$.
\end{enumerate}
}
}

\example Consider the class definitions

\begin{lstlisting}
class A { def f: Int = 1 ; def g: Int = 2 ; def h: Int = 3 }
abstract class B { def f: Int = 4 ; def g: Int }
abstract class C extends A with B { def h: Int }
\end{lstlisting}

Then class \code{C} has a directly bound abstract member \code{h}. It
inherits member \code{f} from class \code{B} and member \code{g} from
class \code{A}.

\ifqualified{
\example\label{ex:compound-b}
Consider the definitions:
\begin{lstlisting}
qualified class Root extends Any { def r1: Root, r2: Int }
qualified class A extends Root { def r1: A, a: String }
qualified class B extends A { def r1: B, b: Double }
\end{lstlisting}
Then ~\lstinline@A with B@~ has members
\lstinline@Root::r1@ of type \code{B}, \lstinline@Root::r2@ of type \code{Int},
\lstinline@A::a:@ of type \code{String}, and \lstinline@B::b@ of type \code{Double},
in addition to the members inherited from class \code{Any}.
}

\subsection{Overriding}
\label{sec:overriding}

A template member $M$ that has the same \ifqualified{qualified} name
as a non-private member $M'$ of a base class (and that belongs to the
same namespace) is said to {\em override} that member.  In this case
the binding of the overriding member $M$ must subsume
(\sref{sec:subtyping}) the binding of the overridden member $M'$.
Furthermore, the overridden definition may not be a class definition.
Method definitions may only override other method definitions (or the
methods implicitly defined by a variable definition). They may not
override value definitions.  Finally, the following restrictions
on modifiers apply to $M$ and $M'$:
\begin{itemize}
\item
$M'$ must not be labeled \code{final}.
\item
$M$ must not be labeled \code{private}.
\item
If $M$ is labeled \code{protected}, then $M'$ must also be
labeled \code{protected}.
\item
If $M'$ is not an abstract member, then
$M$ must be labeled \code{override}.
\item
If $M'$ is labeled \code{abstract} and \code{override}, and $M'$ is a
member of the static superclass of the class containing the definition
of $M$, then $M$ must also be labeled \code{abstract} and
\code{override}.
\end{itemize}

\example\label{ex:compound-a}
Consider the definitions:
\begin{lstlisting}
trait Root { type T <: Root }
trait A extends Root { type T <: A }
trait B extends Root { type T <: B }
trait C extends A with B;
\end{lstlisting}
Then the trait definition \code{C} is not well-formed because the
binding of \code{T} in \code{C} is
~\lstinline@type T <: B@,
which fails to subsume the binding ~\lstinline@type T <: A@~ of \code{T}
in type \code{A}. The problem can be solved by adding an overriding 
definition of type \code{T} in class \code{C}:
\begin{lstlisting}
class C extends A with B { type T <: C }
\end{lstlisting}

\subsection{Modifiers}
\label{sec:modifiers}

\syntax\begin{lstlisting}
  Modifier        ::=  LocalModifier
                    |  private
                    |  protected
                    |  override 
  LocalModifier   ::=  abstract
                    |  final
                    |  sealed
\end{lstlisting}

Member definitions may be preceded by modifiers which affect the
\ifqualified{qualified names, }accessibility and usage of the
identifiers bound by them.  If several modifiers are given, their
order does not matter, but the same modifier may not occur repeatedly.
Modifiers preceding a repeated definition apply to all constituent
definitions.  The rules governing the validity and meaning of a
modifier are as follows.
\begin{itemize}
\item
The \code{private} modifier can be used with any definition in a
template. Private members can be accessed only from within the template
that defines them.  
%Furthermore, accesses are not permitted in
%packagings (\sref{sec:topdefs}) other than the one containing the
%definition. 
Private members are not inherited by subclasses and they
may not override definitions in parent classes.
\code{private} may not be applied to abstract members, and it
may not be combined in one modifier list with
\code{protected}, \code{final} or \code{override}.
\item
The \code{protected} modifier applies to class member definitions.
Protected members can be accessed from within the template of the defining
class as well as in all templates that have the defining class as a base class.
%Furthermore, accesses from the template of the defining class are not
%permitted in packagings other than the one
%containing the definition.  
A protected identifier $x$ may be used as
a member name in a selection \lstinline@$r$.$x$@ only if $r$ is one of the reserved
words \code{this} and
\code{super}, or if $r$'s type conforms to a type-instance of the class
which contains the access.
\item
The \code{override} modifier applies to class member definitions.  It
is mandatory for member definitions that override some other concrete
member definition in a super- or mixin-class. If an \code{override}
modifier is given, there must be at least one overridden member
definition.  

The \code{override} modifier has an additional significance when
combined with the \code{abstract} modifier.  That modifier combination
is only allowed for members of abstract classes.  A member
labeled \code{abstract} and \code{override} must override some
member of the superclass of the class containing the definition.

We call a member of a template {\em incomplete} if it is either
abstract (i.e.\ defined by a declaration), or it is labeled
\code{abstract} and \code{override} and it overrides an incomplete
member of the template's superclass.

Note that the \code{abstract override} modifier combination does not
influence the concept whether a member is concrete or
abstract. A member for which only a declaration is given is abstract,
whereas a member for which a full definition is given is concrete.

\item
The \code{abstract} modifier is used in class definitions. It is
mandatory if the class has incomplete members.  Abstract classes
cannot be instantiated (\sref{sec:inst-creation}) with a constructor
invocation unless followed by mixin constructors or statements which
override all incomplete members of the class.

The \code{abstract} modifier can also be used in conjunction with
\code{override} for class member definitions. In that case the meaning
of the previous discussion applies.
\item
The \code{final} modifier applies to class member definitions and to
class definitions. A \code{final} class member definition may not be
overridden in subclasses. A \code{final} class may not be inherited by
a template. \code{final} is redundant for object definitions.  Members
of final classes or objects are implicitly also final, so the
\code{final} modifier is redundant for them, too.  \code{final} may
not be applied to incomplete members, and it may not be combined in one
modifier list with \code{private} or \code{sealed}.
\item
The \code{sealed} modifier applies to class definitions. A
\code{sealed} class may not be inherited, except if either
\begin{itemize}
\item
the inheriting template is nested within the definition of the sealed
class itself, or
\item
the inheriting template belongs to a class or object definition which
forms part of the same statement sequence as the definition of the
sealed class.
\end{itemize}
\end{itemize}

\example A useful idiom to prevent clients of a class from
constructing new instances of that class is to declare the class
\code{abstract} and \code{sealed}:

\begin{lstlisting}
object m {
  abstract sealed class C (x: Int) {
    def nextC = C(x + 1) {}
  }
  val empty = new C(0) {}
}
\end{lstlisting}
For instance, in the code above clients can create instances of class
\lstinline@m.C@ only by calling the \code{nextC} method of an existing \lstinline@m.C@
object; it is not possible for clients to create objects of class
\lstinline@m.C@ directly. Indeed the following two lines are both in error:

\begin{lstlisting}
  m.C(0)    // **** error: C is abstract, so it cannot be instantiated.
  m.C(0) {} // **** error: illegal inheritance from sealed class.
\end{lstlisting}

\subsection{Attributes}

\syntax\begin{lstlisting}
  AttributeClause ::=  `[' Attribute {`,' Attribute} `]'
  Attribute       ::=  Constr
\end{lstlisting}

Attributes associate meta-information with definitions.  A simple
attribute clause has the form $[C]$ or $[C(a_1 \commadots a_n)]$.
Here, $c$ is a constructor of a class $C$, which must comform to the
class \lstinline@scala.Attribute@. All given constructor arguments
$a_1 \commadots a_n$ must be constant expressions. An attribute clause
applies to the first definition or declaration following it. More than
one attribute clause may precede a definition and declaration. The
order in which these clauses are given does not matter.  It is also
possible to combine several attributres separated by commas in one
clause. Such a combined clause $[A_1 \commadots A_n]$ is equivalent to
a set of clauses $[A_1] \ldots [A_n]$.

The meaning of attribute clauses is implementation-dependent. On the
Java platform, the following attributes have a standard meaning.\bigskip

\lstinline@transient@
\begin{quote}
Marks a field to be non-persistent; this is
equivalent to the \lstinline@transient@
modifier in Java.
\end{quote}

\lstinline@volatile@
\begin{quote}Marks a field which can change its value
outside the control of the program; this
is equivalent to the \lstinline@volatile@
modifier in Java.
\end{quote}

\lstinline@Serializable@
\begin{quote}Marks a class to be serializable; this is
equivalent to inheriting from the 
\lstinline@java.io.Serializable@ interface
in Java.
\end{quote}

\lstinline@SerialVersionUID(<longlit>)@
\begin{quote}Attaches a serial version identifier (a
\lstinline@long@ constant) to a class.
This is equivalent to a the following field
definition in Java:
\begin{lstlisting}[language=Java]
  private final static SerialVersionUID = <longlit>;
\end{lstlisting}
\end{quote}


\section{Class Definitions}
\label{sec:classes}

\syntax\begin{lstlisting} 
  TmplDef          ::= class ClassDef 
  ClassDef         ::= ClassSig {`,' ClassSig} [`:' SimpleType] ClassTemplate 
  ClassSig         ::= id [TypeParamClause] [ClassParamClause] 
  ClassTemplate    ::= extends Template | TemplateBody |
  ClassParamClause ::= `(' [ClassParam {`,' ClassParam}] `)'
  ClassParam       ::= [{Modifier} `val'] Param
\end{lstlisting}

The most general form of class definition is 
~\lstinline@class $c$[$\tps\,$]($ps\,$): $s$ extends $t$@.
Here,
\begin{itemize}
\item[]
$c$ is the name of the class to be defined.
\item[] $\tps$ is a non-empty list of type parameters of the class
being defined.  The scope of a type parameter is the whole class
definition including the type parameter section itself.  It is
illegal to define two type parameters with the same name.  The type
parameter section \lstinline@[$\tps\,$]@ may be omitted. A class with a type
parameter section is called {\em polymorphic}, otherwise it is called
{\em monomorphic}.
\item[] 
$ps$ is a formal value parameter clause for the {\em primary
constructor} of the class. The scope of a formal value parameter includes
the template $t$. However, a formal value parameter may not form 
part of the types of any of the parent classes or members of $t$.
It is illegal to define two formal value parameters with the same name.
The formal parameter section \lstinline@($ps\,$)@ may be omitted, in which case
an empty parameter section \lstinline@()@ is assumed.

If a formal parameter declaration $x: T$ is preceded by a \code{val}
keyword, an accessor definition for this parameter is implicitly added
to the class. The accessor introduces a value member $x$ of $c$ that is
defined as alias of the parameter. The formal paremter declaration may
contain modifiers, which then carry over to the accessor definition.
\item[] 
$s$ is the {\em self type} of the class. Inside the
class, the type of \code{this} is assumed to be $s$.  The self
type must conform to the self types of all classes which are inherited
by the template $t$. The self type declaration `\lstinline@:$s$@' may be
omitted, in which case the self type of the class is assumed to be
equal to \lstinline@$c$[$\tps\,$]@.
\item[] 
$t$ is a
template (\sref{sec:templates}) of the form
\begin{lstlisting}
$sc$ with $mc_1$ with $\ldots$ with $mc_n$ { $\stats$ }   $\gap(n \geq 0)$
\end{lstlisting}
which defines the base classes, behavior and initial state of objects of
the class. The extends clause ~\lstinline@extends $sc$@~ 
can be omitted, in which case
~\lstinline@extends scala.AnyRef@~ is assumed.  The class body
~\lstinline@{$\stats\,$}@~ may also be omitted, in which case the empty body
\lstinline@{}@ is assumed.
\end{itemize}
This class definition defines a type \lstinline@$c$[$\tps\,$]@ and a constructor
which when applied to parameters conforming to types $ps$
initializes instances of type \lstinline@$c$[$\tps\,$]@ by evaluating the template
$t$.

For any index $i$ let $\csig_i$ be a class signature consisting of a class
name and optional type parameter and value parameter sections. Let $ct$
be a class template.
Then a class definition 
~\lstinline@class $\csig_1 \commadots \csig_n$ $ct$@~ 
is a shorthand for the sequence of class definitions
~\lstinline@class $\csig_1$ $ct$; ...; class $\csig_n$ $ct$@.
A class definition 
~\lstinline@class $\csig_1 \commadots \csig_n: T$ $ct$@~ 
is a shorthand for the sequence of class definitions
~\lstinline@class $\csig_1: T$ $ct$; ...; class $\csig_n: T$ $ct$@.

\subsection{Constructor Definitions}\label{sec:constr-defs}

\syntax\begin{lstlisting}
  FunDef          ::=  this ParamClause`=' ConstrExpr
  ConstrExpr      ::=  this ArgumentExprs
                    |  `{' this ArgumentExprs {`;' BlockStat} `}'
\end{lstlisting}

A class may have additional constructors besides the primary
constructor.  These are defined by constructor definitions of the form
~\lstinline@def this($ps\,$) = $e$@.  Such a definition introduces an
additional constructor for the enclosing class, with parameters as
given in the formal parameter list $ps$, and whose evaluation is
defined by the constructor expression $e$.  The scope of each formal
parameter is the constructor expression $e$.  A constructor expression
is either a self constructor invocation \lstinline@this($\args\,$)@ or
a block which begins with a self constructor invocation.  Neither the
signature, nor the self constructor invocation of a constructor
definition may refer to \verb@this@, or refer to value parameters or
members of the enclosing class by simple name.

If there are auxiliary constructors of a class $C$, they define
together with $C$'s primary constructor an overloaded constructor
value. The usual rules for overloading resolution
(\sref{sec:overloaded-defs}) apply for constructor invocations of $C$,
including the self constructor invocations in the constructor
expressions themselves. To prevent infinite cycles of constructor
invocations, there is the restriction that every self constructor
invocation must refer to a constructor definition which precedes it
(i.e. it must refer to either a preceding auxiliary constructor or the
primary constructor of the class).  The type of a constructor
expression must be always so that a generic instance of the class is
constructed.  I.e., if the class in question has name $C$ and type
parameters \lstinline@[$\tps\,$]@, then each constructor must construct an
instance of \lstinline@$C$[$\tps\,$]@; it is not permitted to instantiate formal
type parameters.

\example Consider the class definition

\begin{lstlisting}
class LinkedList[a]() {
  var head = _;
  var tail = null;
  def isEmpty = tail != null;  
  def this(head: a) = { this(); this.head = head; }
  def this(head: a, tail: List[a]) = { this(head); this.tail = tail }
}
\end{lstlisting}
This defines a class \code{LinkedList} with an overloaded constructor of type
\begin{lstlisting}
[a](): LinkedList[a]   $\overload$
[a](x: a): LinkedList[a]    $\overload$
[a](x: a, xs: LinkedList[a]): LinkedList[a]  .
\end{lstlisting}
The second constructor alternative constructs an singleton list, while the
third one constructs a list with a given head and tail.

\subsection{Case Classes}
\label{sec:case-classes}

\syntax\begin{lstlisting} 
  TmplDef  ::=  case class ClassDef
\end{lstlisting}

If a class definition is prefixed with \code{case}, the class is said
to be a {\em case class}.  The primary constructor of a case class may
be used in a constructor pattern (\sref{sec:patterns}).  
The following four restrictions ensure efficient pattern matching for
case classes.
\begin{enumerate}
\item None of the base classes of a case class may be a case
class. 
\item No type may have two different case classes among its base types. 
\item A case class may not inherit indirectly from a
\lstinline@sealed@ class.  That is, if a base class $b$ of a case class $c$
is marked \lstinline@sealed@, then $b$ must be a parent class of $c$.
\item
The primary constructor of a case class may not have any call-by-name
parameters (\sref{sec:parameters}).
\end{enumerate}

A case class definition of ~\lstinline@$c$[$\tps\,$]($ps\,$)@~ with type
parameters $\tps$ and value parameters $ps$ implicitly
generates a function definition for a {\em case class factory}
together with the class definition itself:
\begin{lstlisting}
def c[$\tps\,$]($ps\,$): $s$ = new $c$[$\tps\,$]($ps\,$)
\end{lstlisting}
(Here, $s$ is the self type of class $c$. 
If a type parameter section
is missing in the class, it is also missing in the factory
definition).  

All formal value parameters of a case class are implicitly prefixed
with a \code{val} keyword. Therefore, accessor definitions
(\sref{sec:classes}) for such parameters are generated.

Also implicitly defined are accessor member definitions
in the class that return its value parameters. Every binding
$x: T$ in the parameter section leads to a value definition of
$x$ that defines $x$ to be an alias of the parameter.  
%Every
%parameterless function binding \lstinline@def x: T@ leads to a
%parameterless function definition of $x$ which returns the result
%of invoking the parameter function.  
%The case class may not contain a
%directly bound member with the same simple name as one of its value
%parameters.

Every case class implicitly overrides some method definitions of class
\lstinline@scala.AnyRef@ (\sref{sec:cls-object}) unless a definition of the same
method is already given in the case class itself or a concrete
definition of the same method is given in some base class of the case
class different from \code{AnyRef}. In particular:
\begin{itemize}
\item[] Method ~\lstinline@equals: (Any)boolean@~ is structural equality, where two
instances are equal if they belong to the same class and
have equal (with respect to \code{equals}) primary constructor arguments.
\item[] Method ~\lstinline@hashCode: ()int@~ computes a hash-code
depending on the data structure in a way which maps equal (with respect to
\code{equals}) values to equal hash-codes.
\item[] Method ~\lstinline@toString: ()String@~ returns a string representation which
contains the name of the class and its primary constructor arguments.
\end{itemize}

\example Here is the definition of abstract syntax for lambda
calculus:

\begin{lstlisting}
class Expr;
case class
  Var       (x: String)              extends Expr,
  Apply     (f: Expr, e: Expr)       extends Expr,
  Lambda    (x: String, e: Expr)     extends Expr;
\end{lstlisting}
This defines a class \code{Expr} with case classes
\code{Var}, \code{Apply} and \code{Lambda}. A call-by-value evaluator for lambda
expressions could then be written as follows.

\begin{lstlisting}
type Env = String => Value;
case class Value(e: Expr, env: Env);

def eval(e: Expr, env: Env): Value = e match {
  case Var (x) =>
    env(x)
  case Apply(f, g) =>
    val Value(Lambda (x, e1), env1) = eval(f, env);
    val v = eval(g, env);
    eval (e1, (y => if (y == x) v else env1(y)))
  case Lambda(_, _) =>
    Value(e, env)
}
\end{lstlisting}

It is possible to define further case classes that extend type
\code{Expr} in other parts of the program, for instance
\begin{lstlisting}
case class Number(x: Int) extends Expr;
\end{lstlisting}

This form of extensibility can be excluded by declaring the base class
\code{Expr} \code{sealed}; in this case, the only classes permitted to
extend \code{Expr} are those which are nested inside \code{Expr}, or
which appear in the same statement sequence as the definition of
\code{Expr}.

\section{Traits}

\label{sec:traits}

\syntax\begin{lstlisting}
  TmplDef          ::=  trait ClassDef
\end{lstlisting}

A class definition which starts with the reserved word \code{trait}
instead of \code{class} defines a trait. A trait is a specific
instance of an abstract class, so the \code{abstract} modifier is
redundant for it.  The trait definition must satisfy the following
four restrictions.
\begin{enumerate}
\item There are no value parameters in the trait's primary constructor, nor
      are there secondary constructors.
\item All mixin base classes of the trait are traits.
\item All parent class constructors of the trait
      are primary constructors with empty value
      parameter lists. 
\item All non-empty statements in the trait's template are either
      imports or pure definitions.
\end{enumerate}
A {\em pure} definition can be evaluated without any side effect.
Function, type, class, or object definitions are always pure. A value
definition is pure if its right-hand side expression is pure. A
secondary constructor definition is pure if its right-hand side 
consists only
Pure
expressions are paths, literals, and typed expressions $e: T$ where
$e$ is pure.

These restrictions ensure that the evaluation of the mixin constructor
of a trait has no effect. Therefore, traits may appear several times 
in the base classes of a template, whereas other classes cannot.
%\item Packagings may add interface classes as new base classes to an
%existing class or module.

\example\label{ex:comparable}
The following trait class defines the property of being
ordered, i.e. comparable to objects of some type. It contains an abstract method
\lstinline@<@ and default implementations of the other comparison operators
\lstinline@<=@, \lstinline@>@, and \lstinline@>=@.

\begin{lstlisting}
trait Ord[t <: Ord[t]]: t {
  def < (that: t): Boolean;
  def <=(that: t): Boolean = this < that || this == that;
  def > (that: t): Boolean = that < this;
  def >=(that: t): Boolean = that <= this;
}
\end{lstlisting}

\section{Object Definitions}
\label{sec:modules}
\label{sec:object-defs}

\syntax\begin{lstlisting}
  ObjectDef       ::=  id {`,' id} [`:' SimpleType] ClassTemplate
\end{lstlisting}

An object definition defines a single object of a new class. Its 
most general form is
~\lstinline@object $m$: $s$ extends $t$@. Here,
\begin{itemize}
\item[]
$m$ is the name of the object to be defined.
\item[] $s$ is the {\em self type} of the object. References to $m$
are assumed to have type $s$. Furthermore, inside the template $t$,
the type of \code{this} is also assumed to be $s$.  The type of the
anonymous class defined by $t$ must conform to $s$ and $s$ must
conform to the self types of all classes which are inherited by
$t$. The self type declaration `$:s$' may be omitted, in which case
the self type is assumed to be equal to the anonymous class defined by
$t$.
\item[] 
$t$ is a
template (\sref{sec:templates}) of the form
\begin{lstlisting}
$sc$ with $mc_1$ with $\ldots$ with $mc_n$ { $\stats$ }
\end{lstlisting}
which defines the base classes, behavior and initial state of $m$.
The extends clause ~\lstinline@extends $sc$@~
can be omitted, in which case
~\lstinline@extends scala.AnyRef@~ is assumed.  The class body
~\lstinline@{$\stats\,$}@~ may also be omitted, in which case the empty body
\lstinline@{}@ is assumed.
\end{itemize}
The object definition defines a single object (or: {\em module})
conforming to the template $t$.  It is roughly equivalent to a class
definition and a value definition that creates an object of the class:
\begin{lstlisting}
final class $m\Dollar$cls: $s$ extends $t$;
final val $m$: $s$ = new m$\Dollar$cls;
\end{lstlisting}
(The \code{final} modifiers are omitted if the definition occurs as
part of a block. The class name \lstinline@$m\Dollar$cls@ is not
accessible for user programs.)

There are however two differences between an object definition and a
pair of class and value definitions such as the one given above.  First,
object definitions may appear as top-level definitions in a
compilation unit, whereas value definitions may not.  Second, the
module defined by an object definition is instantiated lazily.  The
~\lstinline@new $m\Dollar$cls@~ constructor is evaluated not at the point
of the object definition, but is instead evaluated the first time $m$
is dereferenced during execution of the program (which might be never
at all). An attempt to dereference $m$ again in the course of
evaluation of the constructor leads to a infinite loop or run-time
error.  Other threads trying to dereference $m$ while the constructor
is being evaluated block until evaluation is complete.

\example
Classes in Scala do not have static members; however, an equivalent
effect can be achieved by an accompanying object definition
E.g.
\begin{lstlisting}
abstract class Point {
  val x: Double;
  val y: Double;
  def isOrigin = (x == 0.0 && y == 0.0);
}
object Point {
  val origin = new Point() { val x = 0.0; val y = 0.0 }
}
\end{lstlisting}
This defines a class \code{Point} and an object \code{Point} which
contains \code{origin} as a member.  Note that the double use of the
name \code{Point} is legal, since the class definition defines the name
\code{Point} in the type name space, whereas the object definition
defines a name in the term namespace.

This technique is applied by the Scala compiler when interpreting a
Java class with static members. Such a class $C$ is conceptually seen
as a pair of a Scala class that contains all instance members of $C$
and a Scala object that contains all static members of $C$.

Let $ct$ be a class template. 
Then an object definition 
~\lstinline@object $x_1 \commadots x_n$ $ct$@~ 
is a shorthand for the sequence of object definitions
~\lstinline@object $x_1$ $ct$; ...; object $x_n$ $ct$@.
An object definition 
~\lstinline@object $x_1 \commadots x_n: T$ $ct$@~ 
is a shorthand for the sequence of object definitions
~\lstinline@object $x_1: T$ $ct$; ...; object $x_n: T$ $ct$@.


\comment{
\example Here's an outline of a module definition for a file system.

\begin{lstlisting}
module FileSystem {
  private type FileDirectory;
  private val dir: FileDirectory

  interface File {
    def read(xs: Array[Byte])
    def close: Unit
  }

  private class FileHandle extends File { $\ldots$ }

  def open(name: String): File = $\ldots$
}
\end{lstlisting}
}

\chapter{Expressions}
\label{sec:exprs}

\syntax\begin{lstlisting}
  Expr            ::=  [Bindings `=>'] Expr
                    |  Expr1
  Expr1           ::=  if `(' Expr `)' Expr [[`;'] else Expr]
                    |  try `{' block `}' [catch Expr] [finally Expr]
                    |  while '(' Expr ')' Expr
                    |  do Expr [`;'] while `(' Expr ')'
                    |  for `(' Enumerators `)' (do | yield) Expr
                    |  return [Expr]
                    |  throw Expr
                    |  [SimpleExpr `.'] id `=' Expr
                    |  SimpleExpr ArgumentExprs `=' Expr
                    |  PostfixExpr [`:' Type1]
                    |  MethodClosure
  PostfixExpr     ::=  InfixExpr [id]
  InfixExpr       ::=  PrefixExpr
                    |  InfixExpr id PrefixExpr
  PrefixExpr      ::=  [`-' | `+' | `~' | `!'] SimpleExpr 
  SimpleExpr      ::=  Literal
                    |  Path
                    |  `(' [Expr] `)'
                    |  BlockExpr
                    |  new Template 
                    |  SimpleExpr `.' id 
                    |  SimpleExpr TypeArgs
                    |  SimpleExpr ArgumentExprs
                    |  XmlExpr
  ArgumentExprs   ::=  `(' [Exprs] ')'
                    |  BlockExpr
  MethodClosure   ::=  `.' Id {`.' Id | TypeArgs | ArgumentExprs}
  BlockExpr       ::=  `{' CaseClause {CaseClause} `}'
                    |  `{' Block `}'
  Block           ::=  {BlockStat `;'} [ResultExpr]
  ResultExpr      ::=  Expr1
                    |  Bindings `=>' Block
  Exprs           ::=  Expr {`,' Expr}
\end{lstlisting}

Expressions are composed of operators and operands. Expression forms are
discussed subsequently in decreasing order of precedence. 

The typing of expressions is often relative to some {\em expected
type}.  When we write ``expression $e$ is expected to conform to
type $T$'', we mean: (1) the expected type of $e$ is
$T$, and (2) the type of expression $e$ must conform to
$T$.

\section{Literals}

\syntax\begin{lstlisting}
  SimpleExpr    ::=  Literal
  Literal       ::=  intLit
                  |  floatLit
                  |  charLit
                  |  stringLit
                  |  symbolLit
                  |  true
                  |  false
                  |  null
\end{lstlisting}

Typing and evaluation of numeric, character, and string literals are
generally as in Java.  An integer literal denotes an integer
number. Its type is normally \code{int}. However, if the expected type
$\proto$ of the expression is either \code{byte}, \code{short}, or
\code{char} and the integer number fits in the numeric range defined
by the type, then the number is converted to type $\proto$ and the
expression's type is $\proto$.  A floating point literal denotes a
single-precision or double precision IEEE floating point number. A
character literal denotes a Unicode character. A string literal
denotes a member of \lstinline@String@.

A symbol literal ~\lstinline@'$x$@~ is a shorthand for the expression
~\lstinline@scala.Symbol("$x$")@. If the symbol literal is followed by
actual parameters, as in ~\lstinline@'$x$($\args\,$)@, then the whole
expression is taken to be a shorthand for
~\lstinline@scala.Symbol("$x$", $\args\,$)@.

The boolean truth values are denoted by the reserved words \code{true}
and \code{false}. The type of these expressions is \code{boolean}, and
their evaluation is immediate. 

The \code{null} literal is of type \lstinline@scala.AllRef@. It
denotes a reference value which refers to a special ``null' object,
which implements methods in class \lstinline@scala.AnyRef@ as follows:
\begin{itemize}
\item
\lstinline@eq($x\,$)@, \lstinline@==($x\,$)@, \lstinline@equals($x\,$)@ return \code{true} iff their
argument $x$ is also the ``null'' object.
\item
\lstinline@isInstanceOf[$T\,$]@ always returns \code{false}.
\item
\lstinline@asInstanceOf[$T\,$]@ returns the ``null'' object itself if
$T$ conforms to \lstinline@scala.AnyRef@, and throws a
\lstinline@NullPointerException@ otherwise.
\item
\code{toString()} returns the string ``null''.
\end{itemize}
A reference to any other member of the ``null'' object causes a
\code{NullPointerException} to be thrown. 

\section{Designators}
\label{sec:designators}

\syntax\begin{lstlisting}
  Designator  ::=  Path
                |  SimpleExpr `.' id
\end{lstlisting}

A designator refers to a named term. It can be a {\em simple name} or
a {\em selection}. If $r$ is a stable identifier of type $T$, the
selection $r.x$ refers to the term member of $r$ that is identified in
$T$ by the name $x$.  For other expressions $e$, $e.x$ is typed as if
it was $(\VAL;y=e\semi y.x)$ for some fresh name $y$. The typing rules
for blocks implies that in that case $x$'s type may not refer to any
abstract type member of $e$.

The expected type of a designator's prefix is always missing.
The
type of a designator is normally the type of the entity it refers
to. However, if the designator is a path (\sref{sec:paths}) $p$,
its type is \lstinline@$p$.type@, provided the expression's expected type is
a singleton type, or $p$ occurs as the prefix of a selection
or type selection.

The selection $e.x$ is evaluated by first evaluating the qualifier
expression $e$. The selection's result is then the value to which the
selector identifier is bound in the object resulting from evaluation of $e$.

\section{This and Super}
\label{sec:this-super}

\syntax\begin{lstlisting}
  SimpleExpr    ::=  [id `.'] this
                  |  [id `.'] super [`[' id `]'] `.' id
\end{lstlisting}

The expression \code{this} can appear in the statement part of a
template or compound type. It stands for the object being defined by
the innermost template or compound type enclosing the reference. If
this is a compound type, the type of \code{this} is that compound type.
If it is a template of an instance creation expression, the type of
\code{this} is the type of that template. If it is a template of a
class or object definition with simple name $C$, the type of this
is the same as the type of \lstinline@$C$.this@.

The expression \lstinline@$C$.this@ is legal in the statement part of an
enclosing class or object definition with simple name $C$. It
stands for the object being defined by the innermost such definition.
If the expression's expected type is a singleton type, or
\lstinline@$C$.this@ occurs as the prefix of a selection, its type is
\lstinline@$C$.this.type@, otherwise it is the self type of class $C$.

A reference \lstinline@super.$m$@ in a template refers to the
definition of $m$ in the actual superclass (\sref{sec:base-classes})
of the template.  A reference \lstinline@$C$.super.$m$@ refers to the
definition of $m$ in the actual superclass of the innermost enclosing
class or object definition named $C$ which encloses the reference. The
definition $m$ referred to via \code{super} or \lstinline@$C$.super@
must be concrete, or the template containing the reference must have an
incomplete (\sref{sec:modifiers}) member $m'$ which overrides $m$.

The \code{super} prefix may be followed by a mixin qualifier
\lstinline@[$M\,$]@, as in \lstinline@$C$.super[$M\,$].$x$@. This is called a {\em mixin
super reference}.  In this case, the reference is to the member of
$x$ in the (first) mixin class of $C$ whose simple name
is $M$. That member may not be abstract.

\example\label{ex:super}
Consider the following class definitions

\begin{lstlisting}
class Root { val x = "Root" }
class A extends Root { override val x = "A" ; val superA = super.x }
class B extends Root { override val x = "B" ; val superB = super.x }
class C extends A with B { 
  override val x = "C" ; val superC = super.x 
}
class D extends A { val superD = super.x }
class E extends C with D { val superE = super.x }
\end{lstlisting}
Then we have:
\begin{lstlisting}
(new A).superA == "Root", (new B).superB == "Root"
(new C).superA == "Root", (new C).superB == "A", (new C).superC == "A"
(new D).superA == "Root", (new D).superD == "A"
(new E).superA == "Root", (new E).superB == "A", (new E).superC == "A",
   (new E).superD == "C", (new E).superE == "C"
\end{lstlisting}
Note that the \code{superB} function returns different results
depending on whether \code{B} is used as defining class or as a mixin class.

\example Consider the following class definitions:
\begin{lstlisting}
class Shape {
  override def equals(other: Any) = $\ldots$;
  $\ldots$
}
trait Bordered extends Shape {
  val thickness: int;
  override def equals(other: Any) = other match {
    case that: Bordered => 
      super equals other && this.thickness == that.thickness
    case _ => false
  }
  $\ldots$
}
trait Colored extends Shape {
  val color: Color;
  override def equals(other: Any) = other match {
    case that: Colored => 
      super equals other && this.color == that.color
    case _ => false
  }
  $\ldots$
}
\end{lstlisting}

Both definitions of \code{equals} are combined in the class
below.
\begin{lstlisting}
trait BorderedColoredShape extends Shape with Bordered with Colored {
  override def equals(other: Any) = 
    super[Bordered].equals(that) && super[Colored].equals(that)
}
\end{lstlisting}

\section{Function Applications}
\label{sec:apply}

\syntax\begin{lstlisting}
  SimpleExpr    ::=  SimpleExpr ArgumentExprs
\end{lstlisting}

An application \lstinline@$f$($e_1 \commadots e_n$)@ applies the function $f$ to the
argument expressions $e_1 \commadots e_n$. If $f$ has a method type
\lstinline@($T_1 \commadots T_n$)U@, the type of each argument
expression $e_i$ must conform to the corresponding parameter type
$T_i$. If $f$ has some value type, the application is taken to be
equivalent to \lstinline@$f$.apply($e_1 \commadots e_n$)@, i.e.\ the
application of an \code{apply} method defined by $f$.

%Class constructor functions
%(\sref{sec:classes}) can only be applied in constructor invocations
%(\sref{sec:constr-invoke}), never in expressions.

Evaluation of \lstinline@$f$($e_1 \commadots e_n$)@ usually entails evaluation of
$f$ and $e_1 \commadots e_n$ in that order. Each argument expression
is converted to the type of its corresponding formal parameter.  After
that, the application is rewritten to the function's right hand side,
with actual arguments substituted for formal parameters.  The result
of evaluating the rewritten right-hand side is finally converted to
the function's declared result type, if one is given.

The case of a formal parameter with a parameterless
method type \lstinline@=> $T$@ is treated specially. In this case, the
corresponding actual argument expression is not evaluated before the
application. Instead, every use of the formal parameter on the
right-hand side of the rewrite rule entails a re-evaluation of the
actual argument expression. In other words, the evaluation order for
\code{def}-parameters is {\em call-by-name} whereas the evaluation
order for normal parameters is {\em call-by-value}.

\section{Type Applications}
\label{sec:type-app}
\syntax\begin{lstlisting}
  SimpleExpr    ::=  SimpleExpr `[' Types `]'
\end{lstlisting}

A type application \lstinline@$e$[$T_1 \commadots T_n$]@ instantiates a
polymorphic value $e$ of type
~\lstinline@[$a_1$ >: $L_1$ <: $U_1 \commadots a_n$ >: $L_n$ <: $U_n$]S@~ with
argument types \lstinline@$T_1 \commadots T_n$@.  Every argument type
$T_i$ must obey corresponding bounds $L_i$ and
$U_i$.  That is, for each $i = 1 \commadots n$, we must
have $L_i \sigma \conforms T_i \conforms U_i \sigma$, where $\sigma$ is the
substitution $[a_1 := T_1 \commadots a_n := T_n]$.  The type
of the application is \lstinline@S$\sigma$@.  

The function part $e$ may also have some value type. In this case
the type application is taken to be equivalent to
~\lstinline@$e$.apply[$T_1 \commadots$ T$_n$]@, i.e.\ the
application of an \code{apply} method defined by $e$.

Type applications can be omitted if local type inference
(\sref{sec:local-type-inf}) can infer best type parameters for a
polymorphic functions from the types of the actual function arguments
and the expected result type.

\section{References to Overloaded Bindings}
\label{sec:overloaded-refs}

If a name $f$ referenced in an identifier or selection is
overloaded (\sref{sec:overloaded-defs}), the context of the reference
has to identify a unique alternative of the overloaded binding. The
way this is done depends on whether or not $f$ is used as a
function.  Let $\AA$ be the set of all type alternatives of
$f$.

Assume first that $f$ appears as a function in an application, as
in \lstinline@$f$($\args\,$)@.  If there is precisely one alternative in
$\AA$ which is a (possibly polymorphic) method type whose arity
matches the number of arguments given, that alternative is chosen.

Otherwise, let $\argtypes$ be the vector of types obtained by
typing each argument with a missing expected type. One determines
first the set of applicable alternatives. A method type alternative is
{\em applicable} if each type in $\argtypes$ is compatible with
the corresponding formal parameter type in the alternative, and, if 
the expected type is defined, the method's result type is compatible to
it.  A polymorphic method type is applicable if local type inference
can determine type arguments so that the instantiated method type is
applicable.

Here, a type $T$ is {\em compatible} to a type $U$ if $T$
conforms to $U$ after applying implicit conversions
(\sref{sec:impl-conv}).

Let $\BB$ be the set of applicable alternatives. It is an error if
$\BB$ is empty. Otherwise, one chooses the {\em most specific}
alternative among the alternatives in $\BB$, according to the
following definition of being ``more specific''.
\begin{itemize} 
\item
A method type \lstinline@($\Ts\,$)$U$@ is more specific than some other
type $S$ if $S$ is applicable to arguments \lstinline@($ps\,$)@ of
types $\Ts$.
\item
A polymorphic method type
~\lstinline@[$a_1$ >: $L_1$ <: $U_1 \commadots a_n$ >: $L_n$ <: $U_n$]T@~ is
more specific than some other type $S$ if $T$ is more
specific than $S$ under the assumption that for
$i = 1 \commadots n$ each $a_i$ is an abstract type name
bounded from below by $L_i$ and from above by $U_i$.
\item
Any other type is always more specific than a parameterized method
type or a polymorphic type.
\end{itemize}
It is an error if there is no unique alternative in $\BB$ which is
more specific than all other alternatives in $\BB$.

Assume next that $f$ appears as a function in a type
application, as in \lstinline@$f$[$\targs\,$]@. Then we choose an alternative in
$\AA$ which takes the same number of type parameters as there are
type arguments in $\targs$. It is an error if no such alternative
exists, or if it is not unique.

Assume finally that $f$ does not appear as a function in either
an application or a type application. If an expected type is given,
let $\BB$ be the set of those alternatives in $\AA$ which are
compatible to it. Otherwise, let $\BB$ be the same as $\AA$.
We choose in this case the most specific alternative among all
alternatives in $\BB$. It is an error if there is no unique
alternative in $\BB$ which is more specific than all other
alternatives in $\BB$.

\example Consider the following definitions:

\begin{lstlisting}
  class A extends B {}
  def f(x: B, y: B) = $\ldots$
  def f(x: A, y: B) = $\ldots$
  val a: A;
  val b: B
\end{lstlisting}
Then the application \lstinline@f(b, b)@ refers to the first
definition of $f$ whereas the application \lstinline@f(a, a)@
refers to the second.  Assume now we add a third overloaded definition
\begin{lstlisting}
  def f(x: B, y: A) = $\ldots$
\end{lstlisting}
Then the application \lstinline@f(a, a)@ is rejected for being ambiguous, since
no most specific applicable signature exists.

\section{Instance Creation Expressions}
\label{sec:inst-creation}

\syntax\begin{lstlisting}
  SimpleExpr     ::=  new Template
\end{lstlisting}

A simple instance creation expression is of the form ~\lstinline@new $c$@~ 
where $c$ is a constructor invocation
(\sref{sec:constr-invoke}).  Let $T$ be the type of $c$. Then $T$ must
denote a (a type instance of) a non-abstract subclass of
\lstinline@scala.AnyRef@ which conforms to its self type
(\sref{sec:classes}). The expression is evaluated by creating a fresh
object of type $T$ which is is initialized by evaluating $c$. The
type of the expression is $T$'s self type (which might be less
specific than $T\,$).

A general instance creation expression is of the form
\begin{lstlisting}
new $sc$ with $mc_1$ with $\ldots$ with $mc_n$ {$\stats\,$}
\end{lstlisting}
where $n \geq 0$, $sc$ as well as $mc_1 \commadots mc_n$ are
constructor invocations (of types $S, T_1 \commadots T_n$, say) and
$\stats$ is a statement sequence containing initializer statements and
member definitions (\sref{sec:members}). The type of such an instance
creation expression is then the compound type
\lstinline@$S$ with $T_1$ with $\ldots$ with $T_n$ {$R\,$}@,
where \lstinline@{$R\,$}@ is
a refinement (\sref{sec:compound-types}) which declares exactly those
members of $\stats$ that override a member of $S$ or $T_1 \commadots
T_n$. \todo{what about methods and overloaded defs?}  For this type to
be well-formed, $R$ may not reference types defined in $\stats$ which
do not themselves form part of $R$.

The instance creation expression is evaluated by creating a fresh
object, which is initialized by evaluating the expression template.

\example Consider the class
\begin{lstlisting}
abstract class C {
  type T; val x: T; def f(x: T): AnyRef
}
\end{lstlisting}
and the instance creation expression
\begin{lstlisting}
C { type T = Int; val x: T = 1; def f(x: T): T = y; val y: T = 2 }
\end{lstlisting}
Then the created object's type is:
\begin{lstlisting}
C { type T = Int; val x: T; def f(x: T): T }
\end{lstlisting}
The value $y$ is missing from the type, since $y$ does not
override a member of $C$.

\section{Blocks}
\label{sec:blocks}

\syntax\begin{lstlisting}
  BlockExpr   ::=  `{' Block `}'
  Block       ::=  [{BlockStat `;'} ResultExpr]
\end{lstlisting}

A block expression ~\lstinline@{$s_1$; $\ldots$; $s_n$; $e\,$}@~ is constructed from a
sequence of block statements $s_1 \commadots s_n$ and a final
expression $e$. The final expression can be omitted, in which
case the unit value \lstinline@()@ is assumed.

%Whether or not the scope includes the statement itself
%depends on the kind of definition.

The expected type of the final expression $e$ is the expected
type of the block. The expected type of all preceding statements is
missing.

The type of a block ~\lstinline@$s_1$; $\ldots$; $s_n$; $e$@~ is usually the type of
$e$.  That type must be equivalent to a type which does not refer
to an entity defined locally in the block. If this condition is
violated, but a fully defined expected type is given, the type of the
block is instead assumed to be the expected type.

Evaluation of the block entails evaluation of its statement sequence,
followed by an evaluation of the final expression $e$, which
defines the result of the block.

\example
Written in isolation, 
the block 
\begin{lstlisting}
{ class C extends B {$\ldots$} ; new C }
\end{lstlisting}
is illegal, since its type
refers to class $C$, which is defined locally in the block.

However, when used in a definition such as 
\begin{lstlisting}
val x: B = { class C extends B {$\ldots$} ; new C }
\end{lstlisting}
the block is well-formed, since the problematic type $C$ can be
replaced by the expected type $B$.

\section{Prefix, Infix, and Postfix Operations}
\label{sec:infix-operations}

\syntax\begin{lstlisting}
  PostfixExpr     ::=  InfixExpr [id]
  InfixExpr       ::=  PrefixExpr
                    |  InfixExpr id PrefixExpr
  PrefixExpr      ::=  [`-' | `+' | `!' | `~'] SimpleExpr 
\end{lstlisting}

Expressions can be constructed from operands and operators.  A prefix
operation $op;e$ consists of a prefix operator $op$, which
must be one of the identifiers `\lstinline@+@', `\lstinline@-@', `\lstinline@!@', or
`\lstinline@~@', and a simple expression $e$.  The expression is
equivalent to the postfix method application $e.op$.

Prefix operators are different from normal function applications in
that their operand expression need not be atomic. For instance, the
input sequence \lstinline@-sin(x)@ is read as \lstinline@-(sin(x))@, whereas the
function application \lstinline@negate sin(x)@ would be parsed as the
application of the infix operator \code{sin} to the operands
\code{negate} and \lstinline@(x)@.

An infix or postfix operator can be an arbitrary identifier. Infix
operators have precedence and associativity defined as follows:

The {\em precedence} of an infix operator is determined by the operator's first
character. Characters are listed below in increasing order of
precedence, with characters on the same line having the same precedence.
\begin{lstlisting}
        $\mbox{\rm\sl(all letters)}$
        |
        ^
        &
        < >
        = !
        :
        + -
        * / %
        $\mbox{\rm\sl(all other special characters)}$
\end{lstlisting}
That is, operators starting with a letter have lowest precedence,
followed by operators starting with `\lstinline@|@', etc.

The {\em associativity} of an operator is determined by the operator's
last character.  Operators ending with a colon `\lstinline@:@' are
right-associative. All other operators are left-associative.

Precedence and associativity of operators determine the grouping of
parts of an expression as follows.
\begin{itemize}
\item If there are several infix operations in an
expression, then operators with higher precedence bind more closely
than operators with lower precedence.
\item If there are consecutive infix
operations $e_0; \op_1; e_1; \op_2 \ldots \op_n; e_n$ 
with operators $\op_1 \commadots \op_n$ of the same precedence, 
then all these operators must
have the same associativity. If all operators are left-associative,
the sequence is interpreted as
$(\ldots(e_0;\op_1;e_1);\op_2\ldots);\op_n;e_n$. 
Otherwise, if all operators are right-associative, the
sequence is interpreted as
$e_0;\op_1;(e_1;\op_2;(\ldots \op_n;e_n)\ldots)$.
\item
Postfix operators always have lower precedence than infix
operators. E.g.\ $e_1;\op_1;e_2;\op_2$ is always equivalent to
$(e_1;\op_1;e_2);\op_2$.
\end{itemize}
A postfix operation $e;\op$ is interpreted as $e.\op$. A
left-associative binary operation $e_1;\op;e_2$ is interpreted as
$e_1.\op(e_2)$. If $\op$ is right-associative, the same operation is
interpreted as ~\lstinline@(val $x$=$e_1$; $e_2$.$\op$($x\,$))@, 
where $x$ is a fresh name.

\section{Typed Expressions}

\syntax\begin{lstlisting}
  Expr1              ::=  PostfixExpr [`:' Type1]
\end{lstlisting}

The typed expression $e: T$ has type $T$. The type of
expression $e$ is expected to conform to $T$. The result of
the expression is the value of $e$ converted to type $T$.

\example Here are examples of well-typed and illegally typed expressions.

\begin{lstlisting}
  1: int               // legal, of type int
  1: long              // legal, of type long
  // 1: string         // illegal
\end{lstlisting}

\section{Method closures}
\syntax\begin{lstlisting}
  MethodClosure   ::=  `.' Id {`.' Id | TypeArgs | ArgumentExprs}
\end{lstlisting}

A method closure $.id$ starts with a period and an identifier, which
may be followed by selections and type- and value-arguments. This
expression is equivalenet to an anonymous function
\lstinline@$x$ => $x.id$@ where $x$ is a fresh parameter name. No type
for $x$ is given; hence this type needs to be inferrable from the
context of the expression.

\example The following method returns the $n$'th column of a given
list of row-lists $xss$, using methods \lstinline@map@,
\lstinline@drop@ and \lstinline@head@ defined in class
\lstinline@scala.List@.
\begin{lstlisting}
def column[T](xss: List[List[T]], n: int): List[T] = 
  xss.map(.drop(i)).map(.head)
\end{lstlisting}
 
\section{Assignments}

\syntax\begin{lstlisting}
  Expr1        ::=  Designator `=' Expr
                 |  SimpleExpr ArgumentExprs `=' Expr
\end{lstlisting}

The interpretation of an assignment to a simple variable ~\lstinline@$x$ = $e$@~
depends on the definition of $x$. If $x$ denotes a mutable
variable, then the assignment changes the current value of $x$ to be
the result of evaluating the expression $e$. The type of $e$ is
expected to conform to the type of $x$. If $x$ is a parameterless
function defined in some template, and the same template contains a
setter function \lstinline@$x$_=@ as member, then the assignment
~\lstinline@$x$ = $e$@~ is interpreted as the invocation
~\lstinline@$x$_=($e\,$)@~ of that setter function.  Analogously, an
assignment ~\lstinline@$f.x$ = $e$@~ to a parameterless function $x$
is interpreted as the invocation ~\lstinline@$f.x$_=($e\,$)@.

An assignment ~\lstinline@$f$($\args\,$) = $e$@~ with a function application to the
left of the ``\lstinline@=@' operator is interpreted as 
~\lstinline@$f.$update($\args$, $e\,$)@, i.e.\
the invocation of an \code{update} function defined by $f$.

\example \label{ex:imp-mat-mul}
Here is the usual imperative code for matrix multiplication.

\begin{lstlisting}
def matmul(xss: Array[Array[double]], yss: Array[Array[double]]) = {
  val zss: Array[Array[double]] = new Array(xss.length, yss.length);
  var i = 0;
  while (i < xss.length) {
    var j = 0;
    while (j < yss(0).length) {
      var acc = 0.0;
      var k = 0;
      while (k < yss.length) {
        acc = acc + xs(i)(k) * yss(k)(j);
        k = k + 1
      }
      zss(i)(j) = acc;
      j = j + 1
    }
    i = i + 1
  }
  zss
}
\end{lstlisting}
Desugaring the array accesses and assignments yields the following
expanded version:
\begin{lstlisting}
def matmul(xss: Array[Array[double]], yss: Array[Array[double]]) = {
  val zss: Array[Array[double]] = new Array(xss.length, yss.length);
  var i = 0;
  while (i < xss.length) {
    var j = 0;
    while (j < yss(0).length) {
      var acc = 0.0;
      var k = 0;
      while (k < yss.length) {
        acc = acc + xss.apply(i).apply(k) * yss.apply(k).apply(j);
        k = k + 1
      }
      zss.apply(i).update(j, acc);
      j = j + 1
    }
    i = i + 1
  }
  zss
}
\end{lstlisting}

\section{Conditional Expressions}

\syntax\begin{lstlisting}
  Expr1          ::=  if `(' Expr `)' Expr [[`;'] else Expr]
\end{lstlisting}

The conditional expression ~\lstinline@if ($e_1$) $e_2$ else $e_3$@~ chooses
one of the values of $e_2$ and $e_3$, depending on the
value of $e_1$. The condition $e_1$ is expected to
conform to type \code{boolean}.  The then-part $e_2$ and the
else-part $e_3$ are both expected to conform to the expected
type of the conditional expression. The type of the conditional
expression is the least upper bound of the types of $e_1$ and
$e_2$.  A semicolon preceding the \code{else} symbol of a
conditional expression is ignored.

The conditional expression is evaluated by evaluating first
$e_1$. If this evaluates to \code{true}, the result of
evaluating $e_2$ is returned, otherwise the result of
evaluating $e_3$ is returned.

A short form of the conditional expression eliminates the
else-part. The conditional expression ~\lstinline@if ($e_1$) $e_2$@~ is
evaluated as if it was ~\lstinline@if ($e_1$) $e_2$ else ()@.  The type of
this expression is \code{unit} and the then-part
$e_2$ is also expected to conform to type \code{unit}.

\section{While Loop Expressions}

\syntax\begin{lstlisting}
  Expr1          ::=  while `(' Expr ')' Expr
\end{lstlisting}

The while loop expression ~\lstinline@while ($e_1$) $e_2$@~ is typed and
evaluated as if it was an application of ~\lstinline@whileLoop ($e_1$) ($e_2$)@~ where
the hypothetical function \code{whileLoop} is defined as follows.

\begin{lstlisting}
  def whileLoop(def c: boolean)(def s: unit): unit  =
    if (c) { s ; while(c)(s) } else {}
\end{lstlisting}

\example The loop 
\begin{lstlisting}
  while (x != 0) { y = y + 1/x ; x = x - 1 }
\end{lstlisting}
Is equivalent to the application
\begin{lstlisting}
  whileLoop (x != 0) { y = y + 1/x ; x = x - 1 }
\end{lstlisting}
Note that this application will never produce a division-by-zero 
error at run-time, since the
expression ~\lstinline@(y = 1/x)@~ will be evaluated in the body of
\code{while} only if the condition parameter is false.

\section{Do Loop Expressions}

\syntax\begin{lstlisting}
  Expr1          ::=  do Expr [`;'] while `(' Expr ')'
\end{lstlisting}

The do loop expression ~\lstinline@do $e_1$ while ($e_2$)@~ is typed and
evaluated as if it was the expression ~\lstinline@($e_1$ ; while ($e_2$) $e_1$)@.
A semicolon preceding the \code{while} symbol of a do loop expression is ignored.

\section{Comprehensions}

\syntax\begin{lstlisting}
  Expr1          ::=  for `(' Enumerators `)' [yield] Expr
  Enumerator     ::=  Generator {`;' Enumerator}
  Enumerator     ::=  Generator
                   |  Expr
  Generator      ::=  val Pattern1 `<-' Expr
\end{lstlisting}

A comprehension ~\lstinline@for ($\enums\,$) yield $e$@~ evaluates
expression $e$ for each binding generated by the enumerators
$\enums$. Enumerators start with a generator, which can be followed by
further generators or filters.  A {\em generator} 
~\lstinline@val $p$ <- $e$@~ 
produces bindings from an expression $e$ which is matched in
some way against pattern $p$. A {\em filter} is an expressions which restricts
enumerated bindings. The precise meaning of generators and filters is
defined by translation to invocations of four methods: \code{map},
\code{filter}, \code{flatMap}, and \code{foreach}. These methods can
be implemented in different ways for different carrier types.  
\comment{As an
example, an implementation of these methods for lists is given in
\sref{cls-list}.}

The translation scheme is as follows. 
In a first step, every generator ~\lstinline@val $p$ <- $e$@, where $p$ is not
a pattern variable, is replaced by
\begin{lstlisting}
val $p$ <- $e$.filter { case $p$ => true; case _ => false }
\end{lstlisting}
Then, the following
rules are applied repeatedly until all comprehensions have been eliminated.
\begin{itemize}
\item
A generator ~\lstinline@val $p$ <- $e$@~ followed by a filter $f$ is translated to
a single generator ~\lstinline@val $p$ <- $e$.filter($x_1 \commadots x_n$ => $f\,$)@~ where
$x_1 \commadots x_n$ are the free variables of $p$.

\item
A for-comprehension 
~\lstinline@for (val $p$ <- $e\,$) yield $e'$@~ 
is translated to
~\lstinline@$e$.map { case $p$ => $e'$ }@.

\item
A for-comprehension
~\lstinline@for (val $p$ <- $e\,$) $e'$@~ 
is translated to
~\lstinline@$e$.foreach { case $p$ => $e'$ }@.

\item
A for-comprehension
\begin{lstlisting}
for (val $p$ <- $e$; val $p'$ <- $e'; \ldots$) yield $e''$ ,
\end{lstlisting}
where \lstinline@$\ldots$@ is a (possibly empty)
sequence of generators or filters,
is translated to
\begin{lstlisting}
$e$.flatmap { case $p$ => for (val $p'$ <- $e'; \ldots$) yield $e''$ } .
\end{lstlisting}
\item
A for-comprehension
\begin{lstlisting}
for (val $p$ <- $e$; val $p'$ <- $e'; \ldots$) $e''$ .
\end{lstlisting}
where \lstinline@$\ldots$@ is a (possibly empty)
sequence of generators or filters,
is translated to
\begin{lstlisting}
$e$.foreach { case $p$ => for (val $p'$ <- $e'; \ldots$) $e''$ } .
\end{lstlisting}
\end{itemize}

\example
the following code produces all pairs of numbers
between $1$ and $n-1$ whose sums are prime.
\begin{lstlisting}
for  { val i <- range(1, n);
       val j <- range(1, i);
       isPrime(i+j)
} yield Pair (i, j)
\end{lstlisting}
The for-comprehension is translated to:
\begin{lstlisting}
range(1, n)
  .flatMap {
     case i => range(1, i)
       .filter { j => isPrime(i+j) }
       .map { case j => Pair(i, j) } }
\end{lstlisting}

\comment{
\example
\begin{lstlisting}
package class List[a] {
  def map[b](f: (a)b): List[b] = match {
    case <> => <>
    case x :: xs => f(x) :: xs.map(f)
  }
  def filter(p: (a)Boolean) = match {
    case <> => <>
    case x :: xs => if p(x) then x :: xs.filter(p) else xs.filter(p)
  }
  def flatMap[b](f: (a)List[b]): List[b] =
    if (isEmpty) Nil
    else f(head) ::: tail.flatMap(f);
  def foreach(f: (a)Unit): Unit =
    if (isEmpty) ()
    else (f(head); tail.foreach(f));
}
\end{lstlisting}

\example
\begin{lstlisting}
abstract class Graph[Node] {
  type Edge = (Node, Node)
  val nodes: List[Node]
  val edges: List[Edge]
  def succs(n: Node) = for ((p, s) <- g.edges, p == n) s
  def preds(n: Node) = for ((p, s) <- g.edges, s == n) p
}
def topsort[Node](g: Graph[Node]): List[Node] = {
  val sources = for (n <- g.nodes, g.preds(n) == <>) n
  if (g.nodes.isEmpty) <>
  else if (sources.isEmpty) new Error(``topsort of cyclic graph'') throw
  else sources :+: topsort(new Graph[Node] {
    val nodes = g.nodes diff sources
    val edges = for ((p, s) <- g.edges, !(sources contains p)) (p, s)
  })
}
\end{lstlisting}
}

\example For comprehensions can be used to express vector 
and matrix algorithms concisely. 
For instance, here is a function to compute the transpose of a given matrix:

\begin{lstlisting}
def transpose[a](xss: Array[Array[a]]) {
  for (val i <- Array.range(0, xss(0).length)) yield
    Array(for (val xs <- xss) yield xs(i))
\end{lstlisting}

Here is a function to compute the scalar product of two vectors:
\begin{lstlisting}
def scalprod(xs: Array[double], ys: Array[double]) {
  var acc = 0.0;
  for (val Pair(x, y) <- xs zip ys) acc = acc + x * y; 
  acc
}
\end{lstlisting}

Finally, here is a function to compute the product of two matrices. Compare with the imperative version of \ref{ex:imp-mat-mul}.
\begin{lstlisting}
def matmul(xss: Array[Array[double]], yss: Array[Array[double]]) = {
  val ysst = transpose(yss);
  for (val xs <- xs) yield
    for (val yst <- ysst) yield 
      scalprod(xs, yst)
}
\end{lstlisting}
The code above makes use of the fact that \code{map}, \code{flatmap},
\code{filter}, and \code{foreach} are defined for members of class
\lstinline@scala.Array@.

\section{Return Expressions}

\syntax\begin{lstlisting}
  Expr1      ::=  return [Expr]
\end{lstlisting}

A return expression ~\lstinline@return $e$@~ must occur inside the
body of some enclosing named method or function $f$. This function
must have an explicitly declared result type, and the type of $e$ must
conform to it. The return expression evaluates the expression $e$ and
returns its value as the result of $f$. The evaluation of any statements or
expressions following the return expression is omitted. The type of 
a return expression is \code{scala.All}.



\section{Throw Expressions}

\syntax\begin{lstlisting}
  Expr1      ::=  throw Expr
\end{lstlisting}

A throw expression ~\lstinline@throw $e$@~ evaluates the expression
$e$. The type of this expression must conform to
\code{Throwable}.  If $e$ evaluates to an exception
reference, evaluation is aborted with the thrown exception. If $e$
evaluates to \code{null}, evaluation is instead aborted with a
\code{NullPointerException}. If there is an active
\code{try} expression (\sref{sec:try}) which handles the thrown
exception, evaluation resumes with the handler; otherwise the thread
executing the \code{throw} is aborted.  The type of a throw expression
is \code{scala.All}.

\section{Try Expressions}\label{sec:try}

\syntax\begin{lstlisting}
  Expr1       ::=  try `{' Block `}' [catch Expr] [finally Expr]
\end{lstlisting}

A try expression ~\lstinline@try { $b$ } catch $e$@~ evaluates the block
$b$.  If evaluation of $b$ does not cause an exception to be
thrown, the result of $b$ is returned. Otherwise the {\em
handler} $e$ is applied to the thrown exception.  Let $\proto$
be the expected type of the try expression.  The block $b$ is
expected to conform to $\proto$.  The handler $e$ is expected
conform to type ~\lstinline@scala.PartialFunction[scala.Throwable, $\proto\,$]@.
The type of the try expression is the least upper bound of the type of
$b$ and the result type of $e$.

A try expression ~\lstinline@try { $b$ } finally $e$@~ evaluates the block
$b$.  If evaluation of $b$ does not cause an exception to be
thrown, the expression $e$ is evaluated. If an exception is thrown
during evaluation of $e$, the evaluation of the try expression is
aborted with the thrown exception. If no exception is thrown during
evaluation of $e$, the result of $b$ is returned as the
result of the try expression. 

If an exception is thrown during
evaluation of $b$, the finally block
$e$ is also evaluated. If another exception $e$ is thrown
during evaluation of $e$, evaluation of the try expression is
aborted with the thrown exception. If no exception is thrown during
evaluation of $e$, the original exception thrown in $b$ is
re-thrown once evaluation of $e$ has completed.  The block
$b$ is expected to conform to the expected type of the try
expression. The finally expression $e$ is expected to conform to
type \code{unit}.

A try expression ~\lstinline@try { $b$ } catch $e_1$ finally $e_2$@~ is a shorthand
for  ~\lstinline@try { try { $b$ } catch $e_1$ } finally $e_2$@.




\section{Anonymous Functions}
\label{sec:closures}

\syntax\begin{lstlisting}
  Expr1           ::=  Bindings `=>' Expr
  ResultExpr      ::=  Bindings `=>' Block
  Bindings        ::=  `(' Binding {`,' Binding `)'
                    |  id [`:' Type1]
  Binding         ::=  id [`:' Type]
\end{lstlisting}

The anonymous function ~\lstinline@($x_1$: $T_1 \commadots x_n$: $T_n$) => e@~ 
maps parameters $x_i$ of types $T_i$ to a result given
by expression $e$. The scope of each formal parameter
$x_i$ is $e$. Formal parameters must have pairwise distinct names.

If the expected type of the anonymous function is of the form
~\lstinline@scala.Function$n$[$S_1 \commadots S_n$, $R\,$]@, the
expected type of $e$ is $R$ and the type $T_i$ of any of the
parameters $x_i$ can be omitted, in which
case~\lstinline@$T_i$ = $S_i$@ is assumed.
If the expected type of the anonymous function is
some other type, all formal parameter types must be explicitly given,
and the expected type of $e$ is missing. The type of the anonymous
function
is~\lstinline@scala.Function$n$[$S_1 \commadots S_n$, $T\,$]@,
where $T$ is the type of $e$. $T$ must be equivalent to a
type which does not refer to any of the formal parameters $x_i$.

The anonymous function is evaluated as the instance creation expression
\begin{lstlisting}
scala.Function$n$[$T_1 \commadots T_n$, $T$] {
  def apply($x_1$: $T_1 \commadots x_n$: $T_n$): $T$ = $e$
}
\end{lstlisting}
In the case of a single formal parameter, ~\lstinline@($x$: $T\,$) => $e$@~ and ~\lstinline@($x\,$) => $e$@~ 
can be abbreviated to ~\lstinline@$x$: $T$ => e@, and ~\lstinline@$x$ => $e$@, respectively.

\example Examples of anonymous functions:

\begin{lstlisting}
  x => x                             // The identity function

  f => g => x => f(g(x))             // Curried function composition

  (x: Int,y: Int) => x + y           // A summation function

  () => { count = count + 1; count } // The function which takes an
                                     // empty parameter list $()$, 
                                     // increments a non-local variable 
                                     // `count' and returns the new value.
\end{lstlisting}

\section{Statements}
\label{sec:statements}

\syntax\begin{lstlisting}
  BlockStat    ::=  Import
                 |  Def
                 |  {LocalModifier} TmplDef
                 |  Expr
                 | 
  TemplateStat ::=  Import
                 |  {AttributeClause} {Modifier} Def
                 |  {AttributeClause} {Modifier} Dcl
                 |  Expr
                 | 
\end{lstlisting}

Statements occur as parts of blocks and templates.  A statement can be
an import, a definition or an expression, or it can be empty.
Statements used in the template of a class definition can also be
declarations.  An expression that is used as a statement can have an
arbitrary value type. An expression statement $e$ is evaluated by
evaluating $e$ and discarding the result of the evaluation. 
\todo{Generalize to implicit coercion?}

Block statements may be definitions which bind local names in the
block. The only modifiers allowed in block-local definitions are modifiers
\code{abstract}, \code{final}, or \code{sealed} preceding a class or
object definition.

With the exception of overloaded definitions
(\sref{sec:overloaded-defs}), a statement sequence making up a block
or template may not contain two definitions or declarations that bind
the same name in the same namespace.  Evaluation of a statement
sequence entails evaluation of the statements in the order they are
written.

\chapter{Pattern Matching}

\section{Patterns}

% 2003 July - changed to new pattern syntax + semantic Burak
%      Nov  - incorporated changes to grammar, avoiding empty patterns
%             definitions for value and sequence patterns
% 2004 Jan  - revert back to original version ?! move regexp to subsect
%      May  - XmlPattern

\label{sec:patterns}

\syntax\begin{verbatim}
  Pattern       ::=  SimplePattern {Id SimplePattern}
                 |   varid `:' Type
                 |   `_' `:' Type
  SimplePattern ::=  varid
                 |   `_'
                 |   literal
                 |   StableId {`(' [Patterns] `)'}
                 |   XmlPattern
  Patterns      ::=  Pattern {`,' Pattern}
\end{verbatim}

For clarity, this section deals with a subset of the Scala pattern language.
The extended Scala pattern language, which is described below, adds more
flexible variable binding and regular hedge expressions.

A pattern is built from constants, constructors, and
variables. Pattern matching tests whether a given value has the shape
defined by a pattern, and, if it does, binds the variables in the
pattern to the corresponding components of the value.  The same
variable name may not be bound more than once in a pattern.

A pattern is built from constants, constructors, variables and regular
operators. Pattern matching tests whether a given value (or sequence
of values) has the shape defined by a pattern, and, if it does, binds
the variables in the pattern to the corresponding components of the
value (or sequence of values).  The same variable name may not be
bound more than once in a pattern.

Pattern matching is always done in a context which supplies an
expected type of the pattern. We distinguish the following kinds of
patterns.

A {\em variable pattern} $x$ is a simple identifier which starts with
a lower case letter.  It matches any value, and binds the variable
name to that value.  The type of $x$ is the expected type of the
pattern as given from outside.  A special case is the wild-card
pattern $\_$ which is treated as if it was a fresh variable.

A {\em typed pattern} $x: T$ consists of a pattern variable $x$ and a
simple type $T$. The type $T$ may be a class type or a compound type;
it may not contain a refinement (\sref{sec:refinements}).  This
pattern matches any value of type $T$ and binds the variable name to
that value.  $T$ must conform to the pattern's expected type. The
type of $x$ is $T$.

A {\em pattern literal} $l$ matches any value that is equal (in terms
of $==$) to it. It's type must conform to the expected type of the
pattern.

A {\em named pattern constant} $r$ is a stable identifier
(\sref{sec:stable-ids}). To resolve the syntactic overlap with a
variable pattern, a named pattern constant may not be a simple name
starting with a lower-case letter. The type of $r$ must conform
to the expected type of the pattern. The pattern matches any value $v$
such that \verb@$r$ == $v$@ (\sref{sec:cls-object}).

A {\em constructor pattern} $c(p_1) \ldots (p_n)$ where $n \geq 0$
consists of an identifier $c$, followed by component patterns $p_1
\commadots p_n$. The constructor $c$ is either a simple name
or a qualified name $r.id$ where $r$ is a stable identifier. It refers
to a (possibly overloaded) function which has one alternative of
result type \verb@class C@, and which may not have other overloaded
alternatives with a class constructor type as result type.
Furthermore, the respective type parameters and value parameters of
(said alternative of) $c$ and of the primary constructor function of
class $C$ must be the same, after renaming corresponding type parameter
names.  If $C$ is monomorphic, then
\verb@C@ must conform to the expected type of the pattern, and the
formal parameter types of $C$'s primary constructor are taken as the
expected types of the component patterns $p_1 \commadots p_n$.  If $C$
is polymorphic, then there must be a unique type application instance
of it such that the instantiation of $C$ conforms to the expected type
of the pattern. The instantiated formal parameter types of $C$'s
primary constructor are then taken as the expected types of the
component patterns $p_1\commadots p_n$.
The pattern matches all objects created from
constructor invocations $c(v_1)\ldots(v_n)$ where each component
pattern $p_i$ matches the corresponding value $v_i$.

An {\em infix operation pattern} \verb@p id p'@ is a shorthand for the
constructor pattern \verb@id_class(p)(p')@.  The precedence and
associativity of operators in patterns is the same as in expressions
(\sref{sec:infix-operations}).

\example Some examples of patterns are:
\begin{enumerate}
\item
The pattern \verb@ex: IOException@ matches all instances of class
\verb@IOException@, binding variable \verb@ex@ to the instance.
\item
The pattern \verb@(x, _)@ matches pairs of values, binding \verb@x@ to
the first component of the pair. The second component is matched
with a wildcard pattern.
\item
The pattern \verb@x :: y :: xs@ matches lists of length $\geq 2$,
binding \verb@x@ to the lists's first element, \verb@y@ to the list's
second element, and \verb@xs@ to the remainder.
\end{enumerate}

%%% new patterns

\subsection{Regular Pattern Matching}

\syntax\begin{lstlisting}
  Pattern         ::=  Pattern1 { `|' Pattern1 }
  Pattern1        ::=  varid `:' Type
                    |  `_' `:' Type
                    |  Pattern2
  Pattern2        ::=  [varid `@'] Pattern3
  Pattern3        ::=  SimplePattern [ '*' | '?' | '+' ]
                    |  SimplePattern { id' SimplePattern }
  SimplePattern   ::=  `_'
                    |  varid
                    |  Literal
                    |  StableId [ `(' [Patterns] `)' ]
                    |  `(' [Patterns] `)'
  Patterns        ::=  Pattern {`,' Pattern}
  id'             ::=  id $\textit{ but not }$ '*' | '?' | '+' | `@' | `|'
\end{lstlisting}

We distinguish between tree patterns and hedge patterns (hedges 
are ordered sequences of trees). A {\em tree pattern} describes 
a set of matching trees (like above). A {\em hedge pattern} describes 
a set of matching hedges. Both kinds of patterns may contain {\em 
variable bindings} which serve to extract constituents of a tree or hedge.

The type of a patterns and the expected types of variables 
within patterns are determined by the context and the structure of the
patterns. The last case ensures that a variable bound 
to a hedge pattern will have a sequence type.

The following patterns are added:

A {\em hedge pattern} $p_1 \commadots p_n$ where $n \geq 0$ is a
sequence of patterns separated by commas and matching the hedge described
by the components. Hedge patterns may appear as arguments to constructor 
applications, or nested within a another hedge pattern if grouped with
parentheses. Note that empty hedge patterns are allowed. The type of tree 
patterns that appear in a hedge pattern is the expected type as 
determined from the enclosing constructor.
A {\em fixed-length argument pattern} is a special hedge pattern where 
where all $p_i$ are tree patterns.

A {\em choice pattern} $p_1 | \ldots | p_n$ is a choice among several
alternatives, which may not contain variable-binding patterns. It
matches every tree and every hedge matched by at least one of its 
alternatives.  Note that the empty sequence may appear as an alternative.  
An {\em option pattern} $p?$ is an abbreviation for $(p| )$. A choice is 
a tree pattern if all its branches are tree patterns. In this case, all 
branches must conform to the expected type and the type 
of the choice is the least upper bound of the branches. Otherwise, 
its type is determined by the enclosing hedge pattern it is part of.

An {\em iterated pattern} $p*$ matches zero, one or more occurrences 
of items matched by $p$, where $p$ may be either a tree pattern or a hedge pattern. $p$ may not 
contain a variable-binding. A {\em non-empty iterated pattern} $p+$ is an 
abbreviation for $(p,p*)$. 

The treatment of the following patterns changes with to the 
previous section:

A {\em constructor pattern} $c ( p )$ consists of a simple type $c$
followed by a pattern $p$.  If $c$ designates a monomorphic case
class, then it must conform to the expected type of the pattern, the
pattern must be a fixed length argument pattern $p_1 \commadots p_n$
whose length corresponds to the number of arguments of $c$'s primary
constructor. The expected types of the component patterns are then
taken from the formal parameter types of (said) constructor.  If $c$
designates a polymorphic case class, then there must be a unique type
application instance of it such that the instantiation of $c$ conforms
to the expected type of the pattern. The instantiated formal parameter
types of $c$'s primary constructor are then taken as the expected
types of the component patterns $p_1\commadots p_n$.  In both cases,
the pattern matches all objects created from constructor invocations
$c(v_1 \commadots v_n)$ where each component pattern $p_i$ matches the
corresponding value $v_i$. If $c$ does not designate a case class, it
must be a subclass of \lstinline@Seq[$T\,$]@. In that case $p$ may be an
arbitrary sequence pattern. Value patterns in $p$ are expected to conform to
type $T$, and the pattern matches all objects whose \lstinline@elements()@
method returns a sequence that matches $p$.

The pattern $(p)$ is regarded as equivalent to the pattern $p$, if $p$
is a nonempty sequence pattern. The empty tuple $()$ is a shorthand
for the constructor pattern \code{Unit}.

A {\em variable-binding} $x @ p$ is a simple identifier $x$
which starts with a lower case letter, together with a pattern $p$. It
matches every item (tree or hedge) matched by $p$, and in addition binds 
it to the variable name. If $p$ is a tree pattern of type $T$, the type 
of $x$ is also $T$.
If $p$ is a hedge pattern enclosed by constructor $c <: $\lstinline@Seq[$T\,$]@,
then the type of $x$ is \lstinline@List[$T\,$]@
where $T$ is the expected type as dictated by the constructor. 

%A pattern consisting of only a variable $x$ is treated as the bound
%value pattern $x @ \_$. A typed pattern $x:T$ is treated as $x @ (_:T)$.
%
Regular expressions that contain variable bindings may be ambiguous,
i.e. there might be several ways to match a sequence against the
pattern. In these cases, the \emph{right-longest policy} applies:
patterns that appear more to the right than others in a sequence take
precedence in case of overlaps.

\example Some examples of patterns are:
\begin{enumerate}
\item
The pattern ~\lstinline@ex: IOException@~ matches all instances of class
\code{IOException}, binding variable \code{ex} to the instance.
\item
The pattern ~\lstinline@Pair(x, _)@~ matches pairs of values, binding \code{x} to
the first component of the pair. The second component is matched
with a wildcard pattern.
\item
The pattern \ \code{List( x, y, xs @ _ * )} matches lists of length $\geq 2$,
binding \code{x} to the list's first element, \code{y} to the list's
second element, and \code{xs} to the remainder, which may be empty.
\item
The pattern \ \code{List( 1, x@(( 'a' | 'b' )+),y,_ )} matches a list that
contains 1 as its first element, continues with a non-empty sequence of 
\code{'a'}s and \code{'b'}s, followed by two more elements. The sequence 'a's and 'b's
is bound to \code{x}, and the next to last element is bound to \code{y}.
\item
The pattern \code{List( x@( 'a'* ), 'a'+ )} matches a non-empty list of
\code{'a'}s. Because of the shortest match policy, \code{x} will always be bound to
the empty sequence.
\item
The pattern \code{List( x@( 'a'+ ), 'a'* )} also matches a non-empty list of
\code{'a'}s. Here, \code{x} will always be bound to
the sequence containing one \code{'a'}
\end{enumerate}


\section{Pattern Matching Expressions}
\label{sec:pattern-match}

\syntax\begin{lstlisting}
  BlockExpr       ::=  `{' CaseClause {CaseClause} `}'
  CaseClause      ::=  case Pattern [`if' PostfixExpr] `=>' Block 
\end{lstlisting}

A pattern matching expression
~\lstinline@case $p_1$ => $b_1$ $\ldots$ case $p_n$ => $b_n$@ \ consists of a number 
$n \geq 1$ of cases. Each case consists of a (possibly guarded) pattern 
$p_i$ and a block $b_i$.  The scope of the pattern variables in $p_i$ is 
the corresponding block $b_i$.

The expected type of a pattern matching expression must in part be
defined. It must be either ~\lstinline@scala.Function1[$T_p$, $T_r$]@ \ or
~\lstinline@scala.PartialFunction[$T_p$, $T_r$]@, where the argument type
$T_p$ must be fully determined, but the result type
$T_r$ may be undetermined.  All patterns are typed
relative to the expected type $T_p$ (\sref{sec:patterns}).  The expected type of
every block $b_i$ is $T_r$.
Let $T_b$ be the least upper bound of the types of all blocks 
$b_i$. The type of the pattern matching expression is
then the required type with $T_r$ replaced by $T_b$
(i.e. the type is either ~\lstinline@scala.Function[$T_p$, $T_b$]@~ or
~\lstinline@scala.PartialFunction[$T_p$, $T_b$]@.

When applying a pattern matching expression to a selector value,
patterns are tried in sequence until one is found which matches the
selector value (\sref{sec:patterns}). Say this case is $\CASE;p_i
\Arrow b_i$.  The result of the whole expression is then the result of
evaluating $b_i$, where all pattern variables of $p_i$ are bound to
the corresponding parts of the selector value.  If no matching pattern
is found, a \code{scala.MatchError} exception is thrown.

The pattern in a case may also be followed by a guard suffix \ \code{if e}\ 
with a boolean expression $e$.  The guard expression is evaluated if
the preceding pattern in the case matches. If the guard expression
evaluates to \code{true}, the pattern match succeeds as normal. If the
guard expression evaluates to \code{false}, the pattern in the case
is considered not to match and the search for a matching pattern
continues.

\comment{
A case with several patterns $\CASE;p_1 \commadots p_n ;\IF; e \Arrow b$  is a
shorthand for a sequence of single-pattern cases $\CASE;p_1;\IF;e \Arrow b
;\ldots; \CASE;p_n ;\IF;e\Arrow b$. In this case none of the patterns
$p_i$ may contain a named pattern variable (but the patterns may contain
wild-cards).
}

In the interest of efficiency the evaluation of a pattern matching
expression may try patterns in some other order than textual
sequence. This might affect evaluation through
side effects in guards. However, it is guaranteed that a guard
expression is evaluated only if the pattern it guards matches.

\example
Often, pattern matching expressions are used as arguments
of the \code{match} method, which is predefined in class \code{Any}
(\sref{sec:cls-object}) and is implemented there by postfix function
application. Here is an example:
\begin{lstlisting}
def length [a] (xs: List[a]) = xs match {
  case Nil =>  0
  case x :: xs1  =>  1 + length (xs1)
}
\end{lstlisting}

In an application of \code{match} such as the one above, the expected
type of all patterns is the type of the qualifier of \code{match}.
In the example above, the expected type of the patterns \code{Nil} and
\code{x :: xs1} would be \code{List[a]}, the type of \code{xs}.


\chapter{Views}\label{sec:views}

Views are user-defined, implicit coercions that are automatically
inserted by the compiler. 

\section{View Definition}

A view definition is a normal function definition with one value
parameter where the name of the defined function is \code{view}.

\example 
The following defines an implicit coercion function from strings to lists of
characters.

\begin{lstlisting}
def view(xs: String): List[char] = 
  if (xs.length() == 0) List()
  else xs.charAt(0) :: xs.substring(1);
\end{lstlisting}

\section{View Application}

View applications are inserted implicitly in two situations.
\begin{enumerate}
\item
Around an expression $e$ of type $T$, if $T$ does not conform to the
expression's expected type $PT$.
\item
In a selection $e.m$ with $e$ of type $T$, if the selector $m$ does not
denote a member of $T$.
\end{enumerate}
In the first case, a view method \code{view} is searched which is
applicable to $e$ and whose result type conforms to $PT$.
If such a method is found, the expression $e$ is converted to 
\lstinline@view($e$)@.

In the second case, a view method \code{view} is searched which is
applicable to $e$ and whose result contains a member named $m$.
If such a method is found, the selection $e.m$ is converted to 
\lstinline@view($e$).m@

\section{Finding Views}\label{sec:finding-views}

Searching a view which is applicable to an expression $e$ of type $T$
is a three-step process.
\begin{enumerate}
\item
First, the set $\AA$ of available views is determined. $\AA$ is the
smallest set such that:
\begin{enumerate}
\item
If a unary method called \code{view} is accessible without qualifier
anywhere on the path of the program tree that leads from $e$ to the
root of the tree (describing the whole compilation unit), then that
method is in the set $\AA$. Methods are accessible without qualifier
because they are locally defined in an enclosing scope, or imported
into an enclosing scope, or inherited by an enclosing class.
\item
If a unary method called \code{view} is a member of an object $C$ such
that there is a base class $C$ of $T$ with the same name as the object
and defined in the same scope, then that method is in the set $\AA$.
\end{enumerate}
\item
Then, among all the methods in $\AA$ the set of all applicable views
$\BB$ is determined. A view method is applicable if it can be applied
to values of type $T$, and another condition is satisfied which
depends on the context of the view application:
\begin{enumerate}
\item
If the view is a conversion to a given prototype $PT$, then the view's
result type must conform to $PT$.
\item
If the view is a conversion in a selection with member $m$, then the
view's result type must contain a member named $m$.
\end{enumerate}
Note that in the determining of view applicability, we do not permit
further views to be inserted. I.e.\ a view is applicable to an
expression $e$ of type $T$ if it can be applied to $e$, without a
further view conversion of $e$ to the view's formal parameter type.
Likewise, a view's result type must conform to a given prototype
directly, no second view conversion is allowed.
\item
It is an error if the set of applicable views $\BB$ is empty.  For
non-empty $\BB$, the view method which is most specific
(\sref{sec:overloaded-refs}) in $\BB$ is selected.  It is an error if
no most specific view exists, or if it is not unique.
\end{enumerate}

\example

Consider the following situation.
\begin{lstlisting}
class A;
class B extends A;
class C;
object B {
  def view(x: B): C = ...
}
object Test with Application {
  def view(x: A): C = ...
  val x: C = new B;
}
\end{lstlisting}
For the expression ~\code{new B}~ there are two available views. The
view defined in object \code{B} is available since its associated
class is (a superclass of) the expression's type \code{B}. The view
defined in object \code{Test} is available since it is accessible
without qualification at the point of the expression ~\code{new
B}. Both views are also applicable since they map values of type
\code{B} to results of type \code{C}. However, the view defined in
object \code{B} is more specific than the view defined in object
\code{Test}.  Hence, the last statement in the example above is
implicitly augmented to
\begin{lstlisting}
val x: C = B.view(new B)
\end{lstlisting}

\section{View-Bounds}

\syntax\begin{lstlisting}
  TypeParam       ::=  id [>: Type] [<% Type]
\end{lstlisting}

A type parameter \code{a} may have a view bound \lstinline@a <% T@
instead of a regular upper bound \lstinline@a <: T@. In that case the
type parameter may be instantiated to any type \code{S} which is
convertible by application of a view method to the view bound
\code{T}. Here, we assume there exists an always available identity
view method
\begin{lstlisting}
def view[a](x: a): a = x .
\end{lstlisting}
Hence, the type parameter \code{a} can always be instantiated to
subtypes of the view bound \code{T}, just as if \code{T} was a regular
upper bound.

View bounds for type parameters behave analogously to upper bounds wrt
to type conformance (\sref{sec:subtyping}), variance checking
(\sref{sec:type-params}), and overriding (\sref{sec:overriding}).

Methods or classes with view-bounded type parameters implicitly take
view functions as parameters. For every view-bounded type parameter
\lstinline@a <% T@ one adds an implicit value parameter
\lstinline@view: a => T@. When instantiating the type parameter
\code{a} to some type \code{S}, the most specific applicable view
method from type \code{S} to type \code{T} is selected, according to
the rules of \sref{sec:finding-views}. This method is then passed as
actual argument to the corresponding view parameter.

Implicit view parameters of a method or class are then taken as
available view methods in its body.

\example Consider the following definition of a trait
\code{Comparable} and a view from strings to that trait.

\begin{lstlisting}
trait Comparable[a] {
  def less(x: a): boolean
}

object StringsAreComparable {
  def view(x: String): Comparable[String] = new Comparable[String] {
    def less(y: String) = x.compareTo(y) < 0
  }
}
\end{lstlisting}

Now, define a binary tree with a method \code{insert} which inserts an
element in the tree and a method \code{elements} which returns a
sorted list of all elements of the tree. The tree is defined for all
types of elements \code{a} that are viewable as \code{Comparable[a]}.

\begin{lstlisting}
trait Tree[a <% Comparable[a]] {
  def insert(x: a): Tree[a] = this match {
    case Empty() => new Node(x, Empty(), Empty())
    case Node(elem, l, r) => 
      if (x == elem) this
      else if (x less elem) Node(elem, l insert x, r)
      else Node(elem, l, r insert x);
  }
  def elements: List[a] = this match {
    case Empty() => List()
    case Node(elem, l, r) => 
      l.elements ::: List(elem) ::: r.elements
  }
}
case class Empty[a <% Comparable[a]]() 
  extends Tree[a];
case class Node[a <% Comparable[a]](elem: a, l: Tree[a], r: Tree[a])
  extends Tree[a];
\end{lstlisting}

Finally, define a test program which builds a tree from all command
line argument strings and then prints out the elements as a sorted
sequence. 

\begin{lstlisting}
object Test {
  import StringsAreComparable.view;

  def main(args: Array[String]) = {
    var t: Tree[String] = Empty();
    for (val s <- args) { t = t insert s }
    System.out.println(t.elements)
  }
}
\end{lstlisting}
Note that the definition ~\lstinline@var t: Tree[String] = Empty();@~
is legal because at that point a view method from \code{String} to
\code{Comparable[String]} has been imported and is therefore
accessible without a prefix. The imported view method is passed as an
implicit argument to the \code{Empty} constructor. 


Here is the \code{Test} program again, this time with implicit views
made visible:
\begin{lstlisting}
object Test {
  import StringsAreComparable.view;

  def main(args: Array[String]) = {
    var t: Tree[String] = Empty($\mbox{\bf StringsAreComparable.view}$);
    for (val s <- args) { t = t insert s }
    System.out.println(t.elements)
  }
}
\end{lstlisting}

And here are the tree classes with implicit views added:

\begin{lstlisting}
trait Tree[a <% Comparable[a]]($\mbox{\bf view}$: a => Comparable[a]) {
  def insert(x: a): Tree[a] = this match {
    case Empty(_) => new Node(x, Empty($\mbox{\bf view}$), Empty($\mbox{\bf view}$))
    case Node(_, elem, l, r) => 
      if (x == elem) this
      else if ($\mbox{\bf view}$(x) less elem) Node($\mbox{\bf view}$, elem, l insert x, r)
      else Node($\mbox{\bf view}$, elem, l, r insert x);
  }
  def elements: List[a] = this match {
    case Empty(_) => List()
    case Node(_, elem, l, r) => 
      l.elements ::: List(elem) ::: r.elements
  }
}
case class Empty[a <% Comparable[a]]($\mbox{\bf view}$: a => Comparable[a]) 
  extends Tree[a];
case class Node[a <% Comparable[a]]($\mbox{\bf view}$: a => Comparable[a], 
                                    elem: a, l: Tree[a], r: Tree[a])
  extends Tree[a];
\end{lstlisting}

Note that views entail a certain run-time overhead because they need
to be passed as additional arguments to view-bounded methods and
classes. Furthermore, every application of a view entails the
construction of an object which is often immediately discarded
afterwards -- see for instance with the translation of \code{(x less elem)} 
in the implementation of method \code{insert} above. It is
expected that the latter cost can be absorbed largely or completely by
compiler optimizations (which are, however, not yet implemented at the
present stage).

\section{Conditional Views}

View methods might themselves have view-bounded type parameters; this
allows the definition of conditional views.

\example The following view makes lists comparable, {\em provided} the
list element type is also comparable.

\begin{lstlisting}
def view[a <% Comparable[a]](xs: List[a]): Comparable[List[a]] = 
  new Comparable[List[a]] {
    def less (ys: List[a]): boolean = 
      !ys.isEmpty 
      && 
      (xs.isEmpty || 
       (xs.head less ys.head) || 
       (xs.head == ys.head) && (xs.tail less ys.tail))
  }
\end{lstlisting}

Note that the condition \code{(xs.head less ys.head)} invokes the
\code{less} method of the list element type, which is unknown at the
point of the definition of the view method. As usual, view-bounded
type parameters are translated to implicit \code{view} arguments. In
this case, the \code{view} method over lists would receive the
\code{view} method over list elements as implicit parameter.

\comment{
This opens up the risk of infinite instantiations because view methods
are be passed to themselves as parameters. For instance, one might try
to define the following ``magic'' universal converter:
\begin{lstlisting}
def view[b, a <% b](x: a): b = x // illegal!
\end{lstlisting}
Application of this view would entail an inifinite expansion with view
parameters. To prevent this situation, we impose the following
restrictions on view definitions and applications.

Suppose a view definition
\begin{lstlisting}
def view [$tps$](x: $T$): $S$
\end{lstlisting}
where the list of type parameters $tps$ might be empty.  The view
definition is legal only if the result type $S$ is not a type variable
in $tps$.  Furthermore, we say the view definition is {\em
contractive} if the parameter type $T$ is not a type variable in
$tps$.  Contractive views can never lead to infinite expansions, as
the view argument type becomes strictly smaller for each nested view
argument.

For non-contractive views we require that every such view can be used
only once on a path of implicit view parameters. That is, a
non-contractive view may not be passed to itself as implicit view
argument, either directly or indirectly.  On the other hand,
contractive view methods may be passed to themselves as arguments. In the
examples above this case arises if a list of lists of strings is seen
as a \code{Comparable}.
}

\chapter{Top-Level Definitions}
\label{sec:topdefs}

\syntax\begin{lstlisting}
  CompilationUnit  ::=  [package QualId `;'] {TopStat `;'} TopStat
  TopStat          ::=  {AttributeClause} {Modifier} TmplDef
                     |  Import
                     |  Packaging
                     |
  QualId           ::=  id {`.' id}
\end{lstlisting}

A compilation unit consists of a sequence of packagings, import
clauses, and class and object definitions, which may be preceded by a
package clause.

A compilation unit ~\lstinline@package $p$; $\stats$@~ starting with a package
clause is equivalent to a compilation unit consisting of a single
packaging ~\lstinline@package $p$ { $\stats$ }@.

Implicitly imported into every compilation unit are, in that order :
the package \code{java.lang}, the package \code{scala}, and the object
\code{scala.Predef} (\sref{cls:predef}). Members of a later import in
that order hide members of an earlier import.

\section{Packagings}\label{sec:packagings}

\syntax\begin{lstlisting}
  Packaging       ::=  package QualId `{' {TopStat `;'} TopStat `}'
\end{lstlisting}

A package is a special object which defines a set of member classes,
objects and packages.  Unlike other objects, packages are not introduced
by a definition.  Instead, the set of members of a package is determined by
packagings.

A packaging \ \code{package p { ds }}\ injects all definitions in
\code{ds} as members into the package whose qualified name is
$p$. If a definition in \code{ds} is labeled \code{private}, it
is visible only for other members in the package.

Selections \code{p.m} from $p$ as well as imports from $p$
work as for objects. However, unlike other objects, packages may not
be used as values. It is illegal to have a package with the same fully
qualified name as a module or a class.

Top-level definitions outside a packaging are assumed to be injected
into a special empty package. That package cannot be named and
therefore cannot be imported. However, members of the empty package
are visible to each other without qualification.

\example The following example will create a hello world program as
function \code{main} of module \code{test.HelloWorld}.
\begin{lstlisting}
package test;

object HelloWord {
  def main(args: Array[String]) = System.out.println("hello world")
}
\end{lstlisting}

\chapter{Local Type Inference}
\label{sec:local-type-inf}

To be completed.

\chapter{XML expressions and patterns}
This chapter describes the syntactic structure of XML expressions and patterns.
It follows as close as possible the XML 1.0 specification \cite{w3c:xml},
changes being mandated by the possibility of embedding Scala code fragments.

\section{XML expressions}
XML expressions are expressions generated by the following production, where the 
opening bracket `<' of the first element must be in a position to start the lexical
XML mode (see \ref{sec::xmlMode}).

\syntax\begin{lstlisting}
XmlExpr ::= Element {Element}
\end{lstlisting}
Well-formedness constraints of the XML specification apply, which
means for instance that start tags and end tags must match, and
attributes may only be defined once, with the exception of constraints
related to entity resolution.

The following productions describe Scala's extensible markup language,
designed as close as possible to the W3C extensible markup language
standard. Only the productions for attribute values and character data
are changed. Scala does not support neither declarations, CDATA
sections nor processing instructions. Entity references are not
resolved at runtime.

\syntax\begin{lstlisting}
Element       ::=    EmptyElemTag
                |    STag Content ETag                                       

EmptyElemTag  ::=    `<' Name {S Attribute} [S] `/>'                         

STag          ::=    `<' Name {S Attribute} [S] `>'                          
ETag          ::=    `</' Name [S] '>'                                        
Content       ::=    [CharData] {Content1 [CharData]}
Content1      ::=    Element
                |    Reference
                |    CDSect
                |    PI
                |    Comment
                |    ScalaExpr
\end{lstlisting}

If an XML expression is a single element, its value is a runtime
representation of an XML node (an instance of a subclass of 
\lstinline@scala.xml.Node@). If the XML expression consists of more
than one element, then its value is a runtime representation of a
sequence of XML nodes (an instance of a subclass of 
\lstinline@scala.Seq[scala.xml.Node]@).

If an XML expression is an entity reference, CDATA section, processing 
instructions or a comments, it is represented by an instance of the 
corresponding Scala runtime class.

By default, beginning and trailing whitespace in element content is removed, 
and consecutive occurrences of whitespace are replaced by a single space
character \U{0020}. This behaviour can be changed to preserve all whitespace
with a compiler option.
\syntax\begin{lstlisting}
Attribute     ::=    Name Eq AttValue	                                 

AttValue      ::=    `"' {CharQ | CharRef} `"'
                |    `'' {CharA | CharRef} `''
                |    ScalaExp

ScalaExpr     ::=    `{' expr `}'

CharData      ::=   { CharNoRef } $\mbox{\rm\em without}$ {CharNoRef}`{'CharB {CharNoRef} 
                                  $\mbox{\rm\em and without}$ {CharNoRef}`]]>'{CharNoRef}
\end{lstlisting}
XML expressions may contain Scala expressions as attribute values or
within nodes. In the latter case, these are embedded using a single opening 
brace `{' and ended by a closing brace `}'. To express a single opening braces 
within XML text as generated by CharData, it must be doubled. Thus, `{{` 
represents the XML text `{` and does not introduce an embedded Scala 
expression.

\syntax\begin{lstlisting}
BaseChar, Char, Comment, CombiningChar, Ideographic, NameChar, S, Reference
              ::=  $\mbox{\rm\em ``as in W3C XML''}$

Char1         ::=  Char $\mbox{\rm\em without}$ `<' | `&'
CharQ         ::=  Char1 $\mbox{\rm\em without}$ `"'
CharA         ::=  Char1 $\mbox{\rm\em without}$ `''
CharB         ::=  Char1 $\mbox{\rm\em without}$ '{'

Name          ::=  XNameStart {NameChar}

XNameStart    ::= `_' | BaseChar | Ideographic 
                 $\mbox{\rm\em (as in W3C XML, but without }$ `:'

\end{lstlisting}
\section{XML patterns}
XML patterns are patterns generated by the following production, where
the opening bracket `<' of the element patterns must be in a position
to start the lexical XML mode (see \ref{sec::xmlMode}).

\syntax\begin{lstlisting}
XmlPattern  ::= ElementPattern {ElementPattern}
\end{lstlisting}
Well-formedness constraints of the XML specification apply.

If an XML pattern is a single element pattern, it expects the type of runtime
representation of an XML tree, and matches exactly one instance of this type
that has the same structure as described by the pattern. If an XML pattern 
consists of more than one element, then it expects the type of sequences
of runtime representations of XML trees, and matches every sequence whose 
elements match the sequence described by the pattern.

XML patterns may contain Scala patterns(\ref{sec:pattern-match}).

Whitespace is treated the same way as in XML expressions. Patterns 
that are entity references, CDATA sections, processing 
instructions and comments match runtime representations which are the
the same.

By default, beginning and trailing whitespace in element content is removed, 
and consecutive occurrences of whitespace are replaced by a single space
character \U{0020}. This behaviour can be changed to preserve all whitespace
with a compiler option.

\syntax\begin{lstlisting}
ElemPattern   ::=    EmptyElemTagP
                |    STagP ContentP ETagP                                    

EmptyElemTagP ::=    '<' Name [S] '/>'
STagP         ::=    '<' Name [S] '>'                          
ETagP         ::=    '</' Name [S] '>'                                        
ContentP      ::=    [CharData] {(ElemPattern|ScalaPatterns) [CharData]}
ContentP1     ::=    ElemPattern
                |    Reference
                |    CDSect
                |    PI
                |    Comment
                |    ScalaPatterns
ScalaPatterns ::=    '{' patterns '}'
\end{lstlisting}


\chapter{The Scala Standard Library}

The Scala standard library consists of the package \code{scala} with a
number of classes and modules. Some of these classes are described in
the following.

\section{Root Classes}
\label{sec:cls-root}
\label{sec:cls-any}
\label{sec:cls-object}

The root of the Scala class hierarchy is formed by class \code{Any}.
Every class in a Scala execution environment inherits directly or
indirectly from this class.  Class \code{Any} has two direct
subclasses: \code{AnyRef} and\code{AnyVal}.

The subclass \code{AnyRef} represents all values which are represented
as objects in the underlying host system. Every user-defined Scala
class inherits directly or indirectly from this class. Furthermore,
every user-defined Scala class also inherits the trait
\code{scala.ScalaObject}.  Classes written in other languages still
inherit from \code{scala.AnyRef}, but not from
\code{scala.ScalaObject}.

The class \code{AnyVal} has a fixed number subclasses, which describe
values which are not implemented as objects in the underlying host
system.

Classes \code{AnyRef} and \code{AnyVal} are required to provide only
the members declared in class \code{Any}, but implementations may add
host-specific methods to these classes (for instance, an
implementation may identify class \code{AnyRef} with its own root
class for objects).

The standard interfaces of these root classes is described by the
following definitions.

\begin{lstlisting}
package scala;
abstract class Any {

  /** Defined equality; abstract here */
  def equals(that: Any): boolean;

  /** Semantic equality between values of same type */
  final def == (that: Any): boolean  =  this equals that

  /** Semantic inequality between values of same type */
  final def != (that: Any): boolean  =  !(this == that)

  /** Hash code */
  def hashCode(): Int = $\ldots$

  /** Textual representation */
  def toString(): String = $\ldots$

  /** Type test */
  def isInstanceOf[a]: Boolean = match {
    case x: a => true
    case _ => false
  }

  /** Type cast */
  def asInstanceOf[a]: a = match {
    case x: a => x
    case _ => if (this eq null) this
              else throw new ClassCastException()
  }

  /** Pattern match */
  def match[a, b](cases: a => b): b = cases(this);
}
final class AnyVal extends Any;
class AnyRef extends Any {
  def equals(that: Any): boolean     = this eq that;
  final def eq(that: Any): boolean   = $\ldots$; // reference equality
}
trait ScalaObject extends AnyRef;
\end{lstlisting}

The type cast operation \verb@asInstanceOf@ has a special meaning (not
expressed in the code above) when its type parameter is a numeric
type.  For any type \lstinline@T <: Double@, and any numeric value
\verb@v@ \lstinline@v.asInstanceIf[T]@ converts \code{v} to type
\code{T} using the rules of Java's numeric type cast operation.  The
conversion might truncate the numeric value (as when going from
\code{Long} to \code{Int} or from \code{Int} to \code{Byte}) or it
might lose precision (as when going from \code{Double} to \code{Float}
or when converting between \code{Long} and \code{Float}).

\section{Value Classes}
\label{cls:value}

Value classes are classes whose instances are not represented as
objects by the underlying host system.  All value classes inherit from
class \code{AnyVal}. Scala implementations need to provide the
value classes \code{Unit}, \code{Boolean}, \code{Double}, \code{Float},
\code{Long}, \code{Int}, \code{Char}, \code{Short}, and \code{Byte}
(but are free to provide others as well).
The signatures of these classes are defined in the following.

\subsection{Class \large{\code{Double}}}

\begin{lstlisting}
package scala;
abstract sealed class Double extends AnyVal {
  def + (that: Double): Double // double addition
  def - (that: Double): Double // double subtraction
  def * (that: Double): Double // double multiplication
  def / (that: Double): Double // double division
  def % (that: Double): Double // double remainder

  def == (that: Double): Boolean // double equality
  def != (that: Double): Boolean // double inequality
  def < (that: Double): Boolean  // double less
  def > (that: Double): Boolean  // double greater
  def <= (that: Double): Boolean // double less or equals
  def >= (that: Double): Boolean // double greater or equals

  def - : Double = 0.0 - this    // double negation
  def + : Double = this
}
\end{lstlisting}

\subsection{Class \large{\code{Float}}}

\begin{lstlisting}
package scala;
abstract sealed class Float extends AnyVal {
  def coerce: Double              // convert to Double

  def + (that: Double): Double;   // double addition
  def + (that: Float): Double     // float addition
  /* analogous for -, *, /, % */

  def == (that: Double): Boolean; // double equality
  def == (that: Float): Boolean;  // float equality
  /* analogous for !=, <, >, <=, >= */

  def - : Float;                  // float negation
  def + : Float
}
\end{lstlisting}

\subsection{Class \large{\code{Long}}}

\begin{lstlisting}
package scala;
abstract sealed class Long extends AnyVal  {
  def coerce: Double              // convert to Double
  def coerce: Float               // convert to Float

  def + (that: Double): Double;   // double addition
  def + (that: Float): Double;    // float addition
  def + (that: Long): Long =      // long addition
  /* analogous for -, *, /, % */

  def << (cnt: Int): Long         // long left shift
  def >> (cnt: Int): Long         // long signed right shift
  def >>> (cnt: Int): Long        // long unsigned right shift
  def & (that: Long): Long        // long bitwise and
  def | (that: Long): Long        // long bitwise or
  def ^ (that: Long): Long        // long bitwise exclusive or

  def == (that: Double): Boolean; // double equality
  def == (that: Float): Boolean;  // float equality
  def == (that: Long): Boolean    // long equality
  /* analogous for !=, <, >, <=, >= */

  def - : Long;                   // long negation
  def + : Long;                   // long identity
  def ~ : Long                    // long bitwise negation
}
\end{lstlisting}

\subsection{Class \large{\code{Int}}}

\begin{lstlisting}
package scala;
abstract sealed class Int extends AnyVal {
  def coerce: Double              // convert to Double
  def coerce: Float               // convert to Float
  def coerce: Long                // convert to Long

  def + (that: Double): Double;   // double addition
  def + (that: Float): Double;    // float addtion
  def + (that: Long): Long;       // long addition
  def + (that: Int): Int;         // int addition
  /* analogous for -, *, /, % */
  
  def << (cnt: Int): Int;         // int left shift
  /* analogous for >>, >>> */

  def & (that: Long): Long;       // long bitwise and
  def & (that: Int): Int;         // int bitwise and
  /* analogous for |, ^ */

  def == (that: Double): Boolean; // double equality
  def == (that: Float): Boolean;  // float equality
  def == (that: Long): Boolean    // long equality
  def == (that: Int): Boolean     // int equality
  /* analogous for !=, <, >, <=, >= */

  def - : Int;                    // int negation
  def + : Int;                    // int identity
  def ~ : Int;                    // int bitwise negation
}
\end{lstlisting}

\subsection{Class \large{\code{Short}}}

\begin{lstlisting}
package scala;
abstract sealed class Short extends AnyVal {
  def coerce: Double              // convert to Double
  def coerce: Float               // convert to Float
  def coerce: Long                // convert to Long
  def coerce: Int                 // convert to Int
}
\end{lstlisting}

\subsection{Class \large{\code{Char}}}

\begin{lstlisting}
package scala;
abstract sealed class Char extends AnyVal {
  def coerce: Double              // convert to Double
  def coerce: Float               // convert to Float
  def coerce: Long                // convert to Long
  def coerce: Int                 // convert to Int

  def isDigit: Boolean;           // is this character a digit?
  def isLetter: Boolean;          // is this character a letter?
  def isLetterOrDigit: Boolean;   // is this character a letter or digit?
  def isWhiteSpace                // is this a whitespace character?
}
\end{lstlisting}

\subsection{Class \large{\code{Short}}}

\begin{lstlisting}
package scala;
abstract sealed class Short extends AnyVal {
  def coerce: Double              // convert to Double
  def coerce: Float               // convert to Float
  def coerce: Long                // convert to Long
  def coerce: Int                 // convert to Int
  def coerce: Short               // convert to Short
}
\end{lstlisting}

\subsection{Class \large{\code{Boolean}}}
\label{sec:cls-boolean}

\begin{lstlisting}
package scala;
abstract sealed class Boolean extends AnyVal {
  def && (def x: Boolean): Boolean; // boolean and
  def || (def x: Boolean): Boolean; // boolean or
  def &  (x: Boolean): Boolean;     // boolean strict and
  def |  (x: Boolean): Boolean      // boolean strict or

  def == (x: Boolean): Boolean      // boolean equality
  def != (x: Boolean): Boolean      // boolean inequality

  def !  (x: Boolean): Boolean      // boolean negation
}
\end{lstlisting}

\subsection{Class \large{\code{Unit}}}

\begin{lstlisting}
package scala;
abstract sealed class Unit extends AnyVal;
\end{lstlisting}

\section{Standard Reference Classes}
\label{cls:reference}

This section presents some standard Scala reference classes which are
treated in a special way in Scala compiler -- either Scala provides
syntactic sugar for them, or the Scala compiler generates special code
for their operations. Other classes in the standard Scala library are
documented by HTML pages elsewhere.

\subsection{Class \large{\code{String}}}

The \verb@String@ class is usually derived from the standard String
class of the underlying host system (and may be identified with
it). For Scala clients the class is taken to support in each case a
method
\begin{lstlisting}
def + (that: Any): String 
\end{lstlisting}
which concatenates its left operand with the textual representation of its
right operand.

\subsection{The \large{\code{Tuple}} classes}

Scala defines tuple classes \lstinline@Tuple$n$@ for $n = 2 \commadots 9$.
These are defined as follows.

\begin{lstlisting}
package scala;
case class Tuple$n$[+a_1, ..., +a_n](_1: a_1, ..., _$n$: a_$n$) {
  def toString = "(" ++ _1 ++ "," ++ $\ldots$ ++ "," ++_$n$ ++ ")"
}
\end{lstlisting}

The implicitly imported \code{Predef} object (\sref{cls:predef}) defines
the names \code{Pair} as an alias of \code{Tuple2} and \code{Triple}
as an alias for \code{Tuple3}.

\subsection{The \large{\code{Function}} Classes}
\label{sec:cls-function}

Scala defines function classes \lstinline@Function$n$@ for $n = 1 \commadots 9$.
These are defined as follows.

\begin{lstlisting}
package scala;
class Function$n$[-a_1, ..., -a_$n$, +b] {
  def apply(x_1: a_1, ..., x_$n$: a_$n$): b;
  def toString = "<function>";
}
\end{lstlisting}

\comment{
There is also a module \code{Function}, defined as follows.
\begin{lstlisting}
package scala;
module Function {
  def compose[a](fs: List[(a)a]): (a)a  = {
    x => fs match {
      case Nil => x
      case f :: fs1 => compose(fs1)(f(x))
    }
  }
}
\end{lstlisting}
}

A subclass of \lstinline@Function1@ represents partial functions,
which are undefined on some points in their domain. In addition to the
\code{apply} method of functions, partial functions also have a
\code{isDefined} method, which tells whether the function is defined
at the given argument:
\begin{lstlisting}
class PartialFunction[-a,+b] extends Function1[a, b] {
  def isDefinedAt(x: a): Boolean
}
\end{lstlisting}

The implicitly imported \code{Predef} object (\sref{cls:predef}) defines the name 
\code{Function} as an alias of \code{Function1}.

\subsection{Class \large{\code{Array}}}\label{cls:array}

The class of generic arrays is given as follows.

\begin{lstlisting}
package scala;
class Array[a](length: int) with Function[Int, a] {
  def length: int;
  def apply(i: Int): a;
  def update(i: Int)(x: a): Unit;
}
\end{lstlisting}

\comment{
\begin{lstlisting}
module Array {
  def create[a](i1: Int): Array[a] = Array[a](i1)
  def create[a](i1: Int, i2: Int): Array[Array[a]] = {
    val x: Array[Array[a]] = create(i1)
    0 to (i1 - 1) do { i => x(i) = create(i2) }
    x
  }
  $\ldots$
  def create[a](i1: Int, i2: Int, i3: Int, i4: Int, i5: Int,
                i6: Int, i7: Int, i8: Int, i9: Int, i10: Int)
    : Array[Array[Array[Array[Array[Array[Array[Array[Array[Array[a]]]]]]]]]] = {
    val x: Array[Array[Array[Array[Array[Array[Array[Array[Array[a]]]]]]]]] = create(i1)
    0 to (i1 - 1) do { i => x(i) = create(i2, i3, i4, i5, i6, i7, i8, i9, i10) }
    x
  }
}
\end{lstlisting}
}

\section{The \large{\code{Predef}} Object}\label{cls:predef}

The \code{Predef} module defines standard functions and type aliases
for Scala programs. It is always implicitly imported, so that all its
defined members are available without qualification. Here is its
definition for the JVM environment.

\begin{lstlisting}
package scala;
object Predef {
  type byte = scala.Byte;
  type short = scala.Short;
  type char = scala.Char;
  type int = scala.Int;
  type long = scala.Long;
  type float = scala.Float;
  type double = scala.Double;
  type boolean = scala.Boolean;
  type unit = scala.Unit;
  
  type String = java.lang.String;
  type NullPointerException = java.lang.NullPointerException;
  type Throwable = java.lang.Throwable;
  // other aliases to be identified

  /** Abort with error message */
  def error(message: String): All = throw new Error(message);

  /** Throw an error if given assertion does not hold. */
  def assert(assertion: Boolean): Unit = 
    if (!assertion) throw new Error("assertion failed");

  /** Throw an error with given message if given assertion does not hold */
  def assert(assertion: Boolean, message: Any): Unit = {
    if (!assertion) throw new Error("assertion failed: " + message);

  /** Create an array with given elements */
  def Array[A](xs: A*): Array[A] = {
    val array: Array[A] = new Array[A](xs.length);
    var i = 0;
    for (val x <- xs.elements) { array(i) = x; i = i + 1; }
    array;
  }

  /** Aliases for pairs and triples */
  type Pair[+p, +q] = Tuple2[p, q];
  def Pair[a, b](x: a, y: b) = Tuple2(x, y);
  type Triple[+a, +b, +c] = Tuple3[a, b, c];
  def Triple[a, b, c](x: a, y: b, z: c) = Tuple3(x, y, z);

  /** Alias for unary functions */
  type Function = Function1;

  /** Some standard simple functions */
  def id[a](x: a): a = x;
  def fst[a](x: a, y: Any): a = x;
  def scd[a](x: Any, y: a): a = y;
}
\end{lstlisting}

\section{Class Node}\label{cls:Node}
\begin{lstlisting}
package scala.xml ;

trait Node {

  /** the label of this node */
  def label: String;              

  /** attribute axis */
  def attribute: Map[ String, String ];

  /** child axis (all children of this node) */
  def child: Seq[Node];         

  /** descendant axis (all descendants of this node) */
  def descendant:Seq[Node] = child.toList.flatMap { 
    x => x::x.descendant.asInstanceOf[List[Node]] 
  } ;

  /** descendant axis (all descendants of this node) */
  def descendant_or_self:Seq[Node] = this::child.toList.flatMap { 
    x => x::x.descendant.asInstanceOf[List[Node]] 
  } ;

  override def equals( x:Any ):boolean = x match {
    case that:Node => 
      that.label == this.label && 
        that.attribute.sameElements( this.attribute ) && 
          that.child.sameElements( this.child )
    case _ => false
  }

 /** XPath style projection function. Returns all children of this node
  *  that are labelled with 'that. The document order is preserved.
  */
    def \(that:Symbol): NodeSeq = {
      new NodeSeq({
        that.name match {

          case "_" => child.toList; 
          case _ =>
            var res:List[Node] = Nil;
            for( val x <- child.elements; x.label == that.name ) {
              res = x::res;
            }
            res.reverse
        }
      });
    }

 /** XPath style projection function. Returns all nodes labelled with the 
  *  name 'that from the descendant_or_self axis. Document order is preserved.
  */
  def \\(that:Symbol): NodeSeq = {
    new NodeSeq(
      that.name match {
        case "_" => this.descendant_or_self;
        case _ => this.descendant_or_self.asInstanceOf[List[Node]].
        filter( x => x.label == that.name );
      })
  }

  /** hashcode for this XML node */
  override def hashCode() = 
    Utility.hashCode( label, attribute.toList.hashCode(), child);

  /** string representation of this node */
  override def toString() = Utility.toXML(this);

}
\end{lstlisting}

