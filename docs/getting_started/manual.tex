\documentclass{article}

%
% a collection of remarks , tips , etc
%


\usepackage{listings}
\usepackage{multirow}

\newcommand{\escala}[0]{EScala }

\title{Getting started with EScala}
\author{Jurgen Van Ham}

\lstset{
  numbers=left,
  frame=single,
  morekeywords={evt,def,class,map,any,beforeExec,afterExec}
}


\begin{document}

\maketitle
\tableofcontents

\section{Target}
The aim of this text is just to collect some useful info to get 
started with \escala. Not to explain all about the normal scala
Also it can help when trying to find problems that might be 
encountered before. The small snippets of code are intended as 
practical examples.

It will just start as a collection of useful tips, I felt to be 
helpful to write some simple applications with EScala. 

There is also some information from the slides of Lucas, and other 
sources (like remarks, etc ) which helped in understanding or getting 
an overview of the possibilities  and the future features.

\section{Examples}
Since examples can be very useful to understand thennddyo language and its
mechanisms, there are at this moment 2 different implementations of an
example of a world with plants and animals (\escala and scala with 
observers) and also a very simple simulation of atoms, this last 
indicates a need for some pattern to avoid 'loops'.

\section{Declaring events}

\subsection{Different types}
The real types are not really what a programmers use, but it is
a good start to understand what the real picture is.

There are primitive and composite types. In the group of primitive
types there are two diffenent types which are not so different after
all. On the one hand there are the \emph{imperative} events (declared
with the {\tt imperative} keyword) that are controlled explicit by 
the code. On the other hand there are the \emph{implicit}
events that decribe events that happen when executing the program.

\begin{tabular}{|c|c|}
\hline
\multirow{2}{*}{primitive}  & imperative \\
                            & primitive  \\
\hline
composite  & declarative \\
\hline
\end{tabular}

\subsubsection{Primitive}
Althought the imperative and implicit don't seem very alike, when
definining a method with a empty body and {\tt Unit} as return type
and the desired parameters (for the event) this method could be called 
in the code. Then a {\tt beforeExec} of that method will result to 
the same as an {\tt imperative} event.

\subsubsection{Declarative}
The only way to use the implicit events is via the declarative events.

\subsubsection{Programmers Views}
For the programmer there are after considering the ideas above, there
are two different groups of  events, which are declared and
used in a different way. The \emph{imperative} events are the most 
obvious since they are in a way similar to what the Observer pattern
offers. The \emph{declarative} events are what makes \escala a different
language from the C\# approach. In the set of declarative events there are also the 
\emph{implicit} events that act like a bridge to AOP.

\begin{description}
\item [imperative] triggered from the code and look similar to a method call (via the apply from Scala)
\item [declarative] they are specified in function of other events. The other events can be also not defined by the programmer eg {\tt afterExec} or {\tt beforeExec}
\end{description}

\subsection{Imperative events}
They are declared with the keyword {\tt imperative} then the keyword 
{\tt evt} followed by a name with an optional argument.

To trigger the events the name of the event is used similar to a method
call. Since handlers in \escala have {\tt Unit} as a return type, this
'method call' will not have a return value.

The leads to declarations like shown below:
\begin{lstlisting}
class MyClass {
  imperative evt somehingHappend[Unit]
  imperative evt dropped[MyClass]
  
  def something = { somethingHappened() }
  def drop() : Unit = { dropped(this) }
}
\end{lstlisting}

\subsection{Declarative Events}
These are constructed from other events. In the most simple form these
other events can be \emph{imperative} events described above or a
(combination of) expressions of other events.

\begin{lstlisting}
evt somethingReallyHappened=somethingHappened
evt disjunctionOfEvents = movedUp || movedDown
evt conjunctionOfEvents = moved && changedColor
evt filteredEvent = someEvent && boolFunction 
\end{lstlisting}


\subsection{Event expressions}

The expressions for a declarative can use the constructions shown
here. The disjunctiona and the filtering are probably the most
obvious. Transforming an events also gets very useful with using
the implicit\footnote{beforeExec and afterExec} events. If $e_1$
and $e_2$ are event expressions or just events new events can
be constructed in the way shown below.

\begin{description}
\item [ $e_1 \&\& e_2$ ] Conjunction of events
\item [ $e_1 || e_2$] Disjunction of events
\item [ $e_1 \backslash e_2$ ] Difference of events
\item [ $ e \&\& f $] Events filtered by a boolean function
\item [ e.map(f) ] Transforming the parameters of an event with a function 'f'
\item [ ? (possiblyNull) ] when an expression could throw a nullpointer exception it could be ignored
\end{description}

\subsection{Implicit Events}
It is also possible to use \emph{implicit} events in an AOP style, for this
there are the keywords {\tt beforeExec} {\tt afterExec} that can be used 
as an event before or after executing a method. In the future there will
be an {\tt afterSet} keyword added that refers the an event that occurs
after setting a field of an object. For now there exists a {\tt Variable[T]}
class that helps to get a similar effect. This idea could probably also lead 
to other events when a variable is accessed in another way.

The arguments for these events are the arguments of the method in the 
case for the {\tt beforeExec}, for the {\tt afterExec} there can be and extra 
parameter for the return value. 

This looks nice in the first place, but most of the time these arguments
will need to be transformed by using {\tt map} with a function to extract
the required information. 

\begin{lstlisting}
class MyClass {
  def aMethod(a : Int) : Float = { .. }

  evt willDo[Int] = 
    beforeExec(aMethod)

  evt hasDone[Float] = 
    afterExec(aMethod) map ((_:Int,r:Float) => r)
}
\end{lstlisting}

\subsection{Observable methods}
The \emph{implicit} events are only available inside the scope of an object,
the reason for this is that knowledge to use these implicit events
is only available for the people who designed this class. This
encapsulation also allows to change the implementation of a class
without affecting 'external' uses of its implicit events.

It is possible to expose these implicit events by marking method with 
the keyword {\tt observable}.  This allows the implicit events for this
methods to be used in other objects. This way the programmer has
control about which methods are exposed for this.

A small remark is that a method without arguments and parenthesis 
seems not be be observable, the fix is just add parenthesis.


\subsection{Filtering events with a boolean function}

When defining declarative events it is possible to add conditions
to filter on the occurences of the event. This can be even on the 
arguments of a beforeExec or afterExec. This allow to limit events 
to those that are really interesting.

\begin{lstlisting}
def aMethod(a :Int) : Unit = { .. }
evt callEvent = 
    beforeExec(aMethod) && 
    ((arg : Int) => arg%2==0)
\end{lstlisting}

\subsection{Transforming arguments}
For declarative events it is possible to transform the arguments
via the {\tt map} operator. This allow to have only useful information
in the arguments of an event. Probably in many cases only the sender of
the event is interesting. This can be done like this.

\begin{lstlisting}
evt preHandler[MyClass]=
   beforeExec(handle) map ((_:Any) => this)
\end{lstlisting}

\subsection{Varlist}
The {\tt scala.events.VarList[T]} can be used to combine many objects
of a type and react to an event sent by one of those objects. For this 
there is an operator {\tt any} that allows a function to select an 
event from those elements.

The result is a disjunction on the selected event from a set of objects.

\begin{lstlisting}
class OthClass {
  imperative evt myEvent 
}
class myClass {
  val lst=new scala.events.VarList[OthClass]
  for (i <- 1 to 10) {lst += new OthClass }
  evt anEvent=lst.any(oc => oc.myEvent)
}
\end{lstlisting}

In the example above the {\tt anEvent} will occur when the myEvent
 occurs in an element of the {\tt lst}.


\subsection{Variable[T]}
This class helps to observe a variable until there are implicit events
for this like the {\tt afterSet} mentioned before. This variable sends 
a {\tt changed} event with 2 parametes, the first is the old value
and the second the new value.

Assignign a value to such variable can be done with the {\tt:=}-operator.
Since the apply method read the value just using the variable as
a normal one will return the value of it.

\begin{lstlisting}
class MyClass {
   def changedTo(old:Int, new:Int) : Unit = { 
     println("T changed "+old+" -> "+new);
   }

   val t=new Variable[Int](5)
   evt aChange=t.changed
   aChange += changedTo _

   t := 6
   println("T is now "+t)
}
\end{lstlisting}



\subsection{Inheritance and Overriding} 
When a subclass overrides an event of its superclass, the subclass could
refer to the event of the superclass with the keyword {\tt super}
this allows to add a filter to such event.

\begin{lstlisting}
class SuperClass {
  var nmbr : Int =5
  evt someEvent 
  ...
}
class SubClass extends SuperClass{
  override evt someEvent = 
    super.someEvent && super.nmbr%2==0
}
\end{lstlisting}


When not specifying an expression for a declarative event
(no imperative keyword) it will be abstract similar to a method without 
a body. It is possible to have handler for such an event that will 
be refined in a subclass.

\section{Handlers}
 A handler must have always an argument list between parenthesis,
even when this is an empty list. Since when there are no parenthesis 
the compiler handles this method in a different way. 

The signature of such a handler method must match the one from the
event it will handle. This allows to read the parameters of the
event via the parameters of the method that handles the event.

A handler also must have the return type {\tt Unit} 

\section{Registering a handler}

And event can 'register' many handlers, to add a handler to the
set of handlers of an event, use the {\tt +=} operator. In a similar
way the {\tt -=} operator can remove a registered handler again. 

The handlers are scala functions, in most cases after the name of 
the handler an underscore is  needed to make it 
a {\tt partially applied function} for the underlying Scala. 
Althought there are other constructions possible.


This means for pratical purposes that such a partially applied 
function can be called with only its arguments and it already 
contains the object where the methods will be called from.

\begin{lstlisting}
evt myEvent

def handler() : Unit = { ..}

myEvent += handler _
\end{lstlisting}

\section{Known causes of problems}

This is just a list of the problems that took me some
time to get solved. It is probably the most growing part
of the text. But I hope it will save time when using \escala.

\subsection{Type of Handler}
A handler should have parenthesis even without arguments and
a return type of Unit. Failing to do so can result in a situation
where the number of handlers grows, because 
unregistering\footnote{event -= handler} doesn't work

\subsection{Null references}
When specifying a declarative event based on events in another object
than can be a null reference (so it causes a null pointer exception)
the question mark 'function' can handle this situation by considering this
as an event that doesn't happen at all.

\subsection{Not unregistering handlers}
When handlers are not unregistered the number can grow which leads to
a decrease in speed. Especially when an application does register and
unregister a lot. An annoying side effect when handlers are not unregistered
is that the objects that contain the handlers will always stay reachable
via the event. This leads to a 'memory leak' that the garbage collector
can't solve. Since via the registered handlers in the event objects 
could still be reachable event they is no other way anymore.


\subsubsection{An easy solution, or not yet}
A solution could be to unregister in some automatic way. The problem
is that at this moment no possible way is known for this.  A weak
reference to a partial applied function would not protect this partial
function, when it is a normal (strong) reference, the object to which
this partial applied function belows will be protected from the
garbage collection process during the time it is registered.

\subsubsection{What is possible at this moment}
A practical approach could be to count the number of reactions by
modifying the {\tt EventsLib.scala} before compiling it by adding
a method to {\tt EventNode[T]} that returns the size of the variable
{\tt \_reactions}. This will need some interpretation, but when this
numbers always increases, it is worth to look into the unregistering
of events.


\end{document}
