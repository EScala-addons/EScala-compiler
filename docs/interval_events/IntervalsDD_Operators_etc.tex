\documentclass[article,colorback,accentcolor=tud4c]{tudreport}
\usepackage{ngerman}

<<<<<<< HEAD
\usepackage[stable]{footmisc} 
\usepackage[ngerman]{hyperref}
\usepackage[utf8]{inputenc}
\usepackage{longtable}
\usepackage{multirow}
\usepackage{booktabs}
\usepackage{pst-all}
\usepackage{amsmath}




% Makros
\newcommand{\before}[1]{\textit{ before(#1) }}
\newcommand{\after}[1]{\textit{after(#1)}}
\newcommand{\aktiv}[1]{\textit{active(#1)}}
\newcommand{\Interval}[1]{\textbf{interval#1}}
\newcommand{\comp}[1]{\textit{comp(#1)}}

\hypersetup{%
  pdftitle={TUD Corporate-Design f"ur {\LaTeX}}, pdfauthor={C. v. Loewenich und
  J. Werner}, pdfsubject={Beispieltext}, pdfview=FitH, pdfstartview=FitV }

\setcounter{seclinedepth}{1}

  \newif\ifTUDmargin\TUDmarginfalse \ifTUDmargin\makeatletter
    \TUD@setmarginpar{2}
  \makeatother\fi
=======
\usepackage[stable]{footmisc}
\usepackage[ngerman]{hyperref}

\usepackage{longtable}
\usepackage{multirow}
\usepackage{booktabs}

\hypersetup{%
  pdftitle={TUD Corporate-Design f"ur {\LaTeX}},
  pdfauthor={C. v. Loewenich und J. Werner},
  pdfsubject={Beispieltext},
  pdfview=FitH,
  pdfstartview=FitV
}

\setcounter{seclinedepth}{1}

%%% Zum Tester der Marginalien %%%
  \newif\ifTUDmargin\TUDmarginfalse
  %%% Wird der Folgende Zeile einkommentiert,
  %%% werden Marginalien gesetzt.
  % \TUDmargintrue
  \ifTUDmargin\makeatletter
    \TUD@setmarginpar{2}
  \makeatother\fi
%%% ENDE: Zum Tester der Marginalien %%%
>>>>>>> DD added

\newlength{\longtablewidth}
\setlength{\longtablewidth}{0.7\linewidth}
\addtolength{\longtablewidth}{-\marginparsep}
\addtolength{\longtablewidth}{-\marginparwidth}

<<<<<<< HEAD
\title{IntervalEvents in EScala: Design}
\subtitle{Michael Kutschke und Frank Englert}


\begin{document}

=======

% \settitlepicture{tudreport-pic}
% \printpicturesize

\title{IntervalEvents in EScala: Design}
\subtitle{Michael Kutschke und Frank Englert}
%\setinstitutionlogo[width]{TUD_sublogo}

\begin{document}
>>>>>>> DD added
\maketitle

\tableofcontents

\section{Grundlegende Definitionen}
<<<<<<< HEAD
Sei \before{\Interval} : Event das Ereignis, das den Beginn des Intervalls int angibt
und \after{\Interval} : Event jenes, welches das Ende von int angibt. Dies kann in
unterschiedlicher Form erfolgen:
\begin{itemize}
\item einmal \before{\_} zu Beginn und einmal \after{\_} am Ende (Standard in der jetzigen Fassung)
\item mehrfaches \before{\_} und ein \after{\_} am Ende (Use case? Wohl meist nicht erw"unscht)
\item verschachtelte, paarweise passende \before{\_} und \after{\_} Ereignisse (Bsp. Exec)
\item beliebige \before{\_} und \after{\_} Events innerhalb des Intervals
\end{itemize}
Ausserdem gebe \aktiv\Interval{} : boolean an, ob das Intervall aktiv ist.
In der jetzigen Fassung bedeutet dies einfach, ob man sich zeitlich innerhalb des
Intervalls befindet. Das Umschalten wude bisher als Sink realisiert. Dies führt
jedoch zu Indeterminismus, sodass e \&\& within(between(e,e')) beim ersten Vorkommen 
von e nicht unbedingt immer ausgeführt wird. Dies ist unschön und sollte vermieden werden.

Daher sollte active eine andere Semantik erhalten und als reaction aktualisiert
werden. So kann sichergestellt werden, dass der Zustand von active zu jedem
Zeitpunkt bekannt ist. Die Semantik ändert sich also wie in Abbildung
\ref{active_bit_behaviour} gezeigt. Dies ist keine Einschränkung, wie im
weiteren gezeigt werden wird.

\begin{figure}[h]
 \centering 
\psset{unit=1cm}
\begin{pspicture}(0,0)(10,2)
\psgrid[subgriddiv=1,griddots=10,gridlabels=7pt,](0,0)(10,2)
%{
%	\pspolygon[fillcolor=greenyellow,fillstyle=solid,linestyle=none](0,0)(0,1)(1,1)(1,0)

	\rput(1.2,1.7){\psframebox*[framearc=.3]{B}}
	\rput(9.2,1.7){\psframebox*[framearc=.3]{A}}
	\rput(0.2,1.55){\psframebox*[framearc=.3]{Früher}}
	\psline[linewidth=1pt]{[-]}(1,1.5)(9,1.5)

	\rput(1.2,0.7){\psframebox*[framearc=.3]{B}}
	\rput(9.2,0.7){\psframebox*[framearc=.3]{A}}
	\rput(0.2,0.55){\psframebox*[framearc=.3]{Aktuell}}
	\psline[linewidth=1pt]{]-]}(1,0.5)(9,0.5)
%}
\end{pspicture}
\caption{Wert des Aktiv-Bits}
\label{active_bit_behaviour}
\end{figure}


Für alle Intervalle muss gelten, dass active nur bei einem \before{\_} oder
\after{\_} Event umgeschaltet wird, wobei \aktiv{\_} nur von einem \before{\_}
Event gesetzt und von einem \after{\_} Event zurückgesetzt werden kann.
Ausserdem sollten ausserhalb des Intervalls keine isolierten Events auftreten.

\section{Operatoren}
Im folgenden werden einige Operatoren sowie ihre mögliche Imlementierung und
evtl. Schwächen des jeweiligen Ansatzes diskutiert.

\subsection{Komplement comp\Interval : Intervall}
\begin{itemize}
\item \before{\comp{\Interval}} <=> \after{\Interval}
\item \after{\comp{\Interval}} <=> \before{\Interval} \&\& ! \aktiv{\Interval}
\item \aktiv{\comp{\Interval}} <=> ! \aktiv{\Interval}
\end{itemize}
Probleme: An den Schnittpunkten ist ein Event laut dieser Definition sowohl in
int als auch \comp{\Interval}. Bei verschachtelten Events gibt es unerwünschte,
isolierte before(\comp{\Interval})-Events.

mögliche Lösungen: Komplement als Methode von Intervall Event überladbar
machen, und in Spezialfällen gesondert behandeln. Für das Schnittpunkt-Problem
gibt es (noch) keine Lösung, ist evtl. auch positiv (siehe Differenz)

Bemerkung: die zusätzlichen Bedingungen an after und active existieren aufgrund
der Nicht-Standard-Intervalle. Die explizite Definition von active ist wichtig damit das Komplement auch anfangs aktiv sein kann.

\subsection{Vereinigung \Interval{1} || \Interval{2} : Intervall}
Die Semantik der ||-Vereinigung soll eine Vereinigung der Zeitpunkte von \Interval{1}
und \Interval{2} sein, also w"are, falls \Interval{1} ein Exec Event wäre, \Interval{1} || \Interval{1} ein Standard Intervall.
\begin{itemize}
\item \before{\Interval{1} || \Interval{2}}  <=> (\before{\Interval{1}} ||
\before{\Interval{2}}) \&\& !
\aktiv{\Interval{1} || \Interval{2}}
\item \after{\Interval{1} || \Interval{2}} <=> ((\before{comp(\Interval{1})}
\&\& ! \aktiv{\Interval 2}) || (\before{\comp{\Interval{2}}} \&\& !
\aktiv{\Interval{1}}) || (\before{\comp{\Interval{1}}} \&\&
\before{\comp{\Interval{2}}})) \textbackslash \before{\Interval{1} ||
\Interval{} 2}
\end{itemize}
Anmerkung: statt \after{\Interval{1}} wurde hier \before{\comp{\Interval{1}}}
verwendet. In der jetzigen Fassung ist dies "aquivalent, allerdings bietet diese Formulierung den
Vorteil, dass falls die Probleme für Komplement gelöst werden, auch zum größten
Teil die Probleme mit Vereinigung verschwinden.

Problem: Verschachtelte Events, äahnlich wie bei Komplement. Eine Lösung für
das Problem mit Komplement löst auch dieses Problem.

\subsection{Schnitt \Interval{1} \&\& \Interval{2} : Intervall}
\begin{itemize}
\item \before{\Interval{1} \&\& \Interval{2}} <=> (((\before{\Interval{1}} \&\& \aktiv{\Interval{2}}) ||
(\before{\Interval{2}} \&\& \aktiv{\Interval{1}}) || (\before{\Interval{1}} \&\& \before{\Interval{2}}))
\textbackslash (\after{\Interval{1}} || \after{\Interval{2}})) \&\& ! \aktiv{\Interval{1} \&\& \Interval{2}}
\item \after{\Interval{1} \&\& \Interval{2}} <=> \before{\comp{\Interval{1}}} ||
\before{\comp{\Interval{2}}}
\item \aktiv{\Interval{1} \&\& \Interval{2}} <=> \aktiv{\Interval{1}} \&\& \aktiv{\Interval{2}}
\end{itemize}
Anmerkungen: zu \before{\comp{\_}} siehe Anm. zur Vereinigung

\subsection{Differenz \Interval{1} \textbackslash\ \Interval{2} : Interval}
Alle Zeitpunkte, die in \Interval{1} liegen, nicht aber in \Interval{2} (ausser Start bzw.
Endzeitpunkt, siehe Anm.). Wegen dieser Einschränkung kann es sinnvoll sein,
eine Bedingung an einen Zeitpunkt über eine Verbindung von within und !within
auszudrücken.

\Interval{1} \textbackslash\ \Interval{2} <=> \Interval{1} \&\& comp(\Interval{2})

Anmerkung: Wie bei Komplement erwähnt, sind die Eckpunkte von \Interval{1} und \Interval{2}
evtl. fälschlicherweise mit in \Interval{1} \textbackslash\ \Interval{2}. Dies
lässt sich nicht lösen, solange keine Möglichkeit gefunden wird, offene Intervalle zu
modellieren (z.B. über explizite before-Trigger o.ä.). U.U. ist dies aber auch
keine schlechte Eigenschaft, so dass ein Wechsel int -> \comp{\Interval{}}
atomar zu einem bestimmten Zeitpunkt stattfindet.

\subsection{within(\Interval{},e) : Event}
within(\Interval{},e) <=> (e \&\& \aktiv{\Interval}) || (e \&\& \before{\Interval})

\subsection{!within(\Interval{},e) : Event}
!within(\Interval{},e) <=> (e \&\& ! \aktiv{\Interval}) \textbackslash\ \before{\Interval}

\subsection{ StrictlyWithin(\Interval{},e) : Event}
StrictlyWithin(\Interval{},e) <=> (e \&\& \aktiv{\Interval}) \textbackslash\ \after{\Interval}

\subsection{ !strictlyWithin(\Interval{},e) : Event }
!strictlyWithin(\Interval{},e) <=> (e \&\& ! \aktiv{\Interval}) || (e \&\& \after{\Interval})

Problem: Auch hier werden bei verschachtelten Intervallen multiple \after{\_} Events mit aufgenommen.

\subsection{weitere Anmerkungen}
Der Umstand, dass sich int und \comp{\Interval{}} zwei Zeitpunkte teilen, mag
unintuitiv erscheinen, allerdings  bietet dies nicht nur Nachteile. So ist int
|| \comp{\Interval{}} wie  erwartet immer aktiv, beginnt und endet niemals. int
\&\& \comp{\Interval{}} ist niemals aktiv  und löst auch niemals before oder
after Ereignisse aus.

  \section{An Events gebundene Werte}
Es ist möglich, beliebige Werte an Events zu binden. Diese Möglichkeit soll
auch für Intervall-Events existieren. Wann im Intervall welche Werte gelten,
hängt von der Art des Intervalls ab. 

Auf den aktellen Wert des Intervalls wird über die Eigenschaft val zugegriffen.
Wenn also nachfolgend <Intervall>.val geschrieben steht, soll auf den aktuellen
Wert des Intervall-Events zugegriffen werden.

Nachfolgend wird das Standartverhalten der Intervall-Operatoren dargestellt.
Natürlich kann das Verhalten der Operatoren bei Bedarf abgeändert werden. Dann
muss beim Intervall-Event-Knoten eine Benutzerdefinierte Merge-Funktion
angegeben werden, die Values der Events aggregiert.

\subsection{Intervall zwischen dem Event B und dem Event A)}
Für between-Intervalle wie in Abbildung \ref{interval-between_b_a} gilt der Wert des
Before-Events bis zum End-Event. Dannach hat das Intervall den Wert null. Wenn der Wert des After-Events benötigt
wird, kann dieser über das Komplement des Intervalls erhalten werden. Beim
Intervall\ref{interval-between_b_a} gilt also:
\[
val(t) = \begin{cases}
null & t < 1 \\
B.val & t >= 1 \; and \; t < 9 \\
null & t >= 9
\end{cases}
\]

\begin{figure}[h]
 \centering 
\psset{unit=1cm}
\begin{pspicture}(0,0)(10,2)
\psgrid[subgriddiv=1,griddots=10,gridlabels=7pt](0,0)(10,1)
%{
%	\pspolygon[fillcolor=greenyellow,fillstyle=solid,linestyle=none](0,0)(0,1)(1,1)(1,0)

	\rput(1.2,0.7){\psframebox*[framearc=.3]{B}}
	\rput(9.2,0.7){\psframebox*[framearc=.3]{A}}
	\psline[linewidth=1pt]{[-]}(1,0.5)(9,0.5)
%}
\end{pspicture}
\caption{Intervall-Event between(B,A)}
\label{interval-between_b_a}
\end{figure}



Wenn die \before{\_}-Events mehrfach auftreten, hat das
between-Intervall als Wert den Wert des ersten \before{\_} oder des ersten
\after{\_}-Events. Ein Beispiel hierfür ist in Abbildung
\ref{interval-between_b_a-multiple} zu sehen. Für
Interval\ref{interval-between_b_a-multiple} gilt \[ val(t) =
\begin{cases}
null & t < 1\\
B.val &  t >= 1 \; and \; t < 7\\
null  & t >= 7
\end{cases}
\]

Für das
Komplement des Interval\ref{interval-between_b_a-multiple} gilt:
\[
comp.val = \begin{cases}
null & t < 7\\
A.val & t >= 7
\end{cases}
 \]
\begin{figure}[h]
 \centering 
\psset{unit=1cm}
\begin{pspicture}(0,0)(10,2)
\psgrid[subgriddiv=1,griddots=10,gridlabels=7pt](0,0)(10,1)
%{
	\rput(1.2,0.75){\psframebox*[framearc=.3]{B}}
	\rput(7.2,0.75){\psframebox*[framearc=.3]{A}}
	\psline[linewidth=1pt]{[-]}(1,0.5)(7,0.5)
	
	\rput(3.2,0.75){\psframebox*[framearc=.3]{B'}}
	\rput(9.2,0.75){\psframebox*[framearc=.3]{A'}}
	\psline[linewidth=1pt,linestyle=dotted]{[-]}(3,0.5)(9,0.5)
%}
\end{pspicture}
\caption{Intervall-Event between(B,A) mit mehrfach auftretenden Before- und
After-Events}
\label{interval-between_b_a-multiple}
\end{figure}
 
 
\subsection{Vereinigung von Intervallen}
Für die Vereinigung von Intervallen ist es sinnvoll, den Wert des letzen
\before{\_}-Events zu speichern. Damit hat man immer Zugriff auf den Wert des
letzten Events, der die Bedingung erfüllt hat. 

\begin{figure}[h]
 \centering 
\psset{unit=1cm}
\begin{pspicture}(0,0)(10,4)
\psgrid[subgriddiv=1,griddots=10,gridlabels=7pt](0,0)(10,3)
%{
	\rput(1.2,2.75){\psframebox*[framearc=.3]{B}}
	\rput(7.2,2.75){\psframebox*[framearc=.3]{A}}
	\psline[linewidth=1pt]{[-]}(1,2.5)(7,2.5)
	
	\rput(3.2,1.75){\psframebox*[framearc=.3]{B'}}
	\rput(9.2,1.75){\psframebox*[framearc=.3]{A'}}
	\psline[linewidth=1pt]{[-]}(3,1.5)(9,1.5)
	
	\rput(1.2,0.75){\psframebox*[framearc=.3]{B}}
	\rput(9.2,0.75){\psframebox*[framearc=.3]{A'}}
	\psline[linewidth=1pt]{[-]}(1,0.5)(9,0.5)
%}
\end{pspicture}
\caption{Vereinigung zweier Intervall-Events: between(B,A) || between(B',A')}
\label{interval-or}
\end{figure}
  
Für das in Abb\ref{interval-or} zu sehende Interval\ref{interval-or}
between(A,B) || between(A',B') gilt also die folgende Belegung des Wertes:
\[
val(t) = \begin{cases}
null & t < 1 \\
A.val & t >=1 \; and t < 3 \\
A'.val & t >=3 \; and t < 9 \\
null & t > 9 \\
\end{cases}
\]

Und für das Komplement von Interval\ref{interval-or} gilt die Belegung:
\[
val(t) = \begin{cases}
null & t < 9 \\
A'.val & t >= 9
\end{cases}
\]
  
  \subsection{Schnittmenge zwischen Intervallen}
Bei der Schnittmenge zwischen zwei Intervall-Events werden die Event-Werte der
beiden aktiven Interval-Events zu einem Tupel zusammengefasst. 

\begin{figure}[h]
 \centering 
\psset{unit=1cm}
\begin{pspicture}(0,0)(10,4)
\psgrid[subgriddiv=1,griddots=10,gridlabels=7pt](0,0)(10,3)
%{
	\rput(1.2,2.75){\psframebox*[framearc=.3]{B}}
	\rput(7.2,2.75){\psframebox*[framearc=.3]{A}}
	\psline[linewidth=1pt]{[-]}(1,2.5)(7,2.5)
	
	\rput(3.2,1.75){\psframebox*[framearc=.3]{B'}}
	\rput(9.2,1.75){\psframebox*[framearc=.3]{A'}}
	\psline[linewidth=1pt]{[-]}(3,1.5)(9,1.5)
	
	\rput(3.2,0.75){\psframebox*[framearc=.3]{(B,B')}}
	\rput(7.2,0.75){\psframebox*[framearc=.3]{(A', A)}}
	\psline[linewidth=1pt]{[-]}(3,0.5)(7,0.5)
%}
\end{pspicture}
\caption{Schnittmenge zweier Intervall-Events: between(B,A) and
between(B',A')}
\label{interval-and}
\end{figure}

Wie in Abb. \ref{interval-and} zu sehen, ergibt sich das folgende Zeitliche
Verhalten für den Wert von Interval\ref{interval-and} \[
val(t)=\begin{cases}
null & t < 3 \\
(B,B') & t >=3 \; and t <= 7 \\
null & t > 7
\end{cases}
\]

Das Komplement von Intervall \ref{interval-and} hat die Belegung: \[
val(t) = \begin{cases}
null & t <= 7\\
(A', A) & t > 7
\end{cases} 
\]

\subsection{Differenz zweier Intervall-Events}
Bei der Differenz zwischen zwei Intervall-Events ist der Wert des Intervalls das
letzte Before-Event vor oder nach dem Schnitt. Für de

\begin{figure}[h]
 \centering 
\psset{unit=1cm}
\begin{pspicture}(0,0)(10,4)
\psgrid[subgriddiv=1,griddots=10,gridlabels=7pt](0,0)(10,3)
%{
	\rput(1.2,2.75){\psframebox*[framearc=.3]{B}}
	\rput(4.2,2.75){\psframebox*[framearc=.3]{A}}
	\psline[linewidth=1pt]{[-]}(1,2.5)(4,2.5)
	
	\rput(6.2,2.75){\psframebox*[framearc=.3]{B''}}
	\rput(9.2,2.75){\psframebox*[framearc=.3]{A''}}
	\psline[linewidth=1pt]{[-]}(6,2.5)(9,2.5)
	
	\rput(3.2,1.75){\psframebox*[framearc=.3]{B'}}
	\rput(5.2,1.75){\psframebox*[framearc=.3]{A'}}
	\psline[linewidth=1pt]{[-]}(3,1.5)(5,1.5)
	
	\rput(1.2,0.75){\psframebox*[framearc=.3]{B}}
	\rput(3.2,0.75){\psframebox*[framearc=.3]{B'}}
	\psline[linewidth=1pt]{[-]}(1,0.5)(3,0.5)
	
	\rput(6.2,0.75){\psframebox*[framearc=.3]{B''}}
	\rput(9.2,0.75){\psframebox*[framearc=.3]{A''}}
	\psline[linewidth=1pt]{[-]}(6,0.5)(9,0.5)
%}
\end{pspicture}
\caption{Schnittmenge zweier Intervall-Events: between(B,A) and
between(B',A')}
\label{interval-diff}
\end{figure}

Für Interval\ref{interval-diff} ergibt sich also folgende Belegung:
\[
val(t)=\begin{cases}
null & t < 1 \\
B & t \in [1,3] \\
null & t \in [3,6] \\
B'' & t\in[6,9]\\
null & t >9
\end{cases}
\]

Für das Komplement von Interval\ref{interval-diff} ergibt sich der zeitliche
Werteverlauf:\[
val(t)= \begin{cases}
null & t < 3\\
B' & t \in [3,5]\\
null & t > 6
\end{cases}
\]
\listoffigures\addcontentsline{toc}{section}{\listfigurename}
=======
Sei before(int) : Event das Ereignis, das den Beginn des Intervalls int angibt und after(int) : Event jenes, welches das Ende von int angibt. Dies kann in unterschiedlicher Form erfolgen:
\begin{itemize}
\item einmal before(\_) zu Beginn und einmal after(\_) am Ende (Standard in der jetzigen Fassung)
\item mehrfaches before(\_) und ein after(\_) am Ende (Use case? Wohl meist nicht erw"unscht)
\item verschachtelte, paarweise passende before(\_) und after(\_) Ereignisse (Bsp. Exec)
\item beliebige before(\_) und after(\_) Events innerhalb des Intervals
\end{itemize}
Ausserdem gebe active(int) : boolean an, ob das Intervall \"aktiv\"\  ist. In der jetzigen Fassung bedeutet dies einfach, ob man sich zeitlich innerhalb des Intervalls befindet. Das Umschalten wude bisher als Sink realisiert. Dies f"uhrt jedoch zu Indeterminismus, sodass e \&\& within(between(e,e')) beim ersten Vorkommen von e nicht unbedingt immer ausgef"uhrt wird. Dies ist unsch"on und sollte vermieden werden.

Daher sollte active eine andere Semantik erhalten und als reaction aktualisiert werden. So kann sichergestellt werden, dass der Zustand von active zu jedem Zeitpunkt bekannt ist. Die Semantik "andert sich also von [----int----] zu ]-----int---]. Dies ist keine Einschr"ankung, wie im weiteren gezeigt werden wird.

Für alle Intervalle muss gelten, dass active nur bei einem before(\_) oder after(\_) Event umgeschaltet wird, wobei active nur von einem before(\_) Event gesetzt und von einem after(\_) Event zurückgesetzt werden kann. Ausserdem sollten ausserhalb des Intervalls keine isolierten Events auftreten.

\section{Operatoren}
Im folgenden werden einige Operatoren sowie ihre m"ogliche Imlementierung und evtl. Schw"achen des jeweiligen Ansatzes diskutiert.

Spezielle Operatoren:
\begin{itemize}
\item merge(int) : macht ein Intervall zu einem Standard-Intervall
\end{itemize}

\subsection{Komplement comp(int) : Intervall}
\begin{itemize}
\item before(comp(int)) <= after(int)
\item after(comp(int)) <= before(int) \&\& ! active(int)
\item active(comp(int)) <= ! active(int)
\end{itemize}
Probleme: An den Schnittpunkten ist ein Event laut dieser Definition sowohl in int als auch comp(int). Bei verschachtelten Events gibt es unerw"unschte, isolierte before(comp(int))-Events.

m"ogliche L"osungen: Komplement als Methode von Intervall Event "uberladbar machen, und in Spezialf"allen gesondert behandeln. F"ur das Schnittpunkt-problem gibt es (noch) keine L"osung, ist evtl. auch positiv (siehe Differenz)

Bemerkung: die zus"atzlichen Bedingungen an after und active existieren aufgrund der Nicht-Standard-Intervalle. Die explizite Definition von active ist wichtig damit das Komplement auch anfangs aktiv sein kann.

\subsection{Vereinigung int1 || int2 : Intervall}
Die Semantik der ||-Vereinigung soll eine Vereinigung der Zeitpunkte von int1 und int2 sein, also w"are, falls int1 ein Exec Event w"are, int1 || int1 ein Standard Intervall.
\begin{itemize}
\item before(int1 || int2)  <= (before(int1) || before(int2)) \&\& ! active(int1 || int2)
\item after(int1 || int2) <= ((before(comp(int1)) \&\& ! active(int2)) || (before(comp(int2)) \&\& ! active(int1)) || (before(comp(int1)) \&\& before(comp(int2)))) \textbackslash before(int1 || int2)
\end{itemize}
Anmerkung: statt after(int1) wurde hier before(comp(int1)) verwendet. In der jetzigen Fassung ist dies "aquivalent, allerdings bietet diese Formulierung den Vorteil, dass falls die Probleme f"ur Komplement gel"ost werden, auch zum gr"o\ss ten Teil die Probleme mit Vereinigung verschwinden.

Problem: Verschachtelte Events, "ahnlich wie bei Komplement. Eine L"osung f"ur das Problem mit Komplement l"ost auch dieses Problem.

\subsection{Schnitt int1 \&\& int2 : Intervall}
\begin{itemize}
\item before(int1 \&\& int2) <= (((before(int1) \&\& active(int2)) || (before(int2) \&\& active(int1)) || (before(int1) \&\& before(int2))) \textbackslash (after(int1) || after(int2))) \&\& ! active(int1 \&\& int2)
\item after(int1 \&\& int2) <= before(comp(int1)) || before(comp(int2))
\item active(int1 \&\& int2) <= active(int1) \&\& active(int2)
\end{itemize}
Anmerkungen: zu before(comp(\_)) siehe Anm. zur Vereinigung

\subsection{Differenz int1 \textbackslash\ int2 : Interval}
Alle Zeitpunkte, die in int1 liegen, nicht aber in int2 (ausser Start bzw. Endzeitpunkt, siehe Anm.). Wegen dieser Einschr"ankung kann es sinnvoll sein, eine Bedingung an einen Zeitpunkt "uber eine Verbindung von within und !within auszudr"ucken.

int1 \textbackslash\ int2 <= int1 \&\& comp(int2)

Anmerkung: Wie bei Komplement erw"ahnt, sind die Eckpunkte von int1 und int2 evtl. f"alschlicherweise mit in int1 \textbackslash\ int2. Dies l"asst sich nicht l"osen, solange keine M"oglichkeit gefunden wird, offene Intervalle zu modellieren (z.B. "uber explizite before-Trigger o."a.). U.U. ist dies aber auch keine schlechte Eigenschaft, so dass ein Wechsel int -> comp(int) atomar zu einem bestimmten Zeitpunkt stattfindet.

\subsection{within(int,e) : Event}
within(int,e) <= (e \&\& active(int)) || (e \&\& before(int))

\subsection{!within(int,e) : Event}
!within(int,e) <= (e \&\& ! active(int)) \textbackslash\ before(int)

\subsection{ StrictlyWithin(int,e) : Event}
StrictlyWithin(int,e) <= (e \&\& active(int)) \textbackslash\ after(int)

Problem: Bei verschachtelten Intervallen werden einzelne Zeitpunkte f"alschlicherweise herausgeschnitten. Deshalb m"oglichst nur auf StandardIntervalle anwenden. z.B. via comp(comp(\_)) oder einen Standard wrapper, siehe unten.

\subsection{ !strictlyWithin(int,e) : Event }
!strictlyWithin(int,e) <= (e \&\& ! active(int)) || (e \&\& after(int))

Problem: Auch hier werden bei verschachtelten Intervallen multiple after(\_) Events mit aufgenommen.

\subsection{weitere Anmerkungen}
Eine andere m"ogliche L"osung f"ur das before(comp(int)) Problem w"are es, statt Komplement zu "uberladen, einen Operator std(int) zu definieren, dessen Semantik ist, dass er die Standard-Entsprechung eines Intervalls zur"uckgibt, d.h. dass erf"ullt sein muss, dass before(int) niemals aktiviert wird, wenn das Intervall schon aktiv ist und das active(int) niemals true zur"uckliefert nach einem after(int) und vor einem before(int). So lie\ss e sich Komplement leicht implementieren ( after(std(int)); before(std(int)) ) und entsprechend die anderen Operatoren.

Der Umstand, dass sich int und comp(int) zwei Zeitpunkte teilen, mag unintuitiv erscheinen, allerdings bietet dies nicht nur Nachteile. So ist int || comp(int) wie erwartet immer aktiv, beginnt und endet niemals. int \&\& comp(int) ist niemals aktiv und löst auch niemals before oder after Ereignisse aus.

  \listoffigures\addcontentsline{toc}{section}{\listfigurename}
>>>>>>> DD added
\end{document}
